%-------------------------------------------------------------------------------
% yum_subsynth
%-------------------------------------------------------------------------------
%
% \file        yum_subsynth.tex
% \library     Documents
% \author      Chris Ahlstrom
% \date        2015-06-07
% \update      2016-02-29 (Leap Day)
% \version     $Revision$
% \license     $XPC_GPL_LICENSE$
%
%     Provides the SUBsynth section of yoshimi-user-manual.tex.
%
%-------------------------------------------------------------------------------

\section{SUBsynth}
\label{sec:subsynth}

   The \textsl{Yoshimi} SUBsynth dialog is a complex dialog for creating a
   subtractive-synthesis instrument,
   "SUBsynth" or "SUBnote" is a simple engine which makes sounds through
   subtraction of harmonics from white noise.  \cite{zyndoc}

\begin{figure}[H]
   \centering 
   \includegraphics[scale=0.9]{bottom-panel/instrument-edit/SUB/SUBsynth-parameters.jpg}
   \caption{SUBsynth Edit Dialog}
   \label{fig:subsynth_edit_dialog}
\end{figure}

   \begin{enumber}
      \item \textbf{AMPLITUDE} (section)
      \item \textbf{BANDWIDTH} (section)
      \item \textbf{FREQUENCY} (section)
      \item \textbf{OVERTONES} (section)
      \item \textbf{FILTER} (section)
      \item \textbf{Harmonics} (section)
      \item \textbf{Clear}
      \item \textbf{C}
      \item \textbf{P}
      \item \textbf{Close}
   \end{enumber}

\subsection{SUBsynth / AMPLITUDE}
\label{subsec:subsynth_amplitude}

   \begin{enumber}
      \item \textbf{Volume}
      \item \textbf{Vel Sens}
      \item \textbf{Pan}
      \item \textbf{Rand}
      \item \textbf{Reset (panning)} (red button)
      \item \textbf{Amplitude Env} (stock sub-panel)
   \end{enumber}

   \setcounter{ItemCounter}{0}      % Reset the ItemCounter for this list.

   \itempar{Volume}{subsynth!amplitude volume}
   SUBsynth Volume.

   Values: \texttt{1 to 127, 64*}

   \itempar{Vel Sens}{subsynth!amplitude vel sense}
   Velocity Sensing function, rightmost/max to disable.

   Values: \texttt{1 to 127, 64*}

   \itempar{Pan}{subsynth!amplitude pan}
   Global panning, leftmost/zero gives random panning.

   Values: \texttt{1 to 127, 64*}

   \itempar{Rand}{subsynth!amplitude random pan}
   Indicator for activation of random panning.

   \itempar{Reset (panning)}{subsynth!amplitude reset pan}
   Reset Panning.

   \itempar{Amplitude Env}{subsynth!amplitude volume}
   Amplitude Envelope.
   See section ...... for this stock sub-panel.

\subsection{SUBsynth / BANDWIDTH}
\label{subsec:subsynth_bandwidth}

   \begin{enumber}
      \item \textbf{BandWidth}
      \item \textbf{B.Width Scale}
      \item \textbf{Bandwidth Env}
   \end{enumber}

   \setcounter{ItemCounter}{0}      % Reset the ItemCounter for this list.

   \itempar{BandWidth}{subsynth!bandwidth}
   SUBsynth Bandwidth.
   Sets the bandwidth of each harmonic.

   Values: \texttt{1 to 127, 40*}

   \itempar{B.Width Scale}{subsynth!bandwidth scale}
   SUBsynth Bandwidth Scale.
   Sets how the bandwidth of each harmonic is increased according to the
   frequency. The default (0) increases the bandwidth linearly according to
   the frequency.

   Values: \texttt{0 to 127???}

   \itempar{Bandwidth Env}{subsynth!bandwidth envelope}
   SUBsynth Bandwidth.

   \begin{enumber}
      \item \textbf{Enabled}
      \item \textbf{A.val}
      \item \textbf{A.dt}
      \item \textbf{R.dt}
      \item \textbf{R.val}
      \item \textbf{Stretch}
      \item \textbf{frcR}
      \item \textbf{C}
      \item \textbf{P}
      \item \textbf{E}
   \end{enumber}

   \setcounter{ItemCounter}{0}      % Reset the ItemCounter for this list.

   \itempar{Enabled}{bandwidth!enable}
   Enable the panel.

   \itempar{A.val}{bandwidth!attack value}
   Attack value.
   We need to figure out what this means.

   Values: \texttt{0 to 127, 64*}

   \itempar{A.dt}{bandwidth!attack time}
   Attack duration. Attack time.

   Values: \texttt{0 to 127, 40*}

   \itempar{R.dt}{bandwidth!release time}
   Release time.

   Values: \texttt{0 to 127, 60*}

   \itempar{R.val}{bandwidth!release value}
   Release Value.
   Actually present only on the Frequency Env sub-panel.

   Values: \texttt{0 to 127, 64*}

   \itempar{Stretch}{bandwidth!stretch}
   Bandwidth Stretch. On lower notes make the bandwidth lower.

   Values: \texttt{0 to 127, 64*}

   \itempar{frcR}{bandwidth!forced release}
   Forced release.
   If this option is turned on, the release will go to the
   final value, even if the sustain level is not reached.

   Also present in this sub-panel are the usual \textbf{C}opy
   and \textbf{P}aste buttons that call up a copy-parameters or
   paste-parameters dialog, as well as a button
   to bring up the editor window.

   Values: \texttt{Off, On*}

\subsection{SUBsynth / FREQUENCY}
\label{subsec:subsynth_frequency}

   \begin{enumber}
      \item \textbf{Detune}
      \item \textbf{FREQUENCY Slider}
      \item \textbf{440Hz}
      \item \textbf{Eq.T}
      \item \textbf{Octave}
      \item \textbf{Detune Type}
      \item \textbf{Coarse Det.}
      \item \textbf{Frequency Env}
   \end{enumber}

   Category - Filter category: Analog/Formant/SVF ????

   \setcounter{ItemCounter}{0}      % Reset the ItemCounter for this list.

   \itempar{Detune}{subsynth!freq detune}
   Frequency Detune.  Fine detune?

   \itempar{FREQUENCY Slider}{subsynth!freq slider}
   Frequency Slider.

   \itempar{440Hz}{subsynth!freq 440hz}
   Frequency 440Hz.
   Fixes the base frequency to 440Hz.
   One can adjust it with detune settings.

   \itempar{Eq.T}{subsynth!freq eq t}
   Frequency Equalize Time.

   \itempar{Octave}{subsynth!freq octave}
   Frequency Octave.
   Octave Shift.

   \itempar{Detune Type}{subsynth!freq detune type}
   Frequency Detune Type.
   Sets the "Detune" and "Coarse Detune" behavior 

   \itempar{Coarse Det.}{subsynth!freq detune coarse}
   Frequency Coarse Detune, "C.Detune".

   \itempar{Frequency Env}{subsynth!freq env}
   Frequency Envelope Stock Sub-Panel.

   \begin{enumber}
      \item \textbf{Enable}
      \item \textbf{A.value} or \textbf{A.val}
      \item \textbf{A.dt}
      \item \textbf{R.dt}
      \item \textbf{R.val}
      \item \textbf{Stretch}
      \item \textbf{frcR}
      \item \textbf{C}
      \item \textbf{P}
      \item \textbf{E}
   \end{enumber}

   See section ....

\subsection{SUBsynth / OVERTONES}
\label{subsec:subsynth_overtones}

The harmonics settings controls the harmonic intensities/relative bandwidth.
Moving the sliders upwards increases the relative bandwidth.  Please note
that, if one increases the number of harmonics, the CPU usage increases. Right
click to set the parameters to default values. 

   \begin{enumber}
      \item \textbf{Overtones Position}
      \item \textbf{Par1}
      \item \textbf{Par2}
      \item \textbf{ForceH}
   \end{enumber}

   \setcounter{ItemCounter}{0}      % Reset the ItemCounter for this list.

   \itempar{Overtones Position}{subsynth!overtone position}
   Subsynth Overtones Position.

   Values: \texttt{Harmonic, ShiftU, ShiftL, PowerU, PowerL, Sine, Power, Shift}

\begin{figure}[H]
   \centering 
   \includegraphics[scale=1.0]{bottom-panel/instrument-edit/SUB/harmonic-type.jpg}
   \caption{Harmonic Type Dropdown}
   \label{fig:harmonic_type_dropdown}
\end{figure}

   \itempar{Par1}{subsynth!overtone par1}
   Subsynth Overtones Par1.

   Values: \texttt{0 to 127}

   \itempar{Par2}{subsynth!overtone par1}
   Subsynth Overtones Par2.

   Values: \texttt{0 to 127}

   \itempar{ForceH}{subsynth!overtone forceh}
   Subsynth Overtones ForceH.

   Values: \texttt{0 to 127}

\subsection{SUBsynth / FILTER}
\label{subsec:subsynth_filter}

   \begin{enumber}
      \item \textbf{Enabled}
      \item \textbf{Filter Params} (stock sub-panel)
      \item \textbf{Filter Env} (stock sub-panel)
      \item \textbf{Stereo}
      \item \textbf{Filter Stages}
      \item \textbf{Mag. Type}
      \item \textbf{Start}
   \end{enumber}

   \setcounter{ItemCounter}{0}      % Reset the ItemCounter for this list.

   \itempar{Enabled}{subsynth!filter enable}
   SUBsynth Filter Enabled.

   \itempar{Filter Params}{subsynth!filter params}
   Filter Params.  See
   \sectionref{subsubsec:filter_parameters_user_interface},
   which describes this stock sub-panel.

   \itempar{Filter Env}{subsynth!filter env}
   Filter Params.  See
   \sectionref{subsubsec:envelope_settings_for_filter},
   which describes this stock sub-panel.

   \itempar{Stereo}{subsynth!filter stereo}
   SUBsynth Stereo.
   Make the instrument stereo. The CPU usage goes up about 2 times.
   Is this really a FILTER item?

   \itempar{Filter Stages}{subsynth!filter stages}
   Filter Stages.  Filter Order.
   Sets the number of filter stages applied to white noise. This parameter
   affects the CPU usage. 

   Values: \texttt{0, 1, 2*, 3, 4, 5???}

   \itempar{Mag. Type}{subsynth!filter mag type}
   Magnitude Type. Type of magnitude settings (Linear/dBs) 

\begin{figure}[H]
   \centering 
   \includegraphics[scale=1.0]{bottom-panel/instrument-edit/SUB/mag-type.jpg}
   \caption{SUBSynth Magnitude Type Dropdown}
   \label{fig:subsynth_magnitude_type_dropdown}
\end{figure}

   Values: \texttt{Linear, -40dB, -60dB, -80dB, -100dB}

   \itempar{Start}{subsynth!filter start type}
   Start Type.
   How to start the filters.

\begin{figure}[H]
   \centering 
   \includegraphics[scale=1.0]{bottom-panel/instrument-edit/SUB/start-type.jpg}
   \caption{SUBsynth Start Type}
   \label{fig:subsynth_start_type}
\end{figure}

   Values: \texttt{Zero, RND, Max.}

\subsection{SUBsynth / Harmonics}
\label{subsec:subsynth_harmonics}

   This section consists of 64 sliders to control the amplitude of the narrow
   noise band at a given harmonic, and 64 sliders to control the bandwidth of
   each band.

   The top row of SUBsynth sliders sets the \textsl{relative} amplitude.  This
   used of the word "relative" is an important distinction, as the overall level
   of the output is normalised; all actual levels will be dependent on whichever
   is the highest.

   The bottom row sets the bandwidth of each harmonic. If one has just the
   fundamental, and drops the bandwith to the minimum, one gets very nearly a
   sinewave.  Set it to maximum and it is very obviously filtered noise.


%-------------------------------------------------------------------------------
% vim: ts=3 sw=3 et ft=tex
%-------------------------------------------------------------------------------
