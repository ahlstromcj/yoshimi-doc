%-------------------------------------------------------------------------------
% yoshimi-user-manual
%-------------------------------------------------------------------------------
%
% \file        yoshimi-user-manual.tex
% \library     Documents
% \author      Chris Ahlstrom
% \date        2015-04-20
% \update      2019-07-21
% \version     $Revision$
% \license     $XPC_GPL_LICENSE$
%
%     This document provides LaTeX documentation for yoshimi.
%
%-------------------------------------------------------------------------------

\documentclass[
 11pt,
 twoside,
 a4paper,
 final                                 % versus draft
]{article}

\frenchspacing

%-------------------------------------------------------------------------------
% docs-structure
%-------------------------------------------------------------------------------
%
% \file        yoshimi-docs-structure.tex
% \library     Documents
% \author      Chris Ahlstrom
% \date        2015-04-20
% \update      2015-09-02
% \version     $Revision$
% \license     $XPC_GPL_LICENSE$
%
%     This "include file" provides LaTeX options for a document.
%
%     Note that enumitem is an extension of enumerate, and comes from
%     Debian's texlive-latex-recommended package.
%
%-------------------------------------------------------------------------------

\usepackage{enumitem}         % setting the whitespace between and within lists
\setlistdepth{9}
% \setlist{nosep}             % spacing around the list
\setlist{noitemsep}           % spacing within the list

% \usepackage[dvipsnames]{xcolor} % provide more colors?

\usepackage{color}            % provide colors?

% \usepackage[usenames,dvipsnames,svgnames,table]{xcolor}

\usepackage{nameref}          % Provide references by name instead of number
\usepackage[colorlinks=true,linkcolor=webgreen,filecolor=webbrown,citecolor=webgreen]{hyperref}
\definecolor{webgreen}{rgb}{0,.5,0}
\definecolor{webbrown}{rgb}{.6,0,0}

\usepackage{url}              % Required for including URLs
\usepackage{hyperref}         % Required for including hyperlinks
\usepackage{amsthm}           % Helps avoid "destination with same
% \usepackage{cleveref}       % identifier" warnings?
\usepackage[hypcap]{caption}  % make labels point to figure, not the caption
% \usepackage{hypcap}         % make labels point to figure, not the caption
\usepackage[pdftex]{graphicx} % Required for including images
\graphicspath{{../images/}}   % Set the default folder for images
\usepackage{float}            % For more control of location of Figures
\usepackage{geometry}         % Page & text layout
\geometry{
  letterpaper,
  top=2.5cm,
  bottom=2.5cm,
  left=2cm,
  right=2cm
}

\usepackage{longtable}        % For making multi-page tables
\usepackage{makeidx}          % For making an index

% This package isn't available easily on CentOS:
%
% \usepackage[subtle]{savetrees} % For tightening document vertical spacing

\hypersetup{                  % HYPERLINKS
% draft,                      % Uncomment removes links (e.g. for B&W printing)
 colorlinks=true,
 breaklinks=true,
% bookmarks=true,
 bookmarksnumbered,
 urlcolor=webbrown,
 linkcolor=blue,              % RoyalBlue
 citecolor=webgreen,
 pdftitle={},
 pdfauthor={\textcopyright},
 pdfsubject={},
 pdfkeywords={},
 pdfcreator={pdfLaTeX},
 pdfproducer={LaTeX with hyperref and ClassicThesis}
}

% Make an "enumber" style that makes all levels of enumerated lists show
% arabic numerals.

\newlist{enumber}{enumerate}{10}
\setlist[enumber]{nolistsep,label=\arabic*.}

% Make "paragraph" a fourth level, and make it shown in the table of
% contents.

\makeatletter
\renewcommand\paragraph{\@startsection{paragraph}{4}{\z@}%
   {-2.5ex\@plus -1ex \@minus -.25ex}%
   {1.25ex \@plus .25ex}%
   {\normalfont\normalsize\bfseries}}
\makeatother
\setcounter{secnumdepth}{4} % how many sectioning levels to assign numbers to
\setcounter{tocdepth}{4}    % how many sectioning levels to show in ToC

% Provide a way of counting user interface items without putting them in an
% enumberation.

\newcounter{ItemCounter}

% Makes a numbered paragraph out of an item, and allows two index entries
% for it.

\newcommand{\itempar}[2] {
   \stepcounter{ItemCounter}
   \textbf{\arabic{ItemCounter}. #1.}
   \index{#1}
   \index{#2}
}

% Provides for two forms of an option, as might be shown in a man page.

\newcommand{\optionpar}[2] {
   \textbf{\texttt{#1}} \textbf{\texttt{#2}} \\
   \index{#1}
   \index{#2}
}

% Now deprecated in preference to \itempar

\newcommand{\settingdesc}[2] {
   \textbf{#1}
   \index{#1}
   \index{#2}
}

% Make a full reference to a figure using its number, its name, and its page
% number.  Very useful if you have a hard-copy of the document to deal with.

\newcommand{\figureref}[1] {
   figure~\ref{#1}
   "\nameref{#1}"
   on page~\pageref{#1}\ignorespaces
}

% Make a full reference to a section using its number, its name, and its page
% number.  Very useful if you have a hard-copy of the document to deal with.

\newcommand{\sectionref}[1] {%
   section~\ref{#1}
   "\nameref{#1}"
   on page~\pageref{#1}\ignorespaces
}

% Make a full reference to a "paragraph"  using its number, its name, and
% its page number.  Very useful if you have a hard-copy of the document to
% deal with.

\newcommand{\paragraphref}[1] {%
   paragraph~\ref{#1}
   "\nameref{#1}"
   on page~\pageref{#1}\ignorespaces
}

% Make a full reference to a table using its number, its name, and its page
% number.  Very useful if you have a hard-copy of the document to deal with.

\newcommand{\tableref}[1] {%
   table~\ref{#1}
   "\nameref{#1}"
   on page~\pageref{#1}\ignorespaces
}

% An attempt to reduce excess vertical space.  Does not work.  See the
% top of yoshimi-user-manual.tex instead.
%
% \setlength{\parindent}{0pt}
% \setlength{\parskip}{0pt}

% Space between floats. \dblfloatsep for 2 column format.
% \setlength{\floatsep}{8pt}

% Space above and below in-line text floats
% \setlength{\intextsep}{8pt}

% Space above float caption
% \setlength{\abovecaptionskip}{8pt}

% Space below float caption
% \setlength{\belowcaptionskip}{8pt}

% Change the fragction of the page that can be filled with graphics from 0.7
% to 0.9.

\renewcommand\floatpagefraction{.9}
\renewcommand\dblfloatpagefraction{.9}
\renewcommand\topfraction{.9}
\renewcommand\dbltopfraction{.9}
\renewcommand\bottomfraction{.9}

\raggedbottom                          % avoid excessive vertical justification

%-------------------------------------------------------------------------------
% vim: ts=3 sw=3 et ft=tex
%-------------------------------------------------------------------------------
         % specifies document structure and layout

% Replacing normal header/footer with a fancier version.  These two symbols of
% document class were showing up as "unused" in the log file.
%
% headinclude,
% footinclude,
%
% So we add the fancyhdr package, clear the default layout, and set it up for
% our wider pages.

\usepackage{fancyhdr}
\pagestyle{fancy}
\fancyhead{}
\fancyfoot{}
\fancyheadoffset{0.005\textwidth}
\lhead{Yoshimi Software Synthesizer}
\chead{}
\rhead{User Manual}
\lfoot{}
\cfoot{\thepage}
\rfoot{}

\makeindex

\begin{document}
\title{Yoshimi Advanced User Manual, v 1.6.1}
\author{Chris Ahlstrom \texttt{(ahlstromcj@gmail.com)} with Will J. Godfrey}
\date{\today}
\maketitle
Provisional
\begin{figure}[H]
   \centering
   \includegraphics[scale=1.00]{1.6.0/Front.png}
\end{figure}

\clearpage                             % moves Contents to next page

\tableofcontents
\listoffigures                         % print the list of figures
\listoftables                          % print the list of tables

% Change the paragraph style to remove indenting and put a line between each
% paragraph.  This could be moved up into the preamble, but then would
% affect the spacing of the TOC and LOF, LOT noted above.
%
% \setlength{\parindent}{0pt}

\setlength{\parindent}{2em}
\setlength{\parskip}{1ex plus 0.5ex minus 0.2ex}

\section{Introduction}
\label{sec:introduction}

   This manual was inspired by a wiki version of a
   \textsl{ZynAddSubFX} manual (see reference \cite{zynwiki}).  That wiki
   shows screen shots and a
   detailed survey of the settings and parameters of \textsl{ZynAddSubFX}.  It
   inspired us to thoroughly document \textsl{Yoshimi}, with the help of
   Will Godfrey, who continues to improve \textsl{Yoshimi} at great pace.
   This manual owes much to the descriptions and diagrams provided by the
   original \textsl{ZynAddSubFX} author, Paul Nasca, as well as some others
   whose names we don't know.

\subsection{Yoshimi And ZynAddSubFX}
\label{subsec:introduction_yoshimi_vs_zyn}

   \textsl{Yoshimi} is an algorithmic MIDI software synthesizer for Linux.
   It synthesizes in real time, can run polyphonic or monophonic, with multiple
   simultaneous patches on one or more MIDI channels, and has broad microtonal
   capability. It includes extensive addititive, subtractive,
   and pad synth capabilities which can be run simultaneously within the
   same patch.  It also has eight audio effects modules.

   This manual describes how to use \textsl{Yoshimi} \cite{yoshimi},
   the software synthesizer derived from the great
   \textsl{ZynAddSubFx} (version 2.4.0) \cite{zynaddsubfx} software
   synthesizer (Copyright 2002-2009 Nasca Octavian Paul).
   Because of their common origin, much of this manual also
   applies to \textsl{ZynAddSubFx} for versions less than 3.0,
   and uses some earlier \textsl{ZynAddSubFX} documentation and diagrams.
   Please note that the references to \textsl{ZynAddSubFx}
   in this manual apply specifically to versions of \textsl{ZynAddSubFx}
   prior to version 3.0.

%  continued for quite a while in its own direction. These include
%  major optimizations for audio and MIDI performance,
%  progressive development of user-level and command-line access to all
%  controls, MIDI learn, and vector control.
%  At the same time, refinement continues, both visually and within the code.
%
%  What are the advantages of
%  \textsl{Yoshimi} versus \textsl{ZynAddSubFX}?
%  At one time \textsl{Yoshimi} had better JACK support
%  than \textsl{ZynAddSubFX}, but that is no longer true.
%  Both projects are in active development, and both are
%  progressing along their respective roadmaps.
%  \textsl{Yoshimi} currently has a considerably
%  revised graphical user-interface,
%  while \textsl{ZynAddSubFX} has a major user-interface upgrade planned.
%  Each project has some features and capabilities that the other lacks.
%  There may be internal differences of importance to
%  a developer, as well as features that aren't so easy to discover.  So,
%  really, ... use them both!
%
%  -- the \textsl{ZynAddSubFx} project is planning
%  a major user-interface change for version 3.0, and many of the file formats
%  will change as well.  Of course, one can still continue to use the old
%  "Zyn" as well.

\subsection{New Features}
\label{subsec:introduction_new_features}

   This section provides an \textsl{ad hoc}, catch-as-catch-can survey of the
   new features of \textsl{Yoshimi}, in no particular order.
   There are new features for the command-line interface, and many internal
   fixes to reduce the likelihood of "xruns", static, clicks, or other
   performance issues.

\subsubsection{MIDI Learn}
\label{subsubsec:new_features_midi_learn}

   \textsl{Yoshimi}, as of version 1.5.0, supports MIDI Learn.
   It is available by right-clicking on the desired parameter widget.
   See \sectionref{subsubsec:concepts_midi_learn},
   and \sectionref{sec:midi_learn}, for more information.

\subsubsection{LV2 Plugin}
\label{subsubsec:new_features_lv2_plugin}

   \textsl{Yoshimi} can run as an LV2 plugin.  Supported features:

   \begin{enumber}
      \item Sample-accurate MIDI timing.
      \item State save/restore support via \textsl{LV2\_State\_Interface}.
      \item Working UI support via \textsl{LV2\_External\_UI\_Widget}.
      \item Programs interface support via \textsl{LV2\_Programs\_Interface}.
      \item Multi-channel audio output.
         'outl' and 'outr' have lv2 index 2 and 3.
         All individual ports numbers start at 4.
   \end{enumber}

   See \sectionref{sec:lv2_plugin}, for more information, but keep in
   mind there is still much more to document concerning the LV2 plugin.

\subsubsection{Control Automation}
\label{subsubsec:new_features_control_automation}

   Controls automation support is a part of a common
   controls interface.  There are significant extensions to the NRPNs that
   \textsl{Yoshimi} handles.
   Sensitivity to MIDI volume change (CC \#7) is variable in
   \textbf{Controllers} in the same way as pan width, etc. The numeric range is
   64 to 127; the default at 96 gives the same sensitivity as before at -12dB
   relative to the GUI controls.  127 gives 0dB and 64 gives -26dB.

   In parallel with this there are more NRPNs supported so that one can perform
   some of these controls via automation. The arrangment looks positively
   steam-punk, but is actually very easy to use, requiring only a utility that
   can send MIDI CCs.  NRPNs aren't special. They are simply a specific pattern
   of CCs.  \textsl{Yoshimi}'s implementation is very forgiving, doesn't mind
   if one stops halfway through (will just get on with other things while it
   waits) and will report exactly what it is doing.
   See \sectionref{sec:nrpns}, for more information, as well as the sections
   noted below.

\subsubsection{MIDI CC}
\label{subsubsec:new_features_midi_cc}

   To help when things don't seem to go right, one can show raw incoming
   CCs. This is enabled from the \textbf{Settings / MIDI} tab.
   These are the values before \textsl{Yoshimi} does any processing.

   MIDI program changes have always been pretty clean from the time Cal first
   introduced them, but now GUI changes are just as clean. While it is
   generally best to change a program when the part is silent, even if a part
   is sounding there is barely a click. There is no interference at all
   with any other sounding parts.

   Sometimes MIDI CCs don't seem to give the results one expects. There is
   a setting that reports all incoming CCs so that one can discover
   what \textsl{Yoshimi} actually sees (which may not be what was expected).

   \index{breath control}
   \index{MIDI CC!breath control}
   At the request of one user, \textsl{Yoshimi} (and only \textsl{Yoshimi})
   has an implementation of CC 2, Breath Control.
   This feature combines volume with filter cutoff.
   See \sectionref{subsubsec:concepts_midi_messages}, for more information.

   \index{legato}
   \index{MIDI CC!legato}
   \textsl{Yoshimi} implements the "legato footswitch" control,
   MIDI CC 68.
   Send this command with a value of 64 and above, and it will
   switch to Legato mode. Send less than 64, and it will revert to whatever it
   was before.  So, if the mode had been Poly, it goes back to that, and it it
   already was Legato, it just stays a Legato.

\subsubsection{Vectors}
\label{subsubsec:new_features_vectors}

   It's probably best to more clearly separate the concept of
   \textsl{parts} versus \textsl{channels} these days.
   \textsl{Yoshimi} can provide up to 64 parts, in blocks of 16. One can
   decide how many one wants to have available using the spin-box alongside
   the channel number.  One can have 16, 32 or 64 parts.  By default, all the
   upper parts are mapped to the same MIDI channel numbers as the lowest ones,
   but have independent voice and patch set values. They cannot normally
   receive independent note or control messages. However, vector control will
   intelligently work with however many are set, as will all the NRPN direct
   part controls.
   See \sectionref{sec:vector}, for more information.

\subsubsection{Bank Support}
\label{subsubsec:new_features_bank_support}

   Bank root directories are better identified, with IDs that can be changed by
   the user in the GUI. This is also made available for selecting over MIDI.
   MIDI only sees banks in the \textsl{current}
   \textbf{root} directory, but all banks are accessible to the GUI.
   One can set up a new bank root path when starting from the
   command line. This takes the form:

   \begin{verbatim}
      $ yoshimi -D /home/(username)/(directory)/(subdirectory)/bank
   \end{verbatim}

   \textsl{Yoshimi} will scan this path for new banks, but
   won't make the root (or any of its banks) current. The final directory
   doesn't have to be \texttt{banks}, but that is tradition.
   When running from the command line there is access to many
   of the system and root, bank, and other settings.

   \textsl{Yoshimi} splits out roots and banks from the main configuration
   file, and creates a new "history" file. The separation means that the
   different functions can be implemented, saved, and loaded at the most
   appropriate time. These files have "yoshimi" as the doc-type, as they are in
   not relevant to \textsl{ZynAddSubFX}.
   See the new Banks sections,
   \sectionref{sec:banks_and_roots}
   and
   \sectionref{subsec:banks_collection_yoshimi},
   for more information.

\subsubsection{Accessibility}
\label{subsubsec:new_features_accessibility}

   An important feature of recent releases of \textsl{Yoshimi} is improved
   accessibility. The effectiveness and usefulness of accessibility will
   shape future complementary interfaces. A number of first-time defaults
   have been changed to make accessibility easier.

   It has always been possible to run \textsl{Yoshimi} headless. Since V 1.5.11
   it possible to \textsl{compile} a headless version with no need for
   any graphical components to be installed.
   Real accessible control is available.

%  When starting from
%  the command line, an argument can be included for a new root path to be defined
%  to point to a set of banks fetched from elsewhere. This will be given the next
%  free ID.

   Once running, all setup can be done within the terminal window.
   Some settings will require a restart.
   There is also extensive control of
   roots, banks, parts and instruments including the ability to list and set
   all of these. One can do things like:

   \begin{verbatim}
      add root /home/music/yoshimi/banks
      set part 4 program 130
   \end{verbatim}

   Additional controls that are taken for granted in the GUI but
   otherwise forgotten are master key shift and master volume.  The
   most important parts of vector control are exposed to the command line.
   For all of this there is extensive error checking and feedback, which can be
   rendered in voice using text-to-speech software.

   There is a partially-sighted person we hear from, and
   a totally blind person (working with a Braille reader/writer) who has
   offered a lot of suggestions, and very much likes vector control.  So
   accessibility \index{accessibility} is an important feature of
   \textsl{Yoshimi}.  See the section that follows.

\subsubsection{Command Line}
\label{subsubsec:new_features_command_line}

   \textsl{Yoshimi} offers great control of one's working environment.
   One can have the graphical user interface, a command-line
   interface, or both, and these settings can be saved.  And both
   interfaces can be disabled, so \textsl{Yoshimi} runs in a headless
   mode, responding in the background to MIDI events.

   The command-line interface can access all top level controls, as well as the
   part editing controls, and can select any effect and effect preset.  It can be
   used to set up Vector Control much more quickly and easily than using NRPNs.
   It allows setting to be made to the various synthesis engines.
   The command-line is also context-sensitive, which, with careful choice
   of command names and abreviations, allows fast access with minimal
   typing.

   Since version 1.5.8 it has been possible to start, stop and select
   different instances for further control.

   \textsl{Yoshimi}'s parser is case-insensitive for commands (but not for
   filenames), and accepts the shortest unambiguous abbreviation. It is
   quite pedantic, and expects spelling to be correct regardless of length.
   Apart from the \texttt{back} commands, it is word-based, so spaces are
   significant.  Some examples:

   "\texttt{s p 4 pr 6}" ("set part 4 program 6"):
   This command sets part 4 to the instrument with ID 6 from the current bank.
   It also leaves one at the part context level and pointed to part 4.
   Additionally, the fact that it loaded an instrument means it will activate
   that part if it was off (and the configuration setting is checked). In most
   cases the words \textbf{program} and \textbf{instrument} are interchangable.

   "\texttt{s ef 1 rev}" ("set effect 1 reverb"):
   This command moves one to the effects context level and sets that part's
   effect number 1 to effect type \textbf{reverb}.

   "\texttt{s pre 2}" ("set preset 2"):
   This command sets preset number 2 (we use numbers here as most preset names
   repeat the effect type).

   "\texttt{..s 6 v 80}" ("up one level, set part 6 volume 80"):
   This command drops one back to part level, switches one to part 6 (but doesn't
   enable it), and sets its volume to 80.

   "\texttt{/s ve cc 93}" ("to top level, set vector control cc 93"):
   This command drops up to the top level, and sets vector control for
   channel 1, X axis to respond to CC 93 leaving one in the vector context.

   Whenever intermediate values are omitted, the default or last-used value
   will be assumed.
   All \textsl{standard} CLI inputs, and the return message numbers,
   should start from 1 with the following exceptions:

    \begin{itemize}
       \item Bank roots
       \item Banks
       \item CCs
       \end{itemize}

   These follow standard MIDI practice in the CLI and in the GUI.

   The CLI prompt always
   shows what level one is on, and the help-lists are also
   context-sensitive, so one doesn't see a lot of irrelevent clutter.
   \textsl{Yoshimi} instrument patches are fully
   compatible with \textsl{ZynAddSubFX} patches, and we have ported across
   new refinements with thanks to the Zyn authors.
   See \sectionref{sec:command_line}, for more information.

\subsubsection{Audio Support}
\label{subsubsec:new_features_audio_support}

   The preferred JACK/ALSA MIDI and audio interfaces are no longer fixed at
   compile time. There are checkboxes on \textbf{Settings} to change them.
   One can also set preferred startup ALSA/JACK MIDI and audio devices.
   These selections will be remembered on the next run.

   \textsl{Yoshimi} will always start even if the audio/MIDI backend called
   for doesn't exist. In this situation, it will try all combinations in this
   order: JACK, ALSA, and null. This enables one to then change the settings
   and try again.

   A significant improvement is the handling of ALSA audio, which is very
   important for some people.  \textsl{Yoshimi} can handle 2-channel 16-bit
   format. Tests have shown that virtually all motherboard sound chipsets will
   handle this setting, but many external ones don't.  So \textsl{Yoshimi}
   initially requests 32-bit 2-channel, then works towards a compromise with the
   hardware.
   See \sectionref{paragraph:menu_yoshimi_settings_alsa_tab}, for details.

\subsubsection{User Feedback}
\label{subsubsec:new_features_user_feedback}
   With any complex software it is important that the user is kept informed
   as to exactly what the current status is, and over time \textsl{Yoshimi}
   has had extensive development of tooltips, warnings and action reports.

   For example, since V 1.5.11 one is now warned if an effect is a modified
   version of a preset and not it's default. This is in the form of changing
   the background color of the preset button to a bright blue.

   When editing an AddSynth voice that is taking it's waveform from another one
   there is a new warning when actually entering the waveform window.

\subsubsection{Miscellany}
\label{subsubsec:new_features_miscellany}

   Where XML files are cross-compatible with \textsl{ZynAddSubFX},
   (such as instrument ones) \textsl{Yoshimi} stamps them with its own major
   and minor version numbers underneath the existing ones, so it is possible
   to tell which version created the files, or whether they were created by \textsl{ZynAddSubFX}. If the new files are not compatible (such as config)
   the doctype and all other details are identfied as purely for
   \textsl{Yoshimi}

   One can direct messages to either \texttt{stderr}
   (the error output of a terminal console) or the \textbf{Reports} window
   on the fly. If one chose \texttt{stderr}, the \textbf{Reports} window is
   inactive.

   One can use the mouse scroll wheel to adjust rotary controls. Holding
   down the Ctrl key gives access to finer adjustment.  Also, horizontal as
   well as vertical mouse movement will adjust the knob.

   All rotary controls and sliders will return
   to their home positions if right-clicked anywhere in the control.
%  In the case of sliders not the peg itself.
   As of version 1.5.8 switches, buttons and
   selectors will also home with a right-click.

   Part-editing windows carry the part number and voice name in the title bar.
   For the AddSynth oscillator window this also includes the voice number.

   When opening an instrument bank, one can tell exactly which synth
   engines are used by each instrument. This is represented by three pale
   background colours: Red = AddSynth; Blue = SubSynth; and Green = PadSynth.

   If the instruments are kits they scanned to find out if any member of the
   kit contains each engine, so that the colors above can be applied.
   This feature is duplicated in the current part, the mixer panel for the
   currently loaded instruments, and in the Instrument Edit window.
   The same colors highlight the engine names when they are enabled with the
   check boxes.

   \textsl{Yoshimi} remembers where major windows were last placed (per
   instance), and if any were left open at shutdown, they will be reopened at
   the same location on the next run.

   Thanks to the \textsl{ZynAddSubFX} developers, \textsl{Yoshimi} has pink
   as well as white noise available on Addsynth voices. Pink noise sounds
   softer.  In \textsl{Yoshimi} version 1.6.0 we add spot noise, which has a
   sort of 'gritty' sound.

   The \textbf{Humanise} feature has had more interest so it's been upgraded.
   It's now a pair of sliders, and its settings can be saved in patch sets.
   It provides a tiny per-note random detune and/or velocity attenuation to an
   \textsl{entire} part (all engines in all kits), but only for that part.

   Audio and MIDI preferences have been improved. If one sets ALSA MIDI
   and JACK audio from the GUI or the command line, the setting can be
   saved and will be reinstated on the next run. These settings are
   per-instance, so if one has multiple sound cards, one can make full use of
   them.  Barring major system failures, there are no circumstances where
   \textsl{Yoshimi} will fail to start.

   We've tested \textsl{Yoshimi} in recovery mode, logged in as root, with
   no X server.

   The command \texttt{/usr/share/bin/yoshimi -A -i} worked perfectly and
   auto-connected the keyboard. Since version 1.5.11 it one can provide
   a compile-time option to build a purely headless version.

   Load and Save dialogues recognise the history lists and offer
   the appropriate first choice. External instrument loads and saves are
   also remembered.
   For saves, on a restart, one is offered the home directory regardless
   of where \textsl{Yoshimi} was launched, but in the case of saving
   external instruments, one is always offered the name of the instrument
   in the currently selected part, prefixed with the home directory.

   There is a specific \textsl{State} menu item ("Save as Default") to save
   the current complete setup as the default. The configuration file is always
   saved to \textsl{Yoshimi}'s configuration directory,
   and will not show in history lists.

   If \textbf{Start With Default State} has been set, and a
   default state has been saved, a complete restart will load this state,
   and a master reset will load this file, instead of doing a
   first-time default reset.
   There is a "gotcha" with this, in that when saving the default state,
   one must \textsl{already} have set the \textbf{Start With Default State}
   switch, otherwise reloading
   the default state works \textsl{once}, but upon re-opening
   \textsl{Yoshimi}, the switch will be
   unchecked \textemdash that is, quite correctly in its previously saved state!

   A final detail with the history lists is that in each list type, the last
   used item will be placed at the top of the list. This is especially useful
   when wanting to continually save/load an item currently being worked on.
   However, since version 1.6.0 it has been possible to freeze these lists, so
   benefiting the (also new) MIDI NRPN control to load settings from these lists.

\subsection{Document Structure}
\label{subsec:introduction_document_structure}

   The structure of any manual is a struggle.
   There's no way to avoid jumping all over the place to
   cover a topic.  The sections are covered roughly
   in the order of the user interface of
   \textsl{Yoshimi}.  To help the reader jump around this manual, multiple
   links and an index are supplied.

   \index{tips!in document}
   Usage tips
   for each of the functions provided in
   \textsl{Yoshimi} are sprinkled throughout this manual.
   Each tip occurs in a section beginning with "Tip:".
   Each tip is provided with an entry in the Index, under the
   main topic "tips".
   \index{bugs!in document}
   Bug notes may also be found.
   Each bug occurs in a sentence beginning with "Bug:".  Each bug is
   provided with an entry in the Index, under the main topic "bugs".
   \index{new!in document}
   New features
   since the last version are flagged with "New:"  We cannot pretend to have
   marked all new developments, as \textsl{Yoshimi} is advancing fast.
   \index{todo!in document}
   To-do items are also present, in the same vein.

\subsection{Yoshimi Mailing List}
\label{subsec:introduction_mailing_list}

   The \textsl{Yoshimi} project used to have an email listserv at
   SourceForge, but the unreliability of the site has prompted a move to a
   new mailing list.  See reference \cite{yoshiminews}.  The team have
   managed to port across all the old yoshimi-user archives to this new
   site.  See reference \cite{yoshiminewsarchive}.

   Subscribe to the \textsl{Yoshimi} mailing list with an e-mail to:
   \url{yoshimi-request@freelists.org} or by visiting
   \url{http://www.freelists.org/list/yoshimi}.

   To post to the list, send an email to: \url{yoshimi@freelists.org}.
   The news archive is at: \url{https://www.freelists.org/archive/yoshimi}.

\subsection{Yoshimi Licensing}
\label{subsec:introduction_yoshimi_licensing}

   \begin{quotation}
      Software licenses are are something I \textsl{really}
      don't want to get involved in - I have much better things to do with my
      time - but I found I was obliged to do so.

      It is possible I'm the only person who knows all the following events, as
      I was the one that instigated them!

      The first time I saw ZynAddSubFX source files they were licensed as GPL
      V2. At that time Zyn had a number of very serious problems, and not much
      was being done about them. Somewhat naively I asked Lars Luthman if he
      would help, as he had offered a couple of small patches previously. His
      response was that he would not do any significant work, as he did not
      agree with the GPL V2 only license.

      I then contacted Paul, explaining the situation and asking if he would
      consider a change in the license to V2 or later. I was actually a bit
      surprised that he immediately agreed. When I next looked at the sources,
      the licenses on the files had indeed been updated, so I passed this
      information on.

      Unfortunately Paul forgot to update the website, but I wasn't especially
      concerned as it was only the files themselves that really mattered.

      While developing Yoshimi after the initial fork, Cal queried the license
      situation. I told him of the conversations I'd had, and passed him a copy
      of the email I'd got from Paul. Later on, Cal - in good faith - wrote new
      sources and placed them under GPL V3. This would be quite compatible with
      V2 or later, but not with V2 strict.

      What I didn't notice until very much later was that Paul had only updated
      half of the text in the sources, leaving the actual licence in an
      ambiguous state.

      To the best of my knowledge, V3 is not compatible with V2 strict, but V2
      or later is. However the \textsl{complete}
      project then becomes downgraded to V2
      strict - although the V2 or later sources (such as all the new root/bank
      code) can independently be freely merged into V3 code.

      I doubt anyone would actually make an issue of this. However, to safeguard
      Yoshimi as a whole, I took it upon myself to change Cal's code to V2 or
      later. I believe it retains the spirit of his wishes, and the only person
      with standing to object -- his daughter - has been totally supportive of
      the work currently being done on Yoshimi.

      Any source code I add will be GPL V2 or later.
      \end{quotation}

      \textbf{Update:} The original change discussion has been located and the
      license for both Zyn and Yoshi is confirmed as GPL V2 or later. Anyone
      wanting to confirm this should look at the Zyn user list archives
      August 2007 and September 2007.


\subsection{Let's Get Started with Yoshimi!}
\label{subsec:introduction_lets_get_started}

   Let us run \textsl{Yoshimi}.
   The first
   thing to do is make sure one has no other sound application running
   (unless one wants to risk blocking \textsl{Yoshimi} or hearing two sounds
   simultaneously, depending on one's sound card and ALSA setup).
   Then start \textsl{Yoshimi}:

\begin{verbatim}
   $ yoshimi
\end{verbatim}

   If JACK is available, it will be used.  Otherwise ALSA will be used.

\begin{figure}[H]
   \centering
%  \includegraphics[scale=0.75]{Yosh-splash-OTW-1.png}
   \includegraphics[scale=0.75]{1.5.0/Splash-1-5-1.png}
   \caption{Yoshimi Splash Screen!}
   \label{fig:yoshimi_splash_screen}
\end{figure}

   One sees a brief message, and then the splash screen.
   We show the new splash screen, \figureref{fig:yoshimi_splash_screen},
   here because it goes away too fast when one runs \textsl{Yoshimi}!
   What fun is that?

   Next shown is the \textsl{Yoshimi} main window,
   as shown in \figureref{fig:yoshimi_main_screen},
   and it persists, of course:

\begin{figure}[H]
   \centering
%  \includegraphics[scale=1.0]{1.3.8/yoshimi-first-screen.jpg}
%  \caption{Yoshimi Main Screen, 1.3.8 and Above}
%  \includegraphics[scale=1.0]{1.4.0/yoshimi-first-screen.png}
   \includegraphics[scale=1.0]{1.6.0/Front.png}
   \caption{Yoshimi Main Screen}
   \label{fig:yoshimi_main_screen}
\end{figure}

    From \textsl{Yoshimi} version 1.5.11, on the very first time Yoshimi is
    started the main window will be overlaid with the following prompt. This
    is for the benefit of very new users who might not know where to start.

\begin{figure}[H]
   \centering
   \includegraphics[scale=0.75]{1.5.11/first_time_info.png}
   \caption{First Time Prompt}
   \label{fig:first_time_info}
\end{figure}


   Note that, if one presses the \texttt{Space} bar while the main window has
   keyboard focus (right after starting \textsl{Yoshimi}), the following prompt
   appears:

\begin{figure}[H]
   \centering
   \includegraphics[scale=0.75]{1.5.0/main_window_spacebar.png}
   \caption{Space Bar Prompt}
   \label{fig:yoshimi_main_spacebar_prompt}
\end{figure}

   Be careful!
   For this manual, the main window is composed of the following sections:

\begin{enumber}
   \item \textbf{Menu}.
   \item \textbf{Top Panel}.
   \item \textbf{Effects Panel}.
   \item \textbf{Bottom Panel}.  Includes the VU-meter at the bottom.
\end{enumber}

   \index{control!reset}
   \index{reset!control}
   Starting with version 1.4.0, a right-click on any control in \textsl{Yoshimi}
   will reset the control's value to its default position.  In version 1.4.1, the \textbf{Vectors} control appears, with some minor rearrangement of the top panel.

   There's a lot going on with \textsl{Yoshimi}, with no way to describe
   it in linear order.  This manual describes how to do useful things in
   each of the sections noted above, while leaving some of the details to be
   described in later sections, to which reference can be made for the details.
   This document depends heavily on index entries and references.
   There is also a "cookbook" at \cite{book}; it is a long way from being
   comprehensive, but still has some useful tips.

   If one downloads the source code for \textsl{Yoshimi}, in the
   \texttt{examples} directory one finds a complete song set,
   \texttt{OutThere.mid} and \texttt{OutThere.xmz}. Together these produce a
   fairly complex 12 part tune that makes \textsl{Yoshimi} work quite hard.
   Also, after installing \textsl{Yoshimi}, one can find a nice, short
   introduction to \textsl{Yoshimi} in Will's document,
   \texttt{/usr/share/doc/yoshimi/The Short Yoshimi Guide.odt},
   along with a number of texts files with information that might not yet be
   present in the long manual.

   One last thing to note is that there is a list of important concepts in
   \sectionref{sec:concepts}, which one should consult if a term is puzzling.

\rhead{\rightmark}         % shows section number and section name

% Configuration Files

%-------------------------------------------------------------------------------
% yum_configuration
%-------------------------------------------------------------------------------
%
% \file        yum_configuration.tex
% \library     Documents
% \author      Chris Ahlstrom
% \date        2016-03-07
% \update      2020-04-19
% \version     $Revision$
% \license     $XPC_GPL_LICENSE$
%
%     Provides descriptions of the configuration files.
%     Not yet part of the document.
%
%-------------------------------------------------------------------------------

\section{Configuration Files}
\label{sec:configuration}

   Let's cover the configuration files, which have expanded in utility in
   recent versions of \textsl{Yoshimi}.
   Understanding these configuration file makes it easier to
   use \textsl{Yoshimi}.
   Also note that all configuration settings are exposed to the command line
   interface as well.


   As with most applications, \textsl{Yoshimi} allows for one to save one's
   work and reload it. In recent versions of \textsl{Yoshimi}, it is possible
   to autoload a default state on startup, so that \textsl{Yoshimi} is
   already configured exactly as desired, with patches loaded and part
   destinations set.
   In addition, \textsl{Yoshimi} now saves settings that have been disabled.
   In this way, they can be re-enabled without having to reconstruct them from
   scratch.

   However, the configuration has changed quite a bit, and configurations from
   \textsl{Yoshimi} 1.4 and earlier will need to be reconstructed. With
   \textsl{Yoshimi} V 1.5.0 the following warning was devised, and will be updated
   with all major version number increments:

\begin{figure}[H]
   \centering
   \includegraphics[scale=1.0]{2.3.0/configwarn.png}
   \caption{Configuration Warning Dialog}
   \label{fig:config_warn_dialog}
\end{figure}

   \textsl{Yoshimi} has a number of different files that make up the current
   configuration.
   Together, they make up the concept of a \textsl{patch set} (also called a
   \textsl{patchset}).
   Sometimes one will see reference to a "session", but that term is too easy
   to confuse with the "session" in "JACK session manager".
%  Here are the file extensions used for saving the \textsl{Yoshimi}
%  patch-set data:

   The last-used file in any configuration section is always at the top of its
   history list.  The main benefit of this new setup is that now all patch
   sets, vectors, scales, MIDI Learn, and state - offer the most recent entry
   when asked to load or save. On first-time use, when there is no history, one
   is offered one's home directory as a location, regardless of where
   \textsl{Yoshimi} was called from.  Presets are not yet included in this
   process.

   When saving these "managed" files, one won't be offered the previous
   last-used configuration unless it was seen during that session, either by
   being loaded, or saved by name.  This protects against accidental
   overwrites....

   For example, you've been working on 'foo' for a whole day, saving as you go,
   then the following day you start up \textsl{Yoshimi}, and immediately have
   a completely new idea 'bar', and start working on it. Without thinking you
   save and hit Enter. Oops, you just wiped out 'foo'. Only you haven't!
   At startup \textsl{Yoshimi} would not see the older file,
   so the save command just offers the home directory to put a new name in.

   Here is a summary of the files.  Please note that the names all start with
   \texttt{yoshimi}.  For example, \texttt{.banks} is really
   \texttt{yoshimi.banks}.

   \begin{itemize}
      \item \texttt{.banks}
         \index{.banks}
         \index{config!.banks}
         Contains information on the accessible instrument banks, and
         information to translate between bank directory names and bank ID
         values.
         Current root and current bank settings have now been moved from
         \texttt{banks} to the \texttt{config} and \texttt{instance}
         files, so the banks file now consists only of the bank structure.
         On first time start up, \textsl{Yoshimi} will look for
         \textsl{ZynAddSubFX} banks as well as \textsl{Yoshimi} banks in the
         usual locations. It \textsl{will not} look for a \textsl{ZynAddSubFX}
         configuration file, as these are no longer relevant.
      \item \texttt{.config}
         \index{.config}
         \index{config!.config}
         Contains the setup information configured in the
         \textbf{Yoshimi / Settings} dialog.
         This is just the basic configuration file.
         Configuration instances are now in place, so the main configuration
         file is common to all, but each instance has its own file for things
         like current bank, JACK/ALSA settings, etc.
         Common overall settings are only visible in the main instance and
         completely hidden in all the later ones.
         In \textsl{Yoshimi} V 1.6.1 there has been further revision of this.
      \item \texttt{.instance(n)}
         \index{.instance(n)}
         \index{config!.instance(n)}
         Contains the current root/bank, MIDI settings, and preferred engines.
         \textsl{Yoshimi} now has instance data separated from the main
         configuration file, with the name \texttt{yoshimi-(n).instance}.
      \item \texttt{.state}
         \index{.state}
         \index{config!.state}
         Contains the information needed to duplicate a complete \textsl{Yoshimi}
         session that was saved.
      \item \texttt{windows}
         \index{windows}
         \index{config!windows}
         Contains the current layout of windows for re-instation at the next
         startup of \textsl{Yoshimi}.
         If there is no such directory
         (\texttt{\textasciitilde/.config/yoshimi/windows}) then the
         keyboard is also opened, alongside the main window, as a help to those
         new to \textsl{Yoshimi}.
         And of course that state will be saved, if present, when
         \textsl{Yoshimi} exits.
         This directory is specific to the GUI, so doesn't really figure in
         this scheme, but it is created or saved when one exits
         \textsl{Yoshimi}.
   \end{itemize}
   The entire config set should then be (ignoring the prepended
   \texttt{yoshimi}):

   \begin{itemize}
      \item \texttt{.config}
      \item \texttt{.instance[n]}
      \item \texttt{windows}
      \item \texttt{.banks}
   \end{itemize}

   Before \textsl{Yoshimi} V 1.7.1 there was a single \texttt{.windows}
   file but now the directory called \texttt{windows} contains individual
   files with the status of each window recognised. This is more reliable and
   less prone to errors under fault conditions. Also, the main config
   directory is more rational.

   Other \textsl{Yoshimi} specific files are:

   \begin{itemize}
      \item \texttt{.xiz}
         \index{.xiz}
         \index{.xiz!Legacy format}
         \index{config!.xiz}
         An Instrument file.  This format is the \textbf{Legacy} or \textbf{Zyn}
         file format.  This format will be supported forever, although some
         backward-compatible refinements might be made as time goes on.
      \item \texttt{.xiy}
         \index{.xiy}
         \index{.xiy!Yoshimi format}
         \index{config!.xiy}
         An Instrument file in the new \textsl{Yoshimi} file format.
         This format includes all the
         controllers, the part mode status (Poly, Mono, Legato) and the
         Humanise settings.
         When loading files, \textsl{Yoshimi} will always look for
         the \texttt{.xiy} version first, and,
         if it can't find it, will then look for the \texttt{.xiz} version.
      \item \texttt{.xly}
         \index{.xly}
         \index{config!.xly}
         \index{MIDI Learn!.xly}
         A MIDI-Learn file for saving the MIDI-Learn settings in force at the
         moment this file is saved.  It is also included in the state file
         (\texttt{.state}).
      \item \texttt{.xmz}
         \index{.xmz}
         \index{config!.xmz}
         All \textsl{Yoshimi} active data; everything except MIDI-Learn.
         This file is called a \textsl{patch set}.
      \item \texttt{.xpz}
         \index{.xpz}
         \index{config!.xpz}
         Presets.
         A preset is a \textsl{Yoshimi} section of data stored by of the copy
         (C) controls. This may be an effect, part of an instrument etc.
      \item \texttt{.xsz}
         \index{.xsz}
         \index{config!.xsz}
         Scale Settings.
      \item \texttt{.xvy}
         \index{.xvy}
         \index{config!.xvz}
         Vector settings. The extension stands for "Xml Vector Yoshimi".
         Vector settings are now included in both the patch sets
         (\texttt{.xmz}) and state files (\texttt{.state}.
         For a good example, see \sectionref{subsection:vector_command_line}.
   \end{itemize}

   In the file-save dialogs, the file extension is determined by the type of
   file being saved, and it doesn't matter if one enters the extension
   explicitly, or not. If it's missing, or it is the wrong one, it will be
   replaced. This is actually true of almost all file saves, and has been for
   quite some time now.

   For vectors (in common with external instruments and patch sets),
   the configuration is saved to the user's home directory.
%  it's up to the user as to where to save.
%  The file filter generally defaults to the
%  either the user home directory, or if \textsl{Yoshimi} was launched from
%  userland, it's the directory it launched from. Then it's the normal
%  file browser selection.
   Once saved, \textbf{Vectors / Options / Recent} is your friend.

 %  \textsl{Yoshimi} can set up critical configuration settings to be writable
%   only by the main instance, but readable (and used) by any others. In the
%   current version of \textsl{Yoshimi}, this applies to \textbf{AddSynth
%   Oscillator Size}, \textbf{Internal Buffer Size}, and \textbf{Alsa
%   Samplerate}. These three must be defined before any other initialisation.

\subsection{Configuration Files / Patch Set}
\label{subsec:configuration_patch_set}

   \index{.xmz}
   \index{config!.xmz}
   \index{patch set}
   \index{file!patch set}
   A patch set is basically a group of instruments related simply by the user
   wanting to have them all loaded at once into \textsl{Yoshimi}.  A patch set
   is stored in a \texttt{.xmz} file.  A patch set is akin to a preset, in that
   it stores a combination of items, that took awhile to set up, for easy
   retrieval later.

   Patch sets are not the full configuration. They carry \textsl{most} of it,
   including almost all of the dynamic settings, but they don't contain the
   configuration settings that \texttt{.state} does.  The patch set format is
   either XML or compressed XML, as explained elsewhere.  The
   \textbf{Patch Sets / Save External...} menu entry saves files with
   the \texttt{.xmz} extension.

   One of the simplest ways to save one's work is to save the bulk of the
\textsl{Yoshimi} dynamic settings.
   This saving can be done through the \textbf{Patch Sets} menu,
   and will result in the creation of
   a \texttt{.xmz} file. Once created, this file will hold the settings for
   all settings within that setup, such as microtonal tunings, all
   patches, system effects, insertion effects, etc.
   See \sectionref{paragraph:menu_yoshimi_settings_main_settings}.
   Patch sets will save all other instruments regardless of whether they are
   activated or not.

   In many cases saving everything in a part is not what is desired.
   Saving a patch later on in an editing session is one such example.
   In order to save a patch, one can either save it from the
   \textbf{Instruments} menu, or through the \textbf{Bank} window.

\subsection{Configuration Files / Config}
\label{subsec:configuration_config}

   \index{.config}
   \index{config}
   \index{file!config}
   Often, one will see the extension \texttt{.config} used in the
   \texttt{\$HOME/.config/yoshimi} directory.  This file once contained
   information to translate between bank directory names and bank ID
   values.  In recent versions of \textsl{Yoshimi}, this file is much
   reduced in size, and its "doctype" is "Yoshimi".

   The \texttt{.config} file is always going to be specific to one machine and
   working modes, so no one will ever want to copy it across even to another
   \textsl{Yoshimi} environment.  Recent patch sets are now no longer stored in
   the main \texttt{.config} file, but in a new \texttt{recent} file.

%  see \sectionref{sec:local_data}.

   The \texttt{.config} file is now a much reduced common startup settings
   file.  It is a single file that every instance can read, but only the
   first one can write.

   The \texttt{.config} file has been separated from \texttt{.instance(n)}.
   It is saved only when the user explicitly calls for it to be saved. If
   it is missing for some reason when restarting, \textsl{Yoshimi} will
   report the fact and take corrective action.

   \begin{verbatim}
      $ yoshimi -a -A
      Yoshimi is starting
      ConfigFile /home/ahlstrom/.config/yoshimi/yoshimi.config not found, will
         use default settings ...
   \end{verbatim}

   The \texttt{.config} file will be readable by all instances of
   \textsl{Yoshimi}, but writable only by the main instance. The relevant
   controls will be hidden from the other instances.  Also, those controls not
   relevant to LV2 are disabled in that mode.  The \texttt{.config} and
   \texttt{.banks} data now reside in separate configuration files.  The banks
   file is saved every time there is a normal exit, so the last-used root and
   bank IDs will always match what that instance thinks is there.  Conversely,
   the main \texttt{.config} file \textsl{doesn't} get saved when one starts a
   new (unknown) instance of \textsl{Yoshimi}, but the config-changed flag is
   set, so one has control over whether any settings are saved.  So now, if
   anything goes wrong with the config files they won't corrupt one's carefully
   organised bank files, and vice-versa.

\subsection{Configuration Files / State}
\label{subsec:configuration_state}

   \index{.state}
   \index{config!.state}
   \index{state}
   \index{file!state}
   Sometimes one will see the extension \texttt{.state} used in the
   \texttt{\$HOME/.config/yoshimi} directory.  These files contain a lot
   more information that is needed to duplicate a \textsl{Yoshimi} session
   that was saved.  This file is a superset of an \texttt{.xmz} file,
   saving everything.
   The state file is accessed from the \textbf{State} menu item in the main
   window.
   Its default name is
   \texttt{\textasciitilde/.config/yoshimi/yoshimi.state}.
   This file can be auto-loaded when \textsl{Yoshimi} starts, if it is
   present and the feature enabled. Otherwise the normal settings are in
   place.

   The advantage of this is that one can set up a complete patch set of
   instruments one commonly uses, with all their settings, including audio
   destination.  Save it to the default state and it will be loaded, along
   with the system settings, every time one starts \textsl{Yoshimi}, if the
   \textbf{Yoshimi / Settings / Switches / Start With Default State}
   setting is checked. To revert the state, simply uncheck the
   \textbf{Yoshimi / Settings / Switches / Start With Default State}
   setting (and change any other needed).

   Since \textsl{Yoshimi} V 1.6.1 one specifically must \textbf{not} save
   the settings when changing this. It is part of the reorganisation.
   State, defaults, and jack session files have all been considerably
   improved, and behave in a more transparent fashion. Also, the instances
   have more controls available that can be actively restored.

   The \textsl{Yoshimi} 'state' file consists of the entire setup, (apart
   from basic configuration settings), instance to currently-loaded
   instrument sets. However, upon investigating some session managers, it
   looks like they don't want (or can't use) some of the configuration
   information because they are expecting to be able to change the entire
   state in \textsl{running} instances.

   \textsl{Yoshimi} now splits the 'instance' data
   from the main configuration.  This solves this session issue
   by saving only the true configuration locally, and to the state save.
   However, the 'instance' data includes things like ALSA/JACK settings.
   Since \textsl{Yoshimi} V1.6.1 one can change these
   \textsl{at the time the instance starts}, but there is a potential
   'gotcha'. If using the JACK session manager (and possibly others) it is
   possible to save a session to switch to ALSA audio when run, and this
   will be done. However, at that point the session manager loses contact
   with \textsl{Yoshimi} and it is no longer possible to save anything
   back to that session for that instance (assuming the session manager can
   handle multiple instances).

\subsection{Configuration Files / Instrument}
\label{subsec:configuration_instrument}

   \index{.xiz}
   \index{config!.xiz}
   \index{instrument}
   \index{file!instrument}
   An Instrument.  These files can have two formats, compressed and
   uncompressed.
   Uncompressed is set by
   \textbf{Yoshimi / Settings / Main Settings / XML Compression Level} set to
   0, and compressed is set by a value greater than 0.

   With the \textbf{Instrument} menu, one can save the file to any
   given location with the \texttt{.xiz} and/or \texttt{.xiy} extension.

   Default instruments are never saved, not even in patch sets and states, but
   if the parts are activated, that fact \textsl{is} saved; it's a part
   feature, not an instrument feature.

\subsection{Configuration Files / Scale}
\label{subsec:configuration_scale}

   \index{.xsz}
   \index{config!.xsz}
   \index{scale}
   \index{file!scale}
   Scale Settings.  These files store microtonal settings that \textsl{Yoshimi}
   can use to produce non-standard musical scales.  Recent scales settings are
   saved and recorded.


% \subsection{Configuration Files / Patch Sets}
% \label{subsec:configuration_patch_sets}

\subsection{Configuration Files / Instance}
\label{subsec:configuration_instance}

   \index{.instance(n)}
   \index{config!.instance(n)}
   \index{instance}
   \index{file!instance}
   A new feature of the \textsl{Yoshimi} configuration.
   It contains the current root/bank, MIDI settings, and preferred engines.
   These instance files are totally independent files, distinguished by a number
   in the file-name.

\subsection{Configuration Files / Banks}
\label{subsec:configuration_banks}

   \index{.banks}
   \index{config!.banks}
   \index{banks}
   \index{file!banks}
   A new feature of the \textsl{Yoshimi} configuration.  Currently each
   \textsl{Yoshimi} instance takes its own copy of the actual files as it starts
   up.  Previously, they could all save, delete, or rename the actual files without
   talking to the other instances, so one could move a file in one instance, and
   then try (and fail) to access it from another.

   For V 1.5.10 this issue was partially resolved by only allowing the main instance
   to make actual file changes (the related icons no longer existing in the
   other instances). However from V 1.7.2 this has been reverted as it was causing
   confusion, especially when \textsl{Yoshimi} was used as a plugin. The issue of
   all instances not being informed of a change made by the one currently in use
   will eventually be resolved.

   With the \textbf{Banks} menu, one can assign a patch to a given slot with
   a bank.  This instrument will remain in that slot for future use until it is
   deleted. To see the physical location of the \texttt{.xiz} file, one
   should check the
   \textbf{Yoshimi / Settings / Banks / Root Dirs}
   (\textsl{File / Settings / Bank\_Root\_Dirs}) window to see the paths for
   banks.

   At startup, after all the configuration is complete, the banks are loaded and
   installed.  On a per instance basis, the first thing this process does is
   look for a \texttt{yoshimi(-n).banks} file. If it can't find that it then
   hunts for a \texttt{yoshimi(-n).config} file. Finally, if that fails it does a
   re-scan for banks. In this way it should be completely backward compatible
   with any previous config files.

   The \texttt{.banks} file is \textsl{saved} every time roots, banks, or
   instruments are changed, and again on a normal exit to catch the current
   root and bank (which don't otherwise trigger a save).  This allows the
   last-used root and bank IDs to always match what that instance thinks is
   there.  Note that one needs to have write permissions to add instruments to
   the bank.  When one saves an instrument to a bank slot, it is given a
   filename with the internal name as the leaf-name.  When one saves an
   instrument to an external file, one is  first offered the internal name
   and the current directory, but one can change it if desired.

   By default, \textsl{Yoshimi} does not assign a bank ID 0 (zero) in any root.
   This feature has an interesting benefit. Several sequencers insist on
   setting a bank with every program change, and if one doesn't give a bank,
   they will try to set 0. However, \textsl{Yoshimi} is smart enough to ignore
   any invalid bank ID and remain with the existing bank number.

   As a further protection against rogue sequencers making assumptions, any
   attempt to set an invalid bank root will also be ignored.  On a
   first-time startup, discovered roots will be given ID numbers starting from 5,
   continuing in steps of 5. This makes it easier to re-arrange them to
   preference. We recommend not using 0.

   On first-time start up, \textsl{Yoshimi}
   looks for \textsl{ZynAddSubFX} banks as well, in the usual locations.
   It will not look for a \textsl{ZynAddSubFX} configuration file, as these
   are no longer relevant to \textsl{Yoshimi}.

   Banks are more thoroughly described in
   \sectionref{subsec:concepts_banks_and_roots}.

\subsection{Configuration Files / .bankdir}
\label{subsec:configuration_bankdir}

   A bit of ancient history has bubbled to the surface.
   When one creates a new bank in \textsl{Yoshimi}, it inserts an empty file in
   \index{config!.bankdir}
   the new bank, called \texttt{.bankdir}.  For example:

   \begin{verbatim}
      ... /banks/Zen Collection/.bankdir
   \end{verbatim}

   The reason for this is that when scanning for banks (especially at startup) it
   looks for this file first. If it can't find it, then it has to go though the
   slower process of looking for at least one completely valid instrument file.
   Running from SSDs, it probably won't make a lot of speed difference but it
   will on a conventional hard drive, especially if one has lots of banks.  So,
   if one wants to get that little startup edge, plonk a copy of this file into
      all your banks.  It's an empty file.

   In modern times, one of the main distros creates a warning for the packagers
   if it sees embedded dot files, and gets very unhappy if these are empty ones.
   The obvious answer is to put something there; \textsl{Yoshimi} now adds a
   \texttt{.bankdir} file that is useful -- when one creates a new bank, this
   file contains a string with the \textsl{Yoshimi} version number it was
   created with, and to add icing to the cake, every time one saves an
   instrument file in a bank, this file is updated and will have the current
   \textsl{Yoshimi} version number. Also, it will be created if it wasn't
   already there.

   So now it is be possible to tell how recently a bank was changed, which may
   have implications if running modern instruments on older \textsl{Yoshimi}'s.
   Also, the complaining distro will be happier because
   the \texttt{.bankdir} file won't be empty.

   For the "Collection" bank, the version number is now 2.0; although its
   instruments were created with \textsl{Yoshimi} sometime before version 1.3.0,
   some of the instrument files contain more information. "Drums" is set to 2.2.1;
   "Companion" is currently set to 2.2.0.

\subsection{Configuration Files / Windows}
\label{subsec:configuration_windows}

   \index{.windows}
   \index{config!.windows}
   \index{windows}
   \index{file!windows}
   No, this term isn't a reference to "that other operating system";
   it is a new feature of the \textsl{Yoshimi} configuration.
   It saves the current layout of application
   windows for re-instation at the next startup
   of \textsl{Yoshimi}.

\subsection{Configuration Files / Format}
\label{subsec:configuration_file_format}

   The Unix \texttt{file} command indicates that the XML files are one of
   two types:

   \begin{itemize}
      \item \textsl{exported SGML document, ASCII text}.
         These files are unindented XML data with an encoding of UTF-8 and
         a DOCTYPE of "ZynAddSubFX-data".
      \item \textsl{gzip compressed data, from Unix}.
         These files can be renamed to end in ".gz", and then run through
         the \texttt{gunzip} program to yield the XML file (but without an
         \texttt{.xml} extension).
   \end{itemize}

   The format depends on the "XML compression level" option discussed in
   \sectionref{paragraph:menu_yoshimi_settings_main_settings}.

   \index{saving settings}
   Saving settings or not:
   If one changes settings, and closes without saving, that means the settings
   remain in place only for the current session. If one has changed anything,
   when one closes \textsl{Yoshimi}, one will be given a second chance to
   save them. If one responds 'No',  the next time \textsl{Yoshimi} starts,
   the old settings will be restored.  An 'undo' feature would get pretty
   crazy very quickly.

   In the \textbf{Settings} window, \textbf{Save Settings}
   refers to the entire window, not just individual tabs. The close buttons are
   actually outside the frame of the tabs.

   \textbf{Close without saving} doesn't mean revert to previous settings; it
   means to use the changes, but don't immediately store them to the
   filesystem.

   In general, the contents are structured a lot like the
   user-interface elements that are used to set them.

\subsection{Configuration Files / MIDI Learn}
\label{subsec:configuration_file_midi_learn}

   \index{.xly}
   \index{config!.xly}
   The MIDI-Learn data crosses the border between static configuration and
   dynamic settings. It is stored in files with an extension of
   \texttt{.xly} ("XML Learn Yoshimi").
   If compression is turned off
   (\textbf{Yoshimi / Settings / Main Settings / XML Compression Level} set to
   0), this file is an XML/SGML file with a MIDILEARN section in it.

   When saving states, if there are any configured MIDI learned lines,
   these lines will also be saved.
   When one reloads the state they will also be restored.
   However, if the state file \textsl{doesn't} have any MIDI learn data,
   it \textsl{will not} clear any settings that are already there.

   Therefore, be aware that if one now re-saves that state, it \textsl{will}
   include such MIDI learned data, and the next time it is loaded,
   it \textsl{will} overwrite any lines that are already there.

   Also note that, during a master reset, the MIDI learn data is the only thing
   that \textsl{is not} cleared (unless the CTRL key is held down at the same
   time).

   These features are designed to give the best protection to a setup
   that could have taken quite a long time to arrange exactly as desired.
   In our experiments, we have discovered that we seem to use pretty much the
   same controls and actions, and the list of our 'preferred' settings is
   slowly increasing.

\section{Local Data Files}
\label{sec:local_data}

   \index{new!Local Data}
   During the development of \textsl{Yoshimi} V 1.7.1/2, a number of files and
   directories have been moved from the configuration directory to the
   standards preferred hidden directory in one's \texttt{home} directory.
   However, one's existing files in the old locations will be copied across, so
   there should be no disruption. For \textsl{Yoshimi} this is
   \texttt{.local/share/yoshimi}

   Here is a summary of the files.

\begin{itemize}
    \item \texttt{presets}
        \index{presets}
        User defined presets (or snapshots) are stored as individual files in
        the \texttt{presets subdirectory}.
   \item \texttt{clipboard}
        \index{clipboard}
        Unnamed presets are stored in the \texttt{clipboard subdirectory} for
        immediate copying to a different location.
   \item \texttt{recent}
         \index{recent}
         Recent instruments, patch sets, etc. are stored in the \texttt{recent}
         file. The last-used file in any section is always at the top of its
         recent history list.
    \item \texttt{found}
        \index{found}
        Editable copies of all the banks found in the \texttt{\textbackslash usr}
        installed locations are stored here.
    \item \texttt{theme}
        \index{theme}
        Imported Yoshimi GUI colour themes are stored in this subdirectory, along
        with the 'Classic' example file.
    \item \texttt{\textasciitilde/.yoshimi\_history}
         \index{\textasciitilde/.yoshimi\_history}
         Specific to command line use, this holds the command-line history.
\end{itemize}

\subsection{Local Data Files / Presets}
\label{subsec:local_data_preset}

   \index{.xpz}
   \index{preset}
   \index{file!preset}
   Have a favourite setting for an envelope, or a difficult-to-reproduce
   oscillator? Then presets are for you! Presets allow for one to save the
   settings for any of the components which support copy/paste operations.
   This is done with preset files (\texttt{.xpz}), which get stored in the
   folders indicated by \textsl{Paths / Preset Dirs...}.
   The key thing about using presets is that one must first
   specify a presets directory!  Otherwise, who knows where they go?
   A good choice for a preset directory is
   \texttt{\textasciitilde/.local/share/yoshimi/presets}.
   In addition, copy-and-paste of synth settings can be done across
   \textsl{Yoshimi} instances.

   In \textsl{Yoshimi}, a
   \textsl{preset} is any collection of settings that can be saved to the
   clipboard or to a file, for later loading elsewhere.

   A preset is a canned version of a \textsl{Yoshimi} sub-setting.  Presets can be
   copied and pasted using the blue \textbf{C} and \textbf{P} user-interface
   buttons associated with many of the \textsl{Yoshimi} dialog windows.  They
   make it easy to save portions of the current settings for later use.  For
   example, resonance settings can be saved.

   The naming convention for a preset file is
   \texttt{presetname.presettype.xpz}, where
   \textsl{presetname} is the name one types into the \textbf{Copy to Preset}
   name field, \textsl{presettype} is the name that appears in the
   \textbf{Type} field, and \textsl{xpz} is the file-extension for compressed
   XML preset files.

   Since \textsl{Yoshimi} V 1.6.0 Preset 'root' directories are arranged in a
   similar way as bank roots, so one can zero in on just those of interest,
   with a 'current' directory instead of the previous 'default' one.
   \textsl{Yoshimi} instances can, of course, have their own current root, but
   the list of directories is common to all.

\subsection{Local Data Files / Clipboard}
\label{subsec:local_data_clipboard}

   \index{.xpz}
   \index{clipboard}
   \index{file!clipboard}
   Since \textsl{Yoshimi} V 2.3.2 a clipboard copy is held as a file is rather
   than holding it in memory. This means all presets are handled exactly the
   same way. Also, the last used copy of any section will be retained over
   \textsl{Yoshimi} restarts. Over time one ends up with a collection of these
   representing the last used entry of that type.

   The clipboard has two benefits. One can quickly copy an entry then paste that
   copy to any number of same-type sub-setting. Also if one has to respond to an
   interruption, on the next time \textsl{Yoshimi} is started one can carry on
   where they left off.

\subsection{Local Data Files / Recent}
\label{subsec:local_data_recent}

   \index{recent}
   \index{local!recent}
   \index{file!recent}
   Recently seen instruments, patch sets, etc. are stored in the
   \texttt{recent} file. For example, if the \textbf{XML Compression} option
   is set to 0, and one exits \textsl{Yoshimi}, then the following file might
   contain the following items (ignoring the XML markup):

   \texttt{\textasciitilde/.local/share/yoshimi/recent}

   \begin{verbatim}
      /home/me/yoshimi-cookbook/sequencer64/b4uacuse/yoshimi-b4uacuse-gm.state
      /home/me/sequencer64/contrib/yoshimi/horse.state
   \end{verbatim}

   \texttt{recent} is a single history file that every instance can read and
   write.

   This is actually quite interesting as there can never be a conflict.  It is
   impossible to have two browser lists open at the same time, and the lists are
   always rebuilt from memory every time they are opened. Similarly, the entries
   are added to every time a new recognised file is loaded or saved and one
   can't physically do two at the same time -- even if one could it would simply
   mean that one very briefly waited for the other, which is not an issue as
   they are not in the realtime thread.

   The \texttt{recent} file is saved only upon a normal exit, as it is
   comparatively unimportant.

\subsection{Local Data Files / Found}
\label{subsec:local_data_found}
   \index{new!Found Banks}
   Banks supplied when \textsl{Yoshimi} is installed are placed in read-only
   locations. This is inconvenient when one wishes to add to, or re-arrange
   them. From V 1.7.1, the first time these banks are seen they are copied into
   the local directory which \textbf{is} writable. For further details see
   \sectionref{subsubsec:banks_and_roots_bank_directories}

\subsection{Local Data Files / Themes}
\label{subsec:local_data_themes}
   \index{new!Themes}

   \textsl{Yoshimi's} themes allow dramatic changes to be easily made to personalise
   the GUI appearance without in any way interfering with its functionality. This
   includes in-line editing of the current displayed user theme. Full details can be
   found in \sectionref{sec:themes}.

\subsection{Local Data Files / CLI history}
\label{subsec:local_data_cli_history}
   \index{Command Line History}
   The CLI history is an outlier and not part of the \textsl{Yoshimi} file
   management. It is a hidden file in one's \texttt{home} directory, maintained
   by the standard \texttt{readline} program.
   It conveniently allows you to easily repeat previous commands, both in the
   current session, and when restarting later ones.


%-------------------------------------------------------------------------------
% vim: ts=3 sw=3 et ft=tex
%-------------------------------------------------------------------------------


% Banks and Roots

%-------------------------------------------------------------------------------
% yum_banks_and_roots
%-------------------------------------------------------------------------------
%
% \file        yum_banks_and_roots.tex
% \library     Documents
% \author      Chris Ahlstrom
% \date        2015-09-06
% \update      2016-03-01
% \version     $Revision$
% \license     $XPC_GPL_LICENSE$
%
%     Provides the concepts.
%
%-------------------------------------------------------------------------------

\section{Banks and Roots}
\label{sec:banks_and_roots}

   In recent versions of \textsl{Yoshimi}, the concepts of banks and roots
   have undergone a fair amount of change, including new features to make
   them easier to manage and easier to automate.  There are a lot of details
   to understand, too many to include along with the descriptions of the
   user-interface elements that control them.
   Therefore, this new section is devoted to banks and roots.
   It is an elaboration of material originally presented in
   \sectionref{subsec:concepts_banks_and_roots}.

   At one time, one could in theory have 1000 roots, 1000 banks and 1000
   presets.  However, now roots and banks were trimmed to what can be
   addressed from MIDI. One can supposedly still have 1000 presets, though.
   Anyway, 128x128x160 = 2621440 instruments should be enough for anyone.

   All root, bank, and instrument IDs are used by MIDI controls, and as of
   version 1.3.6 will also be accessible to the command line.

   The file \texttt{Banks.txt} in the \textsl{Yoshimi} source-code bundle
   makes an important point about a transition (in newer versions)
   to tagging roots (directories) and banks with an ID code:

   \begin{quotation}
      One no longer has the concept of a default root directory, but a current
      one. This can by changed at any time without requiring a restart, so
      there is no longer a need to display the (confusing) contents of all
      roots. Also, roots now have ID numbers associated with them, but no
      changes have been made to the actual directores to achieve this. Instead
      the IDs are stored in the config file. The same ID system is used for
      banks, again without making any file system changes.

      At first run (and whenever new root directories are set) unknown roots
      and banks are given these IDs. Once set they will not change no matter
      how many more roots and banks are later added. One can however, manually
      change root directory IDs in the 'Bank Root Paths window' accessible from
      the 'Paths' item in the main window. Also, there is a new Banks window so
      that these can be set up, moved and renamed in exactly the same way as
      instruments can.  With these IDs, roots and banks can be grouped/ordered
      by function instead of alphabetically. When using the GUI, one will
      always know exactly which root and bank one fetches an instrument from.

      One can quickly step between roots, banks and instruments with the so
      marked buttons in each of these, and if one right-clicks on them one will
      close one window as one opens the other.

      The significance of all this is that one's MIDI sequencer can now
      reliably use these ID numbers to select roots, banks and (already
      available) instruments. That Rosegarden or Muse file one saves today will
      be just as valid in the future, unless one makes the deliberate choice to
      change some IDs. Indeed, one can now start with an 'empty'
      \textsl{Yoshimi}, and
      via MIDI, set roots, banks and load instruments into parts (enabling the
      parts as one does so) swapping banks and roots as necessary. While the
      MIDI file runs it can silently pull instruments from any root/bank into
      any non-sounding part without disturbing the playing ones.

      In Yoshimi / Settings / MIDI one can enable or disable all these features,
      and can define which CCs one wants to use. Bank can be either MSB or LSB
      (as before). Root can be any non-reserved CC but including the one not in
      use for Bank. Also, Extended Program Change now has the same restrictions
      as Root, and these three are all cross-checked against each other. As an
      example, one might set Bank to LSB amd Root to 0 (MSB), effectively
      giving one extended bank control compatible with all sequencers.

      Also, different instances have their own config files so that one can
      have (say) the main instance with current root(9), bank(23) while
      instance 4 has current root(2), bank(6). One can call up instances by
      number and thus access saved settings for that instance. As each instance
      has its own MIDI and audio ports, they can behave more-or-less
      independently.

      In doing all of this we have completely changed the way we manage the
      structure internally, resulting in much greater efficiency, at the cost
      of only a slightly slower startup. Swapping roots performs *no* file
      operations. Swapping banks only fetches the directory list of the newly
      selected bank. Changing an instrument doesn't have to search for a file,
      only load from its already known location.

      If one changes a bank root path, either through the gui or via MIDI, it
      will always reset the current bank to the lowest numbered one it can
      find. This is because there may not be a bank in the new root with the
      same ID, and even if there is, there is no guarantee that it will have
      the same name or contents.

      Also if an attempt is made to reload the same root, nothing will actually
      happen. The same is true of banks. Both of these are kept fully
      up-to-date so there would be no point.

      However, reloading the same *instrument* will be performed every time, as
      one may have changed what is currently loaded without saving it. This
      provides an effective 'restore' operation.

      Finally, it is generally advisable to make root and bank changes on
      channel 1 so that one can more easily keep track of them. However they
      are not channel sensitive as they don't directly affect the sound, so one
      can set them in any convenient channel then perform individual program
      changes on the desired channels.
   \end{quotation}

   It has always been possible to swap and move instruments within a bank, and
   since V 1.3.5 it was possible to swap banks within a bank root, but now one can
   swap/move instruments across any banks and any bank roots. One can also move
   whole banks across bank roots.  These extensions use exactly the same controls
   as before. However, it isn't just a case of changing IDs. Files are actually
   moved, so additional protections and warnings are put in place.

   There are also bank importing and exporting controls and since version 1.5.8
   these have been made available to the CLI with specific IMport and EXPort
   controls.

   A "CC" is a MIDI "continuous controller".
   A MIDI bank change is usually a CC\#0 value of 0, followed by a CC\#32
   value of X, where X is the desired bank number from 0 to whatever.
   (However, in some cases it may simply be a CC\#0 on its own with a value
    of X).  Many synths also require that one send a program change after
   the bank change, to select the program within the bank.

\subsection{Roots}
\label{subsec:banks_and_roots_roots}

   In \textsl{Yoshimi}, a \textsl{root} is a location in which banks can be
   stored.  It is basically a directory, though it ultimately is assigned a
   number by \textsl{Yoshimi}, presumably to be able to access it in an
   automated way.  By choosing a root and making it the current root, one
   can hone in on a smaller collection of banks.

   One cannot reach root paths through the \textbf{Yoshimi / Settings} menu
   any more; it was causing a nightmare of syncing with the other entry
   routes. One can reach it from the Banks window
%  (see \sectionref{subsubsec:menu_instrument_show_banks})
   or the Instruments window,
%  (see\sectionref{subsubsec:menu_instrument_show}),
   and both of the latter also have multiple entry routes.
   Roots can also be reached through the \textbf{Yoshimi / Paths} menu.

\subsection{Banks}
\label{subsec:banks_and_roots_banks}

   Another important concept in \textsl{Yoshimi} is \textsl{banks}.  Instruments
   can be stored in banks. These are loaded and saved automatically by the
   program.  On program start, the last used bank is loaded. A single bank
   can store up to 128 instruments normally, and 160 using extended programs.

%  TODO:  We should be able to walk the user through a session and
%  set up a set of instruments that is tractable under MIDI control.
%  \index{todo!banks/roots}

%  MOVED SOMEWHERE ELSE IN Menu:
%
%  Some minor instructions are provided in
%  \sectionref{subsubsec:menu_instrument_show_root_paths}.

   Also note that, as well as bank and program changes, there is the ability
   to set a MIDI CC to access the voices from 129 to 160 (numbered re 1).
   All the Bank
   controls are contained in a tab in the main \textbf{Settings}
   window, and take immediate effect.

   Bank root directories are identified with ID numbers that can be changed
   by the user in the user-interface. This feature is also made available for
   selection over MIDI.  MIDI only sees banks in the \textsl{current} root
   directory, but all banks are accessible to the user-interface.

\subsubsection{Bank Directories}
\label{subsubsec:banks_and_roots_bank_directories}

   Banks are arranged in directories, with each directory containing a number
   of instrument files.

   Each instrument's file-name should begin with a 4-digit number
   (left-padded with 0's to make it 4 digits long).  This number can serve
   as a MIDI patch number for automated selection of the instrument via a
   MIDI program-change message.

   Unnumbered instruments in a bank will be given a temporary ID starting
   from number 160 and working down. If those numbers already exist then
   they will be skipped over. This can get very confusing. However, if one
   simply loads it and re-saves it to the same instrument slot, it will gain
   that ID and be properly fixed.  One can then move-swap it with others.


%-------------------------------------------------------------------------------
% vim: ts=3 sw=3 et ft=tex
%-------------------------------------------------------------------------------


% Menu

%-------------------------------------------------------------------------------
% yum_menu
%-------------------------------------------------------------------------------
%
% \file        yum_menu.tex
% \library     Documents
% \author      Chris Ahlstrom
% \date        2015-05-11
% \update      2016-06-05
% \version     $Revision$
% \license     $XPC_GPL_LICENSE$
%
%     Provides the Menu section of yoshimi-user-manual.tex.
%
%-------------------------------------------------------------------------------

\section{Menu}
\label{sec:menu}

   At last, we're now ready to describe the user-interface of \textsl{Yoshimi}!
   The \textsl{Yoshimi} menu, as seen at the top of
   \figureref{fig:yoshimi_main_screen},
   is fairly simple, but it is important to understand the
   structure of the menu entries.

\subsection{Menu / Yoshimi}
\label{subsec:menu_yoshimi}

   The \textsl{Yoshimi}
   menu entry contains the sub-items shown in
   \figureref{fig:yoshimi_menu_items}.
   The next few sub-sections discuss the sub-items in the 
   \textsl{Yoshimi} sub-menu.
   (Note that, in \textsl{ZynAddSubFX}, this menu is called the
   \textsl{File} menu.)

\begin{figure}[H]
   \centering 
%  \includegraphics[scale=0.75]{menu/yoshimi-menu-yoshimi.jpg}
   \includegraphics[scale=1.0]{1.4.0/yoshimi-menu-yoshimi.png}
   \caption{Yoshimi Menu Items}
   \label{fig:yoshimi_menu_items}
\end{figure}

   \textbf{New:}
   \index{new!Vector menu entry}
   New with version 1.4.0 of \textsl{Yoshimi} is the \textbf{Vectors} menu
   entry.
   See \sectionref{subsubsec:menu_yoshimi_vectors}, which presents this dialog
   and briefly describes it.

   \textbf{Bug:}
   \index{bugs!menu hot keys don't work}
   There seems to be a bug in that the expected menu hot-keys
   (Alt-Y, Alt-I, Alt-P, and Alt-S) do not work (Yoshimi 1.3.5).

\subsubsection{Menu / Yoshimi / About...}
\label{subsubsec:menu_yoshimi_about}

   There is no \textbf{Help} menu in \textsl{Yoshimi}.  Therefore, the
   \textbf{About} dialog appears in the \textbf{Yoshimi} menu, as shown in
   \figureref{fig:yoshimi_about_dialog}.
   These guys need some acknowledgment for their hard work!
   And they acknowledge the massive groundwork laid by the
   \textsl{ZynAddSubFX} project.

\begin{figure}[H]
   \centering 
%  \includegraphics[scale=1.0]{menu/Yoshimi/yoshimi-about.jpg}
   \includegraphics[scale=1.0]{1.3.8/yoshimi-about.jpg}
   \caption{Yoshimi Menu, About Dialog}
   \label{fig:yoshimi_about_dialog}
\end{figure}

\subsubsection{Menu / Yoshimi / New instance}
\label{subsubsec:menu_yoshimi_new_instance}

   Creates a new instance of \textsl{Yoshimi}.
   If JACK is running,
   start a normal (JACK-using) instance of \textsl{Yoshimi}.
   Then use this menu entry.  \textsl{Yoshimi} will start another instance
   of itself, with an ID of 1.
   This instance can be verified by running a JACK session manager such as
   QJackCtl.

   It is important to note that each instance of \textsl{Yoshimi} has its
   own configuration file.  Each also has its own MIDI and audio ports.
   Thus, these instances are partly independent of each other.

   Opening a new instance creates a copy that has it's own dynamic memory for
   running storage. In the future, some data (such as recent history) will be
   shared between instances. This will be done only where instances actually
   need to be in sync.

   Each instance has it's own file-store in \textsl{Yoshimi}'s configuration
   directory. The means that if one opens a numbered instance, one will get
   back all the settings that were previously used for that instance.

   Instances no longer fight for access to JACK/ALSA audio; they will simply
   try to find another route to a soundcard. Failing to find one,
   they will revert to null audio, but will nonetheless start cleanly.

\subsubsection{Menu / Yoshimi / New instance with id...}
\label{subsubsec:menu_yoshimi_new_instance_with_id}

   Creates a new instance of \textsl{Yoshimi}
   with an ID that is a number.
   See \figureref{fig:yoshimi_instance_dialog}.
   It tries to open a \textsl{Yoshimi} instance based on the configuration
   found in the file
   \texttt{\textasciitilde/.config/\-yoshimi/\-yoshimi.configXX}, where
   \textsl{XX} is the ID one supplied.

\begin{figure}[H]
   \centering 
   \includegraphics[scale=0.75]{menu/Yoshimi/yoshimi-instance-id.jpg}
   \caption{Yoshimi Menu, Instance Dialog}
   \label{fig:yoshimi_instance_dialog}
\end{figure}

   Useful when connecting devices with JACK.
   Start a normal (JACK-using) instance of \textsl{Yoshimi}.
   Then use this menu entry, supply a number as an ID.
   \textsl{Yoshimi} will start another instance
   of itself, with an ID of whatever number one specified.
   This instance can be verified by running a JACK session manager such as
   \textsl{QJackCtl}.

   In a non-JACK setup it won't fail, but in the absence of any specific
   setting, it will have null audio, but (probably) will still connect to ALSA
   MIDI.  Not too useful, but we should test that sometime.

\subsubsection{Menu / Yoshimi / Vectors}
\label{subsubsec:menu_yoshimi_vectors}

   Vectors provide a way of mixing up to four parts in a manner that can be
   automatec, saved, and loaded.  The features of vector control are presented
   in \sectionref{sec:vector}.
   Vector setup and control from the \textsl{Yoshimi} command-line are
   discussed in
   \sectionref{subsec:command_line_command_level}.
   Here, we discuss the vector configuration dialog.
   (As an exercise, one can compare the various functions of the vector dialog
   to the command-line commands one can use to set up the vector
   functionality.)
   The new \textbf{Yoshimi / Vectors} menu entry brings up the following
   dialog:

\begin{figure}[H]
   \centering 
   \includegraphics[scale=0.75]{1.4.0/yoshimi-vectors-dialog.png}
   \caption{Yoshimi Vectors Dialog}
   \label{fig:yoshimi_vectors_dialog}
\end{figure}

   The user-interface items in the vector dialog are:

   \begin{enumber}
      \item \textbf{Top Line}
      \begin{enumber}
         \item \textbf{Base Channel}
         \item \textbf{Options}
      \end{enumber}
      \item \textbf{X Vector}
      \begin{enumber}
         \item \textbf{Controller} (CC Event)
         \item \textbf{Part 1}
         \item \textbf{Part 2}
         \item \textbf{Features}
         \begin{enumber}
            \item \textbf{Feature 1} (Volume)
            \item \textbf{Feature 2} (Pan)
            \item \textbf{Feature 3} (Brightness)
            \item \textbf{Feature 4} (Modulation)
         \end{enumber}
      \end{enumber}
      \item \textbf{Y Vector}
      \begin{enumber}
         \item \textbf{Controller} (CC Event)
         \item \textbf{Part 1}
         \item \textbf{Part 2}
         \item \textbf{Features}
         \begin{enumber}
            \item \textbf{Feature 1} (Volume)
            \item \textbf{Feature 2} (Pan)
            \item \textbf{Feature 3} (Brightness)
            \item \textbf{Feature 4} (Modulation)
         \end{enumber}
      \end{enumber}
      \item \textbf{Bottom Line}
      \begin{enumber}
         \item \textbf{Vector Name}
         \item \textbf{Close}
      \end{enumber}
   \end{enumber}

   Although they are nested, for simplicity we will discuss the unique items
   serially.

   \setcounter{ItemCounter}{0}      % Reset the ItemCounter for this list.

   \itempar{Base Channel}{Vectors!base channel}
   Vector Base Channel.
   This item specifies the MIDI channel on which the vector (of parts) will be
   based.  This channel is the incoming MIDI channel to which the vector setup
   will respond to on all parts.

   Values: \texttt{1 to 16, 1*}

   \itempar{Options}{Vectors!options}
   Vector Options.

   Values: \texttt{1 to 127, 64*}

\begin{figure}[H]
   \centering 
   \includegraphics[scale=0.75]{1.4.0/yoshimi-vectors-options.png}
   \caption{Yoshimi Vectors Options}
   \label{fig:yoshimi_vectors_options}
\end{figure}

   The menu entries provide the actions described in this list:

   \begin{enumber}
      \item \textbf{Load}.
         Brings up a file dialog that let's one pick an arbitary vector file
         (extension \texttt{.xvy}) in an arbitrary directory, or select a
         "Favorites" directory in which to look for vector files.
         The base name of the file is then shown in the \textbf{Vector Name}
         field.
      \item \textbf{Save}.
         Brings up a file dialog that let's one save an arbitary vector file
         (extension \texttt{.xvy}) in an arbitrary directory, or select a
         "Favorites" directory in which to save a vector file.
         The base name of the file is then shown in the \textbf{Vector Name}
         field.
      \item \textbf{Recent}.
         Brings up a short list of the previous vector files dealt with.
      \item \textbf{Clear Chan}.
         Clears the base channel number????
      \item \textbf{Clear All}.
         Clears out the full vector setup, rendering it an "empty" vector setup
         that cannot be saved.
   \end{enumber}

   \itempar{X Vector}{Vectors!x}
   The X Vector.
   This vector provides the minimal setup for a vector.  This setup requires
   \textsl{Yoshimi} to be configured for 32 parts, to be able to fully support
   a two-part vector for every MIDI channel.  This section is disabled until
   selects a \textbf{Controller} event value for it.

   \itempar{Controller}{Vectors!controller}
   Vector Controller CC Event.
   If 0, the section (\textbf{X} or \textbf{Y}) that this value is in is
   disabled.  Otherwise, the number is the MIDI continuous controller (CC)
   event value that is to be used to control the mix of the two parts involved
   in this vector.

   Values: \texttt{1 to 119, 0*}

   \itempar{Part 1}{Vectors!part 1}
   Part 1.
   The top button in the section (\textbf{X} or \textbf{Y}) selects the first
   part to use in the two-part vector.  Clicking this button brings up the
   default bank dialog.  This dialog allows one to select a part (instrument),
   or to select an different bank from which to choose a part (instrument).

   \itempar{Part 2}{Vectors!part 2}
   Part 2.
   The bottom button in the section (\textbf{X} or \textbf{Y}) selects the
   second part to use in the two-part vector.  Clicking this button brings up
   the default bank dialog, just as for the \textbf{Part 1} button.

   \itempar{Feature 1}{Vectors!volume}
   Vector Feature 1, Volume.
   This feature can be disabled, or enabled.  Feature 1 is always fixed as MIDI
   event 7 (volume), and is not reversible.
   When enabled, the volume is traded off between the first part and second part
   as the selected MIDI CC controller event data value changes.
   While the first part increases in volume, the second part decrease in
   volume, and vice versa.

\begin{figure}[H]
   \centering 
   \includegraphics[scale=0.75]{1.4.0/yoshimi-vectors-feature-1.png}
   \caption{Yoshimi Vectors, Feature 1}
   \label{fig:yoshimi_vectors_feature_1}
\end{figure}

   Values: \texttt{Off*, Vol}

   Note that the common theme between all features is that they apply inversely
   to the two parts/instruments that are paired in an \textbf{X}
   or \textbf{Y} vector.

   \itempar{Feature 2}{Vectors!pan}
   Vector Feature 2, Pan.
   This feature can be disabled, enabled, or reversed.
   When enabled, it acts similarly to volume, panning from left to right as the
   data value increases.
   When reversed, it pans from right to left as the data value increases.

\begin{figure}[H]
   \centering 
   \includegraphics[scale=0.75]{1.4.0/yoshimi-vectors-feature-2.png}
   \caption{Yoshimi Vectors, Feature 2}
   \label{fig:yoshimi_vectors_feature_2}
\end{figure}

   Values: \texttt{Off*, Pan, Pan R}

   \itempar{Feature 3}{Vectors!brightness}
   Vector Feature 3, Brightness.
   Brightness here refers to the application of a (we presume) low-pass filter
   with a varying cutoff frequency.
   This feature can be disabled, enabled, or reversed.
   When enabled, it acts similarly to volume, changing the brightness
   from left to right as the data value increases.
   When reversed, it changes the brightness from right to left as the data
   value increases.

\begin{figure}[H]
   \centering 
   \includegraphics[scale=0.75]{1.4.0/yoshimi-vectors-feature-3.png}
   \caption{Yoshimi Vectors, Feature 3}
   \label{fig:yoshimi_vectors_feature_3}
\end{figure}

   Values: \texttt{Off*, Filt, Filt R}

   \itempar{Feature 4}{Vectors!modulation}
   Vector Feature 4, Modulation.
   Modulation here refers to the application of an (we presume) LFO
   (for amplitude or frequency) with a varying modulation depth.
   This feature can be disabled, enabled, or reversed.
   When enabled, it acts similarly to volume, changing the modulation
   from left to right as the data value increases.
   When reversed, it changes the modulation from right to left as the data
   value increases.

\begin{figure}[H]
   \centering 
   \includegraphics[scale=0.75]{1.4.0/yoshimi-vectors-feature-4.png}
   \caption{Yoshimi Vectors, Feature 4}
   \label{fig:yoshimi_vectors_feature_4}
\end{figure}

   Values: \texttt{Off*, Mod, Mod R}

   \itempar{X Vector}{Vectors!x}
   The X Vector.
   This vector provides the maximal setup for a vector.  This setup requires
   \textsl{Yoshimi} to be configured for 64 parts, to be able to fully support
   an additional two-part vector for every MIDI channel.  This section is
   disabled until selects a \textbf{Controller} event value for it.

   Other than that, the \textbf{Y} vector acts like the \textbf{X} vector, and
   all of the sub-items have the same functionality as in the
   \textbf{X} vector.

\begin{figure}[H]
   \centering 
   \includegraphics[scale=0.75]{1.4.0/yoshimi-vectors-saved.png}
   \caption{Yoshimi Vectors Saved as "My First Vector"}
   \label{fig:yoshimi_vectors_saved}
\end{figure}

   If starting a new vector setup, first
   select the base channel (1 to 16) to set the vector on. Next,
   use the \textbf{X} and \textbf{Y} \textbf{Controller}
   spin-boxes to select the incoming CC.
   If you set an invalid one strange things may happen, though it won't
   actually do any harm.

   The instrument buttons bring up the instrument list window in exactly the same
   way as the main part one does, but do not currently the right-click
   windows return feature.

   When setting up a completely new vector, if the mixer or the main part are
   visible, they may be slightly out of sync, but will correct themselves as
   soon as one changes an instrument or part.

   When one \textbf{clears} a vector it doesn't delete loaded voices, nor does
   it change the active status of the part, nor the number of parts available.
   This is because these settings may have been made independent of vector
   control. In short, vector setup will add things but not remove them.

\subsubsection{Menu / Yoshimi / Settings...}
\label{subsubsec:menu_yoshimi_settings}

   The \textsl{Yoshimi Settings} dialog contains four tabs that control the
   major and overall settings of \textsl{Yoshimi}.  At the bottom of this
   dialog are two buttons:
   \textbf{Save and Close} and \textbf{Close Unsaved}.
   \index{Save and Close}
   \index{Close Unsaved}

   Please note that the \textbf{Save and Close} and \textbf{Close Unsaved}
   buttons apply to the \textsl{whole}
   \textbf{Settings} window.
   Furthermore, the "saving" does \textsl{not} refer to preserving the changes
   that have been made
   in any of the tabs for the current \textsl{Yoshimi} session.  Any changes
   made in \textbf{Settings} always remain in place for the current
   \textsl{Yoshimi} session.
   However, the changes to the settings are saved to
   the state file \textsl{only} if \textbf{Save and Close} is clicked.

   \setcounter{ItemCounter}{0}      % Reset the ItemCounter for this list.

   \itempar{Save and Close}{Settings!Save and Close}
   This selection saves the settings made in \textbf{all} of the tabs to the
   state file, and closes the \textsl{Yoshimi} settings dialog.

   \itempar{Close Unsaved}{Settings!Close Unsaved}
   Close Unsaved, Main Settings.

   This selection closes the \textsl{Yoshimi} settings dialog.
   However, note that any changes made in the tabs
   \textsl{are preserved}.  They are preserved for the current
   \textsl{Yoshimi} session, but are not saved to the filesystem.
   
\paragraph{Menu / Yoshimi / Settings / Main Settings}
\label{paragraph:menu_yoshimi_settings_main_settings}

   The Main Settings tab controls the main configuration items that
   follow, which apply to all patches/instruments.
   The main settings are shown in
   \figureref{fig:yoshimi_main_settings_dialog}.

   Some settings only take effect after restarting the synthesizer.
   In Main settings, only items marked with an asterisk ('*')
   need a restart. 
   (This in now only the first two. Some of the other ones had
   prevously been wrongly marked.)

   The settings dialogs are quite different between \textsl{ZynAddSubFX} and
   \textsl{Yoshimi}.  There are some differences even between
   \textsl{Yoshimi} versions earlier than 1.3.5, and the current version
   (currently 1.3.9).

\begin{figure}[H]
   \centering 
%  \includegraphics[scale=0.75]{menu/Yoshimi/yoshimi-settings-main.jpg}
   \includegraphics[scale=0.75]{1.3.9/yoshimi-settings-main.png}
   \caption{Yoshimi Main Settings Tab}
   \label{fig:yoshimi_main_settings_dialog}
\end{figure}

   The following settings exist in the \textsl{Main settings} tab:

   \begin{enumber}
      \item \textbf{AddSynth Oscillator Size} (was "OscilSize")
      \item \textbf{Internal Buffer Size} (new)
      \item \textbf{PADsynth interpolation}
      \item \textbf{Virtual Keyboard Layout}
      \item \textbf{XML compression level}
      \item \textbf{Send reports to}
      \item \textbf{Hide Non Fatal Errors}
      \item \textbf{Log XML headers}
      \item \textbf{Show Splash} (new)
      \item \textbf{Save and Close}
      \item \textbf{Close Unsaved}
   \end{enumber}

   \setcounter{ItemCounter}{0}      % Reset the ItemCounter for this list.

   \itempar{AddSynth Oscillator Size}{Main Settings!oscillator size}
   ADDsynth Oscillator Size (in samples).  This item used to be called
   "OscilSize".  Sets the number of the points of the ADDsynth oscillator.
   Bigger is better, but it takes more CPU time on the start of any note, and
   it may add latency to some processes.

   The default value for \textsl{Yoshimi} is shown marked with an asterisk,
   and the default value for \textsl{ZynAddSubFX} is 512.
   \index{asterisk}
   \index{default!asterisk}
   This asterisk/plus-sign convention is used throughout this manual.
   See \figureref{fig:yoshimi_oscilsize_values},
   shown below for the AddSynth Oscillator Size drop-down element.

   Values: \texttt{256, 512*, 1024, 2048, 4096, 8192, 16384}

\begin{figure}[H]
   \centering 
   \includegraphics[scale=0.75]{menu/Yoshimi/main-oscilsize.jpg}
   \caption[OscilSize Values]{AddSynth Oscillator Size (samples)}
   \label{fig:yoshimi_oscilsize_values}
\end{figure}

   \itempar{Internal Buffer Size}{Main Settings!buffer size}
   This is a new item for version 1.3.6.  It is actually the old
   \textbf{Period Size} field from the \textbf{Alsa} tab.
   It sets the granularity of the sound generation.
   To find out the internal delay in milliseconds, divide the
   buffer-size value by the sample-rate, then multiply the result by 1000:
   For example, \(256 / 44100 * 1000 = 5.8 ms\).

   The default internal buffer size has been reduced from 1024 to 256.  One
   gets better latency that way.  Almost all modern computers can run
   \textsl{Yoshimi} with the current default (smaller) buffer-size value, and
   many will do so at 64 frames (and even 16 frames!)
   without any special precautions.
   
   Note that, for ALSA, if the audio destination is "default",
   then ALSA decides on the buffer size (among other settings), and
   \textsl{Yoshimi} will set it's internal buffer size to match,
   which always seems to be 1024.

   Values: \texttt{64, 128, 256*, 512, 1024}

\begin{figure}[H]
   \centering 
   \includegraphics[scale=0.75]{menu/Yoshimi/main-internalsize.jpg}
   \caption[Internal Size Values]{AddSynth Internal Buffer Size (samples)}
   \label{fig:yoshimi_internalsize_values}
\end{figure}

   \itempar{PADsynth interpolation}{Main Settings!PADsynth Interpolation}
   See \figureref{fig:padsynth_interpolation}, shown below,
   for the interpolation values.
   From an email conversation with Paul Nasca, Will notes that
   the sound improvement with cubic interpolation is quite subtle, and requires
   a well designed audio setup, a PADsynth instrument with a fair amount of
   high-frequency content... and good hearing!

\begin{figure}[H]
   \centering 
   \includegraphics[scale=0.75]{menu/Yoshimi/main-padsynth-interpolation.jpg}
   \caption[PADSynth Interpolation]{PADSynth Interpolation Values}
   \label{fig:padsynth_interpolation}
\end{figure}

   Values: \texttt{Linear(fast)*, Cubic(slow)}

   \itempar{Virtual Keyboard Layout}{Main Settings!Virtual Keyboard Layout}
   The virtual keyboard is useful, but it is difficult to move the mouse
   rapidly to the next key on the virtual keyboard.
   Therefore, \textsl{Yoshimi} supports using the computer keyboard
   to produce notes.

\begin{figure}[H]
   \centering 
   \includegraphics[scale=0.45]{top-panel/ascii-virtual-keyboard.png}
   \caption{QWERTY Virtual Keyboard Layout}
   \label{fig:qwerty_virtual_keyboard}
\end{figure}

   See \figureref{fig:qwerty_virtual_keyboard},
   for the mapping of the computer keyboard to the
   virtual keyboard.
   Three octaves (blue, green, and red) are available, with the dark keys of
   each color representing the "black" keys.
   Note that this is a QWERTY layout.  
   \textsl{Yoshimi} also supports other keyboard layouts.
   See \figureref{fig:virtual_kbd_layout},
   for the virtual keyboard layout settings drop-down.

   Values: \texttt{QWERTY*, Dvorak, QWERTZ, AZERTY}

\begin{figure}[H]
   \centering 
   \includegraphics[scale=0.75]{menu/Yoshimi/main-virtual-kbd-layout.jpg}
   \caption[Virtual Keyboard Layout]{Virtual Keyboard Layout Values}
   \label{fig:virtual_kbd_layout} 
\end{figure}

   \itempar{XML compression level}{Main Settings!XML compression level}
   Gzip Compression level of \textsl{Yoshimi} XML files.
   The settings and instruments of
   \textsl{Yoshimi}
   are preserved in XML files.
   The value of 0 indicates that the XML file is uncompressed.

   In general, 0 is probably the best setting for debugging only.  Setting this
   option makes the XML files a bit larger, perhaps larger by a factor of more
   than 10, making a 10K file into a 180K file.  For a little "wasted" space
   and time, one can view the XML file in a text/programmer's editor.  But, if
   one's system is tight on disk space, higher levels of compression can be
   specified.  Using XML compression can also save file access time which may
   be beneficial if one's computer is borderline on latency.  This setting
   should stay at 3 if one is going to save large patchsets that will later be
   loaded while running. Uncompressing is \textsl{much} faster than file
   loading.

   Values: \texttt{0 to 9, 3*}

   \itempar{Send reports to}{Main Settings!Send Reports Destination}
   Notices and error messages can be sent to the standard error log of
   the terminal in which 
   \textsl{Yoshimi} can be run, or, more usefully, to
   an output console window.
   Now these messages are pushed in reverse order, to avoid scrolling
   and to make the most recent statuses easily visible.
   See \figureref{fig:send_reports_to}.
   It provides a depiction of the selection drop-down.

   Values: \texttt{stderr*, Console Window}

\begin{figure}[H]
   \centering 
   \includegraphics[scale=1.0]{menu/Yoshimi/main-send-reports-to.jpg}
   \caption[Send Reports]{Send Reports To}
   \label{fig:send_reports_to}
\end{figure}

   If the \textbf{Console Window} option is chosen, then the
   \textbf{Reports} button in the effects panel is enabled.
   Pressing the \textbf{Reports} button brings up a small console dialog, as
   shown in \figureref{fig:console_window}.

\begin{figure}[H]
   \centering 
   \includegraphics[scale=1.0]{1.3.9/console-window.png}
   \caption[Yoshimi Console Window]{Console Window}
   \label{fig:console_window}
\end{figure}

   \index{reports!middle-click}
   An interesting feature of the console window is that one can identify
   user-interface elements of the \textsl{Yoshimi} configuration in this window
   by a middle-click on the user-interface element.  The console window figure
   shows the results of middle-clicking on the following bottom-panel knobs in
   order:

   \begin{enumerate}
      \item Velocity Sense (left-click)
      \item Velocity Offset
      \item Pan
      \item Volume
   \end{enumerate}

   \index{reports!middle-click}
   Information about each knob middle-clicked is pushed to the top of the
   windows in reverse order.
   \index{reports!left-click}
   \index{reports!right-click}
   Information about an earlier left-click is shown at the bottom of the figure.
   Each left-click ("Button 1") will increase the parameter represented by the
   knob, and each right-click ("Button3") will decrease the parameter
   represented by the knob, and each change is reported in the console window.
   And, of course, each middle-click changes nothing, but reports the current
   value in the console window.
   This powerful feature provides a way to gather the information needed to
   control the parameter from the command-line.
   The output can be selected, copied, and pasted to a script or archive text
   file for safe-keeping.
   \index{MIDI learn}
   Consider it a form of "MIDI learn" that will be developed in the future.
   And note that it applies to output to \texttt{stderr} as well.

   \itempar{Hide Non Fatal Errors}{Main Settings!Hide Non-Fatal Errors}
   Especially when running from the command line (with reports going there
   too), under some circumstances one can get a swamp of low-level error
   messages (such as XRUNs) that is so large that one cannot work out what is
   going on. This feature disables these error messages; it is a work in
   progress to catch the bulk of them, while still reporting top-level messages
   and ones that cause a forced exit (surely not!)

   \itempar{Log XML headers}{Main Settings!Log XML headers}
   This item sends the information to the console window
   (or \texttt{stderr}) so that
   one can then see what \textsl{ynAddSubFX}/\textsl{Yoshimi}
   version created the file.

   \itempar{Show Splash}{Main Settings!Show Splash}
   This item will speed up the start-up of \textsl{Yoshimi} slightly
   if unchecked, by not showing the splash screen while files are being loaded.

\paragraph{Menu / Yoshimi / Settings / Jack}
\label{paragraph:menu_yoshimi_settings_jack}

   JACK is the "Jack Audio Connection Kit", very useful for increasing audio
   performance and configurability.
   When using the JACK audio backend, instruments can be individually routed
   and sent to the main L/R outputs. This is controlled from the
   panel window,
   \sectionref{subsec:mixer_panel_window},
   and the settings are saved with all the other parameters.

   Direct part outputs carry the Part and Insertion effects, but not the
   System ones.

\begin{figure}[H]
   \centering 
%  \includegraphics[scale=0.75]{menu/Yoshimi/yoshimi-settings-jack.jpg}
%  \includegraphics[scale=0.75]{1.3.8/yoshimi-settings-jack.jpg}
   \includegraphics[scale=0.75]{1.3.9/yoshimi-settings-jack.png}
   \caption[JACK Settings]{JACK Settings Dialog}
   \label{fig:yoshimi_settings_jack}
\end{figure}

   The following items are provided by the Jack settings:

   \begin{enumber}
      \item \textbf{Jack Midi Source}
      \item \textbf{Set as preferred MIDI}
      \item \textbf{Jack Server}
      \item \textbf{Set as preferred audio}
      \item \textbf{Autoconnect audio} (new, 1.3.9)
      \item \textbf{Save and Close}
      \item \textbf{Close Unsaved}
   \end{enumber}

   \setcounter{ItemCounter}{0}      % Reset the ItemCounter for this list.

   \itempar{Jack Midi Source}{JACK!MIDI source}
   Jack MIDI Source.
   It is possible to have more than one JACK MIDI source.  This option
   tells this instance of \textsl{Yoshimi} which JACK
   client to try to auto-connect to for MIDI input.
   This option corresponds to the \textsl{Yoshimi} command line option
   \texttt{--jack-midi(=device)}.

   Values: \texttt{default*, name} name, as in "jackd --name"

   \itempar{Set as preferred MIDI}{JACK!set as preferred MIDI}
   Set as preferred MIDI for JACK.
   This setting determines which MIDI connections a particular instance will
   first attempt. The switches are mutually exclusive across JACK and ALSA,
   so if one checks ALSA for MIDI, it automatically unchecks JACK for MIDI.
   As well as from the GUI, this setting can be set (for instance 0) from the
   command line, both at start-up and once running.

   \itempar{Jack Server}{JACK!server name}
   Jack Server Name.
   It is possible to have more than one JACK server running.  This option
   tells this instance of \textsl{Yoshimi} which JACK server to use.
   This option corresponds to the \textsl{Yoshimi} command line option
   \texttt{--jack-audio(=server)}.

   Values: \texttt{default*, name} name, as in
   \texttt{jackd --name}.

   \itempar{Set as preferred audio}{JACK!set as preferred audio}
   Set as preferred audio for JACK.
   This setting determines which audio connections a particular instance will
   first attempt. The switches are mutually exclusive across JACK and ALSA,
   so if one checks ALSA for audio, it automatically unchecks JACK for audio.
   As well as from the GUI, this setting can be set (for instance 0) from the
   command line, both at start-up and once running.
   Note that any of these setting changes require a restart of \textsl{Yoshimi}
   to take effect.

   \itempar{Autoconnect audio}{JACK!autoconnect audio}
   Sets \textsl{Yoshimi} to connect automatically to the JACK server, just like
   the \texttt{-K} command-line option does.  (However, note that the
   command-line has no way to disable this feature it the configuration has been
   saved.)

\paragraph{Menu / Yoshimi / Settings / ALSA}
\label{paragraph:menu_yoshimi_settings_alsa}

   A significant improvement is to the handling of ALSA audio, which is still
   very important for some people. Until now, \textsl{Yoshimi} has insisted
   on a 2-channel, 16-bit format. Tests have shown that virtually all
   motherboard sound chipsets will handle this, but many external ones don't.

   From \textsl{Yoshimi} 1.3.6 onwards, when using ALSA audio,
   \textsl{Yoshimi} first tries to connect 2 channels at 32 bit depth.  If
   that connection does not succeed, then \textsl{Yoshimi} negotiates
   whatever the soundcard will support.  For example, a card might support
   only 24 bits, and 6 channels.  So \textsl{Yoshimi} will fall back to
   24 bit, and, due to its own limits, will use only channels 1 and 2.
   With external sound modules in mind, endian swaps are also implemented.

   To be able to reliably use ALSA audio, one needs to set a card name, not just
   "Default".  In a terminal window enter the following command:

   \begin{verbatim}
      $ cat /proc/asound/card*/id
   \end{verbatim}

   The result of this command should be something like:

   \begin{verbatim}
      PCH
      K6
   \end{verbatim}

   Go to the ALSA settings tab illustrated below, and in 
   \textsl{Alsa Audio Device} enter, for example, "hw:PCH".
   This ensures one will always connect to this card at startup regardless of
   the order this and other ones.  Another benefit of using this hardware name
   is that ALSA will now use \textsl{Yoshimi}'s internal
   buffer size (256), otherwise ALSA will force \textsl{Yoshimi} to accept its
   default size (usually 1024).

   One can also set the sample rate, but bear in mind that not all cards can use
   all of these.  The sample rates 44100 and 48000 are almost always available.
   If one sets a Midi Device as well (such as a keyboard) Yoshimi will try to
   find and connect to this device at startup.

   To find the MIDI devices available, try:

   \begin{verbatim}
      $ grep Client /proc/asound/seq/clients
   \end{verbatim}

   The result of this command should be something like:

   \begin{verbatim}
      Client info
      Client   0 : "System" [Kernel]
      Client  14 : "Midi Through" [Kernel]
      Client 128 : "TiMidity" [User]
   \end{verbatim}

   It is not obvious how ALSA audio is controlled and who takes command.  If
   one sets a specific audio destination, then \textsl{Yoshimi} makes a
   request.  It's often a negotiation on bit depth and channel count, but
   \textsl{Yoshimi} nearly always gets to decide the buffer size, which is the
   internal buffer size.  However, if the destination is 'default' then ALSA
   decides on the sound card, bit depth, number of channels and the buffer
   size, and \textsl{Yoshimi} will set it's internal buffer size to match.  On
   most machines this seems to be 1024.

\begin{figure}[H]
   \centering 
%  \includegraphics[scale=0.75]{menu/Yoshimi/yoshimi-settings-alsa.jpg}
   \includegraphics[scale=0.75]{1.3.8/yoshimi-settings-alsa.png}
   \caption[ALSA Settings]{ALSA Settings Dialog}
   \label{fig:yoshimi_settings_alsa}
\end{figure}

   \setcounter{ItemCounter}{0}      % Reset the ItemCounter for this list.

   \itempar{Alsa Midi Source}{ALSA!MIDI Source}
   ALSA MIDI Source.
   The purpose of this setting is the same as the command line option
   \texttt{--alsa-midi="name"}.
   It is used so that \textsl{Yoshimi} can auto connect to a MIDI source
   such as a keyboard.  For example, the one that Will has identifies itself
   as name = "Hua Xing".
   A port name, such as "128:0" (for one of the ports provided by
   \textsl{TiMidity}) should work as well.
   If the destination is "default",
   then ALSA decides on the sound card, bit depth, number of channels and the
   buffer size, and \textsl{Yoshimi} will set it's internal buffer size to
   match.  On most machines this always seems to be 1024.

   Values: \texttt{default*}

   \itempar{Set as preferred MIDI}{ALSA!Set as preferred MIDI}
   Set as preferred MIDI for ALSA.
   This setting determines which MIDI connections a particular instance will
   first attempt. The switches are mutually exclusive across JACK and ALSA,
   so if one checks ALSA for MIDI, it automatically unchecks JACK for MIDI.
   As well as from the GUI, this setting can be set (for instance 0) from the
   command line, both at start-up and once running.

   \itempar{Alsa Audio Device}{ALSA!audio device}
   ALSA Audio Device.
   This specifies the sound card to which \textsl{Yoshimi} can connect.
   Normally, this will be an ALSA hardware specification such as
   "hw:0".
   ALSA audio also lets one connect to a sound card by name. For example,
   with a Komplete Audio KA 6 sound card, the device specification is
   "hw:K6". This feature is particularly useful for USB modules, as one can
   never be sure where they appear numerically.

   Values: \texttt{default*}

   \itempar{Set as preferred audio}{ALSA!Set as preferred audio}
   Set as preferred audio for ALSA.
   This setting determines which audio connections a particular instance will
   first attempt. The switches are mutually exclusive across JACK and ALSA,
   so if one checks ALSA for audio, it automatically unchecks JACK for audio.
   As well as from the GUI, this setting can be set (for instance 0) from the
   command line, both at start-up and once running.

   \itempar{Samplerate}{ALSA!sample rate}
   Sample Rate.
   Sets the quality of the sound, higher is better, but it uses more CPU.  One
   can select from a list.  Note that both ALSA and JACK will support the
   192000 rate, if the sound-card supports it.  To find out the internal delay
   in milliseconds, divide the buffer-size value by the Sample Rate and
   multiply the result by 1000 (256 / 44100 * 1000 = 5.8 ms).

   Note that, as of version 1.3.6, the \textbf{Period Size} field has been
   removed from the \textbf{Alsa} tab, and is replaced by the 
   \textbf{Internal Buffer Size} field in the \textbf{Main Settings} tab.
   Note that any of these setting changes require a restart of \textsl{Yoshimi}
   to take effect.
   
   \textsl{ZynAddSubFX}: if one wants a sample-rate that
   is not in the list, select "Custom" and change the value from the right.
   Default is 44100.

   Values: \texttt{192000, 96000, 48000*, 44100}

\paragraph{Menu / Yoshimi / Settings / MIDI}
\label{paragraph:menu_yoshimi_settings_ccs}

   The CC settings tab has been renamed the "MIDI" tab.
   This tab, shown in
   \figureref{fig:yoshimi_settings_cc},
   presents MIDI bank-root, bank, program change, and extended program
   change settings, plus some new values.

   A new feature as of version 1.3.6 is that some changes to the items in this
   tab cause a red \textbf{Pending} button to appear.  Pressing this
   button saves that particular change.

\begin{figure}[H]
   \centering 
%  \includegraphics[scale=0.75]{menu/Yoshimi/yoshimi-settings-ccs.jpg}
%  \includegraphics[scale=0.75]{1.3.8/yoshimi-settings-midi.png}
   \includegraphics[scale=0.75]{1.3.9/yoshimi-settings-ccs.jpg}
   \caption[MIDI Preferences]{MIDI Preferences}
   \label{fig:yoshimi_settings_cc}
\end{figure}

%  TODO:  Could add a figure showing the red Pending button.

   The following items are provided by the MIDI settings tab:

   \begin{enumber}
      \item \textbf{Enable Bank Root Change}
      \item \textbf{Bank Root Change}
      \item \textbf{Bank Change}
      \item \textbf{Enable Program Change}
      \item \textbf{Enable Part On Program Change}
      \item \textbf{Enable Extended Program Change}
      \item \textbf{Extended Program Change}
      \item \textbf{Ignore Reset all CCs}
      \item \textbf{Log incoming CCs}
      \item \textbf{Save and Close}
      \item \textbf{Close Unsaved}
   \end{enumber}

   \setcounter{ItemCounter}{0}      % Reset the ItemCounter for this list.

   The concepts of banks and roots is very useful.
   See \sectionref{subsec:concepts_banks_and_roots}.
   The settings in this tab affect the usage of banks and root changes
   controlled by MIDI messages, thereby making \textsl{Yoshimi} able to
   implement MIDI automation.

   \itempar{Enable Bank Root Change}{MIDI preferences!enable bank root change}
   Enable Bank Root Change.

   Values: \texttt{Off*, On}

   \itempar{Bank Root Change}{MIDI preferences!bank root change}

   Values: \texttt{0*, to 127}

   If enabled, a new reddish button, \textbf{Pending}, appears.
   Once the change has been made in the scroll list, click this button
   to set the change.
   \textbf{Warning:}
   The \textbf{Save and Close} button will not result in the removal of the
   \textbf{Pending} button.
   This result seems counter-intuitive, but the pending button is not removed
   here because, at that point, it still hasn't actually been either set or
   abandoned. It remains available for when the user actually makes up his/her
   mind.

   \itempar{Bank Change}{MIDI preferences!bank change}
   Bank Change.
   Defines which MIDI preferences one wants to use.
   Note that MIDI Controller 0 = CC0 = Bank Select MSB, and MIDI Controller
   32 = CC32 = Bank Select LSB.
   When combined, these Bank Select messages provide \[128*128 = 16384\]
   banks.

   But note that all a Bank Select does is selects the bank for the next
   Program Change event.  The program doesn't change after changing a bank,
   until a Program Change is sent.

   Bank changes can be completely disabled; some hardware
   synthesizers don't play nice with banks.

   Values: \texttt{LSB, MSB*, Off}

   \itempar{Enable Program Change}{MIDI preferences!enable program change}

   Values: \texttt{Off*, On}

   Enables/disables MIDI program change.
   Program changes can be completely disabled, but some hardware synths don't
   play nice!

   \itempar{Enable Part On Program Change}{MIDI preferences!enable part change}

   Values: \texttt{Off*, On}

   The part is automatically enabled if the MIDI program was changed on this
   part.

   \itempar{Enable Extended Program Change}{MIDI preferences!enable extended program change}

   Values: \texttt{Off*, On}

   \itempar{Extended Program Change}{MIDI preferences!extended change}
   If enabled, a new reddish button, \textbf{Pending}, appears.
   Once the change has been made in the scroll list, click this button
   to set the change.

   Values: \texttt{0-127, 110*}

   \itempar{Ignore Reset all CCs}{MIDI preferences!Ignore Reset all CCs}
   Causes \textsl{Yoshimi} to ignore this message.
   \index{CC monitor}
   For example, using \textsl{Yoshimi}'s fairly new CC monitor (see the next
   item), Will found that \textsl{Rosegarden} was sending CC 121 (reset all
   controllers) at the start of some song segments.  Checking this option
   prevents unwanted resets.

   Values: \texttt{Off*, On}

   \itempar{Log incoming CCs}{MIDI preferences!Log incoming CCs}
   This setting is is about the only setting that is never saved. It is there
   solely as an aid for when \textsl{Yoshimi} appears to ignore MIDI commands,
   as it tells one exactly what \textsl{Yoshimi} thinks it received.

   Values: \texttt{Off*, On}

\subsubsection{Menu / Yoshimi / Exit}
\label{subsubsec:menu_yoshimi_exit}

   Simply exits from \textsl{Yoshimi}.
   The user is prompted if unsaved changes exist, as shown in
   \figureref{fig:yoshimi_change_exit}.

   One can sometimes get a false parameters-changed warning if one
   scrolls through one of the menu type entries without actually changing it.
   Better safe than sorry!

\begin{figure}[H]
   \centering 
   \includegraphics[scale=0.75]{menu/Yoshimi/yoshimi-menu-exit-parameters-changed.jpg}
   \caption[Yoshimi Menu, Exit]{Yoshimi Menu, Exit}
   \label{fig:yoshimi_change_exit}
\end{figure}

% We moved some out-of-date menu entries into the following document, and
% created additional sub-sections to include, with up-to-date information.
%
% %-------------------------------------------------------------------------------
% yum_menu
%-------------------------------------------------------------------------------
%
% \file        yum_menu_instruments.tex
% \library     Documents
% \author      Chris Ahlstrom
% \date        2015-05-11
% \update      2016-02-27
% \version     $Revision$
% \license     $XPC_GPL_LICENSE$
%
%     Provides the Menu section of yoshimi-user-manual.tex.
%
%-------------------------------------------------------------------------------

\subsection{Menu / Instruments}
\label{subsec:menu_instrument}

   The \textsl{Yoshimi} Instruments menu lets one select instruments and work
   with banks of instruments.
   \textsl{Yoshimi} stamps instrument XML files with its own major and minor
   version numbers so it is possible to tell which version created the files,
   or whether they were created by \textsl{ZynAddSubFX}.

   When opening an instrument bank one can now tell exactly which synth engines
   are used by each instrument. This is represented by three pale background
   colours:

   \begin{itemize}
      \item \textcolor{red}{Red}: ADDsynth
      \item \textcolor{blue}{Blue}: SUBsynth
      \item \textcolor{green}{Green}: PADsynth
   \end{itemize}

   These new colored engine backgrounds aren't just pretty. They give real
   information about expected processor load, and time taken to be ready when
   loaded:

   \begin{itemize}
      \item \textsl{Processor Load, low to high}: PAD, SUB, then ADD.
      \item \textsl{Time to initialize, low to high}: SUB, ADD, PAD.
   \end{itemize}

   If the instruments are kits they scanned to find out if 
   \textsl{any} member of the kit contains each engine.
   This scanning is duplicated in the current part, the mixer panel for the
   currently loaded instruments, and in the Instrument Edit window the same
   colors highlight the engine names when they are enabled with the check
   boxes. 

   The following sub-menus are provided, as shown in
   \figureref{fig:yoshimi_instrument_menu}.

\begin{figure}[H]
   \centering 
   \includegraphics[scale=1.0]{1.3.8/yoshimi-menu-instrument.jpg}
   \caption{Yoshimi Menu, Instrument}
   \label{fig:yoshimi_instrument_menu}
\end{figure}

   TODO:  Document the many differences in 1.3.8 here.

   \begin{enumber}
      \item \textbf{Clear Instrument...}
      \item \textbf{Open Instrument...}
      \item \textbf{Save Instrument...}
      \item \textbf{Show Instruments...}
      \item \textbf{Show Banks...}
      \item \textbf{Show Root Paths...}
   \end{enumber}

   These menu entries don't appear in the order in which they would normally
   be used.  For simplicity, it is better, especially for \textsl{Yoshimi}
   1.3.5 and above, to summarize how to navigate through these menu items,
   before showing each one in detail.

   \setcounter{ItemCounter}{0}      % Reset the ItemCounter for this list.

   \itempar{Set Current Root Path}{root!set current}
   \index{root!current}
   Instruments are stored in banks, and banks are stored in root directories,
   also known as "roots".  In \textsl{Yoshimi}, there can be a number of
   roots that exist in a user's directory structure, but only one root can be
   the current root.  Thus, the first step is to set up the current root
   directory to point to where we have stored all our banks of instruments.

   \begin{enumber}
      \item In \textsl{Yoshimi}, navigate to the \textbf{Instrument / Show
      Root Paths...} entry in the main menu.
      \item In the \textbf{Bank Root Paths} dialog, select the desired
      bank path by clicking on it.
      \item Click the \textbf{Make Current} button.
%     \item Then click the \textbf{Open Current} button.
      \item Then click the \textbf{Save and Close} button.
   \end{enumber}

   For example, one can make the \texttt{/usr/share/yoshimi/banks} directory
   the current root directory.  This directory is the default location for
   banks when \textsl{Yoshimi} is first installed.
   In the following figure, we set it to the author's local directory,
   \texttt{~/Audio/yoshimi/banks}.
   In general, it is recommended that one copies the default installed
   directories to a local directory, in order to be able to work with them,
   making additions and changes without needing root permissions, and
   without risking the corruption of the default installation.

   For figures and details about the \textbf{Show Root Paths...} menu entry,
   see \sectionref{subsubsec:menu_instrument_show_root_paths}.

   \itempar{Show Current Bank Set}{banks!show}
   Once the current root has been set, one can then see all of the banks
   under that root.  Navigate to the \textbf{Instrument / Show Banks...} menu
   entry and click it.  This brings up a dialog such as the one shown in
   \figureref{fig:show_ca_banks}.  That dialog shows all of the banks that
   exist in the current root directory.  Each is auto-numbered by
   \textsl{Yoshimi} the first time that root directory is accessed, and the
   current bank is highlighted in pink.

   Clicking on a bank with both make that bank the current bank, and opens an
   instruments dialog, such as that shown in
   \figureref{fig:show_alex_j_bank}.

   \itempar{Show Instruments in Current Bank}{instruments!show}
   Once the current root and current bank have been set, another way to show
   the instruments is to navigate to the
   \textbf{Instrument / Show Instruments...} menu entry and click on it.
   Again, this action opens an instruments dialog, such as that shown in
   \figureref{fig:show_alex_j_bank}.

   A left-click on a particular instrument sets that instrument into the
   current Part in force in the main \textsl{Yoshimi} window, where it can
   then be edited, if desired.

   \index{anti-auto-clutter}
   Right-clicking on an instrument causes the instruments list dialog
   to disappear, and be replaced by an instrument dialog.  While this
   behavior might be surprising, it is part of the anti-auto-clutter feature
   of \textsl{Yoshimi}.

   Now that we know how to easily navigate through roots, banks, and
   instruments, we can discuss each of the \textbf{Instrument} menu entries
   in detail.

\subsubsection{Menu / Instrument / Clear Instrument...}
\label{subsubsec:menu_instrument_clear}

   This menu entry brings up a prompt to clear the parameters of the
   instrument that is currently loaded in the current part.

\begin{figure}[H]
   \centering 
   \includegraphics[scale=0.75]{menu/Instrument/clear-instrument.jpg}
   \caption{Clear Instrument Dialog}
   \label{fig:clear_instrument_dialog}
\end{figure}

   \textbf{Bug:}
   \index{bugs!need to clear instrument?}
   Sometime it seems that one needs to clear the instrument if one is
   loading a new instrument to test it out, because some settings seem
   to remain from the previous instrument.

   \textsl{Don't quote us on that.  Maybe Will has fixed that issue by now.}

\subsubsection{Menu / Instrument / Open Instrument...}
\label{subsubsec:menu_instrument_open}

   This menu entry brings up a prompt to open a new instrument.
   This prompt is a file-dialog, and it doesn't depend at all on the settings
   of the current root or the current bank.  It does have a
   \textbf{Favorites} button to help the user get quickly to the desired
   location of instrument files.

\begin{figure}[H]
   \centering 
   \includegraphics[scale=0.75]{menu/Instrument/open-instrument.jpg}
   \caption{Open Instrument Dialog}
   \label{fig:open_instrument_dialog}
\end{figure}

   This dialog has a number of user-interface elements to discuss.

   \begin{enumber}
      \item \textbf{Show}
      \item \textbf{Favorites}
      \item \textbf{Create a new diretory}
      \item \textbf{Instrument List}
      \item \textbf{XML Preview}
      \item \textbf{Preview}
      \item \textbf{Show hidden files}
      \item \textbf{Directory Bar}
      \item \textbf{Filename}
      \item \textbf{OK}
      \item \textbf{Cancel}
   \end{enumber}

   \setcounter{ItemCounter}{0}      % Reset the ItemCounter for this list.

   \itempar{Show}{Open Instrument!show}
   Show types of files.
   This item shows a file filter for selecting instrument files.
   The types of filters are as follows (screen shot not available):

   \begin{enumber}
      \item \textbf{(\{*.xiz\})} (compressed XML files)
      \item \textbf{All Files (*)}
      \item \textbf{Custom Filter}
   \end{enumber}

   \itempar{Favorites}{Open Instrument!favorites}
   Favorite directories.
   Provides a list of options and favorite directories in which to find 
   instrument files.

\begin{figure}[H]
   \centering 
   \includegraphics[scale=0.75]{menu/Instrument/favorites-dropdown.jpg}
   \caption{Favorites Drop-down}
   \label{fig:open_instrument_favorites}
\end{figure}

   \begin{enumber}
      \item \textbf{Add to Favorites}
      \item \textbf{Manage Favorites}
      \item \textbf{File Systems}
      \item \textbf{(Additional favorite directories)}
   \end{enumber}

   \index{Add to Favorites}
   \textbf{Add to Favorites}
   simply adds the currently selected directory shown in the instrument list
   to the list of favorites.

   To add Favorites in the file dialog, navigate to the desired directory.
   Then click \textbf{Favorites}, and select \textbf{Add to Favorites}.

   Once one has a number of favorites set up,
   there is a \textbf{Manage Favorites} that can be used.
   For example, if one needs to get rid of a directory, one can use the
   \textbf{Manage Favorites}
   \index{Manage Favorites}
   dialog, shown in
   \figureref{fig:manage_instrument_favorites} below,
   to do that.

\begin{figure}[H]
   \centering 
   \includegraphics[scale=1.0]{menu/Instrument/manage-favorites.png}
   \caption{Favorites Drop-down}
   \label{fig:manage_instrument_favorites}
\end{figure}

   \textbf{File Systems} \index{File Systems}
   Provides a list of all file systems starting at root ("\texttt{/}").
   This list can be pretty confusing, with a lot of entries.
   But note that one navigates to ("\texttt{/}"), and from there to
   \texttt{/usr/share/yoshimi/banks} to get easy access to all the
   instruments that are preinstalled with
   \textsl{Yoshimi}.
   Generally, one will want to use only
   \textbf{Add to Favorites} and \textbf{Manage Favorites}.

   \itempar{Create Directory}{Open Instrument!create new directory}
   Creates a New Directory.
   This little symbol options a small "New Directory?" dialog (not shown
   here, it is very simple and stock) into which one can type a directory
   name to be added to the current directory of the instrument list.

   \itempar{Instrument List}{Open Instrument!instrument list}
   Provides a list of the instrument files available in the current
   directory.  Also shown are sub-directories (if available)
   that might contain more instruments, and a ("\texttt{../}") entry
   to navigate to the parent directory.

   \itempar{Preview}{Open Instrument!preview checkbox}
   If one thinks the preview feature is not useful, uncheck this check-box.
   so that one doesn't see the preview window.  As a bonus, one can see more
   of the instrument file-name.

   \itempar{Preview pane}{Open Instrument!preview pane}
   XML Preview.
   This box can show the beginning of the XML data of an instrument file.
   \textbf{Bug:}
   \index{bugs!compressed XML preview}
   It seems to show the XML only if the XML is not compressed.

   \itempar{Show hidden files}{Open Instrument!show hidden files}
   Shows file that are hidden.  Not sure how useful this feature is;
   who would hide a \textsl{Yoshimi} instrument file?

   \itempar{Directory Bar}{Open Instrument!directory bar}
   Provides an alternate way to move up through the directory structure.

   \itempar{Filename}{Open Instrument!filename}
   File Name.
   Provides the full path to the instrument file.

   \itempar{OK/Cancel}{Open Instrument!ok/cancel}
   We don't really need to discuss the \textbf{OK} and \textbf{Cancel}
   buttons, do we?

\subsubsection{Menu / Instrument / Save Instrument...}
\label{subsubsec:menu_instrument_save}

   This menu entry brings up a prompt to save a new instrument within the
   user's file system.
   It has all of the user-interface elements of the "Open Instrument"
   dialog shown in
   \figureref{fig:open_instrument_dialog}
   in \sectionref{subsubsec:menu_instrument_open}.
   Like that dialog, it is not dependent on the current root or current bank.
   However, if nothing has changed, then a "Nothing to Save!" prompt (not
   pictured) is shown.

   With \textsl{ZynAddsubFX} and older versions of \textsl{Yoshimi},
   it was possibly to end up with unnamed instruments. Since version
   1.3.4, \textsl{Yoshimi} will trap such an occurrence and name it
   'No Title'; it will not let one save the unedited default sound.

\subsubsection{Menu / Instrument / Show Instruments...}
\label{subsubsec:menu_instrument_show}

   Instruments are stored in banks. The banks (and current bank setting)
   are loaded/saved
   automatically by the program, so one doesn't have to worry about saving the
   banks before the program exits. On program start, the last used bank is
   loaded. A single bank can store up to 128 instruments. 
   However, there is space for a number of additional
   instruments in the bank, the extended-program section, to allow up to 160
   instruments in a bank.

   When the \textbf{Show Instruments...} button is selected, a dialog comes
   up that shows all of the instruments present in the currently-selected
   bank.
   
\begin{figure}[H]
   \centering 
   \includegraphics[scale=0.75]{1.3.6/Alex_J_bank_instruments.png}
   \caption[Instruments in Current Bank]{Instruments in Current Bank 1.3.6}
   \label{fig:show_alex_j_bank}
\end{figure}

   As \figureref{fig:show_alex_j_bank}
   shows, this is a very complex dialog with a lot of options.
   Note how \textsl{Yoshimi} now shows the color codings for the
   synth-sections used in each instrument:
   red for ADDsynth, blue for SUBsynth, and
   green for PADsynth.

   Also note how the numbers at the beginning of the filenames are used as
   an "instrument" or "program" number.  These numbers can be used in MIDI
   Program Change commands.
   
   All of the files with filenames starting with 4-digit numbers will be
   shown in the slot corresponding number.  Those without numbers will start
   with numbers at 129 or above ("extended program change").  One should give
   them numbers by renaming them outside of \textsl{Yoshimi}, then reloading
   the bank.

   \index{extended program}
   Note that MIDI CC
   (see \sectionref{paragraph:menu_yoshimi_settings_ccs})
   can be set to access voices from 129 to 160.
   All the Bank controls in the \textbf{MIDI} settings tab take immediate
   effect when set.
   Bank and program changes can be completely disabled in the settings tab;
   some hardware synths don't play nice with it.

   Learning how to use the Instruments dialog is an important way to make
   instruments easier to manage, and so this will be a long discussion.

%  An important pair of concepts in \textsl{Yoshimi} are
%  \textsl{banks} and \textsl{roots}.  These concepts are described in
%  \sectionref{subsec:concepts_banks_and_roots}.

%  A bank has 3 modes in \textsl{ZynAddSubFX}: 

%  \begin{enumber}
%     \item \textbf{READ}.
%        The instrument is loaded from the bank to the current part.
%     \item \textbf{WRITE}.
%        The instrument is written to the bank.
%     \item \textbf{CLEAR}.
%        The instrument from the bank is cleared (removed).
%  \end{enumber}

%  Pressing the left mouse button on a slot reads/writes/clears the
%  instrument from/to it (according to the current mode).
   
%  Pressing the right mouse button on a slot changes its name.

%  The setup in \textsl{Yoshimi} is a bit different than in
%  \textsl{ZynAddSubFX}.
%  Observe \figureref{fig:show_ca_bank}.
%  It shows a bank loaded from a directory containing customs
%  banks from one of the authors of this document.

   Note that this dialog has been modified in recent versions of
   \textsl{Yoshimi}.

   Here is a list of the user-interface items in the instruments/banks dialog.

   \begin{enumber}
      \item \textbf{Bank Names}
      \item \textbf{Roots}
      \item \textbf{Banks}
      \item \textbf{Instrument and Bank Matrix}
      \item \textbf{SELECT}
      \item \textbf{RENAME}
      \item \textbf{SAVE}
      \item \textbf{DELETE}
      \item \textbf{SWAP}
      \item \textbf{Show synth engines}
         (was \textbf{Show PADsynth status})
      \item \textbf{Close}
   \end{enumber}

   \setcounter{ItemCounter}{0}      % Reset the ItemCounter for this list.

   \itempar{Bank Names}{instruments!bank names}
   Instruments Bank Name.
   Basically, each bank is a directory name, with a number prepended.
   The banks are found under the current root, which is a also a directory
   name, and is the name of the parent directory of a set of banks.
   Here is the Bank Names drop-down list for "my" setup, which has the
   default banks plus a lot of banks found around the Internet:

\begin{figure}[H]
   \centering 
   \includegraphics[scale=0.75]{menu/Instrument/bank-list.jpg}
   \caption[A Sample Bank List]{A Sample Bank List}
   \label{fig:bank_list}
\end{figure}

   And here is the directory listing associated with it, in the order
   produced by the UNIX/Linux "ls -1" (list single-column) command (shown in
   two columns to save space):

   \begin{verbatim}
      Alex_J                        Noises
      Arpeggios                     Organ
      Bass                          Pads
   '  Bells                         Piano
      Brass                         Plucked
      C_Ahlstrom                    RB Zyn Presets
      chip                          README
      Choir_and_Voice               Reed_and_Wind
      Chromatic Percussion          Rhodes
      Cormi_Collection              Splited
      Drums                         Strings
      Drums_DS                      Synth
      Dual                          SynthPiano
      Electric Piano                Test
      Fantasy                       The_Mysterious_Bank
      Flute                         the_mysterious_banks
      folderol collection           Vanilla
      Guitar                        VDX
      Internet Collection           Will_Godfrey_Collection
      Laba170bank                   Will_Godfrey_Companion
      Leads                         Will_J_Godfrey_Collection
      Louigi_Verona_Workshop        x31eq.com
      Misc                          XAdriano Petrosillo
      Misc Keys                     Zen Collection
      mmxgn Collection
   \end{verbatim}

   The first thing to note is that there are only 128 \textsl{Yoshimi} banks
   supported in a \textsl{Yoshimi} root.  The list above takes up about half
   of the available slots, so it might be time to move some of those banks
   to a new root directory.

   The numbers in the drop-down list are generated by \textsl{Yoshimi} the
   first time it sees a new root path or a new bank within the root path.
   Once set, these numbers will never change unless one actually moves them
   around (using the \textbf{SWAP} button).

   The bank number is also the MIDI ID for the bank;
   one can be sure that it will always
   be there for bank changes, no matter how many banks are added later.
   \textsl{Yoshimi} always lists the banks in ID order, not alphabetical
   order, so one can group them sensibly and permanently.
   However, at first-time creation \textsl{Yoshimi} sets the IDs in
   alphabetical order and tries to space them evenly over the range to
   provide some wiggle room.                                        

   Selecting one of the items in this drop-down list selects the bank and
   loads it into the Banks dialog.

   \index{anti-auto-clutter}
   Right-clicking on a bank causes the banks list dialog
   to disappear, and be replaced by the bank dialog.  While this
   behavior might be surprising, it is part of the anti-auto-clutter feature
   of \textsl{Yoshimi}.

   \itempar{Roots}{instruments!roots}
   Instruments Roots Button.
   Shows a list of directories that can serve as "root" directories.
   The "Bank Root Paths" dialog shown in
   \figureref{fig:show_banks_roots} shows
   the system root (e.g. \texttt{/usr/share/yoshimi/banks}) and
   a user's home location for his/her banks and roots.

   \itempar{Banks}{instruments!banks}
   Banks Button.
   This item brings up a Banks dialog showing all of the banks present in the
   current root.
   It is an alternative to using the \textbf{Bank Names} dropdown list.

   \itempar{Instrument and Bank Matrix}{instruments!bank matrix}
   Instruments Bank Matrix.
   Shows the instruments that are in the currently selected bank
   (directory).

   \itempar{SELECT}{instruments!SELECT}
   Instruments SELECT.
   When this button is selected, then clicking on an instrument selects that
   instrument as the instrument for the current Part active in the main
   window.

   \itempar{RENAME}{instruments!RENAME}
   Instruments RENAME.
   When this button is selected, then clicking on a bank brings
   up a small dialog to rename the clicked-on bank.
   However, one might also experience the following warning message:

   \begin{verbatim}
      This instrument file cannot be changed
   \end{verbatim}

   \itempar{SAVE}{instruments!SAVE}
   Instruments SAVE.
   When this button is selected, then clicking on a bank saves
   the instruments as currently configured.
   However, one might also experience the following warning message:

   \begin{verbatim}
      This instrument file cannot be changed
   \end{verbatim}

   \itempar{DELETE}{instruments!DELETE}
   Instruments DELETE.
   Selecting this button and clicking an empty bank entry does nothing.
   Selecting this button and clicking an existing bank entry brings up a
   small dialog asking one if this bank is really to be deleted.
   However, one might also experience the following warning message:

   \begin{verbatim}
      This instrument file cannot be changed
   \end{verbatim}

   \itempar{SWAP}{instruments!SWAP}
   Instruments SWAP.
   Selecting this button, then selecting one bank, and then another,
   swaps the numbering and postion of the selected banks.
   However, one might also experience the following warning message:

   \begin{verbatim}
      This instrument file cannot be changed
   \end{verbatim}

   \itempar{Show synth engines}{instruments!show engines}
   If enabled, then the usage of each of the \textsl{Yoshimi} synthesis
   engines is indicated by color coding, as shown in the figure above.

   \itempar{Close}{instruments!Close}
   Closes the window.

   Here is a more conventional view of instruments, supplied with
   \textsl{Yoshimi}, shown in
   \figureref{fig:show_pads_bank}.

\begin{figure}[H]
   \centering 
%  \includegraphics[scale=0.75]{menu/Instrument/show-pads-bank.jpg}
   \includegraphics[scale=0.75]{1.3.6/show_pads_bank.png}
   \caption[Show Pads Instruments]{Show Pads Instruments}
   \label{fig:show_pads_bank}
\end{figure}

   Note that many of these Pads instruments also use the Add and Sub
   components as well.

\subsubsection{Menu / Instrument / Show Banks...}
\label{subsubsec:menu_instrument_show_banks}

   This menu entry brings up a dialog that shows all of the banks present in
   the current root.

\begin{figure}[H]
   \centering 
   \includegraphics[scale=0.75]{1.3.6/show_CA_banks.png}
   \caption[Show Banks]{Show Banks in Current Root}
   \label{fig:show_ca_banks}
\end{figure}

   This figure illustrates a setup where the installed banks were combined with
   banks downloaded from various web sites.
   The following list shows that the interface elements in the banks dialog
   are slightly different from the instruments dialog.

   \begin{enumber}
      \item \textbf{Roots}
      \item \textbf{Current Bank} (passive display element)
      \item \textbf{Instruments}
      \item \textbf{SELECT}
      \item \textbf{RENAME}
      \item \textbf{ADD}
      \item \textbf{DELETE}
      \item \textbf{SWAP}
      \item \textbf{Close}
   \end{enumber}

   \setcounter{ItemCounter}{0}      % Reset the ItemCounter for this list.

   \itempar{Roots}{banks!roots}
   Banks Roots.
   "Roots" button.
   Shows a list of directories that can serve as "root" directories.

   \itempar{current bank}{banks!current bank}
   \index{current!bank}
   Current Bank.  Simply indicates the current bank via color-highlighting.
   Note that one can left-click on a bank in this dialog to make it the
   current bank.  This setting is saved across \textsl{Yoshimi} restarts.

   \itempar{Instruments}{banks!instruments}
   Banks Instruments.
   \index{current!bank}
   Brings up a banks dialog that shows the instruments in the current bank.

   \itempar{SELECT}{banks!SELECT}
   Banks SELECT.
   When this button is selected, then clicking on a bank makes it the current
   bank.

   (Although we don't show a figure for it, note that some banks provide
   instruments with numbers in the extended program-change range (above
   127) prepended to the file-names.)

   \itempar{RENAME}{banks!RENAME}
   Banks RENAME.
   When this button is selected, then clicking on a bank brings
   up a small dialog to rename the clicked-on bank.
   However, one might also experience the following warning message:

   \begin{verbatim}
      This bank directory cannot be changed
   \end{verbatim}

   \itempar{ADD}{banks!ADD}
   Banks ADD.
   Selecting this button and clicking an empty bank entry brings up a small
   dialog to create a new empty bank name for that entry.
   If one clicks on an existing bank entry, then a small dialog comes up
   stating that the bank number selected is already in use.
   However, one might also experience the following warning message:

   \begin{verbatim}
      This bank directory cannot be changed
   \end{verbatim}

   \itempar{DELETE}{banks!DELETE}
   Banks DELETE.
   Selecting this button and clicking an empty bank entry does nothing.
   Selecting this button and clicking an existing bank entry brings up a
   small dialog asking one if this bank is really to be deleted.
   However, one might also experience the following warning message:

   \begin{verbatim}
      This bank directory cannot be changed
   \end{verbatim}

   \itempar{SWAP}{banks!SWAP}
   Banks SWAP.
   Selecting this button, then selecting one bank, and then another,
   swaps the numbering and postion of the selected banks.
   This button is good for minor reorganization of the bank numbers.

\subsubsection{Menu / Instrument / Show Root Paths...}
\label{subsubsec:menu_instrument_show_root_paths}

\begin{figure}[H]
   \centering 
   \includegraphics[scale=0.75]{menu/Instrument/show-banks-roots.jpg}
   \caption[Show Root Paths]{Show Root Paths}
   \label{fig:show_banks_roots}
\end{figure}

   \setcounter{ItemCounter}{0}      % Reset the ItemCounter for this list.

   \itempar{Add root directory...}{Root Paths!add directory}
   Show Root Paths Add Root Directory.
   To add a bank root path:

   \textsl{Yoshimi} (as installed by Debian Linux) provides a default bank at
   \texttt{/usr/share/yoshimi/banks}.
   To add one's own directory, navigate to "Yoshimi / Instrument / Show Root
   Paths ...".  Then click on "Add root directory...".

   Once selected, one will see that \texttt{/usr/share/yoshimi/banks}
   is marked with an asterisk.  One can select the new root directory,
   and make it current by clicking the "Make current" button.
   Then the Banks dialog will show all the banks in that directory, one bank
   per subdirectory (each subdirectory "is" a bank).

\begin{figure}[H]
   \centering 
   \includegraphics[scale=0.75]{menu/Instrument/new-directory.jpg}
   \caption{Add Root Directory}
   \label{fig:add_root_directory}
\end{figure}

   \itempar{Remove root directory...}{Root Paths!remove directory}
   Show Root Paths Remove Root Directory.
   If a path is selected, then this button is active, and can be used to
   delete the selected path from the "root paths" list.

   \itempar{Make current}{Root Paths!make current}
   Show Root Paths Make Current.
   \index{current!root}
   This button marks the currently-selected path as the "current root" path.

   \itempar{Open current}{Root Paths!open current}
   Show Root Paths Open Current.
   This button opens the current root path.

   \itempar{Change ID}{Root Paths!change ID}
   Show Root Paths Change ID.

   Values: \texttt{0* to 127}

   \textsl{
   We need to know more about how this ID can be used.
   Is it a way to make the path selectable via an extended MIDI control, or
   some other automation method?
   }

%-------------------------------------------------------------------------------
% vim: ts=3 sw=3 et ft=tex
%-------------------------------------------------------------------------------


%-------------------------------------------------------------------------------
% yum_menu_instruments
%-------------------------------------------------------------------------------
%
% \file        yum_menu_instruments.tex
% \library     Documents
% \author      Chris Ahlstrom
% \date        2016-02-27
% \update      2016-02-27
% \version     $Revision$
% \license     $XPC_GPL_LICENSE$
%
%     Provides the Menu / Instruments section of yoshimi-user-manual.tex,
%     for the 1.3.8 and above versions of Yoshimi.  That menu entry has changed
%     a lot.
%
%-------------------------------------------------------------------------------

\subsection{Menu / Instruments}
\label{subsec:menu_instrument}

   The \textsl{Yoshimi} Instruments menu lets one select instruments and work
   with banks of instruments.
   \textsl{Yoshimi} stamps instrument XML files with its own major and minor
   version numbers so it is possible to tell which version created the files,
   or whether they were created by \textsl{ZynAddSubFX}.

   When opening an instrument bank one can now tell exactly which synth engines
   are used by each instrument. This is represented by three pale background
   colours:

   \begin{itemize}
      \item \textcolor{red}{Red}: ADDsynth
      \item \textcolor{blue}{Blue}: SUBsynth
      \item \textcolor{green}{Green}: PADsynth
   \end{itemize}

   These new colored engine backgrounds aren't just pretty. They give real
   information about expected processor load, and time taken to be ready when
   loaded:

   \begin{itemize}
      \item \textsl{Processor Load, low to high}: PAD, SUB, then ADD.
      \item \textsl{Time to initialize, low to high}: SUB, ADD, PAD.
   \end{itemize}

   If the instruments are kits they scanned to find out if 
   \textsl{any} member of the kit contains each engine.
   This scanning is duplicated in the current part, the mixer panel for the
   currently loaded instruments, and in the Instrument Edit window the same
   colors highlight the engine names when they are enabled with the check
   boxes. 

   The following sub-menus are provided, as shown in
   \figureref{fig:yoshimi_instrument_menu}.

\begin{figure}[H]
   \centering 
   \includegraphics[scale=1.0]{1.3.8/yoshimi-menu-instrument.jpg}
   \caption{Yoshimi Menu, Instrument}
   \label{fig:yoshimi_instrument_menu}
\end{figure}

   This new version of the \textbf{Instrument} menu is somewhat different than
   the old version.  It is actually simpler and easier to use, while still
   offering all of the power of the setting up of instruments in
   \textsl{Yoshimi}.

   \begin{enumber}
      \item \textbf{Show Stored...}
      \item \textbf{Load External...}
      \item \textbf{Save External...}
      \item \textbf{Clear}
   \end{enumber}

%  \setcounter{ItemCounter}{0}      % Reset the ItemCounter for this list.

\subsubsection{Menu / Instrument / Show Stored...}
\label{subsubsec:menu_instrument_show}

   Instruments are stored in banks. The banks (and current bank setting)
   are loaded/saved
   automatically by the program, so one doesn't have to worry about saving the
   banks before the program exits. On program start, the last used bank is
   loaded. A single bank can store up to 128 instruments. 
   However, there is space for a number of additional
   instruments in the bank, the extended-program section, to allow up to 160
   instruments in a bank.

   When the \textbf{Show Stored...} button is selected, a dialog comes
   up that shows all of the instruments present in the currently-selected
   bank.
   
\begin{figure}[H]
   \centering 
   \includegraphics[scale=0.75]{1.3.9/instruments_show_stored.png}
   \caption[Show Stored Instruments]{Instruments Stored in Current Banks}
   \label{fig:instruments_show_stored}
\end{figure}

   As \figureref{fig:instruments_show_stored}
   shows, this is a very complex dialog with a lot of options.
   The figure shows a default setup, with the first bank of instruments,
   \textbf{5. Arpeggios}, listed.
   If one drops this list down (shown later), one also observes that the banks
   are numbered in increments of 5, to make it easier for a user to insert his
   or her own bank(s) of instruments.

   Note how \textsl{Yoshimi} now shows the color codings for the
   synth-sections used in each instrument:
   red for ADDsynth, blue for SUBsynth, and
   green for PADsynth.

   Also note how the numbers at the beginning of the filenames are used as
   an "instrument" or "program" number.  These numbers can be used in MIDI
   Program Change commands.
   
   All of the instrument files (such as \texttt{0001-Arpeggio1.xiz})
   with filenames starting with 4-digit numbers will be
   shown in the corresponding slot number.  Those instrument files
   without numbers (or larger numbers?) will start
   with numbers at 129 or above ("Extended Program Change").  One should give
   them numbers by renaming them outside of \textsl{Yoshimi}, then reloading
   the bank.

   \index{extended program}
   Note that MIDI CC
   (see \sectionref{paragraph:menu_yoshimi_settings_ccs})
   can be set to access voices from 129 to 160.
   All the Bank controls in the \textbf{MIDI} settings tab take immediate
   effect when set.
   Bank and program changes can be completely disabled in the settings tab;
   some hardware synths don't play nice with it.

   Learning how to use the Instruments dialog is an important way to make
   instruments easier to manage, and so this will be a long discussion.

%  An important pair of concepts in \textsl{Yoshimi} are
%  \textsl{banks} and \textsl{roots}.  These concepts are described in
%  \sectionref{subsec:concepts_banks_and_roots}.

   Here is a list of the user-interface items in the instruments/banks dialog:

   \begin{enumber}
      \item \textbf{Bank Names}
      \item \textbf{Roots}
      \item \textbf{Banks}
      \item \textbf{Instrument and Bank Matrix}
      \item \textbf{SELECT}
      \item \textbf{RENAME}
      \item \textbf{SAVE}
      \item \textbf{DELETE}
      \item \textbf{SWAP}
      \item \textbf{Show synth engines}
         (was \textbf{Show PADsynth status})
      \item \textbf{Close}
   \end{enumber}

   \setcounter{ItemCounter}{0}      % Reset the ItemCounter for this list.

   \itempar{Bank Names}{instruments!bank names}
   Instruments Bank Name.
   This items is a drop-down list of the available instrument banks in the
   currently-selected \textbf{root} directory.

   Basically, each bank is a directory name, with a number prepended.
   The banks are found under the current root, which is a also a directory
   name, and is the name of the parent directory of a set of banks.
   Here is the Bank Names drop-down list for the default setup, which has the
   default banks provided by the basic \textsl{Yoshimi} installation.

\begin{figure}[H]
   \centering 
%  \includegraphics[scale=0.75]{menu/Instrument/bank-list.jpg}
   \includegraphics[scale=0.75]{1.3.9/instruments_bank_list.png}
   \caption[A Sample Bank List]{A Sample Bank List}
   \label{fig:bank_list}
\end{figure}

   And here is the directory listing associated with it, in the order
   produced by the UNIX/Linux "ls -1" (list single-column) command (shown in
   two columns to save space):

   \begin{verbatim}
      Arpeggios          Pads
      Bass               Plucked
      Brass              Reed_and_Wind
      chip               Rhodes
      Choir_and_Voice    Splited
      Drums              Strings
      Dual               Synth
      Fantasy            SynthPiano
      Guitar             The_Mysterious_Bank
      Misc               Will_Godfrey_Collection
      Noises             Will_Godfrey_Companion
      Organ              
   \end{verbatim}

   The directories (banks) shown above come from the default \textbf{root}
   when \textsl{Yoshimi} and its data files are installed:
   \texttt{/usr/share/yoshimi/banks}.
   If one installed \textsl{Yoshimi} by building the source code, then
   this directory will be
   \texttt{/usr/local/share/yoshimi/banks}.

   The first thing to note is that there are only 128 \textsl{Yoshimi} banks
   supported in a \textsl{Yoshimi} root.  The list above takes up about half
   of the available slots, so it might be time to move some of those banks
   to a new root directory.

   The numbers in the drop-down list are generated by \textsl{Yoshimi} the
   first time it sees a new root path or a new bank within the root path.
   Once set, these numbers will never change unless one actually moves them
   around (using the \textbf{SWAP} button).

   The bank number is also the MIDI ID for the bank;
   one can be sure that it will always
   be there for bank changes, no matter how many banks are added later.
   \textsl{Yoshimi} always lists the banks in ID order, not alphabetical
   order, so one can group them sensibly and permanently.
   However, at first-time creation \textsl{Yoshimi} sets the IDs in
   alphabetical order and tries to space them evenly over the range to
   provide some wiggle room.                                        

   Selecting one of the items in this drop-down list selects the bank and
   loads it into the Banks dialog.

   \index{anti-auto-clutter}
   Right- or left-clicking on a bank in the drop-down list
   causes the instrument list of the previous bank to be replaced by the
   instrument list of the newly-selected bank.

   \itempar{Roots}{instruments!roots}
   Instruments Roots Button.
   Shows a list of directories that can serve as "root" directories.
   The "Bank Root Paths" dialog discussed in
   \sectionref{subsubsec:menu_patch_sets_patch_bank_roots} in
   \figureref{fig:show_patch_banks} shows
   the system root (e.g. \texttt{/usr/share/yoshimi/banks}) and
   a user's home location for his/her banks and roots.

   \itempar{Banks}{instruments!banks}
   Banks Button.
   This item brings up a Banks dialog showing all of the banks present in the
   current root.
   It is an alternative to using the \textbf{Bank Names} drop-down list to
   select a bank.  It is also a way to reorganize and renumber the
   banks without using the Linux console or a file-explorer application to do
   so.

   \itempar{Instrument and Bank Matrix}{instruments!bank matrix}
   Instruments Bank Matrix.
   Shows the instruments that are in the currently selected bank
   (directory).

   The next few items are selector buttons that determine what happens when one
   clicks on an instrument name.

   \itempar{SELECT}{instruments!SELECT}
   Instruments SELECT.
   When this button is selected, then clicking on an instrument selects that
   instrument as the instrument for the current Part active in the main
   window.  In the main window of \textsl{Yoshimi}, that instrument name will
   appear in the currently-selected \textbf{Part}.  If \textsl{Yoshimi} is
   writing to a console window then each part, when clicked, will be shown:

   \begin{verbatim}
		yoshimi> Loaded 64 "Hyper Organ1" to Part 0
		Loaded 65 "Hyper Arpeggio" to Part 0
		Loaded 10 "Arpeggio11" to Part 0
		Loaded 41 "Soft Arpeggio4" to Part 0
		Loaded 67 "Glass Arpeggio1" to Part 0
   \end{verbatim}

   (Remember that "Part 0" in the console is actually "Part 1" in the main
   window.)

   \itempar{RENAME}{instruments!RENAME}
   Instruments RENAME.
   When this button is selected, then clicking on a bank brings
   up a small dialog to rename the clicked-on bank.
   However, one will see the following warning message if trying to rename a
   file that is in a directory not modifiable by normal users:

   \begin{verbatim}
      ! Could not rename instrument 39 to Soft Arpeggio5 [Close]
   \end{verbatim}

   Note that, as soon as this operation is done, the auto-selector (green
   check-box) moves back to the \textbf{SELECT} button.

   \itempar{SAVE}{instruments!SAVE}
   Instruments SAVE.
   When this button is selected, then clicking on a bank saves
   the instruments as currently configured.
   A prompt like the following will appear:

   \begin{verbatim}
      ? Overwrite the slot no. 43 ?  [No/Yes]
   \end{verbatim}

   However, if one answers yes, and the instrument is in a non-modifiable
   directory, then one will see the following error message:

   \begin{verbatim}
      ! Could not save to this location [Close]
   \end{verbatim}

   \itempar{DELETE}{instruments!DELETE}
   Instruments DELETE.
   Selecting this button and clicking an empty bank entry does nothing.
   Selecting this button and clicking an existing bank entry brings up a
   small dialog asking one if this bank is really to be deleted.

   \begin{verbatim}
      ? Clear the slot no. 68?  [No/Yes]
   \end{verbatim}

   However, if one answers yes, and the instrument is in a non-modifiable
   directory, then one will see the following error message:

   \begin{verbatim}
      ! Could not clear this location  [Close]
   \end{verbatim}

   \itempar{SWAP}{instruments!SWAP}
   Instruments SWAP.
   Selecting this button, then selecting one instrument, and then another,
   swaps the numbering and postion of the selected instruments.
   However, one might also experience the following warning message:

   \begin{verbatim}
      ! Could not swap these locations  [Close]
   \end{verbatim}

   Note that all of the above error messages are also shown in the console, if
   it is where \textsl{Yoshimi} is running.  For example:

   \begin{verbatim}
      40 Failed to remove /usr/local/share/yoshimi/banks/Arpeggios/0041-Soft
      Arpeggio3.xiz Permission denied
   \end{verbatim}

   \itempar{Show synth engines}{instruments!show engines}
   If enabled, then the usage of each of the \textsl{Yoshimi} synthesis
   engines is indicated by color coding, as shown in the figure above.

   \itempar{Close}{instruments!Close}
   Closes the window.

%  Here is a more conventional view of instruments, supplied with
%  \textsl{Yoshimi}, shown in
%  \figureref{fig:show_pads_bank}.
% 
% \begin{figure}[H]
%    \centering 
% %  \includegraphics[scale=0.75]{menu/Instrument/show-pads-bank.jpg}
%    \includegraphics[scale=0.75]{1.3.6/show_pads_bank.png}
%    \caption[Show Pads Instruments]{Show Pads Instruments}
%    \label{fig:show_pads_bank}
% \end{figure}
% 
%    Note that many of these Pads instruments also use the Add and Sub
%    components as well.

\subsubsection{Menu / Instrument / Load External...}
\label{subsubsec:menu_instrument_load}

   This menu entry simply brings up a file dialog, allowing the user to
   navigate to an arbitrary directory, and then to a solitary instrument file
   (\texttt{*.xiz}), and load it into the current Part.

   These "xiz" files are normally found in a \texttt{banks} directory, but this
   operation allows access to instruments that are not located in a bank.

\subsubsection{Menu / Instrument / Save External...}
\label{subsubsec:menu_instrument_save}

   This menu entry simply brings up a file dialog, allowing the user to
   navigate to an arbitrary directory, and then save the current Part
   to a solitary instrument file (\texttt{*.xiz}).

\subsubsection{Menu / Instrument / Clear}
\label{subsubsec:menu_instrument_clear}

   This menu entry simply clears the instrument that is loaded into the current
   Part.  This converts the instrument to a 
   \textsl{Simple Sound} patch.

%-------------------------------------------------------------------------------
% vim: ts=3 sw=3 et ft=tex
%-------------------------------------------------------------------------------


%-------------------------------------------------------------------------------
% yum_menu_patch_sets
%-------------------------------------------------------------------------------
%
% \file        yum_menu_patch_sets.tex
% \library     Documents
% \author      Chris Ahlstrom
% \date        2016-02-27
% \update      2018-03-11
% \version     $Revision$
% \license     $XPC_GPL_LICENSE$
%
%     Provides the Menu / Instruments section of yoshimi-user-manual.tex,
%     for the 1.3.8 and above versions of Yoshimi.  That menu entry has changed
%     a lot.
%
%-------------------------------------------------------------------------------

\subsection{Menu / Patch Sets}
\label{subsubsec:menu_patch_sets}

   This new menu entry is part of the very nice reorganization and simplification
   of the handling of roots and banks in the new \textsl{Yoshimi}.  The
   \textbf{Patch Sets} menu replaces the old \textbf{Parameters} menu.  Do you
   like the new name?  The patch set saves all of the settings, including effects
   and instruments.  Patch sets will save all other instruments regardless of
   whether they are activated or not.  Default instruments are never saved, not
   even in patch sets, but if the parts are activated that fact \textsl{is}
   saved.  It is a part feature, not an instrument feature.

   \textsl{Yoshimi} stamps its configuration XML files with its own major and
   minor version numbers so it is possible to tell which version created the
   files, or whether they were created by \textsl{ZynAddSubFX}.

   The main dialog is somewhat similar in layout and function to the
   dialog shown in
   \figureref{fig:instruments_show_stored},
   for managing instruments in a selected bank.

\subsubsection{Menu / Patch Sets / Show Patch Banks...}
\label{subsubsec:menu_patch_sets_show_patch_banks}

   The \textbf{Banks} window has had some button shuffling, and one can
   import and export banks as well.

\begin{figure}[H]
   \centering
%  \includegraphics[scale=0.75]{menu/Instrument/show-banks-roots.jpg}
%  \includegraphics[scale=0.75]{1.3.9/patch_sets_show_patch_banks.png}
   \includegraphics[scale=0.75]{1.5.7/Bank.png}
   \caption[Show Patch Banks]{Show Patch Banks}
   \label{fig:show_patch_banks}
\end{figure}

   Here is a list of the user-interface items in the patch-banks dialog:

   \begin{enumber}
      \item \textbf{Roots}
      \item \textbf{current bank}
      \item \textbf{Instruments}
      \item \textbf{Bank Matrix}
      \item \textbf{SELECT}
      \item \textbf{RENAME}
      \item \textbf{SAVE}
      \item \textbf{DELETE}
      \item \textbf{SWAP}
      \item \textbf{IMPORT}
      \item \textbf{EXPORT}
      \item \textbf{Close}
   \end{enumber}

   \setcounter{ItemCounter}{0}      % Reset the ItemCounter for this list.

   \itempar{Roots}{Roots!root directories}
   Show Patch Banks, Root Directories.
   To add a bank root path, delete a bank root path, or manage bank root path,
   press this button.  The result is somewhat similar to a file dialog,
   and is described in detail in
   \sectionref{subsubsec:menu_patch_sets_patch_bank_roots}, later in
   this sub-chapter.

   \itempar{current bank}{banks!current bank}
   This item is highlighted in pink, and the bank that is actually the current
   bank is also highlighted in pink.  There is no action associated with this
   user-interface element; it merely indicates the currently-selected bank.

   \itempar{Instrument}{banks!instrument}
   This button brings up an instruments window similar
   to the one shown in
   \figureref{fig:instruments_show_stored}, which shows
   the instruments collected in the currently-selected bank.
   Clicking on a bank in the dialog also brings up the instruments window.

   \itempar{Bank Matrix}{banks!matrix}
   This view shows all of the banks available in the current root.
   Left-clicking on a bank in the dialog brings up the Instruments window for
   that bank.
   Right-clicking on a bank in the dialog brings up the Instruments window for
   that bank, but also closes the banks window, to reduce clutter.

   \itempar{IMPORT}{banks!IMPORT}
   There are a number of benefits to using the IMPORT/EXPORT buttons
   rather than dealing with the directories externally.
   One has far greater control where things go when
   importing, and it's much easier to identify the bank to export.

   When importing or exporting,
   \textsl{Yoshimi} refuses to overwrite exisiting banks or
   directories. That is a flat refusal for exporting, but for importing it will
   add a numeric suffix to the name.

   Importing will copy in \textsl{only} files that
   \textsl{Yoshimi} understands, but will notify
   if there were other unrecognised types in there.
   Exporting just dumps out the entire bank contents.

   There are a number of banks in the wild that contain all sorts of extraneous
   stuff, usually copyright notices; one should use only the instrument
   text fields, provided for exactly that purpose.
   Oh, and one bank Will found had subdirectories with pictures,
   and they weren't small!

   In the main part \textbf{Instrument Edit} window there is a new
   \textbf{Default} button top right.
   See \sectionref{subsec:bottom_panel_instrument_edit}.

   We hope this encourages people
   to fill in the Author and Copyright information.
   To set it up, fill in the text field as normal,
   then, while holding down the Ctrl key, click on the button
   (left or middle mouse click) . This text will now be stored in
   one's
   \textsl{Yoshimi} configuration directory,
   and whenever one creates a new instrument, just
   click on the \textbf{Default} button, and the saved text will be
   filled in.

   \itempar{EXPORT}{banks!EXPORT}
   Export of banks is described in the previous section.

   The buttons \textbf{SELECT}, \textbf{RENAME}, \textbf{SAVE},
   \textbf{DELETE}, and \textbf{SWAP} behave similarly to the same buttons in
   the Instruments window, as
   described in the discussion at
   \sectionref{subsubsec:menu_instrument_show}.

\subsubsection{Menu / Patch Sets / Load External...}
\label{subsubsec:menu_patch_sets_load}

   This menu entry simply brings up a file dialog, allowing the user to
   navigate to an arbitrary directory, and then to a solitary instrument file
   (\texttt{*.xmz}), and load it into the current set of parts.

\begin{figure}[H]
   \centering
   \includegraphics[scale=0.75]{menu/Parameters/open-parameters.jpg}
   \caption{Load Patch Set}
   \label{fig:yoshimi_menu_open_parameters}
\end{figure}

   These "xmz" files are normally found in a \texttt{banks} directory, but this
   operation allows access to banks that are not located in a particular root.

   When an "xmz" file is loaded, all of the instruments it contains are
   loaded sequentially into the Parts.  Thus, a number of instruments are loaded
   at once.  So, a patch set is a list of instruments that are related by
   being necessary for a given tune, rather than by being located in a
   particular bank.

%  In patch sets, \textsl{Yoshimi} will save named-but-disabled patches.
%  Currently, \textsl{ZynAddSubFX} does not, so be aware when transferring
%  data between the two synthesizers.

\subsubsection{Menu / Patch Sets / Save External...}
\label{subsubsec:menu_patch_sets_save}

   This menu entry simply brings up a file dialog, allowing the user to
   navigate to an arbitrary directory, and then save the current Part
   to a solitary instrument file (\texttt{*.xiz}).

   In patch sets, \textsl{Yoshimi} will save named-but-disabled patches.
   Currently, \textsl{ZynAddSubFX} does not, so be aware when transferring
   data between the two synthesizers.

\begin{figure}[H]
   \centering
   \includegraphics[scale=0.75]{menu/Parameters/save-parameters.jpg}
   \caption{Save Patch Set}
   \label{fig:yoshimi_menu_save_parameters}
\end{figure}

   Patch set saves include everything that is not part of the main
   configuration, and so saved patch sets
   includes \textbf{Master Volume} and \textbf{Detune}
   \textbf{Part} destinations, \textbf{Humanise},
   and more.
   If nothing has changed, then the following dialog is shown.

\begin{figure}[H]
   \centering
   \includegraphics[scale=0.75]{menu/Parameters/nothing-to-save.jpg}
   \caption{Patch Set, Nothing to Save}
   \label{fig:yoshimi_menu_nothing_to_save_parameters}
\end{figure}

\subsubsection{Menu / Patch Sets / Recent Sets}
\label{subsubsec:menu_patch_sets_recent_sets}

   This menu entry brings up a dialog box with a list of the recent patch sets
   that have been loaded.  This item makes it easy to move around one's
   frequently-used banks.

\subsubsection{Menu / Patch Sets / Patch Bank Roots}
\label{subsubsec:menu_patch_sets_patch_bank_roots}

   \textsl{Yoshimi} (as installed by Debian Linux) provides a default bank at
   \texttt{/usr/share/yoshimi/banks}.
   To add one's own directory, click on the \textbf{Roots} button.
   It brings up the following dialog.

\begin{figure}[H]
   \centering
   \includegraphics[scale=0.75]{1.3.9/patch_sets_bank_root_paths.png}
   \caption{Bank Root Paths}
   \label{fig:bank_root_paths}
\end{figure}

   This dialog has a number of buttons, some of which will be disabled if no
   directory in the list is selected.

   Then click on the "Add root directory..." button.  In the file dialog that
   appear, one can use the \textbf{Create Directory} button to make a new
   directory, if desired:

\begin{figure}[H]
   \centering
   \includegraphics[scale=0.75]{menu/Instrument/new-directory.jpg}
   \caption{New Root Directory?}
   \label{fig:new_root_directory}
\end{figure}

   Otherwise, once can add an existing directory to the list.

   \setcounter{ItemCounter}{0}      % Reset the ItemCounter for this list.

   \itempar{Add root directory...}{Root Paths!add directory}
   Bank Root Paths, Add Root Directory.

   Once selected, one will see that
   \texttt{/usr/share/yoshimi/banks} or
   \texttt{/usr/local/share/yoshimi/banks}
   is marked with an asterisk.  One can select the new root directory via the
   file dialog that appears, and then make it the current root by clicking the
   \textbf{Make current} button.  Then the Banks dialog will show all the banks
   in that directory, one bank per subdirectory (each subdirectory "is" a
   bank).

   \itempar{Remove root directory...}{Root Paths!remove directory}
   Bank Root Paths, Remove Root Directory.
   If a path is selected, then this button is active, and can be used to
   delete the selected path from the "root paths" list. Note that this does
   \textbf{not} delete actual files, just removes the entry from the list.

   \itempar{Make current}{Root Paths!make current}
   Bank Root Paths, Make Current.
   \index{current!root}
   This button marks the currently-selected path as the "current root" path.

   \itempar{Open current}{Root Paths!open current}
   Bank Root Paths, Open Current.
   This button opens the current root path.
   (Does this work?)

   \itempar{Change ID}{Root Paths!change ID}
   Bank Root Paths, Change ID.
   This ID can be used to make the bank selectable via an extended MIDI
   control.

   Values: \texttt{0* to 127}

%-------------------------------------------------------------------------------
% vim: ts=3 sw=3 et ft=tex
%-------------------------------------------------------------------------------


% Removed.  Basically replaced by the Patch Sets menu.
%
% \subsection{Menu / Parameters}

\subsection{Menu / Paths}
\label{subsec:menu_paths}

   This menu entry provides a more direct way to set up the Bank Root and
   the Presets directories.  It contains the following items:

   \begin{enumber}
      \item \textbf{Bank Root Dirs...}
      \item \textbf{Preset Dirs...}
   \end{enumber}

   \paragraph{Bank Root Dirs...}
   The Paths Bank Root Dirs dialog is described in
   \sectionref{subsubsec:menu_patch_sets_patch_bank_roots},
   which shows
   \figureref{fig:bank_root_paths}, and describes this dialog in full.

   \paragraph{Preset Dirs...}
   The \textsl{Yoshimi} preset directories are the locations where
   presets can be found.  When first installed, the system
   preset directory is one of the following, depending on whether
   \textsl{Yoshimi} was installed via a package manager or via source code:

   \begin{verbatim}
      /usr/share/yoshimi/presets
      /usr/local/share/yoshimi/presets
   \end{verbatim}
   
   The user can provide additional directories for the presets, up to a limit
   of 128 directories (the same limit as for roots and banks).
   These directories are useful for containing copies of the system
   presets that one can modify safely, and for providing custom
   presets designed by the user.

   The following items are provided by the preset directory settings:

   \begin{enumber}
      \item \textbf{Preset list}
      \item \textbf{Add preset directory...}
      \item \textbf{Remove preset directory...}
      \item \textbf{Make default}
      \item \textbf{Save and Close}
      \item \textbf{Close Unsaved}
   \end{enumber}

\begin{figure}[H]
   \centering 
   \includegraphics[scale=0.75]{menu/Yoshimi/yoshimi-settings-presets-dirs.jpg}
   \caption[Preset Dirs Tab]{Yoshimi Preset Dirs Dialog}
   \label{fig:yoshimi_presets_dirs_tab}
\end{figure}

   \setcounter{ItemCounter}{0}      % Reset the ItemCounter for this list.

   \itempar{Preset list}{Yoshimi Presets!Preset List}
   This interface element contains a list of preset directories.
   By default, the only directory present is the installed preset directory.
   For example, \texttt{/usr/share/yoshimi/presets}.
   
   \textbf{Tip:}
   \index{tip!preset directory}
   If there is no directory in this dialog, then one must
   add one, otherwise there is no place to store the presets.
   So make this one of the first items specified when first running
   \textsl{Yoshimi}!

   Another example would be this project; let YOSHIMI-DOC be the directory
   where this project is stored.  Then one can add
   \texttt{YOSHIMI-DOC/config/yoshimi/presets} to this list, using the
   button described next.

   \itempar{Add preset directory...}{Yoshimi Presets!Add Directory}
   Use this button and dialog to add a preset directory to the list, for
   easy access.

   Press the \textbf{Add preset directory...} button, revealing the
   following dialog.

\begin{figure}[H]
   \centering 
   \includegraphics[scale=0.75]{menu/Yoshimi/presets-add-a-preset-directory.jpg}
   \caption[Add Preset Directory]{Add a Preset Directory}
   \label{fig:presets_add_a_preset_directory}
\end{figure}

   Navigate to the desired directory, select it, and press the \textbf{Ok}
   button.  (There is no need to press the \textbf{Save and Close} button;
   the directory is added as soon as \textsl{OK} is clicked.  However, one
   tends to want to click it anyway, to be sure.)
   \textsl{Important}:  Restart \textsl{Yoshimi} to use the preset directory.

   \itempar{Remove preset directory...}{Yoshimi Presets!Remove Directory}
   Select one of the preset directories in the preset list, then press this
   button to remove the preset directory from the list of preset
   directories.  It is removed immediately, with no need to confirm the
   deletion, click an OK button, or click a Save button.

   \itempar{Make default presets}{Yoshimi Presets!Make Default}
   Make Default Presets Directory.
   Select one of the preset directories in the preset list, then press this
   button to make the preset directory the default preset directory.
   It should be a directory for which one has write permissions.
   By default, it is \texttt{\textasciitilde/.config/yoshimi/presets}.

\subsection{Menu / Scales}
\label{subsec:menu_scales}

   \textsl{Yoshimi} is a microtonal synthesizer, and is capable of a wide
   range of microtonal scales.

   At present, we're not too experienced with this feature.

\begin{figure}[H]
   \centering 
   \includegraphics[scale=0.9]{menu/yoshimi-menu-scales.jpg}
   \caption{Yoshimi Menu, Scales}
   \label{fig:yoshimi_menu_scales}
\end{figure}

   \begin{enumber}
      \item \textbf{Show Settings...}
      \item \textbf{Load...}
      \item \textbf{Save...}
      \item \textbf{Recent Scales...}
      \item \textbf{Clear}
   \end{enumber}

\subsubsection{Menu / Scales / Show Settings}
\label{subsec:menu_scales_show}

\begin{figure}[H]
   \centering 
   \includegraphics[scale=0.75]{menu/Scales/scale-settings-microtonal.jpg}
   \caption{Yoshimi Menu, Scales Settings}
   \label{fig:yoshimi_menu_scales_settings}
\end{figure}

\paragraph{Scales Basic Settings}
\label{paragraph:menu_scales_basic_settings}

   This item controls the micro-tonal capabilities of \textsl{Yoshimi} and
   some other settings related to tuning. 
   The last entry in the tunings list represents one octave.
   All other notes are deduced from these settings.

   \setcounter{ItemCounter}{0}      % Reset the ItemCounter for this list.

   \itempar{Microtonal}{Enable Microtonal}
   Enable Microtonal Scales.
   When disabled, the synthesizer will use equal-temperament, 12 notes per
   octave.  Otherwise, one can input any scale one desires.

   Values: \texttt{Off*, On}

   \itempar{"A" Freq}{"A"}
   Frequency of the "A" Note.
   Sets the frequency of the "A" key. The standard is 440.0 Hz.

   Values: \texttt{440*}

   \itempar{"A" Note}{"A" MIDI}
   Sets the MIDI Value of the "A" Note.

   Values: \texttt{0 to 127, 69*}

   \itempar{Invert Keys}{keys}
   Allows the keys to be inverted, so that higher-valued keys play lower
   notes.

   Values: \texttt{Off*, On}

   \itempar{Center}{center}
   Center for Inverted Keys.
   This is the center where the notes frequencies are turned upside-down if
   \textbf{Invert keys} is enabled.
   If the center is 60, the note 59 will become 61, 58 will become 62, 61
   will become 59, and so on.

   Values: \texttt{0 to 127, 60*}

   \itempar{Name}{mapping}
   Name of the Mapping.
   For example, the default mapping is called "12tET".

   \itempar{Shift}{key shift}
   Key Shift.
   Shift the scale. If the scale is tuned to A, one can easily tune it to
   another key.

   Values: \texttt{-63 to 64, 0*}

   \itempar{Comment}{comment}
   Comment for Key Mapping.
   Provides a comment or a description of the scale.
   By default, this is "Equal Temperament 12 notes per octave".

   \itempar{Tunings}{tuning}
   Tunings.
   Here one can input a scale by entering all the tunings for one octave. 
   One can enter the tunings in two ways: 

   \begin{enumber}
      \item As the number of cents (1200 cents=1 octave) as a float number
         like "100.0", "123.234"
      \item As a proportion like "2/1" which represents one octave, "3/2" a
         perfect fifth, "5734/6561".  "2/1" is equal to "1200.0" cents.
   \end{enumber}

   The default is a series of values:
   \texttt{0100.0, 0200.0, ..., 1100.0, 2/1}.

   \itempar{Retune}{retune}
   Retune button.
   This button retunes the synthesizer according to the settings of
   the \textbf{Tunings} and \textbf{Keyboard Mapping} lists.
   All other changes operate immediately.

   \itempar{nts./oct}{Notes per Octave}
   Notes Per Octave.

   Values: \texttt{12*} (range not yet known)

   \itempar{Import .SCL file}{scale file}
   Import Scala files.
   Scala is a powerful application for experimentation with musical tunings
   (intonation scales, micro-tonal,...etc.). From its home page \cite{scala},
   one can download more than 2800 scales which one can import directly into
   \textsl{Yoshimi}.  Note that the zip file \textsl{must} be unzipped with
   the \texttt{-aa} ("autoconvert") option.  However, we have converted it to a
   a much smaller tar file (it crams 18 Mb of files into an sub-500 Kb file),
   which can be untarred directly into
   one's configuration directory to create a
   \texttt{\textasciitilde/.config/yoshimi/scales} directory chock full of
   scales.

    \begin{verbatim}
      $ cd ~/.config/yoshimi/
      $ tar xf yoshimi-scales.tar.xz
    \end{verbatim}

    Note that a Scala file cannot be loaded directly.  It must be imported.

\begin{figure}[H]
   \centering 
   \includegraphics[scale=0.75]{menu/Scales/import-scl-file.jpg}
   \caption{Yoshimi Menu, Scales, Import File}
   \label{fig:yoshimi_menu_scales_import_file}
\end{figure}

   \itempar{Import .scl file}{scl file}
   This item is a standard file dialog for reading
   a \texttt{*.scl} file.

\begin{figure}[H]
   \centering 
   \includegraphics[scale=0.75]{menu/Scales/import-kbm-file.jpg}
   \caption{Yoshimi Menu, Scales, Import Keyboard Map}
   \label{fig:yoshimi_menu_scales_import_keyboard_map}
\end{figure}

   \itempar{Import .kbm file}{kbm file}
   This item is a standard file dialog for reading
   a \texttt{*.kbm} file.

   \itempar{Close, Scales Dialog}{close scales}

   The items related to the \textbf{Keyboard Mapping} are discussed
   separately in the next section.

\paragraph{Keyboard Mapping}
\label{paragraph:menu_scales_keyboard_mapping}

   One can set the MIDI keyboard mapping to scale-degree mapping.
   This is used if the scale has more or less than 12 notes/octave.
   One can enable the mapping by pressing the \textbf{ON} check-box.

   \begin{enumber}
      \item \textbf{ON}
      \item \textbf{First Note}
      \item \textbf{Last Note}
      \item \textbf{Midle Note}
      \item \textbf{Map}
      \item \textbf{Map Size}
   \end{enumber}

   \setcounter{ItemCounter}{0}      % Reset the ItemCounter for this list.

   \itempar{ON}{scales!on}

   Values: \texttt{Off*, On}

   \itempar{First Note}{scales!first note}
   First MIDI Note Number.
   Keys below this value are ignored.

   Values: \texttt{0* to 127}

   \itempar{Last Note}{scales!last note}
   Last MIDI Note Number.
   Keys above this value are ignored.

   Values: \texttt{0 to 127*}

   \itempar{Middle Note}{scales!middle note}
   Middle note where scale-degree 0 is mapped to;
   the middle note represents the note where the formal octave starts.
   Note the misspelling of "middle".

   Values: \texttt{0 to 127*}

   \itempar{Map}{scales!map}
   Scales map.  This is the input field where the mappings are entered.
   The numbers represent the order (degree) entered on
   \textbf{Tunings Input} field, with the first value being 0.
   This number must be less than the number of notes per octave (since
   the values start at 0).
   If one doesn't want a key to be mapped, one enters an "x" instead of a
   number.

   Values: \texttt{0 to 11}

   \itempar{Map Size}{scales!map size}
   Provides the size of the scale-map.

   Values: \texttt{12}

   In the current version of \textsl{Yoshimi}, up to 25 recently used scales are
   now stored in the new history file
   (\texttt{yoshimi.history}), and can be quickly reinstalled with a
   mini-browser in exactly the same way as patch sets.

\subsubsection{Menu / Scales / Load}
\label{subsec:menu_scales_load}

\begin{figure}[H]
   \centering 
   \includegraphics[scale=0.75]{menu/Scales/open-scales.jpg}
   \caption{Yoshimi Menu, Open Scales}
   \label{fig:yoshimi_menu_open_scales}
\end{figure}

   If the format of the scales file is not correct, then the following prompt
   will appear.

\begin{figure}[H]
   \centering 
   \includegraphics[scale=0.75]{menu/Scales/failed-to-load-scl-file-vice-xsz.jpg}
   \caption{Yoshimi Menu, Failed to Load Scales}
   \label{fig:yoshimi_menu_failed_to_load_scales}
\end{figure}

\subsubsection{Menu / Scales / Save}
\label{subsec:menu_scales_save}

   This dialog opens a stock file-dialog to allow the saving of
   \texttt{*.xsz} files.
   If one has imported a scale from an \texttt{*.scl} file, and one
   wants direct access to it from the \textbf{Scales / Recent Scales} menu, one
   must first save the imported file as an \texttt{*.xsz} files.

\subsubsection{Menu / Scales / Recent Scales...}
\label{subsec:menu_scales_recent_scales}

   Once some scale file has been loaded (or imported and saved), then it
   becomes available in this list, for more convenient access to it.

\begin{figure}[H]
   \centering 
   \includegraphics[scale=0.75]{menu/Scales/recent-scales.png}
   \caption{Yoshimi Menu, Recent Scales}
   \label{fig:yoshimi_menu_recent_scales}
\end{figure}

\subsubsection{Menu / Scales / Clear}
\label{subsec:menu_scales_clear}

   This menu entry simply resets the \textsl{Yoshimi} scale back to it default,
   the twelve-tone equally-tempered scale.

\subsection{Menu / State}
\label{subsec:menu_state}

   \textsl{Yoshimi} state is saved in files with the extension
   \texttt{.state}.  These files are also XML files.

   \textsl{Yoshimi} "state" will include the system settings, as well as all
   patches. Some of these settings (such as Oscillator Size) can only be
   realised on a reload if loading via the command line at startup. However,
   the ones that can't be dynamically changed will be set, and if the
   configuration is then saved, will be set on the next load -- this is not
   ideal. We are working on it, but don't expect improvements soon!

   \begin{enumber}
      \item \textbf{Load}
      \item \textbf{Save}
      \item \textbf{Recent States...}
   \end{enumber}

   As the following figures show, state files are normally stored in the
   user's \texttt{.config/yoshimi/yoshimi.state} file.

   \setcounter{ItemCounter}{0}      % Reset the ItemCounter for this list.

   \itempar{State Load}{State!Load}
   Provides a way to load a previously-saved \textsl{Yoshimi} state file.

\begin{figure}[H]
   \centering 
   \includegraphics[scale=0.75]{menu/State/load-state-file.jpg}
   \caption{Yoshimi Menu, State Load}
   \label{fig:yoshimi_menu_state_load}
\end{figure}

   This item is a standard \textsl{Yoshimi} file dialog.
   Note that XML text is shown in the preview pane, but, if XML compression is
   set, then a large question mark is all that would be shown.

   \itempar{State Save}{State!Save}
   Provides a way to save a new or modified \textsl{Yoshimi} state file.

\begin{figure}[H]
   \centering 
   \includegraphics[scale=0.75]{menu/State/save-state-file.jpg}
   \caption{Yoshimi Menu, State Save}
   \label{fig:yoshimi_menu_state_save}
\end{figure}

   This item is a standard \textsl{Yoshimi} file dialog.

   \itempar{Recent States}{State!Recent}
   This item brings up a list of states to select.
   The Recent States dialog will not come up if there are no states that have
   yet been managed.

%-------------------------------------------------------------------------------
% vim: ts=3 sw=3 et ft=tex
%-------------------------------------------------------------------------------


% Musical scales

%-------------------------------------------------------------------------------
% yum_menu
%-------------------------------------------------------------------------------
%
% \file        yum_menu.tex
% \library     Documents
% \author      Chris Ahlstrom
% \date        2017-09-24
% \update      2017-09-24
% \version     $Revision$
% \license     $XPC_GPL_LICENSE$
%
%     Provides the Scales section of yoshimi-user-manual.tex.
%
%-------------------------------------------------------------------------------

\section{Scales}
\label{sec:Scales}

   \textsl{Yoshimi} is a microtonal synthesizer, and is capable of a wide
   range of microtonal scales.  Many improvements have been made to the scales,
   including the user-interface, performance, accuracy of calculations,
   and adherence to the Scala (\cite{scala}) specification.
   in version 1.5.2 and above. At the request of users, since version 1.5.8 some controls have been made accessible to MIDI-learn, and these have the familiar pale blue border.
   \index{LV2!scales}
   For users of the LV2 plugin, any changes in scale
   settings are reported back so that the plugin host can be aware of the
   change.

\subsection{Scales / Command Line}
\label{subsec:scales_command_line}

	One can now fully control scales from the CLI.
	For tunings, either ratios or floating point numbers can be entered.
	Ratios are in the form
	\texttt{n1/n2} to a maximum of normal integer range.
   If just a numerator is set, it will be regarded as \texttt{n/1}.
   Floating point numbers \textsl{must}
   include the decimal point and at least one digit (or zero) on either side.
   The numbers are padded out with leading and trailing zeros in the form
   \texttt{nnnn.nnnnnn}.

   \index{scales!non-sounding notes}
   In keyboard maps, non-sounding notes should be entered as an 'x' instead of
   the key number.

   CLI tunings and keymaps are entered in CSV format.
   Tuning:

   \begin{verbatim}
      0076.049000, 0193.156860, 0310.264710, 5/4, 0503.421570,
      0579.470570, 0696.578430, 25/16, 0889.735290, 1006.843140,
      1082.892140, 2/1
   \end{verbatim}

   Keymap:

   \begin{verbatim}
      0, 1, 2, 3, x, 5, 6, 7, x, 9, 10, 11
   \end{verbatim}

   The tuning/keymap sizes are generated internally by counting the number of
   entries in the strings.

   When saving scales, for floating point numbers, \textsl{Yoshimi} includes
   the text it was derived from. This has accuracy benefits,
   but also reassures less experienced users,
   because the values they enter won't seem to change on
   re-loading.  The stored value is still saved for backward compatibility with
   older versions of \textsl{Yoshimi}.

   \index{scales!shift}
   Scale shift provides an offset to the scale start position, and only makes a
   difference in uneven interval sizes.

   Normally (for the even tempered scale) the scale starts on 'A', and, as the
   intervals are all identical, changing the octave start will make no
   difference. However, if one has (say) a 5-note pentatonic scale, the
   intervals will be very different and the scale shift will effectively
   determine the key of the scale.

\begin{figure}[H]
   \centering
   \includegraphics[scale=0.9]{menu/yoshimi-menu-scales.jpg}
   \caption{Yoshimi Menu, Scales}
   \label{fig:yoshimi_scales}
\end{figure}

   \begin{enumber}
      \item \textbf{Show Settings...}
      \item \textbf{Load...}
      \item \textbf{Save...}
      \item \textbf{Recent Scales...}
      \item \textbf{Clear}
   \end{enumber}

\subsection{Scales / Show Settings}
\label{subsec:scales_show}

\begin{figure}[H]
   \centering
   \includegraphics[scale=0.75]{1.6.0/Scales.png}
   \caption{Yoshimi Menu, Scales Settings}
   \label{fig:yoshimi_scales_settings}
\end{figure}

\subsubsection{Scales Basic Settings}
\label{subsubsec:scales_basic_settings}

   This item controls the microtonal capabilities of \textsl{Yoshimi} and
   some other settings related to tuning.
   The last entry in the tunings list represents one octave.
   All other notes are deduced from these settings.

   \begin{enumber}
      \item \textbf{Enable Microtonal}
      \item \textbf{(Ref.) Freq.}
      \item \textbf{(Ref.) Note}
      \item \textbf{Invert Keys}
      \item \textbf{Center}
      \item \textbf{Name}
      \item \textbf{Shift}
      \item \textbf{Comment}
      \item \textbf{Tunings}
      \item \textbf{Retune}
      \item \textbf{Keyboard Mapping}
      \item \textbf{ON}
      \item \textbf{First note}
      \item \textbf{Middle note}
      \item \textbf{Last note}
      \item \textbf{nts./oct.}
      \item \textbf{Import .scl file}
      \item \textbf{Map Size}
      \item \textbf{Import .kbm file}
      \item \textbf{Close}
   \end{enumber}

   \setcounter{ItemCounter}{0}      % Reset the ItemCounter for this list.

   \itempar{Enable Microtonal}{Enable Microtonal}
   Enable Microtonal Scales.
   When disabled, the synthesizer will use equal-temperament, 12 notes per
   octave.  Otherwise, one can input any scale one desires.
   In \textsl{Yoshimi} V 1.6.1 this was revised to more correctly identify
   the frequency and note settings. The ranges have also been reduced to
   reduce the risk of possible damage to audio equipment. They are still
   well outside any reasomable requirement.

   Values: \texttt{Off*, On}

   \itempar{(Ref.) Freq}{Frequency}
   Frequency of the refenence note.
   Sets the frequency of the reference key. The standard is "A", 440.0 Hz.

   Values: \texttt{30 to 1100, 440*}

   \itempar{(Ref.) Note}{MIDI note number}
   Sets the MIDI value of the refencence note. This is usually number 69,
   which is A4

   Values: \texttt{24 to 84, 69*}

   \itempar{Invert Keys}{keys}
   Allows the keys to be inverted, so that higher-valued keys play lower
   notes.

   Values: \texttt{Off*, On}

   \itempar{Center}{center}
   Center for Inverted Keys.
   This is the center where the notes frequencies are turned upside-down if
   \textbf{Invert keys} is enabled.
   If the center is 60, the note 59 will become 61, 58 will become 62, 61
   will become 59, and so on.

   Values: \texttt{0 to 127, 60*}

   \itempar{Name}{mapping}
   Name of the Mapping.
   For example, the default mapping is called "12tET".

   \itempar{Shift}{key shift}
   Key Shift.
   Shift the scale. If the scale is tuned to A, one can easily tune it to
   another key.

   Values: \texttt{-63 to 64, 0*}

   \itempar{Comment}{comment}
   Comment for Key Mapping.
   Provides a comment or a description of the scale.
   By default, this is "Equal Temperament 12 notes per octave".

   \itempar{Tunings}{tuning}
   Tunings.
   Here one can input a scale by entering all the tunings for one octave.
One can enter the tunings in two ways:

   \begin{enumber}
      \item As the number of cents (1200 cents=1 octave) as a float number
         like "100.0", "123.234"
      \item As a proportion like "2/1" which represents one octave, "3/2" a
         perfect fifth, "5734/6561".  "2/1" is equal to "1200.0" cents.
   \end{enumber}

   The default is a series of values:
   \texttt{0100.0, 0200.0, ..., 1100.0, 2/1}.

   \itempar{Keyboard Mapping}{keyboard}
   The items related to the \textbf{Keyboard Mapping} are discussed
   separately in the next section.

   \itempar{Retune}{retune}
   Retune button.
   This button retunes the synthesizer according to the settings of
   the \textbf{Tunings} and \textbf{Keyboard Mapping} lists.
   The \textbf{Retune} button is needed if one
   changes any of the actual scale settings. However, it's not needed for key
   mappings or any other controls, all of which operate immediately.

   \itempar{Notes/oct}{Notes per Octave}
   Notes Per Octave.
   This value is affected by changes to the \textbf{Tunings} mapping.

   Values: \texttt{12*}

   \itempar{Import .SCL file}{scale file}
   Import Scala files.
   Scala is a powerful application for experimentation with musical tunings
   (intonation scales, micro-tonal,...etc.). From its home page \cite{scala},
   one can download more than 2800 scales which one can import directly into
   \textsl{Yoshimi}.  Note that the zip file \textsl{must} be unzipped with
   the \texttt{-aa} ("autoconvert") option.  However, we have converted it to a
   a much smaller tar file (it crams 18 Mb of files into an sub-500 Kb file),
   which can be untarred directly into
   one's configuration directory to create a
   \texttt{\textasciitilde/.config/yoshimi/scales} directory chock full of
   scales.

    \begin{verbatim}
      $ cd ~/.config/yoshimi/
      $ tar xf yoshimi-scales.tar.xz
    \end{verbatim}

    Note that a Scala file cannot be loaded directly.  It must be imported.

\begin{figure}[H]
   \centering
   \includegraphics[scale=0.75]{menu/Scales/import-scl-file.jpg}
   \caption{Yoshimi Menu, Scales, Import File}
   \label{fig:yoshimi_scales_import_file}
\end{figure}

%  \itempar{Import .scl file}{scl file}
   This item is a standard file dialog for reading
   a \texttt{*.scl} file.

\begin{figure}[H]
   \centering
   \includegraphics[scale=0.75]{menu/Scales/import-kbm-file.jpg}
   \caption{Yoshimi Menu, Scales, Import Keyboard Map}
   \label{fig:yoshimi_scales_import_keyboard_map}
\end{figure}

   \itempar{Map Size}{scales!map size}
   Map Size.
   This value is affected by changes to the \textbf{Keyboard Mapping}.

   Values: \texttt{12*}

   \itempar{Import .kbm file}{kbm file}
   This item is a standard file dialog for reading
   a \texttt{*.kbm} file.

   \itempar{Close, Scales Dialog}{close scales}

\subsubsection{Keyboard Mapping}
\label{subsubsec:scales_keyboard_mapping}

   One can set the MIDI keyboard mapping to scale-degree mapping.
   This is used if the scale has more or less than 12 notes/octave.
   One can enable the mapping by pressing the \textbf{ON} check-box.

   \begin{enumber}
      \item \textbf{ON}
      \item \textbf{First Note}
      \item \textbf{Last Note}
      \item \textbf{Middle Note}
      \item \textbf{Map}
      \item \textbf{Map Size}
   \end{enumber}

   \setcounter{ItemCounter}{0}      % Reset the ItemCounter for this list.

   \itempar{ON}{scales!on}
   This item enables the \textbf{Keyboard Mapping} list.

   Values: \texttt{Off*, On}

   \itempar{First Note}{scales!first note}
   First MIDI Note Number.
   Sets the MIDI note value to use for the first note of the scale.
   MIDI notes below this value are ignored.

   Values: \texttt{0* to 127}

   \itempar{Middle Note}{scales!middle note}
   Sets the MIDI note value to use for the middle note of the scale.
   This is the note where the scale-degree 0 setting is mapped;
   the middle note represents the note where the formal octave starts.

   Values: \texttt{0 to 127, 60*}

   \itempar{Last Note}{scales!last note}
   Last MIDI Note Number.
   Sets the MIDI note value to use for the last note of the scale.
   Keys above this value are ignored.

   Values: \texttt{0 to 127*}

   \itempar{Map}{scales!map}
   Scales map.  This is the input field where the mappings are entered.
   The numbers represent the order (degree) entered on
   \textbf{Tunings Input} field, with the first value being 0.
   This number must be less than the number of notes per octave (since
   the values start at 0).
   If one doesn't want a key to be mapped, one enters an "x" instead of a
   number.

   Values: \texttt{0 to 11}

   \itempar{Map Size}{scales!map size}
   Provides the size of the scale-map.

   Values: \texttt{12}

   In the current version of \textsl{Yoshimi}, up to 25 recently used scales are
   now stored in the new history file
   (\texttt{yoshimi.history}), and can be quickly reinstalled with a
   mini-browser in exactly the same way as patch sets.

\subsection{Scales / Load}
\label{subsec:scales_load}

\begin{figure}[H]
   \centering
   \includegraphics[scale=0.75]{menu/Scales/open-scales.jpg}
   \caption{Yoshimi Menu, Open Scales}
   \label{fig:yoshimi_open_scales}
\end{figure}

   If the format of the scales file is not correct, then the following prompt
   will appear.

\begin{figure}[H]
   \centering
   \includegraphics[scale=0.75]{menu/Scales/failed-to-load-scl-file-vice-xsz.jpg}
   \caption{Yoshimi Menu, Failed to Load Scales}
   \label{fig:yoshimi_failed_to_load_scales}
\end{figure}

   Note that the loading and saving of scales is fully available in the
   command-line as well.

\subsection{Scales / Save}
\label{subsec:scales_save}

   This dialog opens a stock file-dialog to allow the saving of
   \texttt{*.xsz} files.
   If one has imported a scale from an \texttt{*.scl} file, and one
   wants direct access to it from the \textbf{Scales / Recent Scales} menu, one
   must first save the imported file as an \texttt{*.xsz} files.

   Note that the loading and saving of scales is fully available in the
   command-line as well.

   In the past, every time one saved and reloaded a scale, there was a
   degradation in the accuracy of the scales.  This issue has been fixed, since
   people are very sensitive to pitch intervals.

\subsection{Scales / Recent Scales...}
\label{subsec:scales_recent_scales}

   Once a scale file has been loaded (or imported and saved), then it
   becomes available in this list, for more convenient access to it.

\begin{figure}[H]
   \centering
   \includegraphics[scale=0.75]{menu/Scales/recent-scales.png}
   \caption{Yoshimi Menu, Recent Scales}
   \label{fig:yoshimi_recent_scales}
\end{figure}

\subsection{Scales / Clear}
\label{subsec:scales_clear}

   This menu entry simply resets the \textsl{Yoshimi} scale back to it default,
   the twelve-tone equally-tempered scale.

\subsection{Scales / Reference Pitch...}
\label{subsec:scales_reference_pitch}
   A note about relationships between the refernce note frequency, note number and
   key shifts.

   The reference note frequency should be regarded as an absolute value, with the
   note number being specifically a MIDI representation.
   Master keyshift, and Part keyshifts are relative values that modify this for
   convenience.

   As an example, some very old wood-framed pianos were tuned with C4=256 cycles
   per second (Hz). This can't be changed as the frame couldn't withstand the
   extra stress, so to play alongside such a piano one would have to set the
   absolute value 'A' frequency at 430.581Hz. Better still would be to set note
   number 60 to a frequency of 256! If you have the reports window open you will see the following responses, making it quite clear what has changed.

   \begin{verbatim}
      Scales Ref note 60 (C4)
      Scales (C4) Frequency Value 256.000000
   \end{verbatim}

   A singer might then say they couldn't sing at that pitch, so one would then use
   the Master Keyshift to change to a key they were more comfortable with.

   When playing \textsl{Yoshimi} in a band where there is a minor pitch
   discreapancy it would be best to use the Master Detune to match, rather than
   altering the reference frequency.

   Since V 1.5.11 reference frequency range has been limited, and revised further
   since V 1.6. The range is 30Hz to 1100Hz which covers
   the range B0 to C6. Previously it could be set anywhere between 1Hz and 2000Hz.
   Not only did this put most notes right out of the audio spectrum, but it was
   potentially damaging for some audio equipment.

   With the various key shift and octave controls it is still possible to cover
   every part of the audio spectrum.

%-------------------------------------------------------------------------------
% vim: ts=3 sw=3 et ft=tex
%-------------------------------------------------------------------------------


% Settings Index and Descriptions

%-------------------------------------------------------------------------------
% yum_settings
%-------------------------------------------------------------------------------
%
% \file        yum_settings.tex
% \library     Documents
% \author      Chris Ahlstrom
% \date        2015-05-15
% \update      2015-07-17
% \version     $Revision$
% \license     $XPC_GPL_LICENSE$
%
%     Provides the Settings section of yoshimi-user-manual.tex, which covers
%     stock settings user-interface items.
%
%-------------------------------------------------------------------------------

\section{Stock Settings Elements}
\label{sec:stock_settings_elements}

   This section collects all of the setting values one will find for
   audio parameters in the \textsl{Yoshimi} GUI.
   Sometimes the labels and tool-tips
   in the application are a bit too brief to understand. 
   One will find their meanings in this section.

   This section also covers the sub-panels that provide the settings.
   By describing these deep details here, we can refer to them when
   describing how to set up specific sounds in
   \textsl{Yoshimi}.

   Much of this material comes from
   \url{http://sourceforge.net/zynaddsubfx/Doc}
   and has been reorganized in minor ways.

\subsection{Settings Features}
\label{sec:stock_settings_ui_features}

   This section notes some minor interface and synthesizer features that may
   be seen thoughout \textsl{Yoshimi}.

\subsubsection{Title Bars}
\label{subsubsec:stock_settings_elements_title_bars}

   The title bars of all editing windows display both the part number and the
   current name of the instrument one are working on. In the ADDsynth
   Oscillator Editor, one also sees the voice number of the oscillator one is
   editing.

\subsubsection{Color Coding}
\label{subsubsec:stock_settings_elements_color_coding}

   A GUI enhancement for \textsl{Yoshimi 1.3.5} is color-coded
   identification of an instrument's use of Add, Sub, and Pad synth engines,
   no matter where in the instrument's kit they may be. This can be
   enabled/disabled in the mixer panel. It does slow down
   \textsl{Yoshimi}'s startup,
   but due to the banks reorganisation (done some time ago) it causes no
   delay in changing banks/instruments once \textsl{Yoshimi} is
   up and running.  Some
   saved instruments seem to have had their Info section corrupted. Yoshimi
   can detect this and step over it to find the true status. Also, if one
   resaves the instrument, not only will the PADsynth status be restored, but
   ADDsynth and SUBsynth will be included, allowing a faster scan next time.

\subsubsection{Knobs}
\label{subsubsec:stock_settings_elements_knobs}

   Visual knobs are used for modifying numerical parameters.
   Horizontal, as well as vertical, mouse movements will adjust the knob.
   Holding down Ctrl provides finer adjustment.
   One can also use the mouse scroll wheel to adjust rotary controls.

\subsubsection{Automation}
\label{subsubsec:stock_settings_elements_automation}

   In \textsl{Yoshimi 1.3.5}, a number of existing, as well as new features
   have come together to give much greater flexibility (especially for
   automation) using standard MIDI messages. These are:

   \begin{enumber}
      \item \textbf{NRPNs}
      \item \textbf{ZynAddSubFX controls}
      \item \textbf{Independent part control}
      \item \textbf{16, 32 or 64 parts}
      \item \textbf{Vector Control}
      \item \textbf{Direct part stereo audio output}
   \end{enumber}

   \setcounter{ItemCounter}{0}      % Reset the ItemCounter for this list.

   \itempar{NRPNs}{automation!NRPNs}
   NRPNs can handle individual bytes appearing in either order, and usually the
   same with the data bytes. Increment and decrement is also supported as
   graduated values for both data LSB and MSB. Additionally, ALSA sequencer's
   14-bit NRPN blocks are supported.

   \itempar{ZynAddSubFx controls}{automation!controls}
   System and Insertion Effect controls are fully supported, with extensions
   to allow one to set the effect type and (for insertion effects) the
   destination part number.

   \itempar{Part control}{automation!part control}
   Independent part control enables one to change instrument, volume, pan, or
   indeed any other available control of just that part, without affecting any
   others that are receiving the same MIDI channel. This can be particularly
   interesting with multiply layered sounds. There are more extensions planned.

   \itempar{16/32/64 Parts}{automation!16/32/64 parts}
   With 32 and 64 parts, it helps to think of 2 or 4 rows of 16. When one
   saves a parameter block, the number of parts is also saved, and will be
   restored when one reloads.  By default each \textsl{column} has the same
   MIDI channel number, but these can be independently switched around, and
   by setting (say) number 17 taken right out of normal access.

   In tests, \textsl{compiling} for 64 parts compared with 16 parts increased
   processor load by a very small amount when \textsl{Yoshimi} was idling,
   but this becomes virtually undetectable once one has 8 or more instruments
   actually generating output. In normal use, selecting the different formats
   makes no detectable difference, but using the default 16 reduces clutter
   when one doesn't need the extras.

   \itempar{Vector control}{automation!vector control}
   Vector control is based on these parts columns, giving one either 2 (X
   only) or 4 (X + Y) instruments in this channel. Currently the vector
   CCs one set up can (as inverse pairs) vary any combination of volume, pan,
   and filter cut-off.  More will be added.  To keep the processor load
   reasonable it pays to use fairly simple instruments, but if one has
   sufficient processing power, it would be theoretically possible to set up
   all 16 channels with quite independent vector behavior!

   \itempar{Direct part audio}{automation!part audio}
   Direct part audio is JACK-specific, and allows one to apply further
   processing to just the defined part's audio output (which can still output
   to the main L+R if one wants). This setting is saved with parameter
   blocks. Currently it is only set in the mixer panel window, but it will also
   eventually come under MIDI direct part control.  Again, to reduce
   unnecessary clutter, part ports are only registered with JACK if they are
   both enabled, and set for direct output. However, once set they will remain
   in place for the session to avoid disrupting other applications that may
   have seen them.

\subsection{Filter Settings}
\label{subsec:filter_settings}

   This section describes filtering at a high level, in terms of frequency
   responses and other concepts of filtering.
   The end of this section covers a user interface used in filter settings.
   It is a stock-panel re-used in other user-interface elements.
   See \sectionref{subsubsec:filter_parameters_user_interface}
   if one is in a hurry.

   \textsl{Yoshimi}
   offers several different types of filters, which can be used to
   shape the spectrum of a signal. The primary parameters that affect the
   characteristics of the filter are the cutoff, resonance, filter stages, and
   the filter type.

   Filter stages are the number of times that this filter is applied in
   series. So, if this number is 1, one simply has this one filter. If it is
   two, the sound first passes the filter, and the results then pass the same
   filter again. In \textsl{ZynAddSubFX}, the wetness is applied after all
   stages were passed.

\subsubsection{Filter Type}
\label{subsubsec:filter_type}

   \index{filter!type}
   A filter removes or attenuates frequency elements or tones from a signal.
   Filtering changes the character of a signal.

   The basic analog filters that \textsl{Yoshimi} and \textsl{ZynAddSubFX}
   offer are shown in \figureref{fig:basic_filter_types}, with
   the center frequency being marked by the red
   line. The state variable filters should look quite similar.

\begin{figure}[H]          % keep figure closer, requires the 'float' package
   \centering 
   \includegraphics[scale=0.5]{zyn/zyn_filter_types_filter0.png}
   \caption[Basic Filter Types]{Filter Types, Yoshimi/ZynAddSubFX}
   \label{fig:basic_filter_types} 
\end{figure}

   \begin{enumber}
      \item A \textbf{low-pass} filter makes the sound more muffled.
      \item A \textbf{band-pass} filter makes the sound more tone-like, and
         sometimes more penetrating, if the total energy in the passband is
         preserved as the bandwidth decreases.
      \item A \textbf{high-pass} filter makes the sound seem sharper or more
         strident.
   \end{enumber}

\subsubsection{Filter Cutoff}
\label{subsubsec:filter_cutoff}

   \index{filter!cutoff}
   The filter cutoff value determines which frequency marks the changing
   point for the filter. In a low pass filter, this value marks the point
   where higher frequencies begin to be attenuated.

\subsubsection{Filter Resonance}
\label{subsubsec:filter_resonance}

   \index{filter!Q}
   \index{filter!resonance}
   The resonance of a filter determines how much excess energy is present at
   the cutoff frequency. In \textsl{Yoshimi} and \textsl{ZynAddSubFX},
   this is represented by the Q-factor,
   which is defined to be the cutoff frequency divided by the bandwidth. In
   other words higher Q values result in a much more narrow resonant spike.

   The Q value of a filter affects how concentrated
   the signal’s energy is at the cutoff frequency. The result of differing Q
   values are shown in figure~\ref{fig:low_q_vs_high_q} on
   page~\pageref{fig:low_q_vs_high_q}.
   For many classical analog sounds, high Q values were used on sweeping
   filters. A simple high Q low pass filter modulated by a strong envelope is
   usually sufficient to get a good sound.

\begin{figure}[H]
   \centering 
   \includegraphics[scale=0.5]{zyn/low_q_high_q_filter1.png}
   \caption[Low Q vs. High Q]{The Effect of the Q Value}
   \label{fig:low_q_vs_high_q} 
\end{figure}

\subsubsection{Filter Stages}
\label{subsubsec:filter_stages}

   \index{filter!stages}
   \index{filter!order}
   The number of stages in a given filter describes how sharply it is able to
   make changes in the frequency response.
   The more stages, the sharper the filter.
   However, each added stage increases the processor time needed to make the
   filter calculation.

\begin{figure}[H]
   \centering 
   \includegraphics[scale=0.5]{zyn/2_pole_8_pole_filter2.png}
   \caption[2 Pole vs. 8 Pole Filter]{The Effect of the Order of a Filter}
   \label{fig:2_pole_vs_8_pole_filter}
\end{figure}

   The affect of the order of the filter can be seen in the figure above.
%  in figure~\ref{fig:2_pole_vs_8_pole_filter}
%  on page~\pageref{fig:2_pole_vs_8_pole_filter}.
   This is roughly synonymous with the number of stages of the filter. For
   more complex patches, it is important to realize that the extra sharpness
   in the filter does not come for free, as it requires many more
   calculations being performed. This phenomena is the most visible in
   SUBsynth, where it is easy to need several \textsl{hundred} filter stages
   to produce a given note.

   There are different types of filters. The number of poles define what will
   happen at a given frequency. Mathematically, the filters are functions which
   have poles that correspond to that frequency. Usually, two poles mean that
   the function has more "steepness", and that one can set the exact value of
   the function at the poles by defining the "resonance value". Filters with
   two poles are also often referred to as \textsl{Butterworth Filters}.

   For the interested, functions having poles means that we are given a
   quotient of polynomials. The denominator has degree 1 or 2, depending on the
   filter having one or two poles. In the file \texttt{DSP/AnalogFilter.cpp},
   \texttt{AnalogFilter :: computefiltercoefs()} sets the coefficients
   (depending on the filter type), and
   \texttt{AnalogFilter :: singlefilterout()} shows
   the whole polynomial (in a formula where no quotient is needed).

\subsubsection{Filter Parameters User Interface}
\label{subsubsec:filter_parameters_user_interface}

\begin{figure}[H]
   \centering 
   \includegraphics[scale=1.0]{subpanels/Filter_Params.png}
   \caption[Filter Parameters Sub-panel]{Stock Filter Parameters Sub-Panel}
   \label{fig:filter_parameters_subpanel} 
\end{figure}

   The user interface for filter parameters is a small stock sub-panel that
   is re-used in a number of larger dialog boxes, as shown in the figure
   above.  Let's describe each item of this sub-panel.

%  \figureref{fig:filter_parameters_subpanel}.

\begin{figure}[H]
   \centering 
   \includegraphics[scale=1.0]{bottom-panel/instrument-edit/ADD/filter-category.jpg}
   \caption[Filter Categories Dropdown]{Filter Categories, Dropdown Box}
   \label{fig:filter_categories_dropdown} 
\end{figure}

   \begin{enumber}
      \item \textbf{Category}
      \item \textbf{Filter Type}
      \item \textbf{C.freq}
      \item \textbf{Q}
      \item \textbf{V.SnsA}
      \item \textbf{freq.tr}
      \item \textbf{gain}
      \item \textbf{St}
      \item \textbf{C}
      \item \textbf{P}
   \end{enumber}

   \setcounter{ItemCounter}{0}      % Reset the ItemCounter for this list.

   \itempar{Category}{filter!category}
   Determines the category of filter to be used.
   There are three categories of filters
   (as shown in the dropdown element shown in
   \figureref{fig:filter_categories_dropdown}).

\begin{enumber}                     % enumber is our arabic numbering style
   \item \textbf{Analog} (the default)
   \item \textbf{Formant}
   \item \textbf{StVarF}
\end{enumber}

   An \textbf{analog} filter
   \index{filter!analog}
   is one that approximates a filter that is based on
   a network of resistors, capacitors, and inductors.

   A \textbf{formant} filter
   \index{filter!formant}
   is a more complex kind of filter that acts a lot
   like the human vocal tract, allowing for sounds that
   are a bit like human voices.

   A \textbf{state variable} ("StVarF") filter
   \index{filter!state variable}
   \index{filter!StVarF}
   \index{StVarF}
   is a type of active filter.
   The frequency of operation and the Q factor can be varied independently.
   This and the ability to switch between different filter responses make the
   state-variable filter widely used in analogue synthesizers.

   Values: \texttt{Analog*, Formant, StVarF}

   \itempar{Filter Type}{filter!type}
   Selects the type of filter to be used, such as high-pass, low-pass,
   and band-pass.
   See the dropdown element in \figureref{fig:filter_type_dropdown}.

\begin{figure}[H]
   \centering 
   \includegraphics[scale=1.0]{bottom-panel/instrument-edit/ADD/filter-filtertype.jpg}
   \caption[Filter Type Dropdown]{Type of Filter Passband, Dropdown Box}
   \label{fig:filter_type_dropdown} 
\end{figure}

   Values: \texttt{LPF1, HPF1, LPF2*, HPF2, BPF2, NF2, PkF2, LSh2, HSh2}

   \itempar{C.freq}{cutoff frequency}
   \index{center frequency}
   Cutoff frequency or center frequency.
   This items has various definitions in the literature. 
   Usually it refers to the frequency at which the level
   drops to 3 Db below the maximum level.
   In various dialogs, this value is the
   center frequency of the filter or the base position in
   a vowel's sequence.

   Values: \texttt{0 to 127, 90*}

   \itempar{Q}{resonance level}
   The level of resonance for the filter. 
   It indicates a measure of the sharpness of a filter.
   The higher the Q, the sharper the filter.
   Generally, a higher Q value leads to a louder, more tonal
   affect for the filter.
   Note that some filter types might ignore this parameter.

   \itempar{V.SnsA}{velocity sensing amount}
   Velocity sensing amount for filter cutoff.
   Velocity sensing amount of the filter.

   TODO.

   Values: \texttt{0 to 127, 64*}

   \itempar{V.Sns}{velocity sensing function}
   \index{filter!velocity sensing function}
   Velocity sensing function of the filter.
   Set the amplitude of the velocity sensing.

   Values: \texttt{0 to 127, 64*}

   \itempar{freq.tr}{frequency tracking amount}
   \index{filter!frequency tracking amount}
   Filter Frequency Tracking Amount.
   When this parameter is positive, higher note
   frequencies shift the filter’s cutoff frequency higher.
   For the filter frequency tracking knob, left is negative, middle is
   zero, and right is positive.

   Values: \texttt{0 to 127, 64*}

   \itempar{gain}{filter!gain}
   Filter gain.
   Additional gain/attenuation for a filter.
   Also described as the filter output gain/damping factor.

   Values: \texttt{0 to 127, 64*}

   \itempar{St}{filter!stages}
   Filter stages.
   The more filter stages applied to a signal, the stronger (in general) the
   filtering.
   It is the number of additional times the filter will be applied (in
   order to create a very steep roll-off, such as 48 dB/octave).
   This dropdown
   element is shown in
   \figureref{fig:filter_stage_dropdown}.
   Obviously, the more stages used, the more calculation-intensive the
   filter will be.  This should also increase the latency (lag) of the
   filter.

   % \item \textbf{C} (See elsewhere)\\
   % \item \textbf{P} (See elsewhere)\\

\begin{figure}[H]
   \centering 
   \includegraphics[scale=1.0]{bottom-panel/instrument-edit/ADD/filter-stages.jpg}
   \caption[Filter Stage Dropdown]{Filter Stage Dropdown}
   \label{fig:filter_stage_dropdown} 
\end{figure}

   Also present in this sub-panel are the usual \textbf{C}opy
   and \textbf{P}aste buttons that call up a copy-parameters or
   paste-parameters dialog.

\subsection{LFO Settings}
\label{subsec:lfo_settings}

   \textsl{Yoshimi} provides LFOs for it amplitude, frequency, and filtering
   functions.
   "LFO" means Low Frequency Oscillator. These oscillators are not used to make
   sounds by themselves, but they change parameters cyclically as a sound
   plays.

   LFOs are, as the name says, oscillators with, compared to the frequency of
   the sound, low frequency. They often appear in order to control the
   effect.

\subsubsection{LFO Basic Parameters}
\label{subsubsec:lfo_basic_parameters}

\begin{figure}[H]
   \centering 
   \includegraphics[scale=0.75]{zyn/basic_parameters_lfo0.png}
   \caption[Basic LFO Parameters]{Basic LFO Parameters}
   \label{fig:basic_parameters_lfo} 
\end{figure}

   \begin{enumber}
      \item \textbf{Delay}.
      \item \textbf{Start Phase}.
      \item \textbf{Frequency}.
      \item \textbf{Depth}.
   \end{enumber}

   The LFOs has some basic parameters (see
   \figureref{fig:basic_parameters_lfo}.

   \setcounter{ItemCounter}{0}      % Reset the ItemCounter for this list.

   \itempar{Delay}{LFO!delay}
   LFO Delay.
   This parameter sets how much time takes since the start of the note to
   start the cycling of the LFO.
   When the LFO starts, it has a certain position called "start phase".

   \itempar{Start Phase}{LFO!start phase}
   LFO Start Phase.
   The angular position at which a LFO waveform will start.

   \itempar{Frequency}{LFO!frequency}
   LFO Frequency.
   How fast the LFO is (i.e. how fast the parameter controlled by
   the LFO changes.)

   \itempar{Depth}{LFO!depth}
   LFO Depth.
   The amplitude of the LFO (i.e. how much the parameter is controlled by
   the LFO changes.)

\subsubsection{LFO Function}
\label{subsubsec:lfo_function}

   \index{LFO!shape}
   \index{LFO!type}
   \index{LFO!function}
   Another important additional LFO parameter is the shape or type of the
   LFO. There are many LFO Types that vary according to the function used to
   generate the LFO. \textsl{Yoshimi} supports the LFO shapes shown in
   \figureref{fig:types_of_lfo}.

\begin{figure}[H]
   \centering 
   \includegraphics[scale=0.75]{zyn/types_of_lfo1.png}
   \caption[LFO Functions]{LFO Types, Shapes, or Functions}
   \label{fig:types_of_lfo}
\end{figure}

\subsubsection{LFO Randomness}
\label{subsubsec:lfo_randomness}

\begin{figure}[H]
   \centering 
   \includegraphics[scale=0.75]{zyn/randomness_in_lfo2.png}
   \caption[LFO Randomization]{LFO Randomization}
   \label{fig:randomness_in_lfo}
\end{figure}

   \index{LFO!randomness}
   Another parameter is the LFO Randomness. It modifies the LFO amplitude or
   the LFO frequency at random. In \textsl{Yoshimi}
   one can choose how much the LFO
   frequency or LFO amplitude changes by this parameter.
   Observe \figureref{fig:randomness_in_lfo}.
   It shows some examples of randomness and how it changes the shape of a
   triangle LFO.

\subsubsection{LFO, More Settings}
\label{subsubsec:lfo_more_settings}

   Other settings are available as well.

   \index{LFO!continuous mode}
   Continous mode: If this mode is used, the LFO will not start from "zero" on
   each new note, but it will be continuous. This is very useful if one
   applies on filters to make interesting sweeps.

   \index{LFO!stretch}
   Stretch: It controls how much the LFO frequency changes according to the
   note’s frequency. It can vary from negative stretch (the LFO frequency is
   decreased on higher notes) to zero (the LFO frequency will be the same
   on all notes) to positive stretch (the LFO frequency will be
   increased on higher notes).

\subsubsection{LFO User Interface Panels}
\label{subsubsec:lfo_user_interface_panels}

   \setcounter{ItemCounter}{0}      % Reset the ItemCounter for this list.

\begin{figure}[H]
   \centering 
   \includegraphics[scale=1.0]{subpanels/Amplitude_LFO.png}
   \caption[Amplitude LFO Sub-Panel]{Amplitude LFO Sub-Panel}
   \label{fig:amplitude_lfo}
\end{figure}

   In \textsl{Yoshimi}, LFO parameters are available for amplitude, filters,
   and frequency.  They all have essentially the same interface elements.
   Note that
   \figureref{fig:amplitude_lfo} shows an example of an LFO stock sub-panel.

These parameters are:

   \begin{enumber}
      \item \textbf{Freq}
      \item \textbf{Depth}
      \item \textbf{Start}
      \item \textbf{Delay}
      \item \textbf{A.R}
      \item \textbf{F.R}
      \item \textbf{C} or \textbf{C.}
      \item \textbf{Str}
      \item \textbf{Type}
      \item \textbf{C} (copy)
      \item \textbf{P} (paste)
   \end{enumber}

   \setcounter{ItemCounter}{0}      % Reset the ItemCounter for this list.

   \itempar{Freq}{LFO!frequency}
   LFO Frequency.
   This parameter varies from 0 to 1.
   TODO: We still need to figure out what that scale means, however.

   Values: \texttt{0 to 1, 0.63*}

   \itempar{Depth}{LFO!depth}
   \index{LFO!amount}
   LFO Depth.  Also called "LFO Amount".

   Values: \texttt{0* to 127}

   \itempar{Start}{LFO!starting phase}
   LFO Start Phase. If this knob is at the lowest value, the LFO Start
   Phase will be random.

   Values: \texttt{0 = random to 127, 64*}

   \itempar{Delay}{LFO!delay}
   LFO Delay.

   Values: \texttt{0* to 127}

   \itempar{A.R}{LFO!amplitude randomness}
   LFO Amplitude Randomness.

   Values: \texttt{0* to 127}

   \itempar{F.R}{LFO!frequency randomness}
   LFO Frequency Randomness.

   Values: \texttt{0* to 127}

   \itempar{C}{LFO!continuous mode}
   LFO Continous Mode.

   Values: \texttt{Off*, On}

   \itempar{Str}{LFO!stretch}
   LFO Stretch. See the image in
   \figureref{fig:amplitude_lfo}.
   It shows the LFO stretch is set to zero,
   though the tooltip would show it to be 64.

   Values: \texttt{0 to 127, 64*}

   \itempar{Type}{LFO!type}
   LFO Function.

   Values: \texttt{SINE*, TRI, SQR, R.up, R.dn, E1dn, E2dn}

   Also present in this sub-panel are the usual \textbf{C}opy
   and \textbf{P}aste buttons that call up a copy-parameters or
   paste-parameters dialog.

\begin{figure}[H]
   \centering 
   \includegraphics[scale=1.0]{bottom-panel/instrument-edit/ADD/lfo-function-type.jpg}
   \caption[LFO Type Dropdown]{LFO Function Type Dropdown Element}
   \label{fig:lfo_function_type_dropdown}
\end{figure}

   \itempar{Type}{LFO!function type}
   LFO Type (or Shape, or Function).
   The various shapes of LFO functions are shown in
   \figureref{fig:types_of_lfo}.
   The values that can be selected are shown in
   \figureref{fig:lfo_function_type_dropdown}.

   Values: \texttt{SINE*, TRI, SQR, R.up, R.dn, E1dn, E2dn}

   % \item \textbf{C}
   % \item \textbf{P}

   Also present in this sub-panel are the usual \textbf{C}opy
   and \textbf{P}aste buttons that call up a copy-parameters or
   paste-parameters dialog.

   For reference,
   \figureref{fig:filter_lfo}
   shows the LFO sub-panel for a filter, and
   \figureref{fig:frequency_lfo_subpanel}
   shows the LFO sub-panel for frequency.

\subsubsection{Filter LFO Sub-panel}
\label{subsubsec:filter_lfo_sub_panel}

   % Obtain sub-items from checklist

\begin{figure}[H]
   \centering 
   \includegraphics[scale=1.0]{subpanels/Filter_LFO.png}
   \caption[Filter LFO Sub-Panel]{Filter LFO Sub-Panel}
   \label{fig:filter_lfo}
\end{figure}

   \begin{enumber}
      \item \textbf{Enable} (present on some versions of this sub-panel).
      \item \textbf{Freq.}
      \item \textbf{Depth}
      \item \textbf{Start}
      \item \textbf{Delay}
      \item \textbf{Str.}
      \item \textbf{C.}
      \item \textbf{A.R.}
      \item \textbf{F.R.}
      \item \textbf{Type}
      \item \textbf{C}
      \item \textbf{P}
   \end{enumber}

   \setcounter{ItemCounter}{0}      % Reset the ItemCounter for this list.

   \itempar{Enable}{lfo!enable}
   Enable the panel.  (Present on some versions of this sub-panel).

   \itempar{Freq.}{lfo!frequency}
   LFO Frequency.

   Values: \texttt{0 to 1, 0.64*}

   \itempar{Depth}{lfo!depth}
   LFO Amount.

   Values: \texttt{0* to 127}

   \itempar{Start}{lfo!start phase}
   LFO Startphase (leftmost is random).

   Values: \texttt{0 to 127, 64*}

   \itempar{Delay}{lfo!delay}
   LFO Delay.

   Values: \texttt{0* to 127}

   \itempar{Str.}{lfo!stretch}
   LFO Stretch.

   Values: \texttt{0 to 127, 64*}

   \itempar{C.}{lfo!continuous}
   Continuous LFO.

   Values: \texttt{Off*, On}

   \itempar{A.R.}{lfo!random amplitude}
   LFO Amplitude Randomness.

   Values: \texttt{0* to 127}

   \itempar{F.R.}{lfo!random frequency}
   LFO Frequency Randomness.

   Values: \texttt{0* to 127}

   \itempar{Type}{lfo!type}
   LFO Type.

\begin{figure}[H]
   \centering 
   \includegraphics[scale=1.0]{bottom-panel/instrument-edit/ADD/lfo-function-type.jpg}
   \caption[LFO Function Types]{LFO Function Type Dropdown}
   \label{fig:frequency_lfo_dropdown}
\end{figure}

   Values: \texttt{SINE*, TRI, SQR, R.up, R.dn, E1dn, E2dn}

   \itempar{C}{lfo!copy}
   Copy to Clipboard/Preset.

   \itempar{P}{lfo!paste}
   Paste from Clipboard/Preset.

\subsubsection{Frequency LFO Sub-panel}
\label{subsubsec:frequency_lfo_sub_panel}

\begin{figure}[H]
   \centering 
   \includegraphics[scale=1.0]{subpanels/Frequency_LFO.png}
   \caption[Frequency LFO Sub-Panel]{Frequency LFO Sub-Panel}
   \label{fig:frequency_lfo_subpanel}
\end{figure}

   This panel is basically identical to the Filter LFO panel described
   in the previous section.

% Move this section upward?

\subsection{Envelope Settings}
\label{subsec:envelope_settings}

   Envelopes control how the amplitude, the frequency, or the filter changes
   over time.  The general envelope generator has four sections:

   \begin{enumber}
      \item \textbf{Attack}.
         \label{ref:attack}
         \index{attack}
         \index{envelope!attack}
         The attack is the initial envelope response. 
         It begins when the key for the note is first held down
         (at Note On).
         The volume starts at 0, and rises fast or slowly until a peak value.
         In \textsl{Yoshimi}, the attack is always linear.
      \item \textbf{Decay}
         \label{ref:decay}
         \index{decay}
         \index{envelope!decay}
         When the attack is at its highest value, it immediately begins
         to decay to the sustain value.  The decay can be fast or slow.
         The attack and decay together can be used to produce something like
         horn blips, for example.
      \item \textbf{Sustain}
         \label{ref:sustain}
         \index{sustain}
         \index{envelope!sustain}
         This is the level at which the parameter stays while the key is
         held down, i.e. until a Note Off occurs.
      \item \textbf{Release}
         \label{ref:release}
         \index{release}
         \index{envelope!release}
         When the key is released, the sound decays, either fast or slowly,
         until it is off (the volume is 0).
   \end{enumber}

   The ADSR envelope generally controls the amplitude of the sound.
   In \textsl{Yoshimi},
   amplitude envelopes can be linear or logarithmic.

   Together, these values are called "ADSR".  

\begin{figure}[H]
   \centering 
   \includegraphics[scale=0.75]{zyn/ADSR_envelope1.png}
   \caption{ADSR Envelope (Amplitude)}
   \label{fig:adsr_envelope_depiction}
\end{figure}

   Figure~\ref{fig:adsr_envelope_depiction} on
   page~\pageref{fig:adsr_envelope_depiction}
   shows a depiction of an ADSR envelope.
   The ADSR is mostly applied to amplitude envelopes.

\begin{figure}[H]
   \centering 
   \includegraphics[scale=0.75]{zyn/ASR_frequency_envelope2.png}
   \caption{ASR Envelope, Frequency}
   \label{fig:asr_envelope_depiction}
\end{figure}

   Frequency envelopes control the frequency (more exactly, the pitch) of the
   oscillators. The following image depicts the stages of these envelopes.

   For frequency envelopes, a simpler form of envelope is used.
   This envelope is an ASR envelope, shown in
   \figureref{fig:asr_envelope_depiction}.
   The dotted line represents the real pitch of the sound without the envelope.
   The frequency envelopes are divided into 3 stages:

   \begin{enumber}
      \item Attack. 
      It begins at the Note On. The frequency starts from a certain value and
      glides to the real frequency of the note.
      \item Sustain.
      The frequency stays the same during the sustain period.
      \item Release.
      This stage begins on Note Off and glides the frequency of the note to a
      certain value.
   \end{enumber}

\subsubsection{Amplitude Envelope Sub-Panel}
\label{subsubsec:amplitude_envelope_subpanel}

\begin{figure}[H]
   \centering 
   \includegraphics[scale=1.0]{subpanels/Amplitude_Env.png}
   \caption[Amplitude Envelope Sub-Panel]{Amplitude Envelope Sub-Panel}
   \label{fig:amplitude_env}
\end{figure}

   \begin{enumber}
      \item \textbf{A.dt}
      \item \textbf{D.dt}
      \item \textbf{S.val}
      \item \textbf{R.dt}
      \item \textbf{Str}
      \item \textbf{L}
      \item \textbf{frcR}
      \item \textbf{C}
      \item \textbf{P}
      \item \textbf{E}
   \end{enumber}

   \setcounter{ItemCounter}{0}      % Reset the ItemCounter for this list.

   \itempar{A.dt}{envelope!attack time}
   Attack duration, attack time.
   TODO: determine the units of time at play for ADSR durations.

   Values: \texttt{0* to 127}

   \itempar{D.dt}{envelope!decay time}
   Decay duration, decay time.

   Values: \texttt{0 to 127, 44*}

   \itempar{S.val}{envelope!sustain value}
   Sustain value.
   This is the (relative?) level at which the envelope will settle
   while the note is held down.
   The only stage that always remains defined is the Sustain, where the
   envelopes freezes until a Note Off event.

   Values: \texttt{0 to 127*}

   \itempar{R.dt}{envelope!release time}
   Release time.

   Values: \texttt{0 to 127, 25*}

   \itempar{Str}{envelope!stretch}
   Stretch.
   How the envelope is stretched according the note.
   Envelope Stretch means that, on lower notes, the envelope will be longer.
   On the higher notes the envelopes are shorter than lower notes. In the
   leftmost value, the stretch is zero. The rightmost use a stretch of 200\%;
   this means that the envelope is stretched about 4 times peroctave.

   Values: \texttt{0 to 127, 64*}

   \itempar{L}{envelope!linear}
   Linear envelope.
   If this option is set, the envelope is linear, otherwise, it will be
   logarithmic.

   Values: \texttt{Off*, On}

   \itempar{frcR}{envelope!forced release}
   Forced release.
   This means that if this option is turned on, the release will go to the
   final value, even if the sustain stage is not reached. Usually, this must
   be set.

   Values: \texttt{Off, On*}

   If this option is turned on, the release will go to the
   final value, even if the sustain level is not reached.

   Also present in this sub-panel are the usual \textbf{C}opy
   and \textbf{P}aste buttons that call up a copy-parameters or
   paste-parameters dialog.

   \itempar{C}
   Copy to Clipboard/Preset.

   \itempar{P}
   Paste from Clipboard/Preset.

   \itempar{E}
   Amplitude Envelope Window.

\subsubsection{Envelope Settings}
\label{subsubsec:envelope_settings}

   Amplitude Envelope Window.

\begin{figure}[H]
   \centering 
   \includegraphics[scale=1.0]{bottom-panel/instrument-edit/ADD/amplitude-envelope.jpg}
   \caption{Amplitude/Filter/Frequency Envelope Editor}
   \label{fig:amplitude_envelope_editor}
\end{figure}

   \begin{enumber}
      \item \textbf{Graph Window}
      \item \textbf{FreeMode}
      \item \textbf{C}
      \item \textbf{P}
      \item \textbf{Close}
   \end{enumber}

   \itempar{FreeMode}{envelope!freemode enable}
   Freemode Enable.

   Values: \texttt{Off*, On} \\

\subsubsection{Freemode Envelope Settings}
\label{subsubsec:freemode_envelope_settings}

   The envelopes are parts that control a parameter (frequencies) of a sound. 

   For all envelopes, there is a mode that allows the user to set an arbitrary
   number of stages and control points. This mode is called Freemode.
   The only stage that always remains defined is the Sustain, where the
   envelopes freezes until a Note Off event.
   The Freemode envelope editor has a separate window to set the parameters and
   controls.

   The main concept of the freemode editor window is the
   \textsl{control point}.
   \index{control point}
   One can move the points using the mouse. In the right on the
   window, it shows the total duration of the envelope. If the mouse button
   is pressed on a control point, it will be shown the duration of the
   stage where the point is.

   \figureref{fig:amplitude_envelope_freemode}
   shows an example of the stock freemode envelope editor, with
   freemode enabled.

\begin{figure}[H]
   \centering 
   \includegraphics[scale=1.0]{bottom-panel/instrument-edit/ADD/amplitude-envelope-freemode.jpg}
   \caption{Amplitude/Filter/Frequency Envelope Freemode Editor}
   \label{fig:amplitude_envelope_freemode}
\end{figure}

   All of the envelope editors have some common controls.

   \begin{enumber}
      \item \textbf{Graph Window}
      \item \textbf{Add point}
      \item \textbf{E}
      \item \textbf{FreeMode}
      \item \textbf{Add point}
      \item \textbf{Delete point}
      \item \textbf{Sust}
      \item \textbf{Stretch}
      \item \textbf{L}
      \item \textbf{frcR}
      \item \textbf{Close}
      \item \textbf{C}
      \item \textbf{P}
   \end{enumber}

   \setcounter{ItemCounter}{0}      % Reset the ItemCounter for this list.

   \itempar{E}{envelope!editor}
   Editor.  Graph Window.
   Shows a window with the real envelope shape and the option to convert to
   freemode to edit it.
   The envelope editor shows a window in which one can view and modify the
   detailed envelope shape, or convert it to "freemode" to edit it almost
   without restriction.
   By default, only the \textsl{Freemode} button/checkbox is visible.

   If an envelope has the FreeMode mode enabled, it allows one to edit the
   graph of the envelope directly. Select a point from the graph and move it.
   Notice that \textsl{only the line before the currently edited point of the
   envelope} changes its duration.

   If a point is being dragged, the text on the right shows the duration of
   the line before it. Otherwise, the text shows the total duration of the
   envelope. 

   If the envelope doesn't have the FreeMode mode enabled, it doesn't allow
   one to move the points; the envelope window is then useful only to see
   what happens if one changes the ADSR settings.

   \itempar{FreeMode}{envelope!freemode}
   FreeMode.  Provides a mode where completely arbitrary envelopes may be
   drawn.

   Values: \texttt{Off*, On}

   Actually, the envelopes aren't completely arbitrary, as the sustain
   section is always flat, and its duration corresponds with the duration
   the note is held down.
   When this mode is enabled, the rest of the controls shown in
   \figureref{fig:amplitude_envelope_freemode}
   appear, and are described in the following paragraphs.

   \itempar{Add point}{envelope!add point}
   Add point.
   Provides a way to add a data point to the Freemode envelope.
   It adds the point after the currently-selected point. One can select a
   point by clicking on it.

   \itempar{Delete point}{envelope!delete point}
   Delete point.
   Provides a way to delete the current data point from the Freemode envelope.

   \itempar{Sust}{envelope!sustain}
   Sustain point.
   Sets the sustain point.
   The sustain point is shown using the yellow line.
   If the point is at 0, then sustain is disabled.

   Values: \texttt{0, 1, 2*}

   \begin{enumber}
      \item{0} means that sustain is disabled, and the envelope immediately
      starts dying, even if the note is held.
      \item{1} seems to mean the sustain curve follows its course while the
      note is held.
      \item{2} seems to mean that extra sustain kicks in after the note is
      released.
   \end{enumber}

   It is difficult to determine the difference between 1 and 2.

   \itempar{Stretch}{envelope!stretch}
   Envelope Stretch.
   How the envelope is stretched according the note. On the higher notes the
   envelopes are shorter than lower notes. At the leftmost value, the
   stretch is zero. The rightmost sets a stretch of 200\%; this means that the
   envelope is stretched about four times/octave.

   \itempar{L}{envelope!linear}
   Envelope Linear.
   This setting is only available in the amplitude envelope.
   If enabled, the envelope is linear.
   If not enabled, the envelope is logarithmic (dB).

   Values: \texttt{Off*, On}

   \itempar{frcR}{envelope!forced release}
   Forced Release.
   This means that if this option is turned on, the release will go to the
   final value, even if the sustain stage is not reached. Usually, this must
   be set.
   When the key is released, the position of the envelope jumps directly to
   the point after the release point. If the release is disabled, the
   envelope position jumps to the last point on release.

   Values: \texttt{Off*, On}

   \itempar{Close}{envelope!close dialog}
   Close Dialog.

   Also present in this sub-panel are the usual \textbf{C}opy
   and \textbf{P}aste buttons that call up a copy-parameters or
   paste-parameters dialog, as well as a button
   to bring up the editor window.

\subsubsection{Envelope Settings, Frequency}
\label{subsubsec:envelope_settings_for_frequency}

   These envelopes controls the frequency (more exactly, the pitch) of the
   oscillators.
   Observe \figureref{fig:asr_envelope_depiction}.
   It depicts the stages of these envelopes.
   The dotted line represents the real pitch of the sound without the envelope.

   The frequency envelopes are divided into 3 stages:
   attack (see \ref{ref:attack});
   sustain (see \ref{ref:sustain});
   and
   release (see \ref{ref:release}).

   One question to answer is:  
   can the attack and release go in the opposite directions, or do the knob
   ranges prohibit this?

\begin{figure}[H]
   \centering 
   \includegraphics[scale=1.0]{subpanels/Frequency_Env.png}
   \caption[Frequency Envelope Sub-Panel]{Frequency Envelope Sub-Panel}
   \label{fig:frequency_env}
\end{figure}

   \begin{enumber}
      \item \textbf{Enable} (present on some versions of this sub-panel).
      \item \textbf{A.value} or \textbf{A.val}
      \item \textbf{A.dt}
      \item \textbf{R.dt}
      \item \textbf{R.val} (present on some versions of this sub-panel).
      \item \textbf{Stretch}
      \item \textbf{frcR}
      \item \textbf{C}
      \item \textbf{P}
      \item \textbf{E}
   \end{enumber}

   For Frequency Envelopes the interface has the following parameters:

   \setcounter{ItemCounter}{0}      % Reset the ItemCounter for this list.

   \itempar{Enable}{envelope!enable}
   Enable the panel.  (Present on some versions of this sub-panel).

%  \itempar{A.value}{envelope!attack value}

   \itempar{A.val}{envelope!attack value}
   Attack value.
   We need to figure out what this means.

   Values: \texttt{0 to 127, 64*}

   \itempar{A.dt}{envelope!attack time}
   Attack duration. Attack time.

   Values: \texttt{0 to 127, 40*}

   \itempar{R.dt}{envelope!release time}
   Release time.

   Values: \texttt{0 to 127, 60*}

   \itempar{R.val}{envelope!release value}
   Release Value.
   Actually present only on the Frequency Env sub-panel.

   Values: \texttt{0 to 127, 64*}

   \itempar{Stretch}{envelope!stretch}
   Envelope Stretch.
   Envelope Stretch (on lower notes make the envelope longer).

   Values: \texttt{0 to 127, 64*}

   \itempar{frcR}{envelope!forced release}
   Forced release.

   Values: \texttt{Off, On*}

   If this option is turned on, the release will go to the
   final value, even if the sustain level is not reached.

   Also present in this sub-panel are the usual \textbf{C}opy
   and \textbf{P}aste buttons that call up a copy-parameters or
   paste-parameters dialog, as well as a button
   to bring up the editor window.

\subsubsection{Envelope Settings for Filter}
\label{subsubsec:envelope_settings_for_filter}

   This envelope controls the cutoff frequency of the filters.
   The filter envelopes are divided into 4 stages:

% attack (see \ref{ref:attack});
% decay (see \ref{ref:decay});
% sustain (see \ref{ref:sustain});
% and
% release (see \ref{ref:release}).

   \begin{enumber}
      \item Attack.
         It begins at the Note On.
         The cutoff frequency starts from a certain value and glides to another
         value.
      \item Decay.
         The cutoff frequency continues to glide to the real cutoff frequency
         value of the filter (dotted line).
      \item Sustain.
         The cutoff frequency stays the same during the sustain period (dotted
         line).
      \item Release.
         This stage begins on Note Off and glides the filter cutoff frequency
         of the note to a certain value.
   \end{enumber}

\begin{figure}[H]
   \centering 
   \includegraphics[scale=1.0]{subpanels/Filter_Env.png}
   \caption[Filter Envelope Sub-Panel]{Filter Envelope Sub-Panel}
   \label{fig:filter_env}
\end{figure}

   \begin{enumber}
      \item \textbf{A.value}
      \item \textbf{A.dt}
      \item \textbf{D.val}
      \item \textbf{D.dt}
      \item \textbf{R.dt}
      \item \textbf{Stretch}
      \item \textbf{frcR}
      \item \textbf{L}
   \end{enumber}

   Filter Envelopes has the following parameters:

   \setcounter{ItemCounter}{0}      % Reset the ItemCounter for this list.

   \itempar{A.value}{envelope!attack value}
   Attack Value.  Starting Value.
   We need to figure out what this means.

   Values: \texttt{0 to 127, 64*}

   \itempar{A.dt}{envelope!attack time}
   Attack Duration.  Attack Time.

   Values: \texttt{0 to 127, 40*}

   \itempar{D.val}{envelope!decay value}
   Decay Value.

   Values: \texttt{0 to 127, 64*}

   \itempar{D.dt}{envelope!decay time}
   Decay Duration.  Decay Time.

   Values: \texttt{0 to 127, 70*}

   \itempar{R.dt}{envelope!release time}
   Release time.

   Values: \texttt{0 to 127, 60*}

   \itempar{Stretch}{envelope!stretch}
   Stretch.
   Envelope Stretch (on lower notes make the envelope longer).

   Values: \texttt{0 to 127, 64*}

   \itempar{frcR}{envelope!forced release}
   Forced Release.

   Values: \texttt{Off, On*}

   If this option is turned on, the release will go to the
   final value, even if the sustain level is not reached.

   Also present in this sub-panel are the usual \textbf{C}opy
   and \textbf{P}aste buttons that call up a copy-parameters or
   paste-parameters dialog, as well as a button that bring up the editor
   window.

   Addition picture and GUI items for ADDsynth version?

   Figure: bottom-panel/instrument-edit/ADD/ADDsynth-filter-envelope.jpg

   \itempar{L}{envelope!linearity}
   If this option is set, the envelope is linear, otherwise, it will be
   logarithmic.

   Values: \texttt{Off*, On}

   % \item \textbf{C}
   % \item \textbf{P}
   % \item \textbf{E}

\subsubsection{Formant Filter Settings}
\label{subsubsec:formant_filter_settings}

   This window allows one to change most of the parameters of the formant
   filter. 

\begin{figure}[H]
   \centering 
   \includegraphics[scale=0.75]{zyn/formant_filter.png}
   \caption[Formant Filter Editor]{Formant Filter Editor Dialog}
   \label{fig:formant_filter_editor}
\end{figure}

   \begin{enumber}
      \item \textbf{Category}
      \item \textbf{Num.Formants}
      \item \textbf{Fr.Sl.}
      \item \textbf{Vw.Cl.}
      \item \textbf{C.f.}
      \item \textbf{Oct.}
      \item \textbf{Vowel no}
      \item \textbf{Formant}
      \item \textbf{freq}
      \item \textbf{Q}
      \item \textbf{amp}
      \item \textbf{Seq Size}
      \item \textbf{S.Pos}
      \item \textbf{Vowel}
      \item \textbf{Strtch}
      \item \textbf{Neg Input}
   \end{enumber}

\paragraph{Formant Parameters}
\label{paragraph:formant_parameters}

   \itempar{Num.Formants}{formant!number}
   Number of Formants Used.

   Values:  \texttt{0 to xxx?}

   \itempar{Fr.Sl.}{formant!slowness}
   Formant Slowness.

   Values:  \texttt{0 to xxx?}

   This parameters prevents too-fast morphing between vowels.

   \itempar{Vw.Cl.}{formant!clearness}
   Vowel "Clearness".

   Values:  \texttt{0 to xxx?}

   Sets how much the vowels are kept "clear",
   that is, how much the "mixed" vowels are avoided.

   \itempar{C.f.}{formant!cf}
   Center Frequency.

   Values:  \texttt{0 to xxx?}

   The center frequency of the graph.

   \itempar{Oct.}{formant!octaves}
   Number of Octaves.

   Values:  \texttt{0 to xxx?}

   The number of octaves in the graph.

\paragraph{Formant Vowel Parameters}
\label{paragraph:formant_vowel_parameters}

   \itempar{Vowel no}{formant!vowel number}
   The number of the current vowel.
   This number means what?

   Values:  \texttt{0 to xxx?}

   \itempar{Formant}{formant!number}
   The current formant.

   Values:  \texttt{0 to xxx?}

   \itempar{freq}{formant!frequency}
   The frequency of the current formant.

   Values:  \texttt{0 to xxx?}

   \itempar{Q}{formant!Q}
   The Q (resonance depth or bandwidth) of the current formant.

   Values:  \texttt{0 to xxx?}

   \itempar{amp}{formant!amplitude}
   Amplitude of the current formant.

   Values:  \texttt{0 to xxx?}

\paragraph{Formant Sequence Parameters}
\label{paragraph:formant_sequence_parameters}

   The sequence represents what vowel is selected to sound according to the
   input from the filter envelopes and LFO's.
 
   \itempar{Seq Size}{formant!seq size}
   Sequence Size.
   The number of vowels in the sequence.

   Values:  \texttt{0 to xxx?}

   \itempar{S.Pos}{formant!seq position}
   Sequence Position.
   The current position of the sequence.

   Values:  \texttt{0 to xxx?}

   \itempar{Vowel}{formant!vowel position}
   The vowel from the current position.

   Values:  \texttt{0 to xxx?}

   \itempar{Strtch}{formant!seq stretch}
   How the sequence is stretched.
   This number means what?

   Values:  \texttt{0 to xxx?}

   \itempar{Neg Input}{formant!reversed}
   Negative Input.
   If enabled, the sequence is reversed.

   Values:  \texttt{0 to xxx?}

\subsubsection{Controller Settings}
\label{subsubsec:controller_settings}

TODO.

\subsection{Clipboard Presets}
\label{subsec:clipboard_presets}

In many of the settings panels, there are buttons
labelled \textbf{C}, \textbf{P}, and \textbf{E},
\textbf{E} is the editor window, discussed in 
section \ref{subsubsec:freemode_envelope_settings}
\textbf{C} and \textbf{P} are the clipboard/present copy and paste
dialogs, respectively.

\subsubsection{Clipboard/Preset Copy}
\label{subsubsec:clipboard_copy}
\index{subsubsec:clipboard!copy}

   Note that \figureref{fig:copy_to_clipboard}
   shows an example of the copying dialog for the clipboard.

\begin{figure}[H]
   \centering 
   \includegraphics[scale=0.75]{bottom-panel/instrument-edit/ADD/copy-to-clipboard-preset.jpg}
   \caption[Copy to Clipboard]{Copy to Clipboard/Presets}
   \label{fig:copy_to_clipboard} 
\end{figure}

   \setcounter{ItemCounter}{0}      % Reset the ItemCounter for this list.

   \itempar{Type}{clipboard!copy type}
   Clipboard type for copying.
   This field indicates the context (e.g h. "envamplitude") or name of the
   clipboard to which the data will be copied.

   \itempar{Clipboard list}{clipboard!list}
   Clipboard list.

   \itempar{Copy to Preset}{preset!copy}
   Clipboard to preset.
   Provides a way to specify the preset to which this data should be
   copied.

   To save to a preset, type the desired name of the setting.  This entry
   will enable this button.  When the button is pressed, the preset will
   be saved to the default directory,
   as specified per \paragraphref{paragraph:menu_yoshimi_settings_preset_dirs}.
   (Be sure to set up a default directory where ordinary users have write
   permissions!)
   The file-name of the of the preset will be a hidden file such as

   \begin{verbatim}
      .ADnoteParameters.xpz
   \end{verbatim}

   There is no way in \textsl{Yoshimi} to change this name.  One
   must do it using file system commands.

   \itempar{Copy to Clipboard}{clipboard!copy}
   Preset to Clipboard.

\subsubsection{Clipboard/Preset Paste}
\label{subsubsec:clipboard_paste}
\index{subsubsec:clipboard!paste}

   Observe \figureref{fig:paste_to_clipboard}.
   It shows an example of the pasting dialog for the clipboard.

\begin{figure}[H]
   \centering 
   \includegraphics[scale=0.75]{bottom-panel/instrument-edit/ADD/paste-from-clipboard-preset.jpg}
   \caption[Paste from Clipboard]{Paste from Clipboard/Presets}
   \label{fig:paste_to_clipboard} 
\end{figure}

   \begin{enumber}
      \item \textbf{Add point}
      \item \textbf{Type}
      \item \textbf{Clipboard list}
      \item \textbf{Paste from Preset}
      \item \textbf{Paste from Clipboard}
   \end{enumber}

   \setcounter{ItemCounter}{0}      % Reset the ItemCounter for this list.

   \itempar{Type}{clipboard!paste type}
   Clipboard type for pasting.  
   This field indicates the context (e.g h. "envamplitude") or name of the
   clipboard to which the data will be copied.

   \itempar{Clipboard list}{clipboard!list}
   Clipboard list.

   \itempar{Paste from Preset}{preset!paste}
   Paste from preset.
   Provides a way to specify the preset to which this data should be
   copied.

   \itempar{Paste from Clipboard}{clipboard!paste}
   Clipboard to preset.

%-------------------------------------------------------------------------------
% vim: ts=3 sw=3 et ft=tex
%-------------------------------------------------------------------------------


% Top Panel User-Interface items

%-------------------------------------------------------------------------------
% yum_top_panel
%-------------------------------------------------------------------------------
%
% \file        yum_top_panel.tex
% \library     Documents
% \author      Chris Ahlstrom
% \date        2015-05-29
% \update      2018-05-26
% \version     $Revision$
% \license     $XPC_GPL_LICENSE$
%
%     Provides the Top Panel section of yoshimi-user-manual.tex.
%
%-------------------------------------------------------------------------------

\section{Top Panel}
\label{sec:top_panel}

   The \textsl{Yoshimi} top panel provides quick access to some major
   features of the application.
   The top panel is shown in
   \figureref{fig:yoshimi_main_screen}.

   Here are the major elements of the top panel.

   \begin{enumber}
      \item \textbf{Stop!}
      \item \textbf{Reset}
%     \item \textbf{Panel}
      \item \textbf{Mixer Panel}
%     \item \textbf{VirKbd}
      \item \textbf{Virtual Keyboard}
      \item \textbf{Vectors}
      \item \textbf{Midi Learn}
      \item \textbf{S}
      \item \textbf{Shift}
      \item \textbf{Detune}
%     \item \textbf{Reset Detune}
      \item \textbf{Volume}
   \end{enumber}

   \setcounter{ItemCounter}{0}      % Reset the ItemCounter for this list.

   \itempar{Stop!}{top panel!stop all sound}
   Stop!
   This button causes \textsl{Yoshimi} to
   "Cease all sound immediately!"
   Useful when MIDI input suddenly stops due to a bug in the MIDI source.

   \itempar{Reset}{top panel!reset}
   \index{master reset}
   Master Reset.
   Resets \textsl{Yoshimi} to its default state, when no default configuration
   files exist.  If there is a saved default state and the \textbf{start with
   default} option is set, then a reset will reload that file. For any other
   situation it will set the first-time defaults.

   \index{master reset, ctrl}
   If the Ctrl key is held down while doing a master reset, then
   MIDI Learn will also be cleared.

   \itempar{Mixer Panel}{top panel!mixer panel}
   This button brings up a panel that shows a "mixer" view
   of all of the parts that have been created in the current
   state of \textsl{Yoshimi}.

   For the details of this panel,
   see \sectionref{subsec:mixer_panel_window}.

   \itempar{Virtual Keyboard}{top panel!virtual keyboard}
   This button brings up the virtual keyboard, which is a way to enter
   MIDI information without a real MIDI keyboard.
   It also provides a way to use the computer keyboard for faster
   playing.  See \sectionref{subsec:virtual_keyboard}.

   \itempar{Vectors}{top panel!vectors}
   Provides the recent new feature of vector control.  This has been moved
   to this button from the \textbf{Yoshimi} menu.  See
   \sectionref{subsec:vector_dialogs}.

   \itempar{Midi Learn}{top panel!midilearn}
   When pressed, the \textsl{Yoshimi} Midi Learn editing window is opened.
   This button used to be for opening the \textsl{Yoshimi} Reports window,
   but that is seldom changed so they've been swapped.

   \itempar{S}{top panel!stereo}
   Stereo Button.
   This toggles the main audio output between stereo and mono, but doesn't
   affect individual part ones.
   It is useful for checking balance between the two on the fly while playing.
   This is new from V 1.5.11.

   \itempar{Shift}{top panel!key shift}
   Master Key Shift.
   This is the key-shift (transpose) that applies to all parts, in units of
   semitones.
   In recent versions of \textsl{Yoshimi}, this range has been extended.
   Also note that the master key shift can be set via the user-interface, the
   command-line, or by MIDI NRPN commands.

   Values: \texttt{-36 to 36, 0*}

   Also see the \textbf{Key Shift} item in
   \sectionref{sec:bottom_panel} \hspace{6 pt}for more information.

   \itempar{Detune}{top panel!detune}
   Detune.  Provides a global fine detune functionality.
   The fine detune mapping to the knob values shown below is
   -64 to 63 cents.

   Values: \texttt{0 to 127, 64*} (float)

   \itempar{Volume}{top panel!overall volume}
   Volume, Master Volume.
   Controls the overall volume of all sounds generated by
   \textsl{Yoshimi}.

   Values: \texttt{0 to 127, 90*}

\subsection{Mixer Panel Window}
\label{subsec:mixer_panel_window}
\index{Panel}

   The \textsl{Mixer Panel} button opens the "Mixer" window.
   The mixer provides a global view of the most important
   adjustable parameters of all of the defined parts.
   There are two views, a 2x8 view and a 2x16 view.
   See \figureref{fig:yoshimi_part_panel_2x8}, which
   shows the 2x8 view.

   The Panel Window allows one to edit some important part parameters
   (instrument/volume/panning/etc.) and it acts like a mixer. Also, since
   V 1.5.11 this window shows VU-meters for each part.
   To make a part the current part, left-click on its \textbf{Edit} button.
   To edit an instrument, right-click on the \textbf{Edit} button for that
   instrument.

   When using the JACK audio backend, parts can be individually routed or sent
   to the main L/R outputs, either by themselves, or working with the main Left
   and Right outputs at the same time.  This is controlled from the panel
   window, and the settings are saved with all the other parameters.

   The individual \textsl{Direct Part} outputs will have the part effects, and any
   \textbf{Insertion} effects that are linked to them, but not the
   \textbf{System} effects.

   \textsl{Yoshimi} used to register all parts with JACK by default, but that
   is a bit much now that 64 parts are available, so now \textsl{Yoshimi}
   uses an "on demand" model.

   In the mixer panel window one will see a field just above the
   \textbf{Edit} button.
   This field determines the audio destination on a part-by-part basis,
   defaulting to just the main L+R pair. The direct part outputs are only
   exposed on parts that are active, and have the destination set to either
   \textbf{part} or \textbf{both}.
   Once activated, they will remain in place for the entire session, even if
   the part is later disabled or routed to main only. This is so that other
   programs won't see links suddenly disappear, although they will become
   silent.  This setting is preserved in Yoshimi's patch sets and will be
   re-instated when next loaded.

\begin{figure}[H]
   \centering
%  \includegraphics[scale=0.75]{top-panel/yoshimi-panel-2x8.jpg}
%  \includegraphics[scale=0.75]{top-panel/yoshimi-panel-2x8-1_5.jpg}
%  \includegraphics[scale=0.75]{1.5.0/Mixer8.png}
%  \includegraphics[scale=0.75]{1.5.7/Mixer8.png}
   \includegraphics[scale=0.75]{1.5.11/Mixer8.png}
   \caption[Yoshimi Mixer Panel]{Yoshimi Mixer Panel, 2x8 View}
   \label{fig:yoshimi_part_panel_2x8}
\end{figure}

   Note that there is also a 1x16 version of this dialog (not shown).
   This dialog has been further updated; as well as the \textbf{Solo}
   control, (described below) it now presents separate left and right
   VU meter bars.

   \setcounter{ItemCounter}{0}      % Reset the ItemCounter for this list.

   \itempar{Part Summary}{Part Section, 1 to 16}
   Parts View or Summary.

   \itempar{Enable part}{parts!enable}
   Enable/Disable the part. The check-box enables/disables the part.
   When the part is disabled, its controls are greyed out.

   Values: \texttt{Off*, On}

   \itempar{Part name}{parts!name}
   Instrument name. Click on this box to change the instrument (it will
   open up the \textbf{Edit} window.

   \itempar{Volume Slider}{parts!volume}
   Volume Bar.
   Changes the volume of the part.

   \itempar{VU-meter display}{parts!meter}
   Shows the level of the part when playing.

   \itempar{Panning Knob}{parts!panning}
   Panning Dial-Button.
   Changes the panning of the part.

   Values: \texttt{0 (left) to 64* (center) to 127 (right)}

   \itempar{Channel}{parts!channel}
   Receive from MIDI channel.
   Changes the MIDI channel assigned to the part.

   Values: \texttt{Ch1*, Ch2, ..., Ch16}

   \itempar{Main}{parts!destination}
   Set Audio Destination.
   Sets the audio for this part to be routed to the main audio output, to
   the audio specified by the part setup, or to both outputs.
   This option requires that \textsl{Yoshimi} use JACK audio.  If running
   ALSA, this option is disabled (greyed out).
   The part's audio destination (JACK) is saved with the patch sets, and
   so is the number of available parts.  (\textsl{ZynAddSubFX} will still
   load these files, but it ignores any settings it doesn't recognise. If
   one re-saves in \textsl{ZynAddSubFX}, the settings will be lost.)

   Values: \texttt{Main, Part, or Both}

   \itempar{Edit}{parts!edit}
   The Edit button provides two function
   Left mouse button: Part select.
   Right mouse button: Instrument edit.
   This setup is a bit unintuitive, but the tooltips make it clear
   which click one might want to use.

   \itempar{Solo}{parts!solo}
   \index{channel switcher}
   \index{parts!channel switcher}
   There are two commands that change the way
   \textsl{Yoshimi} responds to incoming MIDI,
   so that only one of a group of instruments will see note-on events, but all
   of the group will see note-off ones. These commands
   are both in the \textbf{Mixer Panel}, as shown below.
\begin{figure}[H]
   \centering
   \includegraphics[scale=0.75]{1.5.11/MixerSwitch8.png}
   \caption[Yoshimi Mixer Panel]{Yoshimi Mixer Panel, Solo}
   \label{fig:yoshimi_part_panel_solo}
\end{figure}
   They are referred as a \textsl{Solo} feature.
   The \textbf{Solo} settings are saved in patch sets, which saves a
   little frustration when loading one's current favorite patch set.

   Values: \texttt{Off*, Row, Col, Loop, TwoWay}

   For these modes, if one has a programmable MIDI controller, one can set it
   up to activate a specific part, or to increment/decrement which part in the
   set is active.  The \textbf{Solo} drop-down list enables the feature for
   either \textbf{Row} or \textbf{Col}umn mode, and also makes the CC spin-box
   visible.
   One uses this spin-box to set which incoming CC changes the part that gets
   new notes.
   The \textsl{value} this CC sends performs the actual change, instantly and
   silently. Most importantly it leaves any existing notes sounding through a
   note off release and the effects tail.

   \index{solo!row}
   \textbf{Row} means that all of the first 16 channels will be set to channel
   1, but with only one active, and one's CC will dial up any of the parts,
   disabling the others.
   In \textbf{Row} mode the whole of the first 16 parts are ostensibly
   receiving on channel 1.  This mode is most useful if one wants to play live
   through a piece with multiple instrument changes while playing. It works
   best with a foot switch that internally stores a channel number and
   increments/decrements it with every press, then sends it.

   Although this uses all of the first 16 parts, one can set the number of
   parts to 32, so that one can use the 17+ row for normal 1 through 16
   channels. Also, if one has \textbf{vector control} set up, \textbf{Solo}
   intelligently recognises this fact, and, for each vector it finds, it will
   switch in/out the whole vector column appropriately.

   Note that \textbf{Row} is recommended only for automation.

   \index{solo!column}
   For running \textbf{Solo} in \textbf{Column} mode, one needs to have 32 or
   (preferably) 64 parts set; with this setup, one can have up to 4 parts
   switched per channel, and independently of each other. However, this works
   more like vector control in that switching has to be in groups of 16. For
   example, to control the channel 4 column one would send 4, 20, 36, 52 to
   select the wanted part. This usage is more appropriate for post recording
   MIDI automation.

   For both of these modes, if one has a programmable MIDI controller, one can
   set it up to activate a specific part, or the increment/decrement which part
   in the set is active.

   \index{solo!loop}
   \textbf{Loop}.  Loop mode is a variation on the Row mode.
   With this mode, if one sends \textsl{any} value (except zero)
   via the designated CC,
   it will increment the active part by one, rolling round to 1 after 16.
   This should make even the dumbest foot controller usable.

   To keep it all lightweight, one needs to load and activate all the patches
   and parts wanted, but that could be obtained from a saved patch set, and the
   channels are only changed from the very first time one sends the CC.  To look
   really clever, the whole lot can be embedded in a MIDI file.

   One can play a piece that needs to live-switch between 10 instruments, using
   a footswitch to do the channel changes. The device holds a channel
   number (starting with zero) and increments/decrements it depending on which
   switch is pressed, then sends the resulting CC.

   At the \textsl{2017 Linux Audio Conference},
   it became obvious there was a possible problem with the \textbf{Loop}
   feature. This issue arose from the 'bounce' of a cheap footswitch that would
   then send two changes instead of one. It is now resolved by adding a
   debounce timer of about 60ms so that a second pulse inside that time will be
   ignored.

   \index{solo!TwoWay}
   \textbf{TwoWay}.
   A further development suggested at that time has now also been implemented. This
   is the addition of a \textbf{TwoWay} option.
   This option works in a similar way to \textbf{Loop}, but
   a value between 1 and 63 will step down instead of up, so that if one does make
   a mistake it can quickly be rectified.

   With a reasonable MIDI controller one can usually set a couple of foot switches
   to report the same CC but with different values. Alternatively stick to their
   native values and pass them through something like \textsl{QmidiRoute} to do the
   translation.

   \itempar{Parts Layout}{parts!change layout}
   \index{parts!2x8}
   \index{parts!1x16}
   Changes the layout of the panel to the other layout, either
   \textbf{Change to 2 x 8} or
   \textbf{Change to 1 x 16}.

   \itempar{Close}{parts!close}
   Close the window.

\subsection{Virtual Keyboard}
\label{subsec:virtual_keyboard}

   This section describes the detailed usage of the
   \textsl{Yoshimi} virtual keyboard.
   The virtual keyboard lets one play notes using the keyboard/mouse. There is
   no MIDI requirement.

   Using the computer keyboard: The keyboard is split into three octaves.
   It may happen that the keys will not trigger a note-on;
   this happens when another widget has the keyboard focus.
   To play using the computer keyboard, click on the virtual keyboard.

   Using the mouse: One can use the mouse to play.  If one presses the
   Shift key while pressing the mouse button, the keys will be not released
   when the mouse button is released.  If one presses the \textbf{Stop!} or
   "panic" button from the \textsl{ZynAddSubFX}/\textsl{Yoshimi} main window,
   all keys are released.

\begin{figure}[H]
   \centering
%  \includegraphics[scale=1.0]{top-panel/yoshimi-virtual-keyboard.jpg}
   \includegraphics[scale=1.0]{1.5.0/Keyboard.png}
   \caption{Yoshimi Virtual Keyboard}
   \label{fig:yoshimi_virtual_keyboard}
\end{figure}

\subsubsection{Virtual Keyboard, Basics}
\label{subsubsec:virtual_keyboard_basics}

   \begin{enumber}
      \item \textbf{Pwh}
%     \item \textbf{R}
      \item \textbf{Midi Channel}
      \item \textbf{Velocity}
      \item \textbf{Velocity}
      \item \textbf{Octave}
      \item \textbf{Key Oct}
      \item \textbf{Maps Oct}
      \item \textbf{Controller}
      \item \textbf{Cval}
      \item \textbf{Close}
   \end{enumber}

   \itempar{Pwh}{vkdb!pitch bend}
   Pitch bend knob. Pitch wheel.
   This item is now a slider control.  To reset it to the middle position,
   right-click within the slider.

%  Press the \textbf{R} button to reset it.
%
%  \itempar{R}{vkdb!reset pitch bend}
%  Reset Pitch Bend.

   \itempar{Midi Channel}{vkdb!midi channel}
   MIDI Channel.
   Sets the MIDI channel for the virtual keyboard.

   Values: \texttt{1* to 16}

   \itempar{Velocity}{vkdb!velocity}
   Velocity of Notes.
   Sets the note-on velocity for the virtual keyboard.

   Values: \texttt{1 to 127, 100*}

   \itempar{Velocity}{vkdb!velocity randomness}
   Velocity Randomness.

   Values: \texttt{0* to 127}

   \itempar{Octave}{vkdb!qwerty}
   Transposes all of the virtual keyboard notes by the given number of
   octaves.

   Values: \texttt{1, 2*, 3, 4, 5}

   \itempar{Key Oct}{vkdb!key octave}
   Transposes the upper keys (the numbers and the "qwert" keys);
   the range of these keys is from C-4 to A-5 (replace the '5' with the octave).
   Look at the tooltips as a reminder.

   Values: \texttt{1, 2*, 3, 4, 5}

   \itempar{Maps Oct}{vkdb!maps octave}
   Transposes the lower keys ("sdghj" and "zxcvb"); the range of these keys is
   from C-3 to E-4 (replace the '4' with the octave).  Look at the tooltips as a
   reminder.

   Values: \texttt{1, 2*, 3, 4, 5}

   \itempar{Controller}{vkdb!controller}
   Keyboard Controller.

   Values: \texttt{01:Mod.Wheel, 07:Volume, 10:Panning,
      11:Expression, 64:Sustain, 65:Portamento, 71:Filter Q,
      74:Filter Freq*, 75:Bandwidth, 76:FM Gain,
      77:Res.c.freq, 78:Res.bw.}

   Sets the controller to be changed according to the \textbf{Cval}
   controller.
   See \sectionref{subsubsec:virtual_keyboard_controllers}.

   \itempar{Cval}{vkdb!controller value}
   Controller value.
   Changes the controller value.
   This item consists of two parts.  The top part shows a tiny
   number representing the current value of the selected controller.
   The bottom part is a combination value-bar and slider that one
   can move up and down with the mouse, to change the controller value.
   Note that the \textbf{Cval} value might not reflect the
   internal value of the controller when one changes the controller.

   Values: \texttt{1 to 127, 96*}

   \itempar{Close}{vkdb!close}
   Close button.

\subsubsection{Virtual Keyboard, ASCII Mapping}
\label{subsubsec:virtual_keyboard_ascii}

   In addition to this virtual keyboard, the QWERTY (or Dvorak, or AZERTY)
   keyboards can be used to produce notes.
   The computer keyboard layout is shown in
   \figureref{fig:qwerty_virtual_keyboard},
   From lowest octave to highest, the colors are blue, then green, then red.
   The "white" keys are the light colors, and the "black" keys are the
   deeper colors.
   The range of the keys on the "zxcvb..." row is C3 to E4.
   The range of the keys on the "qwert..." row is C4 to A5.
   These octave ranges can be adjusted.
   The computer keyboard will produce notes only when the virtual keyboard
   has focus.
%  Also note that we replaced the monopoly symbol with the monopolist
%  symbol.  On X11 systems, this key is known as the "Super" key.

\subsubsection{Virtual Keyboard, Controllers}
\label{subsubsec:virtual_keyboard_controllers}

   This section (will give) a brief overview of the controller's that this
   window supports.

   \begin{enumber}
      \item \textbf{01: Mod. Wheel}
      \item \textbf{07: Volume}
      \item \textbf{10: Panning}
      \item \textbf{11: Expression}
      \item \textbf{64: Sustain}
      \item \textbf{65: Portamento}
      \item \textbf{71: Filter Q}
      \item \textbf{74: Filter Freq.}
      \item \textbf{75: Bandwidth}
      \item \textbf{76: FM Gain}
      \item \textbf{77: Res. c. freq}
      \item \textbf{78: Res. bw.}
   \end{enumber}

      The following figure shows the corresponding drop-down list of controller
      values, each preceded by its MIDI control number, re 1.

\begin{figure}[H]
   \centering
%  \includegraphics[scale=0.75]{menu/Instrument/virtual-keyboard-controllers.jpg}
   \includegraphics[scale=0.75]{top-panel/virtual-keyboard-controller.png}
   \caption{Virtual Keyboard Controllers}
   \label{fig:virtual_keyboard_controllers}
\end{figure}

   \setcounter{ItemCounter}{0}      % Reset the ItemCounter for this list.

   \itempar{Mod. Wheel}{controllers!modulation wheel}
   Sets the MIDI modulation value.  This control will
   only have an effect on certain instruments.  (It has no effect on the
   "Simple Sound", for example).

   \itempar{Volume}{controllers!volume}
   Controls the overall volume of the instrument being played by the virtual
   keyboard.

   \itempar{Panning}{controllers!panning}
   Controls the left-right location of the sounds played by the virtual
   keyboard.

   \itempar{Expression}{controllers!expression}
   Controls the expression.  This probably can have different effects depending
   on the instrument.  For example, with the "Simple Sound", this control is a
   lot like volume.

   \itempar{Sustain}{controllers!sustain}
   Controls the sustain duration.  This works even with the "Simple Sound".
   Using it makes even this virtual keyboard capable of some "virtuoso"
   expression.

   \itempar{Portamento}{controllers!portamento}
   Controls the time of transition from one pitch to another.
   Using it makes even this virtual keyboard capable of some "virtuoso"
   expression.

   \itempar{Filter Q}{controllers!filter q}
   Controls the sharpness of the filters used in an instrument.
   Generally requires a complex instrument to take effect.
   For example, try this control with the "Weird Pad" instrument in the
   "Fantasy" bank.

   \itempar{Filter Freq}{controllers!filter frequency}
   Controls the center frequency of the filters used in an instrument.
   Generally requires a complex instrument to take effect.
   For example, try this control with the "Weird Pad" instrument in the
   "Fantasy" bank.

   \itempar{Bandwidth}{controllers!bandwidth}
   Controls the frequency bandwidth of the filters used in an instrument.

   \itempar{FM Gain}{controllers!fm gain}
   TODO.
   Haven't found a sound that exercises this control.
   Haven't looked all that hard yet.

   \itempar{Res. c. freq}{controllers!resonance center frequency}
   Resonance Center Frequency.
   Applies only if the part has resonance set up.

   \itempar{Res. bw}{controllers!resonance bandwidth}
   Resonance Bandwidth.
   Applies only if the part has resonance set up.

%-------------------------------------------------------------------------------
% vim: ts=3 sw=3 et ft=tex
%-------------------------------------------------------------------------------


% Middle Panel Effects User-Interface items

%-------------------------------------------------------------------------------
% yum_effects
%-------------------------------------------------------------------------------
%
% \file        yum_effects.tex
% \library     Documents
% \author      Chris Ahlstrom
% \date        2015-06-05
% \update      2018-05-26
% \version     $Revision$
% \license     $XPC_GPL_LICENSE$
%
%     Provides the effectsxx section of the manual.
%
%-------------------------------------------------------------------------------

\section{Effects}
\label{sec:effects}

   The \textsl{Yoshimi} \textbf{Effects} panel provides a number of special
   effects that can be applied to parts.
   Effects are, generally, black boxes that transform audio signals in a
   specified way. More exactly, the only input data for an effect in
   \textsl{Yoshimi} is an array of samples.
   The output is the transformed array of samples.

   As described, effects have no information about anything else. For
   example, key presses are not recognised. Therefore, pressing a key does
   not initiate the LFO. Phase knobs will always be relative to a global LFO,
   dependent only on the system time.

   \textsl{Wetness} determines the mix of the results of the effect and its
   input.  This mix is made at the effects output. If an effect is wet, it
   means that nothing of the input signal is bypassing the effect. If it is
   dry, then the effect has no effect.

   \textsl{Interpolation}
   means that, if one MIDI-learns the controls, one can now automate them
   smoothly instead of the somewhat gritty previous behaviour.
   This does not change processor demand when running at 64 frames, a short
   number of frames.
   This interpolation is especially effective on "saw" sounds with the
   frequency control on the EQ low pass filter.


   The \textbf{Effects} panel is shown in
   \figureref{fig:yoshimi_main_screen}.
   Note that these effects have been incorporated into a separate
   guitar-effects project called \textsl{Rakkarrak} \cite{rakarrack}.

   There are two types of effects: System effects and Insertion effects.
   Insertion effects have a sub-type for part effects. The effects themselves
   behave in the same way but with slightly different 'outer' controls.

   The System effects apply to all parts and allows one to set the amount of
   effect that applies to each part. Also, it is possible to send the output
   of one system effect to another system effect. In the user interface this
   is shown as "source -\textless destination". For example:
   The \textbf{0 -\textless 1} knob controls how
   much of the system effect 0 is sent to system effect 1.

   Insertion effects are described in
   \sectionref{subsubsec:effects_paneltypes_insertion}.

\subsection{Effects / Panel Types}
\label{subsec:effects_paneltypes}

   There are three variations of Effects sub-panels:

   \begin{itemize}
      \item \textbf{System Effects}.
      \item \textbf{Insertion Effects}.
      \item \textbf{Part/Instrument Effects}.
   \end{itemize}

   Here are the major elements of the main effects panel, which shows the
   System and Insertion effects tabs.

\begin{figure}[H]
   \centering
   \includegraphics[scale=0.5]{2.3.0/system.png}
   \caption{System Effects Dialog}
   \label{fig:system_effects_dialog}
\end{figure}

   \begin{enumber}
      \item \textbf{System Effects Tab}
      \item \textbf{Effect Number}
      \item \textbf{Effect Name}
      \item \textbf{On}
      \item \textbf{Send to}
      \item \textbf{C}
      \item \textbf{P}
      \item \textbf{Effects Panel}
      \item \textbf{Insertion Effects Tab}
      \item \textbf{Reports}
   \end{enumber}

   \setcounter{ItemCounter}{0}      % Reset the ItemCounter for this list.

   \itempar{System Effects Tab}{effects!system tab}
   System Effects Tab.
   The items in this tab are described in the next few paragraphs.

   \itempar{Effect Number}{effects!number}
   Effect Number.
   Up to 8 effects can be supported at one time by one part.

   \itempar{Effect Name}{effects!name}
   Effect Name.

   Values: \texttt{No Effect*, Reverb, Echo, Chorus, Phaser, AlienWah,
      Distortion, EQ, DynFilter}

\begin{figure}[H]
   \centering
   \includegraphics[scale=1.0]{effects-panel/system-effects-selections.jpg}
   \caption{Effects Names}
   \label{fig:effects_names}
\end{figure}

\itempar{Effect On}{effects!on-off}
   Effect enable checkbox. Used to temporarily disable a particular system
   effect without altering the settings. New in \textsl{Yoshimi} V1.6.0

   Values: \texttt{On*, Off}

   \itempar{Send to}{effects!send to}
   Effects Send To.
   Each knob controls how much of the system effect indicated by the left
   number is sent to the system effect indicated by the right number.

   Values: \texttt{Next Effect, Part Out, Dry Out}

\begin{figure}[H]
   \centering
   \includegraphics[scale=0.75]{2.3.0/syseffsend.png}
   \caption{Effects, Send To}
   \label{fig:effects_send_to}
\end{figure}

   \itempar{C}{effects!copy dialog}
   Copy-to-clipboard Dialog.

\begin{figure}[H]
   \centering
   \includegraphics[scale=0.75]{2.3.0/copy.png}
   \caption{Effects / Copy To Clipboard}
   \label{fig:effects_copy_to_clipboard}
\end{figure}

   \itempar{P}{effects!paste dialog}
   Paste-from-clipboard Dialog.

\begin{figure}[H]
   \centering
   \includegraphics[scale=0.75]{2.3.0/paste.png}
   \caption{Effects / Paste From Clipboard}
   \label{fig:effects_paste_from_clipboard}
\end{figure}

   \itempar{Effects Panel}{effects!panel}
   Effects Panel.
   This area is filled by the controls for the selected effect.

   \itempar{Insertion Effects Tab}{effects!insertion tab}
   Insertion Effects Tab.
   The items in this tab are described below,
   in the \ref{subsubsec:effects_paneltypes_insertion}
   sub-section.

   \iffalse
   VVVV How did this get here? It's quite out of place!
   \itempar{Reports}{effects!reports}
   Effects Reports.

\begin{figure}[H]
   \centering
   \includegraphics[scale=1.0]{effects-panel/reports.jpg}
   \caption{Effects / Reports}
   \label{fig:effects_reports}
\end{figure}
\fi

   The next sub-sections show the variations on the effects panels, using the
   DynFilter effect as the subject effects panel.

\subsubsection{Effects / Panel Types / System }
\label{subsubsec:effects_paneltypes_system}

   The first variation
   appears when one enables an effect in the
   \textbf{System Effects}
   panel of the main \textsl{Yoshimi} dialog.  It contains the standard
   controls for the given effect, plus the following interface items
   (as previously described).

\begin{figure}[H]
   \centering
%   \includegraphics[scale=1.0]{2.1.0/system_effect_example.png}
   \includegraphics[scale=0.5]{2.3.0/system.png}
   \caption{Sample System Effects Dialog}
   \label{fig:sample_system_effects_dialog}
\end{figure}

   \begin{enumber}
      \item \textbf{Effect number}
      \item \textbf{Effect name}
      \item \textbf{On}
      \item \textbf{Send To}
      \item \textbf{C}
      \item \textbf{P}
   \end{enumber}

\subsubsection{Effects / Panel Types / Insertion }
\label{subsubsec:effects_paneltypes_insertion}

   The second effects variation
   appears when one enables an effect in the
   \textbf{Insertion Effects}
   panel of the main \textsl{Yoshimi} dialog.
   It contains the same standard controls for the given effect, but with
   a different header containing the following interface items.

\begin{figure}[H]
   \centering
%   \includegraphics[scale=1.0]{2.1.0/insertion_effect_example.png}
   \includegraphics[scale=0.5]{2.3.0/insertion.png}
   \caption{Insertion Effects Dialog}
   \label{fig:insertion_effects_dialog}
\end{figure}

   \begin{enumber}
      \item \textbf{Effect number}
      \item \textbf{Effect selection}
      \item \textbf{To}
      \item \textbf{C}
      \item \textbf{P}
   \end{enumber}

   The insertion effects apply to one part or to the master output.
   One may use more    than one insertion effect for one part or the
   master output. If using more than one effect, the effects with smaller
   indexes will be applied first (first, insertion effect 1 occurs, then
   effect 2, and so on).

   \setcounter{ItemCounter}{0}     % Reset the ItemCounter for this list.

   \itempar{To}{effects!to}
   Send the Effect To.

\begin{figure}[H]
   \centering
   \includegraphics[scale=1.0]{bottom-panel/instrument-edit/Effects/part-selection-dropdown.png}
   \caption{Part Selection Dropdown}
   \label{fig:sample_part_selection_dropdown}
\end{figure}
   The first option sends the effect to the main L/R outputs rather than
   just one part.
   \index{insertion effect!master out}
   Note that if \textsl{Yoshimi} is set for 32 or 64 parts, the dropdown list
   will be extended to include them.

\subsubsection{Effects / Panel Types / Instrument }
\label{subsubsec:effects_paneltypes_instrument}

   There is also a "part" or "instrument" effects window which is accessed
   by going to the main window, clicking the \textbf{Edit} button in the
   bottom panel to open the edit dialog, and then clicking the
   \textbf{Effects} button there.  The part effects window has the
   same layout as System and Insertion effects; it is now almost identical
   to Insertion effects.

   It contains the standard controls for the given effect, plus the
   following interface items.

   \begin{enumber}
      \item \textbf{Effect number}
      \item \textbf{Effect selection}
      \item \textbf{To} (part-selection-dropdown.png)
      \item \textbf{C}
      \item \textbf{P}
      \item \textbf{Bypass}
      \item \textbf{Close}
   \end{enumber}

   "To" Values: \texttt{Next Effect* , Part Out, Dry Out}\\
   \\
   The default is to pass each effect (combined with the incoming signal) on to the next,
   forming a daisy chain of effects.\\
   If it is set to Part Out, it breaks the chain and blocks any higher numbered effects.\\
   If it is set to Dry Out, it sends the incoming signal component directly where part
   effects are added together, so it won't be passed on to later effects. However, the
   output of the effect itself \textsl{is} passed on to later ones. Consequently, if this
   effect is bypassed it also effectively bypasses all later ones.

\begin{figure}[H]
   \centering
   \includegraphics[scale=0.75]{2.3.0/part_effect.png}
   \caption{Sample Instrument Effects Dialog}
   \label{fig:sample_instrument_effects_dialog}
\end{figure}

   \index{effects!bypass}
   Note the extra \textbf{Bypass} check-box.  If the \textbf{Bypass} item is
   checked, then the effect is not used; it is taken out of the circuit.  This
   user-interface item only appears if one clicks the \textbf{Edit} button for
   a Part, and then clicks the \textbf{Effects} button in the \textbf{Edit}
   window.

   Also be aware that the layout of some of the effects dialogs have been modified
   in the latest revisions of \textsl{Yoshimi}.
   This dialog form is reversed (top and bottom) compared with very old
   \textsl{Yoshimi} versions, and slightly simplified in appearance. This was done to
   more closely match the layout of the System and Insertion Effects.
   Also, in the actual effects, some control positions and sizes have been changed
   to improve readability.

   From here on we only show the effects inserts themselves, as they are identical
   across Part, Insertion, and System. The only difference being \textbf{D/W} becoming
   \textbf{Vol} in System effects.

\subsection{Effects / Upgrade}
\label{subsec:effects_upgrade}
   Since V 1.5.11 there is an indication that effect controls have been altered. As can
   be seen, the normal pale cyan background of the preset selector becomes a strong blue.
   This colour change will also apply to any effects that have been saved.
   \begin{figure}[H]
   \centering
   \includegraphics[scale=0.75]{2.3.0/effects_warning.png}
%   \caption{Effects Edit, No Effect}
   \label{fig:effects_warning}
\end{figure}

   This change will take place if one alters any of the controls after a preset has been
   selected. The rationale here is that one can make such changes, then save the
   Instrument/Patchset that this effect is in.
   When re-loading, one would be quite likely to forget that changes have been made and
   experimentally switch to different presets or even different effects.
   Previously, at this point those changes have been lost and one might be puzzled that
   the sound has changed (possibly quite subtly) when returning to the same preset.

   With the new upgrade one is warned about this. It even applies when loading very old
   Instruments and Patchsets as the effects are checked against the known defaults as
   they are installed.

\subsection{Effects / None}
\label{subsec:effects_edit_none}

\begin{figure}[H]
   \centering
   \includegraphics[scale=0.75]{2.3.0/no_effect.png}
   \caption{Effects Edit, No Effect}
   \label{fig:effects_edit_none}
\end{figure}

\subsection{Effects / DynFilter}
\label{subsec:effects_edit_dynfilter}

   A dynamic filter is, as the name says, a filter which changes its
   parameters dynamically, dependent on the input and current time. In
   \textsl{Yoshimi}, frequency is the only variable parameter. It can be
   used as an "envelope following filter" (sometimes referenced "Auto Wah" or
   simply "envelope filter").

\subsubsection{Effects / DynFilter / Circuit}
\label{subsubsec:effects_edit_dynfilter_circuit}

   Though this filter might look a bit complicated, it is actually easy. We
   divide the parameters into two classes:

   \begin{itemize}
      \item Filter Parameters are the ones obtained when one clicks on Filter.
         They give the filter its basic settings.
      \item Effect Parameters are the other ones that control how the filter
         changes.
   \end{itemize}

   The filter basically works like this: The input signal is passed through a
   filter which dynamically changes its frequency. The frequency is an
   additive of:

   \begin{itemize}
      \item The filter’s base frequency.
      \item An LFO from the effect parameters.
      \item The "amplitude" of the input wave.
   \end{itemize}

   The amplitude of the input wave is not the current amplitude, but the so
   called "Root Mean Square (RMS)" value. This means that we build a mean on
   the current amplitude and the past values. How much the new amplitude
   takes influence is determined by the Amplitude Smoothness (see below).

   RMS value plays an important role in the term \textsl{loudness}.
   A fully distorted signal can sound 20dB louder due to its higher RMS value.
   This filter takes this into account, depending on the smoothness.

\begin{figure}[H]
   \centering
   \includegraphics[scale=0.25]{zyn/effects/dynamic.png}
   \caption{Dynamic Filter Circuit Diagram}
   \label{fig:dynfilter_circuit_diagram}
\end{figure}

\subsubsection{Effects / DynFilter / User Interface}
\label{subsubsec:effects_edit_dynfilter_ui}

\begin{figure}[H]
   \centering
%  \includegraphics[scale=1.0]{bottom-panel/instrument-edit/Effects/effects-edit-dynfilter.jpg}
   \includegraphics[scale=0.75]{2.3.0/dynfilter.png}
   \caption{Effects Edit, DynFilter}
   \label{fig:effects_edit_dynfilter}
\end{figure}

   This figure shows the Part/Instrument variation of the DynFilter sub-panel.
   The System/Insertion variation has the following elements.

   \begin{enumber}
      \item \textbf{Preset}
      \item \textbf{Filter}
      \item \textbf{Vol} (system/insertion) or \textbf{D/W} (part/instrument)
      \item \textbf{Pan}
      \item \textbf{Freq}
      \item \textbf{Rnd}
      \item \textbf{BPM}
      \item \textbf{LFO Type}
      \item \textbf{St.df}
      \item \textbf{LfoD}
      \item \textbf{A.S.}
      \item \textbf{A.M.}
      \item \textbf{Inv.}
   \end{enumber}

   The five controls in the middle of the middle panel
   (\textbf{Freq}, \textbf{Rnd}, \textbf{BPM}, \textbf{LFO Type}, and \textbf{St.df})
   control the LFO.

   BPM is new from \textsl{Yoshimi} V 2.1.0 and locks the frequency control to the
   incoming MIDI clock. See \sectionref{subsubsec:bpm_and_frequency}\ for further details.

   In \textbf{DynFilter}, the \textbf{Gain} control, and most of the formant
   filter ones only operate as one \textsl{releases} the mouse button, and the
   scroll wheel cannot be used at all.  Investigating, it was found that this
   was specifically done because these controls create significant noise when
   adjusted, and effects are real time, so that's a lot of noise. (When filters
   are applied elsewhere, the result is next-note so one doesn't hear the
   changes).

   Which is more desirable:
   (1) Noise when ever the control is moved, scroll wheel capability and fully
   responsive GUI;
   (2) Noise only when the control is released, no scroll wheel, filter graphs
   only updated on control release.
   To be determined.

   Let's start with the user-interface elements present in the
   System/Insertion variation of this effect.

   \setcounter{ItemCounter}{0}      % Reset the ItemCounter for this list.

   \itempar{Preset}{dynfilter!preset}
   DynFilter Preset.

\begin{figure}[H]
   \centering
   \includegraphics[scale=1.0]{2.3.0/dynfilter_presets.png}
   \caption{DynFilter Presets}
   \label{fig:effects_dynfilter_presets}
\end{figure}

   Values: \texttt{WahWah, AutoWah, Sweep, VocalMorph1, VocalMorph2}

   \itempar{Filter}{dynfilter!filter}
   DynFilter Filter.

   This small button brings up Filter Params stock sub-panel item.
   This stock user-interface item is shown and described in
   \sectionref{subsubsec:filter_parameters_user_interface}.

   \itempar{Vol}{dynfilter!volume}
   DynFilter Volume.

   Values: \textsl{0 to 127}

   If the effect is used as a System effect, then this control appears.

   \itempar{D/W}{dynfilter!dry/wet}
   DynFilter Dry/Wet Mix Setting.

   Values: \textsl{0 to 127}

   If the effect is used as an Insertion effect, then this control appears.
   "Dry" means the unprocessed signal and "wet" means the processed signal.

   \itempar{Pan}{dynfilter!pan}
   DynFilter Panning.

   Values: \textsl{0 to 127}

   After the input signal has passed through the filter, Pan can apply
   panning.

   \itempar{Freq}{dynfilter!lfo freq}
   DynFilter LFO Frequency.

   Values: \textsl{0 to 127}

   \itempar{Rnd}{dynfilter!lfo randomness}
   DynFilter LFO Randomness.

   Values: \textsl{0 to 127}

   \itempar{LFO Type}{dynfilter!lfo type}
   DynFilter LFO Type.

   \itempar{St.df}{dynfilter!lfo stdf}
   DynFilter LFO.
   Left/right channel phase shift.

   \itempar{LfoD}{dynfilter!lfo depth}
   DynFilter LFO Depth.
   This control is one that helps define the mix of the LFO and the
   amplitude.

   \itempar{A.S}{dynfilter!a.s.}
   DynFilter A.S.
   This control is one that helps define the mix of the LFO and the
   amplitude.
   A.S sets the Amplitude Sensing (i.e. how much influence the amplitude
   shall have).

   \itempar{A.M}{dynfilter!a.m.}
   DynFilter A.M.
   Changes the rate at which the amplitude changes the filter. The higher
   one sets this value, the more slowly will the filter react.

   \itempar{Inv.}{dynfilter!a.inv}
   DynFilter A.Inv.
   If set, negates the (absolute) RMS value. This will lower the filter
   frequency instead of increasing it. Note that this will not have much
   effect if the effects input is not very loud.

\subsubsection{Effects / DynFilter / NRPN Values}
\label{subsubsec:effects_edit_dynfilter_nrpn}

   Effects can be controlled via "non-registered parameter numbers", or NRPNs.
   This section will eventually (we hope)
   detail the NRPN values supported by the DynFilter effect.

   For more information on the concept of NRPNs, see
   \sectionref{subsubsec:concepts_midi_nrpn}.

% ----------------------------------------------------------

\subsection{Effects / AlienWah}
\label{subsec:effects_edit_alienwah}

   AlienWah is a nice effect done by Paul Nasca. It resembles a vocal morpher
   or wahwah a bit, but it is more strange. That's why he called it "AlienWah"
   The effect is a feedback delay with complex numbers.

   The AlienWah effect is a special, dynamic formant filter.
   Paul Nasca named it AlienWah because it sounded "a bit like
   wahwah, but more strange". The result of the filter is a sound varying
   between the vocals "Ahhhhh" (or "Uhhhhh") and "Eeeeee".

\subsubsection{Effects / AlienWah / Circuit}
\label{subsubsec:effects_edit_alienwah_circuit}

   No diagram, just a description of AlienWah.

   Hint: Keep in mind that Effects that can be controlled by LFO can also be
   controlled arbitrarily: Set the LFO depth to zero and manipulate the phase
   knob (e.g. with NRPNs or maybe via OSC in the future).

   The way that the filter moves between the two vocals is mainly described
   by an LFO. Paul Nasca has stated the - slightly simplified - formula (for
   i2 = -1 and R \textless 1) as:

   \[fb=R*(cos(a)+i*sin(a))\]

   \[yn=yn-delay*R*(cos(a)+i*sin(a))+xn*(1-R).\]

   The input xn has the real part of the samples from the wavefile and the
   imaginary part is zero. The output of this effect is the real part of
   \texttt{yn}.
   \texttt{a} is the phase.

\subsubsection{Effects / AlienWah / User Interface}
\label{subsubsec:effects_edit_alienwah_ui}

\begin{figure}[H]
   \centering
%  \includegraphics[scale=1.0]{bottom-panel/instrument-edit/Effects/effects-edit-alienwah.jpg}
   \includegraphics[scale=0.75]{2.3.0/alienwah.png}
   \caption{Effects Edit, AlienWah}
   \label{fig:effects_edit_alienwah}
\end{figure}

   \begin{enumber}
      \item \textbf{Preset}
      \item \textbf{Phase}
      \item \textbf{Vol} or \textbf{D/W}
      \item \textbf{Pan}
      \item \textbf{Freq}
      \item \textbf{Rnd}
      \item \textbf{BPM}
      \item \textbf{LFO type}
      \item \textbf{St.df.}
      \item \textbf{Dpth}
      \item \textbf{Fb.}
      \item \textbf{Delay}
      \item \textbf{L/R}
   \end{enumber}

   \setcounter{ItemCounter}{0}      % Reset the ItemCounter for this list.

   \itempar{Preset}{alienwah!preset}
   AlienWah Preset.

   Values: \texttt{AlienWah 1, AlienWah 2, AlienWah 3, AlienWah 4}

   \itempar{Phase}{alienwah!phase}
   The phase of the AlienWah.
   See \texttt{a} in the above formula.
   This lets one set where the vocal is between
   "Ahhhhh" and "Eeeeee".

   \itempar{Vol}{alienwah!volume}
   AlienWah Volume.

   Values: \texttt{0 to 127}

   The volume control is present in this effect when used as a System
   effect.

   \itempar{D/W}{alienwah!dry/wet}
   AlienWah Dry/Wet.

   Values: \texttt{0 to 127}

   The \textbf{Vol} control is replaced by this control if the effect is
   used as an Insertion effect.

   \itempar{Freq}{alienwah!lfo frequency}
   LFO Frequency.

   Values: \texttt{0 to 127}

   Determines the LFO’s frequency in relative units.

   \itempar{Rnd}{alienwah!lfo randomness}
   LFO Amplitude Randomness.

   Values: \texttt{0 to 127}

   Part of the LFO definition.

   \itempar{BPM}{alienwah!lfo sync}
   Locks the frequency control to the incoming MIDI clock. New from
   \textsl{Yoshimi} V 2.1.0 See
   \sectionref{subsubsec:bpm_and_frequency}\ for further details.

   \itempar{LFO type}{alienwah!lfo shape}
   Set the LFO shape.

   Values: \texttt{SINE, TRI}

   Part of the LFO definition.
   Note that the LFO in other contexts has ramps and exponential shapes that
   are not present here.

   \itempar{St.df}{alienwah!phase diff}
   AlienWah Left/Right Chanell Phase Difference.

   Values: \texttt{0 to 127}

   Part of the LFO definition.
   Sets the phase difference between LFO for left/right channels.
   \textbf{St.df} lets one determine how much left and right LFO are phase
   shifted.  64.0 means stereo, higher values increase the right LFO
   relatively to the left one.

   \itempar{Dpth}{alienwah!depth}
   LFO depth.

   Values: \texttt{0 to 127}

   \textbf{Dpth} is a multiplier to the LFO. Thus, it determines
   the LFO's amplitude and its influence.

   \itempar{Delay}{alienwah!delay}
   Amount of delay before the feedback.

   Values: \texttt{1 to 100}

   If this value is low, the sound becomes more of a "wah-wah" effect.

   \itempar{Fb}{alienwah!feedback}
   AlienWah Feedback.

   Values: \texttt{0 to 127}

   \index{todo!alienwah feedback}
   TODO: What is the effect of the AlienWah feedback setting?

   \itempar{L/R}{alienwah!l/r}
   Determines how the left/right channels are routed to output:

   \begin{itemize}
      \item \textsl{Leftmost/0}. Left to left and right to right.
      \item \textsl{Middle/64}. Left+right to mono.
      \item \textsl{Rightmost/127}. Left to right, and right to left.
   \end{itemize}

   L/R applies crossover at the end of every stage. This is currently not
   implemented for the Analog Phaser.

   \itempar{Subtract}{alienwah!subtract}
   The output is inverted

\subsubsection{Effects / AlienWah / NRPN Values}
\label{subsubsec:effects_edit_alienwah_nrpn}

Effects can be controlled via "non-registered parameter numbers", or NRPNs.
This section details the value supported by the AlienWah effect.

\subsection{Effects / Chorus}
\label{subsec:effects_edit_chorus}

   In a chorus, many people sing together. Even if each of them sings at
   exactly the same frequency, all their voices usually sound different. We
   say they have a different timbre. Timbre is the way we perceive sound and
   makes us differ between different music instruments. This is, physically,
   achieved by varying both the amplitude envelope and the frequency
   spectrum. Multiple sounds with slightly different timbres make a sound
   more shimmering, or powerful. This is called the chorus effect.

   The chorus effect can be achieved by multiple people singing together. In
   a concert, there are many instruments, resulting in the same effect. When
   making electronic music, we only have an input wave and need to generate
   these different timbres ourselves.
   \textsl{Yoshimi} therefore simply plays
   the sound, pitch modulated by an LFO, and adds this to the original sound.
   This explains the diagram below: The multiple pitches are generated by a
   delayed version of the input. This version is being pitched by an LFO.
   More detailed, this pitch is generated by varying the reading speed of
   the delayed sound; the variation amount is controlled by an LFO.

   Related effects to Chorus are Flangers. Flangers can be described as
   Chorus with very short LFO delay and little LFO depth. One can imagine a
   flanger as two copies of a sound playing at almost the same time. This
   leads to interference, which can be clearly heard. It is popular to apply
   flangers to guitars, giving them more "character".

\subsubsection{Effects / Chorus / Circuit}
\label{subsubsec:effects_edit_chorus_circuit}

\begin{figure}[H]
   \centering
   \includegraphics[scale=0.25]{zyn/effects/chorus.png}
   \caption{Chorus Circuit Diagram}
   \label{fig:chorus_circuit_diagram}
\end{figure}

   First, crossover is applied.
   The \textbf{Freq}, \textbf{Rnd}, \textbf{LFO Type}, \textbf{St.df},
   \textbf{Depth} knobs control the LFO
   for the pitch. If the depth is set to zero, the pitch will not be changed
   at all.

   Delay is the time that the delayed sound is delayed "on average". Note
   that the delay also depends on the current pitch.

   After the correct element of the sound buffer is found using the LFO, the
   Fb knob lets one set how loud it shall be played. This is mostly redundant
   to the D/W knob, but we have not applied panning and subtraction yet.

   Next, the signal can be negated. If the \textbf{Subtract}
   checkbox is activated, the amplitude is multiplied by -1.

   Finally, \textbf{Pan} lets one apply panning.

\subsubsection{Effects / Chorus / User Interface}
\label{subsubsec:effects_edit_chorus_ui}

\begin{figure}[H]
   \centering
%  \includegraphics[scale=1.0]{bottom-panel/instrument-edit/Effects/effects-edit-chorus.jpg}
   \includegraphics[scale=0.75]{2.3.0/chorus.png}
   \caption{Effects Edit, Chorus}
   \label{fig:effects_edit_chorus}
\end{figure}

   \begin{enumber}
      \item \textbf{Freq}
      \item \textbf{Rnd}
      \item \textbf{BPM}
      \item \textbf{LFO type}
      \item \textbf{St.df.}
      \item \textbf{Dpth}
      \item \textbf{Delay}
      \item \textbf{Fb.}
      \item \textbf{L/R}
      \item \textbf{Subtract}
   \end{enumber}

   \setcounter{ItemCounter}{0}      % Reset the ItemCounter for this list.

   \itempar{Freq}{chorus!lfo freq}
   Chorus LFO Frequency.

   \itempar{Rnd}{chorus!lfo randomness}
   Chorus LFO randomness.

   \itempar{LFO type}{chorus!lfo type}
   Set the LFO shape.

   \itempar{BPM}{chorus!lfo sync}
   Locks the frequency control to the incoming MIDI clock. New from
   \textsl{Yoshimi} V 2.1.0 See
   \sectionref{subsubsec:bpm_and_frequency}\ for further details.

   \itempar{St.df}{chorus!l/r phase}
   The phase difference between LFO for left/right channels .

   \itempar{Dpth}{chorus!lfo depth}
   Chorus LFO depth.

   \itempar{Delay}{chorus!delay}
   Delay of the chorus.
   If one uses low delays and LFO depths, this will result in a flanger
   effect.

   \itempar{Fb}{chorus!feedback}
   Chorus Feedback.

   \itempar{L/R}{chorus!l/r routing}
   How the left/right channels are routed to output:

      \begin{enumber}
         \item leftmost. Left to left and right to right.
         \item middle. Left+right to mono.
         \item rightmost. Left to right, and right to left.
      \end{enumber}

   \itempar{Subtract}{chorus!subtract}
   The Chorus output is inverted

\subsubsection{Effects / Chorus / NRPN Values}
\label{subsubsec:effects_edit_chorus_nrpn}

Effects can be controlled via "non-registered parameter numbers", or NRPNs.
This section details the value supported by the Chorus effect.

\subsection{Effects / Distortion}
\label{subsec:effects_edit_distortion}

   Distortion means, in general, altering a signal. Natural instruments
   usually produce sine-like waves. A wave is transformed in an unnatural way
   when distortion is used. The most distorted waves are usually pulse waves.
   It is typical for distortion to add overtones to a sound. Distortion often
   increases the power and the loudness of a signal, while the dB level is
   not increased. This is an important topic in the Loudness War.

   As distortion increases loudness, distorted music can cause ear damage at
   lower volume levels. Thus, one might want to use it carefully.
   Distortion can happen in many situations when working with audio. Often,
   this is not wanted. In classical music, for example, distortion does not
   occur naturally. However, distortion can also be a wanted effect. It is
   typical for Rock guitars, but also present in electronic music, mostly in
   Dubstep and DrumNBass.

   The basic components of distortion are mainly

   \begin{itemize}
      \item A preamplifier.
      \item The waveshaping function.
      \item Filters.
   \end{itemize}

   Preamplification changes the volume before the wave is shaped, and is
   indeed the amount of distortion. For example, if one clips a signal, the
   louder the input gets, the more distortion one will get. This can have
   different meanings for different types of distortions, as described below.

   The filters are practical. A reason for using them afterwards is that
   distortion can lead to waves with undesired high frequency parts. Those
   can be filtered out using the LPF. A reason for using filters before
   applying is to achieve multiband distortion.

   The topic of types of distortion is discussed in the
   Oscillator Section.

   Note that one can use the Oscillator editor in order
   to find out what the distortion effect does. Also note that while the
   Oscillator editor’s distortion is limited to some oscillators one can
   produce in the Oscillator editor, the distortion effect can be used on
   every wave that one can generate with \textsl{Yoshimi}.

\subsubsection{Effects / Distortion / Circuit}
\label{subsubsec:effects_edit_distortion_circuit}

   We explain the functionality in a diagram and list the components below.

\begin{figure}[H]
   \centering
   \includegraphics[scale=0.333]{zyn/effects/distort.png}
   \caption{Distortion Circuit Diagram}
   \label{fig:distortion_circuit_diagram}
\end{figure}

   Negation is the first thing to happen. If the \textbf{Neg Checkbox} is
   activated, the amplitude is multiplied by -1.

   \index{distortion effect!panning}
   Panning is applied. Note, however, that one must activate the Stereo
   Checkbox, labelled \textbf{St}, before it will work.

   Pre-amplification is done next. The amount can be changed using the Drive
   nob. Indeed, this is the amount of distortion. For example, if one clips a
   signal, the louder the input gets, the more distortion one will get. This
   can have different meanings for different types of distortion, as
   described above.

   HPF and LPF are filters with 2 poles. Whether they are used before or
   after the waveshape, depends on the checkbox labeled PF.

   The next step is the wave shape. This defines how the wave is actually
   modified. The Type ComboBox lets one define how. We will discuss some
   types below.

   After the wave shape, we scale the level again. This is called output
   amplification. One can change the value using the Level knob.

   Crossover is the last step. This is controlled by the knob LR Mix and
   means that afterwards, a percentage of the left side is applied to the
   right side, and, synchronously, the other way round. It is a kind of
   interpolation between left and right. If one sets the LR Mix to 0.0, one
   will always have a stereo output.

\subsubsection{Effects / Distortion / User Interface}
\label{subsubsec:effects_edit_distortion_ui}

\begin{figure}[H]
   \centering
%  \includegraphics[scale=1.0]{bottom-panel/instrument-edit/Effects/effects-edit-distortion.jpg}
   \includegraphics[scale=0.75]{2.3.0/distortion.png}
   \caption{Effects Edit, Distortion}
   \label{fig:effects_edit_distortion}
\end{figure}

   \begin{enumber}
      \item \textbf{Drive}
      \item \textbf{Level}
      \item \textbf{Type}
      \item \textbf{Neg.}
      \item \textbf{LPF}
      \item \textbf{HPF}
      \item \textbf{St.}
   \end{enumber}

   \setcounter{ItemCounter}{0}      % Reset the ItemCounter for this list.

   \itempar{Drive}{distortion!drive}
   Set the amount of distortion.

   \itempar{Level}{distortion!level}
   Amplify or reduce the signal after distortion.

   \itempar{Type}{distortion!type}
   Set the function of the distortion (like arctangent, sine).

   \itempar{Neg}{distortion!negate}
   Negates the amplitude (invert the signal).

   \itempar{LPF}{distortion!lpf}
   Low Pass Filter.

   \itempar{HPF}{distortion!hpf}
   High Pass Filter.

   \itempar{St}{distortion!stereo}
   Set the distortion mode (stereo or mono, checked is stereo).

\subsubsection{Effects / Distortion / NRPN Values}
\label{subsubsec:effects_edit_distortion_nrpn}

   Effects can be controlled via "non-registered parameter numbers", or NRPNs.
   This section details the value supported by the Distortion effect.

\subsection{Effects / Echo}
\label{subsec:effects_edit_echo}

   The echo effect, also known as delay effect, simulates the natural
   reflection of a sound. The listener can hear the sound multiple times,
   usually decreasing in volume. Echos can be useful to fill empty parts of
   songs with.

\subsubsection{Effects / Echo / Circuit}
\label{subsubsec:effects_edit_echo_circuit}

   The good circuit diagram is shown in an old printout we have, but the
   current version of the Echo description at
   http://zynaddsubfx.sourceforge.net/Doc/ shows a
   junk file.  So Paul Nasca's description will have to suffice.

% \begin{figure}[H]
%    \centering
%    \includegraphics[scale=0.25]{zyn/effects/echo.png}
%    \caption{Echo Circuit Diagram}
%    \label{fig:echo_circuit_diagram}
% \end{figure}

   The echo is basically implemented as the addition of the
   current sound and a delayed version of it. The delay is implemented as in
   the picture below. First, we add the delayed signal to the effect input.
   Then, they pass an LP1. This shall simulate the effect of dampening, which
   means that low and especially high frequencies get lost earlier over
   distance than middle frequencies do. Next, the sound is delayed, and then
   it will be output and added to the input.

   The exact formula in the source code for the dampening effect is as
   follows:

   \[Y(t)=(1-d)*X(t)+d*Y(t-1)\]

   where t be the time index for the input buffer, d be the dampening amount
   and X,Y be the input, respective the output of the dampening. This solves
   to

   \[Y(z)=Z(Y(t))=(1-d)*X(z)+d*Y(z)*z-1 <==> H(z)=Y(z)X(z)=1-d1-d*z-1\]

   which is used in \(Y(z)=H(z)*X(z)\). So H(z) is indeed a filter, and by
   looking at it, we see that it is an LP1. Note that infinite looping for
   d=1 is impossible.

\subsubsection{Effects / Echo / User Interface}
\label{subsubsec:effects_edit_echo_ui}

\begin{figure}[H]
   \centering
%  \includegraphics[scale=1.0]{bottom-panel/instrument-edit/Effects/effects-edit-echo.jpg}
   \includegraphics[scale=0.75]{2.3.0/echo.png}
   \caption{Effects Edit, Echo}
   \label{fig:effects_edit_echo}
\end{figure}

   TODO (yoshimi):  Pan lets one apply panning of the input.

   \begin{enumber}
      \item \textbf{Delay}
      \item \textbf{LRdl.}
      \item \textbf{LRc.}
      \item \textbf{Fb.}
      \item \textbf{Damp}
      \item \textbf{BPM}
   \end{enumber}

   \setcounter{ItemCounter}{0}      % Reset the ItemCounter for this list.

   \itempar {Delay}{echo!delay}
   The delay time of one echo.

   \itempar {LRdl}{echo!l/r delay}
   Left-Right-Delay.
   The delay between left/right channels.
   If it is set to the middle, then both sides are delayed equally. If
   not, then the left echo comes earlier and the right echo comes (the
   same amount) later than the average echo; or the other way round.
   Set the knob to 0 to hear on the right first.

   \itempar {LRc}{echo!crossover}
   Echo Crossover.
   The "crossing" between left/right channels.

   \itempar {Fb}{echo!feedback}
   Echo feedback.
   Feedback describes how much of the delay is added back to the input.
   Set Fb. to the maximum to hear an infinite echo, or to the minimum to
   just hear a single repeat.

   \itempar {Damp}{echo!damp}
   Echo damping.
   How high frequencies are damped in the Echo effect.
   The Damp value lets the LP1 reject higher frequencies earlier if
   increased.

   \itempar{BPM}{echo!lfo sync}
   Locks the delay time to the incoming MIDI clock. New from
   \textsl{Yoshimi} V 2.1.0 See
   \sectionref{subsubsec:bpm_and_frequency}\ for further details.

\subsubsection{Effects / Echo / NRPN Values}
\label{subsubsec:effects_edit_echo_nrpn}

   Effects can be controlled via "non-registered parameter numbers", or NRPNs.
   This section details the value supported by the Echo effect.  TODO.

\subsection{Effects / EQ}
\label{subsec:effects_edit_eq}

   EQ is a parametric equaliser.
   An equaliser is a filter effect that applies different volume to different
   frequencies of the input signal. This can, for example, be used to "filter
   out" unwanted frequencies. \textsl{Yoshimi}’s implementations follow the
   "Cookbook formulae for audio EQ" (\cite{cookbookeq})
   by Robert Bristow-Johnson.

   On the equaliser graph there are 3 white
   vertical bars for 100Hz, 1kHz, 10kHz.

\subsubsection{Effects / EQ / Circuit}
\label{subsubsec:effects_edit_eq_circuit}

\iffalse
 No such figure:

 \begin{figure}[H]
    \centering
    \includegraphics[scale=0.25]{zyn/effects/eq.png}
    \caption{EQ Circuit Diagram}
    \label{fig:eq_circuit_diagram}
 \end{figure}
\fi

\subsubsection{Effects / EQ / User Interface}
\label{subsubsec:effects_edit_eq_ui}

\begin{figure}[H]
   \centering
%  \includegraphics[scale=1.0]{bottom-panel/instrument-edit/Effects/effects-edit-eq.jpg}
   \includegraphics[scale=0.75]{2.3.0/eq.png}
   \caption{Effects Edit, EQ}
   \label{fig:effects_edit_eq}
\end{figure}

   We describe all parts of the GUI here. The term passband (or often just
   "band") refers to the amount of frequencies which are not
   significantly attenuated by the filter.

   \begin{enumber}
      \item \textbf{Gain}
      \item \textbf{Graph}
      \item \textbf{Band}
      \item \textbf{Type}
      \item \textbf{Freq}
      \item \textbf{Gain}
      \item \textbf{Q}
      \item \textbf{Stages}
   \end{enumber}

   Global:

   \setcounter{ItemCounter}{0}      % Reset the ItemCounter for this list.

   \itempar{Gain Master}{eq!master gain}
   Amplifies or reduces the overall signal that passes through EQ.

   \itempar{Graph}{eq!graph}
   Shows the complete frequency response of all the active EQ bands and the
   master \textbf{Gain} setting.

   \itempar{Band}{eq!band}
   Set the current frequency band number (or filter).
   Band lets one choose the passband number. Multiple passbands define one
   filter. This is important if one wants multiple filters to be called
   after each other. Note that filters are commutative.

   Values: \texttt{0*, 1, ... 7}

   Bands:

   \itempar{Type}{eq!filter type}
   Set the type of the filter.

   Values: \texttt{Off*, Lp1, Hp1, Lp2, Hp2, Bp2, N2, Pk, Lsh, Hsh}

   Note that, for certain values of the \textbf{Type} parameter, the
   \textbf{Gain} and/or \textbf{Q} controls will not be available.

   \itempar{Freq}{eq!filter freq}
   The frequency of the filter.
   Freq describes the frequencies where the filter has its poles. For some
   filters, this is called the "cutoff" frequency. Note, however, that a
   bandpass filter has two cutoff frequencies.

   \itempar{Gain (Filter)}{eq!filter gain}
   The gain of an individual filter.
   Gain is only active for some filters
   (\textbf{Pk}, \textbf{Lsh}, and \textbf{Hsh},
   and it sets the amount of a special
   peak these filters have. Note that for those filters, using the
   predefined gain makes them ineffective.

   \itempar{Q}{eq!filter q}
   The Q (resonance, or bandwidth) of the filter.
   Resonance lets one describe a peak at the given frequency for filters
   with 2 poles. This can be compared to real physical objects that have
   more gain at their resonant frequency.

   \itempar{Stages}{eq!stages}
   Number of additional times the filter will be applied (in
   order to do very steep roll-off - eg. 48 dB/octave).
   Stages lets one define multiple filter stages. This is equivalent to
   having multiple copies of the same filter in sequence.

   Values: \texttt{0*, 1, ... 4}

\subsubsection{Effects / EQ / NRPN Values}
\label{subsubsec:effects_edit_eq_nrpn}

   Effects can be controlled via "non-registered parameter numbers", or NRPNs.
   This section details the value supported by the EQ effect.

   TODO.

\subsection{Effects / Phaser}
\label{subsec:effects_edit_phaser}

   The Phaser is a special dynamic filter. The result is a sweeping sound,
   which is often used on instruments with a large frequency band, like
   guitars or strings. This makes it typical for genres like rock or funk,
   where it is often modulated with a pedal, but also for giving strings a
   warm, relaxing character.

\subsubsection{Effects / Phaser / Circuit}
\label{subsubsec:effects_edit_phaser_circuit}

   We explain the functionality in a diagram and list the components below.

\begin{figure}[H]
   \centering
   \includegraphics[scale=0.6]{1.5.11/effects-drawing-phaser.png}
   \caption{Phaser Circuit}
   \label{fig:phaser_circuit}
\end{figure}

   The audio signal is split into two paths. One path remains unchanged. The
   other one is sent to a delay line. The delay time (the so-called phase) is
   made dependent on the frequency. Therefore, an all-pass filter is applied
   to the signal, which preserves the amplitude, but determines the delay
   time. At the end, both paths are added.

   Yoshimi offers different types of phasers:

   \begin{itemize}
      \item \textbf{Analog and "normal" phasers}.
      Analog phasers are more complicated.
      They sound punchier, while normal phasers sound more fluent. However,
      analog filters usually need more filter stages to reach a
      characteristic sound.
      \item \textbf{Sine and triangle filters}.
      Note that an analog triangle filter
      with many poles is a barber pole filter and can be used to generate
      Shepard Tones, i.e. tones that seem to increase or decrease with time,
      but do not really.
      \item \textbf{The LFO function can be squared}.
      This is only available for the Analog phaser and converts the triangle
      wave into a hyper sine wave. This approximates a triangle for the top
      half, and more rounded sine-like bottom half. The sine squared is simply
      a faster sine wave.
   \end{itemize}

%  TODO: Barber is deactivated, since PLFOtype is only 0 or 1?

   For the normal phaser,
   \figureref{fig:effects_edit_phaser}, below, shows the controls referred to
   in this list of steps.

   \begin{enumber}
      \item First, the LFO is generated.
         There are 4 controls
         (\textbf{Freq}, \textbf{Rnd}, \textbf{LFO} type, \textbf{St.df})
         that define the LFO.
      \item \textbf{Phase} and \textbf{Depth} are added in the usual way.
      \item If \textbf{hyper} is set, then the LFO function is squared.
      \item Next, this modulates the input signal amplitude.
      \item The \textbf{Analog} setting decides whether the phaser is analog
            or "normal".
            For the analog phaser (see the \textbf{Analog} check-box),
            \textbf{L/R} is not implemented.
            Conversely, for the normal phaser, \textbf{hyper} and \textbf{dist}
            are not available.
      \item \textbf{Pan} applies panning to the original input in every loop.
      \item Next, phasing is applied - barber-pole type for \textbf{Analog} only.
      \item Then, based on the setting of \textbf{Stages}, further phasing
            stages are applied.
            For \textbf{Analog} only the \textbf{dist} control sets the amount of
            distortion when applying the phasing stages.
      \item \textbf{Fb} applies feedback next. The last sound buffer element is (after
            phasing) multiplied by this value and then added back in. For the
            normal filter, the value is added before, and, for analog, after the
            first phasing stage.
      \item Finally, the \textbf{Sub.} option inverts the signal, multiplying it
            by -1.
   \end{enumber}

\subsubsection{Effects / Phaser / User Interface}
\label{subsubsec:effects_edit_phaser_ui}

\begin{figure}[H]
   \centering
   \includegraphics[scale=0.75]{2.3.0/phaser.png}
   \caption{Effects Edit, Phaser}
   \label{fig:effects_edit_phaser}
\end{figure}

%  TODO. Include the item-paragraphs for each GUI element.

   \begin{enumber}
      \item \textbf{Preset.}
      \item \textbf{Phase}
      \item \textbf{Depth}
      \item \textbf{L/R}
      \item \textbf{Dist}
      \item \textbf{D/W}
      \item \textbf{Pan}
      \item \textbf{Freq}
      \item \textbf{Start}
      \item \textbf{Rnd}
      \item \textbf{BPM}
      \item \textbf{LFO type}
      \item \textbf{St.df}
      \item \textbf{Fb}
      \item \textbf{Analog}
      \item \textbf{Hyper}
      \item \textbf{Sub.}
      \item \textbf{Stages}

   \end{enumber}

%   The extra fields that are shown if the effect is an insertion effect are
%   not shown.  They are still need to be described.
%   What extra fields ???

\setcounter{ItemCounter}{0}

   \itempar{Preset}{phaser!preset}
   Phaser Presets.

\begin{figure}[H]
   \centering
   \includegraphics[scale=1.0]{2.3.0/phaser_presets.png}
   \caption{Phaser presets}
   \label{fig:phaser-presets}
\end{figure}
   Values: \texttt{Phaser 1 - 6, APhaser 1 - 6}

   \itempar{Phase}{phaser!phase}
   Phaser Phase.

   Values: \texttt{0\% to 100\%}

   \itempar{Depth}{phaser!depth}
   Phaser Depth. Phaser LFO Depth?

   Values: \texttt{0\% to 100\%}

   \itempar{L/R}{phaser!l/r}
   L/R. How the left/right channels are routed to output:

      \begin{enumber}
         \item leftmost. Left to left and right to right.
         \item middle. Left+right to mono.
         \item rightmost. Left to right, and right to left.
      \end{enumber}

   \itempar{Dist}{phaser!dist}
   Phaser Distortion.
   Ranges from 0\% to 100\%.

   \itempar{D/W}{phaser!dry/wet}
   Phaser Dry/Wet.
   This knob sets the effect volume.  The dry value ranges from 0 dB down to
   "inf" (infinity) dB, while the wet value is the complementary range, from
   "inf" dB to 0 dB.  Confusing?  The tooltip tells the user exactly what the
   settings are.

   \itempar{Pan}{phaser!pan}
   Phaser Panning.
   Ranges from 100\% left to centered to 100\% right.

   \itempar{Freq}{phaser!freq}
   Phaser Freq.
   Set the Phaser LFO frequency.
   Ranges from 0.0 Hz to 30.68 Hz.

   \itempar{Start}{phaser!start}
   Phaser Start.
   Set the start of the Phaser phase relative to MIDI sync.
   Ranges from 0 to 178.6\%

   \itempar{Rnd}{phaser!randomness}
   Phaser Randomness.
   Set the Phaser LFO randomness.
   Ranges from 0.0\% to 100\% percent.

   \itempar{BPM}{phaser!lfo sync}
   Locks the frequency control to the incoming MIDI clock. New from
   \textsl{Yoshimi} V 2.1.0 See
   \sectionref{subsubsec:bpm_and_frequency}\ for further details.

   \itempar{LFO}{phaser!lfo type}
   Phaser LFO Type.

   Values: \texttt{SINE, TRI}

   \itempar{St.df}{phaser!stereo phase diff}
   Left/Right Channel Phase Shift.
   The phase difference between LFO for left/right channels.
   Ranges from -180 degrees (left 180) to equal to +180 degrees (right 180).
   The actual end values can differ a little from 180.

   \itempar{Fb}{phaser!feedback}
   Phaser Feedback.
   Ranges from -99\% to 99\%.

   \itempar{Analog}{phaser!analog}
   Phaser Analog.
   Checking this box emulates an "FET"  (Field-effect transistor).

   Values: \texttt{Off*, On}

   \itempar{Hyper}{phaser!hyper}
   Phaser Hyper.
   Checking this box sets the "hyper-sine" mode.

   Values: \texttt{Off*, On}

   \itempar{Sub.}{phaser!Subtract}
   Phaser Subtract.
   Checking this box inverts the output so it tends to subtract from the incoming rather than adding.

   Values: \texttt{Off*, On}

   \itempar{Stages}{phaser!stages}
   Phaser Stages.

   Values: \texttt{1*, 2, ... 12}

\subsubsection{Effects / Phaser / NRPN Values}
\label{subsubsec:effects_edit_phaser_nrpn}

Effects can be controlled via "non-registered parameter numbers", or NRPNs.
This section details the value supported by the Phaser effect.

\subsection{Effects / Reverb}
\label{subsec:effects_edit_reverb}

   A Reverberation actually expresses the effect of many echoes being played
   at the same time. This can happen in an enclosed room, where the sound can
   be reflected in different angles. Also, in nature, thunder approximates
   reverb, because the sound is reflected in many different ways, arriving
   at the listener at different times.

   In music, reverb is popular in many ways. Reverb with large room size
   can be used to emulate sounds like in live concerts. This is useful for
   voices, pads, and hand claps. A small room size can simulate the sound
   board of string instruments, like guitars or pianos.

\subsubsection{Effects / Reverb / Circuit}
\label{subsubsec:effects_edit_reverb_circuit}

   As mentioned, a reverb consists of permanent echo. The reverb in
   \textsl{Yoshimi} is more complex than the echo. After the delaying, comb
   filters and then allpass filters are being applied. These make the
   resulting sound more realistic. The parameters for these filters depend on
   the roomsize. For details, consider the information about Freeverb.

\begin{figure}[H]
   \centering
   \includegraphics[scale=0.25]{zyn/effects/reverb.png}
   \caption{Reverb Circuit Diagram}
   \label{fig:reverb_circuit_diagram}
\end{figure}

\subsubsection{Effects / Reverb / User Interface}
\label{subsubsec:effects_edit_reverb_ui}

   The user-interface for the Reverb effect depends on whether it is used as a
   System effect or an Insertion effect. When used as a System effect \textbf{D/W} becomes \textbf{Vol}.
\iffalse
   Observr \figureref{fig:effects_edit_reverb}, where
   the Insertion mode is shown.  In the System mode, only the light-blue
   portion of the user-interface appears.
\fi
\begin{figure}[H]
   \centering
%  \includegraphics[scale=1.0]{bottom-panel/instrument-edit/Effects/effects-edit-reverb.jpg}
   \includegraphics[scale=0.75]{2.3.0/reverb.png}
   \caption{Effects Edit, Reverb}
   \label{fig:effects_edit_reverb}
\end{figure}

   \begin{enumber}
      \item \textbf{Preset}
      \item \textbf{Type}
      \item \textbf{R.S.}
      \item \textbf{D/W}
      \item \textbf{Pan}
      \item \textbf{Time}
      \item \textbf{I.del}
      \item \textbf{I.delfb}
      \item \textbf{BW}
      \item \textbf{E/R}
      \item \textbf{LPF}
      \item \textbf{HPF}
      \item \textbf{Damp}
   \end{enumber}
\iffalse
???
  There is a fourth type we have screen captures for, but we can't seem
  to navigate to them now!  Are these now out-of-date screen captures?
This belongs elsewhere
   \begin{enumber}
      \item \textbf{FX No.}
      \item \textbf{bypass}
      \item \textbf{EffType}
      \item \textbf{Send To}
      \item \textbf{C}
      \item \textbf{P}
      \item \textbf{Close}
   \end{enumber}
\fi
   \setcounter{ItemCounter}{0}      % Reset the ItemCounter for this list.

   \itempar{Preset}{reverb!preset}
      Reverb Preset.

\begin{figure}[H]
   \centering
   \includegraphics[scale=1.0]{2.3.0/reverb_presets.png}
   \caption{Reverb Preset Dropdown}
   \label{fig:reverb_preset_dropdown}
\end{figure}

   Values: \texttt{Cathedral 1, Cathedral 2, Cathedral 3, Half 1, Half 2,
              Room 1, Room 2, Basement, Tunnel, Echoed 1, Echoed 2, Very Long
               1, Very Long 2}

   \itempar{Type}{reverb!type}
   Reverb Type.
   The combobox lets one select a reverb type.

\begin{figure}[H]
   \centering
   \includegraphics[scale=1.0]{2.3.0/reverb_type.png}
   \caption{Reverb Type Dropdown}
   \label{fig:reverb_type_dropdown}
\end{figure}

   \begin{itemize}
      \item Freeverb is a preset. It was proposed by Jezar at Dreampoint.
      \item Bandwidth has the same parameters for the comb and allpass
         filters, but it applies a unison before the LPF/HPF. The unison’s
         bandwidth can be set using BW.
      \item Random chooses a random layout for comb and allpass each time the
         type or the roomsize is being changed.
   \end{itemize}

   Values: \texttt{Random, Freeverb, Bandwidth}

   \itempar{R.S}{reverb!room size}
   Reverb Room Size.
   The room size defines parameters only for the comb and allpass filters.

   \itempar{D/W}{reverb!dry/wet}
   Reverb Dry/Wet Setting.
   This setting controls much of the original signal is mixed with the
   reverb effect.

   \itempar{Pan}{reverb!pan}
   Reverb Panning.
   Pan lets one apply panning. This is the last process to happen.

   \itempar{Time}{reverb!time}
   Reverb Time.
   Set the duration of late reverb.
   Time controls how long the whole reverb takes, including how slowly
   the volume is decreased.

   \itempar{I.del}{reverb!initial delay}
   Reverb Initial Delay.
   The initial delay (I.del) is the time which the sounds need at least to
   return to the user.

   \itempar{I.delfb}{reverb!initial delay feedback}
   Reverb Initial Delay Feedback.
   Sets the initial delay feedback.
   The initial delay feedback (I.delfb) says how much
   of the delayed sound is added to the input.
   It is not recommended to use this setting together with
   low initial delays).

   \itempar{BW}{reverb!bandwidth}
   Reverb Bandwidth.

   \itempar{E/R}{reverb!e/r}
   Reverb E/R.
   Early Reflection (not currently implemented).

   \itempar{LPF}{reverb!lpf}
   Reverb Lowpass Filter.
   This filter is applied before the comb filters.

   \itempar{HPF}{reverb!hpf}
   Reverb Highpass Filter.
   This filter is applied before the comb filters.

   \itempar{Damp}{reverb!damp}
   Reverb Damp.
   Damp determines how high frequencies are damped during the
   reverberation.  The dampening control (Damp) currently only allows to
   damp low frequencies. Its parameters are used by the comb and allpass
   filters.
\iffalse
??? This doesn't belong here.
   \itempar{FX No}{reverb!fx no.}
   Reverb FX Number.

   Values: \texttt{1 to 8?}

   \itempar{bypass}{reverb!fx bypass}
   Reverb FX Bypass.

   Values: \texttt{Off*, On}

   \itempar{EffType}{reverb!eff type}
   Reverb Effect Type.

   Values: \texttt{Reverb, EQ, Echo, etc. TODO}

   \itempar{Send To}{reverb!send to}
   Reverb Send To.
   This user-interface drop-down is shown only in the
   \textbf{Part / Edit / Effects} version of the effects panel.

   Values: \texttt{Next Effect, Part Out, Dry Out}

   \itempar{C}{reverb!copy}
   Reverb Copy.

   \itempar{P}{reverb!paste}
   Reverb Paste.

   \itempar{Close}{reverb!close}
   Close Window.
\fi
\subsubsection{Effects / Reverb / NRPN Values}
\label{subsubsec:effects_edit_reverb_nrpn}

   Effects can be controlled via "non-registered parameter numbers", or NRPNs.
   This section details the values supported by the Reverb effect.

   TODO:  detail the values supported by the Reverb effect.
   \index{todo!reverb nrpn values}

%-------------------------------------------------------------------------------
% vim: ts=3 sw=3 et ft=tex
%-------------------------------------------------------------------------------


% Bottom Panel User-Interface items

%-------------------------------------------------------------------------------
% yum_bottom_panel
%-------------------------------------------------------------------------------
%
% \file        yum_bottom_panel.tex
% \library     Documents
% \author      Chris Ahlstrom
% \date        2015-06-06
% \update      2016-03-01
% \version     $Revision$
% \license     $XPC_GPL_LICENSE$
%
%     Provides the bottom_panel section of yoshimi-user-manual.tex.
%
%-------------------------------------------------------------------------------

\section{Bottom Panel}
\label{sec:bottom_panel}

\subsection{Bottom Panel Controls}
\label{subsec:bottom_panel_controls}

   The \textsl{Yoshimi} bottom panel provides quick access to some major
   features of the application.
   The bottom panel is shown in
   \figureref{fig:yoshimi_main_screen}.

   Here are the major elements of the bottom panel.

   \begin{enumber}
      \item \textbf{Part}
      \item \textbf{of}
      \item \textbf{Instrument Name}
      \item \textbf{Edit} (Instrument Edit Button)
      \item \textbf{Midi}
      \item \textbf{Mode}
      \item \textbf{Enabled}
      \item \textbf{Portamento}
      \item \textbf{Velocity Sens}
      \item \textbf{Velocity Offset}
      \item \textbf{Pan}
      \item \textbf{Pan Reset Button}
      \item \textbf{Volume}
      \item \textbf{Controllers}
      \item \textbf{Minimum Note}
      \item \textbf{Maximum Note}
      \item \textbf{m}
      \item \textbf{R}
      \item \textbf{M}
      \item \textbf{Key Shift}
      \item \textbf{Key Limit}
      \item \textbf{System Effect Sends 1}
      \item \textbf{System Effect Sends 2}
      \item \textbf{System Effect Sends 3}
      \item \textbf{System Effect Sends 4}
      \item \textbf{Sound Meter}
   \end{enumber}

   \setcounter{ItemCounter}{0}      % Reset the ItemCounter for this list.

   \itempar{Part}{bottom panel!part number}
   Part Number.

   Values: \texttt{1 to 16; 1 to 32; 1 to 64 }

   Show and set current part.  The maximum number of values depends on the
   \textbf{Part of} selection.

   \itempar{of}{bottom panel!part maximum}
   Maximum Number of Parts.

   Values: \texttt{16*, 32, 64}

   \textsl{Yoshimi} now has up to 64 parts in blocks of 16. One can now decide
   how many one wants to have available using this user-interface item.  By
   default these are wraped around the normal MIDI channels, so that parts 1,
   17, 33, and 49 all respond to channel 1 messages. This was originally
   implemented for Vector Control, working with up to four sounds on a channel
   (similar to the Yamaha SY hardware series).

   However, these additional parts have other less obvious uses. One of these
   is getting far more than 16 completely independent tracks addressed by just
   the 16 channels. Most tunes run with instruments having a relatively narrow
   pitch range, and this is what we can make use of.

   As an example, in \textsl{Yoshimi}'s main window select 64 parts, then on
   part 1 set (say) Steel Bass and maximum note as 52 (E).  Next select part 17
   and enable it (easiest to use the mixer panel for this) set Tunnel Piano,
   the *minimum* note as 53 and maximum as 71 (B).  Finally, enable part 33,
   set Rushes, and set it's minimum note as 72, but key shift down an octave.
   With a 61 note keyboard that gives one quite a useful working range, on just
   one channel.

   However, the idea really comes into its own with a sequencer like Rosegarden
   where one can record multiple parts over the full MIDI range and track them
   to the same channel. Also, in Rosegarden the parts can be separately named,
   and identified as Bass and Treble in the notation editor. This makes it very
   convenient for those wanting a more formal musical layout.

   So, with very little effort, one can now have 48 tracks playing at once!
   Ummm, one does need a decent processor though :)

   Yes, one could run more instances of Yoshimi on different MIDI ports, but
   where's the fun in that?

   By default, all the upper parts (numbers greater than 16)
   are mapped to the same MIDI channel
   numbers as the lowest ones, but have independent voice and parameter
   settings. They cannot normally receive independent note or control
   messages. However, vector control will intelligently work with however
   many one has set, as will all the NRPN direct part controls.
   See \sectionref{subsection:vector_control}.

   This item is a fairly new feature of \textsl{Yoshimi} (as of version
   1.3.5).
   
   \itempar{Instrument Name}{bottom panel!instrument name}
   Instrument Name.
   Left-click to open the Bank window.
   Right-click to change the name of the current instrument.
   If one changes the name of the instrument, be sure to select
   \textbf{Menu / Instrument / Save Instrument} to preserve that change.

   The name now has color-coding to indicate the instrument's use of
   ADDsynth, SUBsynth, or PADsynth.  One can see the "red" color for ADDsynth
   in the figure for the bottom panel.  "Blue" would indicate SUBsynth, and
   "green" would indicate PADsynth.

   \itempar{Edit}{bottom panel!instrument edit}
   Instrument Edit button.
   This button brings up the instrument-edit dialog shown in
   \figureref{fig:instrument_edit_dialog}.

   This dialog provides a very broad overview of the instrument, and
   provides access to far more detailed dialogs to edit the instrument.
   This dialog is explained in detail in
   \sectionref{subsec:bottom_panel_instrument_edit}.

   \itempar{Midi}{bottom panel!MIDI channel}
   MIDI Channel.

   Values: \texttt{1 to 16}

   \itempar{Mode}{bottom panel!mode}
   Mode. Poly.
   Sets the mode (polyphonic/monophonic/legato).
   In polyphonic mode, multiple simultaneous notes are supported.
   (How many at maximum?).
   In monophonic mode, only one note is supported.
   In legato mode, the sound flows smoothly from note to note without
   any breaks.

   Values: \texttt{Poly, Mono, Legato}

   \itempar{Enabled}{bottom panel!instrument enable}
   Enable the part. If the Part is disabled it doesn't use CPU time.

   Values: \texttt{Off*, On}

   \itempar{Portamento}{bottom panel!portamento enable}
   Enable/disable the portamento.
   One can set the duration and other parameters by opening the Controllers
   window.

   Values: \texttt{Off*, On}

   \itempar{Velocity Sens}{bottom panel!velocity sensing}
   Velocity Sensing Function.

   Values: \texttt{0 to 127, 64*}

   \itempar{Velocity Offset}{bottom panel!velocity offset}
   Velocity Offset.

   Values: \texttt{0 to 127, 64*}

   \itempar{Pan}{bottom panel!pan}
   Pan.

   Values: \texttt{0 to 127, 64*}

   \itempar{Pan (reset)}{bottom panel!pan reset}
   Reset Pan to Middle (64).

   \itempar{Volume}{bottom panel!volume}
   Instrument Volume.

   Values: \texttt{0 to 127, 96*}
 
   The default volume for ADD parts (overall) and SUB parts is 96; the
   default volume for SUB parts is 90; the ADD voice volumen is 100; and
   effects volumes vary heavily with the effect.

   \itempar{Minimum Note}{bottom panel!minimum note}
   Minimum note the part receives.

   Values: \texttt{0* to 127}

   \itempar{Maximum Note}{bottom panel!maximum note}
   Maximum note the part receives.

   Values: \texttt{0 to 127*}

   \itempar{m}{bottom panel!m}
   Minimum Note Capture Button.

   Set minimum note to last note played.

   \itempar{R}{bottom panel!R}
   Minimum and Maximum Note Reset Button.

   Reset the minimum key to 0 and the maximum key to 127.

   \itempar{M}{bottom panel!M}
   Maximum Note Capture Button.
   Sets the maximum note to the last pressed key.

   \itempar{Key Shift}{bottom panel!key shift}
   Key Shift.

   Values: \texttt{-12 to 12, 0*}

   \itempar{Key Limit}{bottom panel!key limit}
   Maximum keys for this part.

   Values: \texttt{0 to 55, 15*}

   \itempar{System Effect Sends 1, 2, 3, and 4}{bottom panel!system effect sends}

   TODO:  Describe how these sends work.
   \index{todo!SES123}

   Values: \texttt{0 to 127*}

   \itempar{Sound Meter}{bottom panel!sound meter}
   VU Meter.  Sound Meter.

   This discussion of "Audio Output and Levels"
   comes from \texttt{Output Levels.txt}.

   At the bottom of the main window there is a pair of horizontal grids
   representing a bargraph type display. The upper one is for the left hand
   channel and the lower one for the right hand one. The grid divisions each
   represent 1 dB, and the brighter divisions are therefore 5 dB. The thicker
   bright divisions therefore being 10 dB. The overall scale range is -48 dB to
   0 dB.

   As the output level rises pale blue strips will light up in these grids.
   These fast responding bars are the peak levels and should never be allowed
   to go above 0 dB, otherwise the output is likely to be clipped and distorted.
   There is also a pair of boxes on the end of these grids which will show the
   highest peak level seen. If clipping has happened the box background will
   change from black to red.

   To clear clip and peak level indication click on this area.

   As well as the peak level, the display shows a much slower responding RMS
   level, as a yellow line on top of the blue bar. This gives and indication of
   the apparent accoustic power.

   If one opens the panel window one will see vertical bargraphs for each
   individual part. On these, the faint bars are 5dB steps and the bright ones
   10dB. The peak leve isn't shown numerically, but if one exceeds 0dB a thick
   red line will appear at the top of the bargraph. This is also cleared from
   the box in the main window.

   (More information to come).

\paragraph{Tip: Using the VU Meter}
\label{paragraph:tips_using_the_vu_meter}
\index{tips!vu meter}

   The VU meter topic is very interesting, because one of the problems
   is a tendency to overdrive by way of sustain pedal.  At the last test it
   showed up in the output before it showed up in the VU meter, so
   the VU meter should help a lot in analysis.

   One way to avoid overdrive is to keep polyphony to 20 on each patch (two
   or three patches per \textsl{Yoshimi} instance, with two or three
   \textsl{Yoshimi} simultaneous instances depending on the patch).

   Another item which helps a lot is compression (for example, the Calf
   multiband compressor is amazingly good.

\subsection{Bottom Panel / Controllers}
\label{subsec:bottom_panel_controllers}

\begin{figure}[H]
   \centering 
   \includegraphics[scale=1.0]{bottom-panel/controllers-dialog.jpg}
   \caption{Controllers Dialog}
   \label{fig:controllers_dialog}
\end{figure}

   \begin{enumber}
      \item \textbf{Exp MWh}
      \item \textbf{ModWh}
      \item \textbf{Exp BW}
      \item \textbf{BwDepth}
      \item \textbf{PanWdth}
      \item \textbf{FltQ}
      \item \textbf{FitCut}
      \item \textbf{Vol Rng}
      \item \textbf{PWheelB.Rng}
      \item \textbf{Expr}
      \item \textbf{FMamp}
      \item \textbf{Vol}
      \item \textbf{Sustain}
      \item \textbf{Resonance} (section)
      \item \textbf{PortaMento} (section)
      \item \textbf{Reset all controllers}
      \item \textbf{Close}
   \end{enumber}

   \setcounter{ItemCounter}{0}      % Reset the ItemCounter for this list.

   \itempar{Exp MWh}{controllers!expression mod wheel}
   Exponential Modulation Wheel.
   Changes the modulation scale to exponential.

   Values: \texttt{Off*, On}

   \itempar{ModWh}{controllers!mod wheel depth}
   Modulation Wheel Depth.

   Values: \texttt{0 to 127, 80*}

   \itempar{Exp BW}{controllers!exp bandwidth controller}
   Exponential Bandwidth Controller.
   Changes the bandwidth scale to exponential.

   Values: \texttt{Off*, On}

   \itempar{BwDepth}{controllers!bandwidth depth}
   Bandwidth Depth.

   Values: \texttt{0 to 127, 64*}

   \itempar{Exp BW}{controllers!bandwidth depth}
   Exponental Bandwidth.
   Changes the bandwidth scale to exponential.

   Values: \texttt{0 to 127, 64*}

   \itempar{PanDpth}{controllers!panning depth}
   Panning Depth.

   Values: \texttt{0 to 64*}

   \itempar{FltQ}{controllers!filter Q depth}
   Filter Q (resonance) Depth.

   Values: \texttt{0 to 127, 64*}

   \itempar{FltCut}{controllers!filter cutoff depth}
   Filter Cutoff Frequency Depth.

   Values: \texttt{0 to 127, 64*}

   \itempar{Vol Rng}{controllers!volume range}
   Volume Range.

   Values: \texttt{64 to 127, 64*}

   \itempar{PWheelB.Rng}{controllers!pitch wheel range}
   Pitch Wheel Bend Range (cents).
   100 cents = 1 halftone.

   Values: \texttt{-6400 to 6400, 200*}

   \itempar{Expr}{controllers!expression}
   Expression Enable.
   Enable/disable expression.

   Values: \texttt{Off, On*}

   \itempar{FMamp}{controllers!fm amplitude}
   FM Amplitude Enable.
   Enable/disable receiving Modulation Amplitude controller (76).

   Values: \texttt{Off, On*}

   \itempar{Vol}{controllers!volume enable}
   Volume Enable.

   Values: \texttt{Off, On*}

   Enable/disable receiving volume controller.
   Sensitivity to MIDI volume change (CC7) is now variable in 'Controllers'
   in the same way as pan width etc. The nnumeric range is 64 to 127; the
   default at 96 gives the same sensitivity as before at -12dB relative to
   the GUI controls. 127 gives 0dB and 64 gives -26dB

   \itempar{Sustain}{controllers!sustain pedal enable}
   Sustain Pedal Enable.
   Enable/disable sustain pedal.

   Values: \texttt{Off, On*}

   \itempar{Reset all controllers}{controllers!reset all}
   Reset All Controllers.

   \itempar{Close}{controllers!close}
   Close Window.

\subsubsection{Bottom Panel / Controllers / Resonance}
\label{subsubsec:bottom_panel_controllers_resonance}

   \setcounter{ItemCounter}{0}      % Reset the ItemCounter for this list.

   \itempar{CFdepth}{controllers!resonance CF depth}
   Resonance Center Frequency Depth,
   Center Frequency Controller Depth.

   Values: \texttt{0 to 127, 64*}

   \itempar{BWdepth}{controllers!resonance BW depth}
   Resonance Bandwidth Depth,
   Resonance Bandwidth Controller Depth.

   Values: \texttt{0 to 127, 64*}

\subsubsection{Bottom Panel / Controllers / Portamento}
\label{subsubsec:bottom_panel_controllers_portamento}

   \setcounter{ItemCounter}{0}      % Reset the ItemCounter for this list.

   \itempar{Rcv}{controllers!portamento receive}
   Portamento Receive,
   Receive Portamento Controllers.
   Determines if the part receives Portamento On/Off (65) controller.

   Values: \texttt{Off, On*}

   \itempar{Proprt.}{controllers!portamento proportional}
   Portamento Proportional,
   Enable Proportional Portamento (over fixed portamento).

   Values: \texttt{Off*, On}

   \itempar{time}{controllers!portamento time}
   Portamento time.
   The duration of the portamento.

   Values: \texttt{0 to 127, 64*}

   \itempar{t.dn/up}{controllers!portamento time, down/up}
   Portamento Time Stretch (up/down).

   Values: \texttt{0 to 127, 64*}

   \itempar{threshx100 cnt.}{controllers!portamento threshold}
   Threshold of the Portamento.

   Values: \texttt{0 to 127, 3*}

   Minimum or maximum difference of notes in order
   to do the portamento (x 100 cents).
   It represents the minimum or the maximum number of halftones (or hundred
   cents) required to start the portamento. The difference is computed
   between the last note and current note.

   The threshold refers to the frequencies and not to MIDI notes (one should
   consider this if one uses microtonal scales).

   \itempar{th.type}{controllers!portamento threshold type}
   Threshold Type (min/max).
   Checked means that the portamento activates when the difference of
   frequencies is above the threshold ("thresh"); not checked is for below
   the threshold.

   Values: \texttt{Off, On*}

   \itempar{Propt.}{controllers!portamento proportional}
   Proportional Portamento.
   If set, the portamento is proportional to ratio of frequencies.

   Values: \texttt{Off, On*}

   \itempar{Prp.Rate}{controllers!portamento rate}
   Distance required to double change from nonproportional
   portamento time.
   The ratio needed to double the time of portamento.

   Values: \texttt{0 to 127, 80*}, requires \textbf{Proprt.} = \texttt{On}

   \itempar{Prp.Depth}{controllers!portamento depth}
   The difference from nonproportional portamento.

   Values: \texttt{0 to 127, 90*}, requires \textbf{Proprt.} = \texttt{On}

\subsection{Bottom Panel Instrument Edit}
\label{subsec:bottom_panel_instrument_edit}

   The main instrument-editing dialog is relatively simple, and provides for
   editing information that identifies the instrument, and buttons to access
   the more complex dialogs of the ADDsynth, SUBsynth, PADsynth, Kit Edit,
   and Effects components.

%  \includegraphics[scale=1.0]{bottom-panel/edit-instrument.jpg}

\begin{figure}[H]
   \centering 
   \includegraphics[scale=1.0]{1.3.5/edit-instrument.png}
   \caption{Instrument Edit Dialog}
   \label{fig:instrument_edit_dialog}
\end{figure}

   This dialog provides a very broad overview of the instrument, and
   provides access to far more detailed dialogs to edit the instrument.
   This dialog is called up by the \textbf{Edit} button on the bottom panel
   of the main \textsl{Yoshimi} main screen.

   \begin{enumber}
      \item \textbf{Type}
      \item \textbf{Author and Copyright}
      \item \textbf{Comments}
      \item \textbf{ADDsynth}
      \begin{enumber}
         \item \textbf{Enabled}
         \item \textbf{Edit}
      \end{enumber}
      \item \textbf{SUBsynth}
      \begin{enumber}
         \item \textbf{Enabled}
         \item \textbf{Edit}
      \end{enumber}
      \item \textbf{PADsynth}
      \begin{enumber}
         \item \textbf{Enabled}
         \item \textbf{Edit}
      \end{enumber}
      \item \textbf{Kit Edit}
      \item \textbf{Effects}
      \item \textbf{Rnd. Det.}, now replaced by a \textbf{Humanise} slider.
      \item \textbf{Close}
   \end{enumber}

   The ADDsynth, SUBsynth, PADsynth, Kit Edit, and Effects
   dialogs are detailed in separated sections, as they are all
   very complex dialogs with many sub-dialogs.

   \setcounter{ItemCounter}{0}      % Reset the ItemCounter for this list.

   \itempar{Type}{edit!category}
   Instrument Type.
   Instrument Category.

   This dropdown dialog allows one to tag the type of instrument, to
   indicate what category of instruments it fits into.
   The following figure shows the types.

\begin{figure}[H]
   \centering 
   \includegraphics[scale=1.0]{bottom-panel/edit-instrument-type.jpg}
   \caption{Instrument Type Drop-down List}
   \label{fig:instrument_type_dropdown}
\end{figure}

   Values: \texttt{Piano, Chromatic Percussion, Organ, Guitar, Bass,
              Solo Strings, Ensemble, Brass, Reed, Pipe,
              Synth Lead, Synth Pad, Synth Effects, Ethnic,
              Percussive, Sound Effects}

   \itempar{Author and Copyright}{edit!author/copyright}
   This field provides space for identifying the author, copyright, and
   license for the part.

   \itempar{Comments}{edit!comments}
   Allows free-form comments and notes to be entered.

   \itempar{ADDsynth}{edit!addsynth}

   \begin{enumber}
      \item \textbf{Enabled}.
      Enables this synth type to be used in the part/instrument.
      When enabled, its marker color, red, is shown.
      \item \textbf{Edit}.
      Brings up the editing dialog presented in
      \figureref{fig:addsynth_edit_dialog}.
      There one will find a full discussion of that dialog.
   \end{enumber}

   \itempar{SUBsynth}{edit!subsynth}

   \begin{enumber}
      \item \textbf{Enabled}.
      Enables this synth type to be used in the part/instrument.
      When enabled, its marker color, blue, is shown.
      \item \textbf{Edit}.
      Brings up the editing dialog presented in
      \figureref{fig:subsynth_edit_dialog}.
      There one will find a full discussion of that dialog.
   \end{enumber}

   \itempar{PADsynth}{edit!padsynth}

   \begin{enumber}
      \item \textbf{Enabled}.
      Enables this synth type to be used in the part/instrument.
      When enabled, its marker color, green, is shown.
      \item \textbf{Edit}.
      Brings up the editing dialog presented in
      \figureref{fig:padsynth_edit_dialog}.
      There one will find a full discussion of that dialog.
   \end{enumber}

   \itempar{Kit Edit}{edit!kit}
   Brings up the editing dialog presented in
   \figureref{fig:kit_edit_dialog}.
   There one will find a full discussion of that dialog.

   \itempar{Effects}{edit!effects}
   Brings up the editing dialog presented in
   \figureref{fig:effects_edit_none}.
   There one will find a full discussion of that dialog.

   \itempar{Rnd. Det.}{edit!rnd det}
   Small Random Detune, Humanise.
   \index{humanise}
   \index{edit!humanise}
   In the most recent versions of \textsl{Yoshimi}, this
   item has been replaced by a \textbf{Humanise} slider.
   This value is an experimental feature.  It lends some complexity or
   piquancy to the part.

   Values: \texttt{0* to 20}

   \itempar{Close}{edit!close}
   Closes the Edit window.

%-------------------------------------------------------------------------------
% vim: ts=3 sw=3 et ft=tex
%-------------------------------------------------------------------------------


% ADDsynth User-Interface items

%-------------------------------------------------------------------------------
% yum_addsynth
%-------------------------------------------------------------------------------
%
% \file        yum_addsynth.tex
% \library     Documents
% \author      Chris Ahlstrom
% \date        2015-06-07
% \update      2021-11-28
% \version     $Revision$
% \license     $XPC_GPL_LICENSE$
%
%     Provides the ADDsynth section of yoshimi-user-manual.tex.
%
%-------------------------------------------------------------------------------

\section{ADDsynth}
\label{sec:addsynth}

   The \textsl{Yoshimi} ADDsynth (also spelled "ADsynth" or "AddSynth")
   dialog is a complex dialog for creating an
   instrument.  This is the most complex, most advanced and most
   sophisticated part of the synthesizer and allows one to edit the
   parameters that apply to all the voices of ADDsynth.
   AddSynth is the most complex and feature filled engine,
   and so is split up into various context levels.
   Although based on well-known additive synthesis,
   \textsl{Yoshimi} extends that considerably.

   ADDsynth, a primarily additive synthesis engine, is one of the three major
   synthesis engines available in \textsl{Yoshimi}/\textsl{ZynAddSubFX}.
   The basic concept of this engine is the summation of a collection of voices,
   each of which consists of oscillators.

   "ADDsynth" (sometimes spelled "ADsynth") or "ADnote" is a complex engine
   which makes sounds by adding a number of voices. Each one has filters,
   envelopes, LFOs, morphing, modulation, resonance, etc.
   Each voice includes a very powerful
   waveform generator with up to 128 sine/non-sine harmonics. One can use
   Fourier synthesis, or if one doesn't like it, one can use
   wave-shaping/filtering of functions. This engine includes anti-aliasing.
   Modulation includes ring modulation, phase modulation, and more.
   The modulators can have any shape.
   \cite{zyndoc}

   There are two oscillators per voice: the \index{voice oscillator} voice
   oscillator and the \index{modulation oscillator} modulation oscillator. For
   each of these two oscillators, there are three alternatives: one can have a
   locally defined oscillator (internal), an oscillator defined in a
   lower-numbered voice or use a lower-numbered voice itself as an oscillator
   (that voice must be enabled).

   In the voice window, each of the two oscillators has a \textbf{Source} and a
   \textbf{Local Oscillator}. The \textbf{Source} gives the choice of previous
   voices as oscillators and the \textbf{Local Oscillator} gives the choice
   of previously defined oscillators and the current (internal) oscillator.

   The sum of the voices are passed through filters and amplification to
   produce the final sound. This could lead one to think that ADDsynth is just
   a bunch of minor post-processing, and at this level much of the sound
   generation is hidden.

\begin{figure}[H]
   \centering
%  \includegraphics[scale=1.0]{2.0/AddSynth.png}
   \includegraphics[scale=0.65]{2.1.2/global.png}
   \caption{ADDsynth Edit/Global Dialog}
   \label{fig:addsynth_edit_dialog}
\end{figure}

   The major sections of this dialog are listed:

   \begin{enumber}
      \item \textbf{AMPLITUDE} (stock section)
      \item \textbf{FILTER} (stock section)
      \item \textbf{FREQUENCY} (stock section)
      \item \textbf{Show Voice Parameters} (section)
      \item \textbf{Show Voice List} (section)
      \item \textbf{Resonance} (stock section)
      \item \textbf{C}
      \item \textbf{P}
      \item \textbf{Close}
   \end{enumber}

   This complex dialog is best described section by section.
   Many of the sub-sections are stock sub-panels described elsewhere
   in this document.  References to those sections are included.

\subsection{ADDsynth / AMPLITUDE}
\label{subsec:addsynth_amplitude}

   \begin{enumber}
      \item \textbf{Volume}
      \item \textbf{Vel Sens}
      \item \textbf{Pan}
      \item \textbf{Rand}
      \item \textbf{Width}

       The controls above are discussed in detail in
       \sectionref{subsec:volume_panning}. However their values for ADDsynth
       are as below.

      \item \textbf{Amplitude Env}
         The Amplitude Env panel is described in detail in
         \sectionref{subsubsec:amplitude_envelope_subpanel}.
      \item \textbf{Amplitude LFO}
         The Amplitude LFO panel is described in detail in
         \sectionref{subsubsec:lfo_user_interface_panels}.

      \item \textbf{Stereo}
      \item \textbf{Rnd Grp}
      \item \textbf{D.Pop}
      \item \textbf{P.Str.}
      \item \textbf{P.t}
      \item \textbf{P.Stc.}
      \item \textbf{P.Vel.}
      \item \textbf{Detune}
      \item \textbf{Octave}
      \item \textbf{relBW}
      \item \textbf{Detune Type}
      \item \textbf{Coarse Detune}
   \end{enumber}

   Note the two sub-panels, mentioned above, that are described elsewhere.
   They will not be discussed in detail below.

   \setcounter{ItemCounter}{0}      % Reset the ItemCounter for this list.

   \itempar{Volume}{addsynth!volume}

   Values: \texttt{-60dB to 19.4dB, -3.8dB*}

   \itempar{Vel Sens}{addsynth!vel sens}

   Values: \texttt{-48dB to -0.8dB, disabled, -6.02dB*}

   \itempar{Pan}{addsynth!pan}

   Values: \texttt{100\% left to 100\% right, centered*}

   \itempar{Rand}{addsynth!random pan}

   Values: \texttt{off*, on}

   \itempar{Width}{addsynth!random width}

   Values: \texttt{0 to 100\%* }

   Next, we skip the \textbf{Amplitude Env} and \textbf{Amplitude LFO}
   panels, which are described elsewhere, as noted above.

   \itempar{Stereo}{addsynth!Stereo}
   ADDsynth Stereo.
   Stereo can be enabled.
   When disabled, all the voices will also have panning disabled.
   \textbf{Stereo} determines whether the whole of this engine is to be Stereo or
   Mono. If the box is not checked (i.e. Mono), there will be no \textsl{spread}
   to the sound, but the Panning controls will not be affected, and neither will
   the stereo spread of any part effects.

   Values: \texttt{Off, On*}

   \itempar{Rnd Grp}{addsynth!group}
   ADDsynth Random Group.
   \textbf{Rnd Grp} Normally each voice> has it's own harmonic amplitude
   random element. If this is checked, then all voices that use the same waveform
   oscillator will be grouped so they have the same harmonic randomness, assuming
   this has been set in the Waveform window.
   It disables harmonic amplitude randomness of voices with a common oscillator.
   There are many per-voice random elements that can give the sound more
   'depth'. When this control is checked, the voices all sound together
   instead.

   Values: \texttt{Off*, On}

   \itempar{D. Pop}{addsynth!de-pop}
   ADDsynth De-Pop.
   This sets the time of a very short attack ramp up,
   to suppress the click some sounds might produce.

   \itempar{P.Str}{addsynth!punch strength}
   ADDsynth Punch Strength.
   The punch strength of a note in ADDsynth is a constant amplification to
   the output at the start of the note, with its length determined by the
   punch time and stretch and the amplitude being determined by the punch
   strength and velocity sensing. The \textbf{relBW}
   control in the frequency panel is
   effectively a multiplier for detuning all voices within an ADnote.

   Values: \texttt{0* to 127}

   \itempar{P.t}{addsynth!punch time}
   ADDsynth Punch Time (duration).
   Sets the punch effect duration (from 0.1 ms to 100 ms on an A note, 440Hz).

   Values: \texttt{0 to 127, 64*}

   \itempar{P.Stc}{addsynth!punch stretch}
   ADDsynth Punch Stretch.
   Sets the punch effect stretch according to frequency. On lower-frequency
   notes, punch stretch makes the punch effect last longer.

   Values: \texttt{0 to 127, 64*}

   \itempar{P.Vel}{addsynth!punch vel sens}
   ADDsynth Punch Velocity Sensing.
   The higher this value, the higher the effect of velocity on the punch of
   the note.

   Values: \texttt{0 to 127, 72*}

   \itempar{Detune}{addsynth!detune}
   ADDsynth Detune.
   Detunes the engine by a small amount.

   Values: \texttt{0 to 127, 72*}

   \itempar{Octave}{addsynth!octave}
   ADDsynth Octave.
   Allows a change of the overall pitch up or down by up to eight octaves.

   Values: \texttt{0 to 8, 0*}

   \itempar{relBW}{addsynth!relbw}
   ADDsynth relBW (relative bandwidth).
   Modifies the amount individual voice and unison detunes can deviate
   from the overall value.

   Values: \texttt{0 to 127, 72*}

   \itempar{Detune Type}{addsynth!detune type}
   ADDsynth Detune Type.
   The menu options control the amount of detune of the slider and whether it is a
   linear change or exponential (i.e. has more effect the further away it is from
   the centre position).

   Values: \texttt{0 to 127, 72*}

   \itempar{Coarse Detune}{addsynth!coarse detune}
   ADDsynth Coarse Detune.
   The inner single arrows make a coarse step change to the detune, and the outer
   double ones change by ten times as much.

   Values: \texttt{0 to 127, 72*}

\subsection{ADDsynth / FILTER}
\label{subsec:addsynth_filter}

   The ADDsynth FILTER block consists solely of sub-panels
   described in detail in the sections noted below.  The
   sub-panels of the FILTER section are:

   \begin{enumber}
      \item \textbf{Filter Params}
      \item \textbf{Filter Env}
      \item \textbf{Filter LFO}
   \end{enumber}

   \setcounter{ItemCounter}{0}      % Reset the ItemCounter for this list.

   \itempar{Filter Params}{addsynth!filter params}
   ADDsynth Filter Parameters.
   The Filter Params panel is described in detail in
   \sectionref{subsubsec:filter_parameters_user_interface}.

   \itempar{Filter Env}{addsynth!filter env}
   ADDsynth Filter Envelope.
   The Filter Env panel is described in detail in
   \sectionref{subsubsec:envelope_settings_for_filter}.

   \itempar{Filter LFO}{addsynth!filter lfo}
   The Filter LFO panel is described in detail in
   \sectionref{subsubsec:lfo_user_interface_panels}.

\subsection{ADDsynth / FREQUENCY}
\label{subsec:addsynth_frequency}

   \begin{enumber}
      \item \textbf{Detune}
      \item \textbf{FREQUENCY slider}
      \item \textbf{Octave}
      \item \textbf{RelBW}
      \item \textbf{Frequency Env}.
         A stock sub-panel described in
         \sectionref{subsubsec:envelope_settings_for_frequency}.
      \item \textbf{Frequency LFO}
         A stock sub-panel described in
         \sectionref{subsubsec:frequency_lfo_sub_panel}.
      \item \textbf{Detune Type}
      \item \textbf{Coarse det.}
   \end{enumber}

   \setcounter{ItemCounter}{0}      % Reset the ItemCounter for this list.

   \itempar{Detune}{addsynth!detune value}
   ADDsynth Detune Value.
   This display box shows the value of the detune as selected by the
   frequency slider described below.

\begin{figure}[H]
   \centering
   \includegraphics[scale=1.0]{bottom-panel/instrument-edit/ADD/frequency-detune-type.jpg}
   \caption{ADDsynth Frequency Detune Type}
   \label{fig:addsynth_freq_detune_type}
\end{figure}

   This value defines the number of cents that define the range of the
   \textbf{FREQUENCY} slider, that is 35 cents, 10 cents, 100 cents (one
   semitone), or 1200 cents (1 octave), below and above the main
   frequency.  The default is 35 cents.  The 1200-cents setting provides a
   whole octave of detuning in either direction.

   The "L" stands for "linear", and the "E" for "exponential", to
   describe how the detune slider acts.

   Values: \texttt{Default*, L35cents, L10cents, E100cents, E1200cents}

   \itempar{FREQUENCY slider}{addsynth!freq slider}
   ADDsynth Fine Detune (cents), a slider control.
   While the detune type dropdown and the octave selection provide a coarse
   selection of detune, the slider allows for a finer selection of detune,
   up to roughly one-third of a semitone.

   Values:
      \texttt{-35.00 to 35.00},
      \texttt{-10.00 to 10.00},
      \texttt{-100.00 to 100.00},
      \texttt{-1200.00 to 1200.00}

   \itempar{Octave}{addsynth!octave}
   ADDSynth Octave.
   The octave setting changes the frequency by octaves.

   Values: \texttt{-8 to 7, 0*}

   \itempar{RelBW}{addsynth!relative bw}
   ADDSynth Relative Bandwidth.
   Bandwidth: how the relative fine detune of the voice is changed.

   Values: \texttt{0 to 127, 64*}

   \itempar{Frequency Env}{addsynth!frequency env}
   ADDsynth Frequency Envelope.
   The Frequency Env panel is described in detail in
   \sectionref{subsubsec:envelope_settings_for_frequency}.

   \itempar{Frequency LFO}{addsynth!frequency lfo}
   The Frequency LFO panel is described in detail in
   \sectionref{subsubsec:lfo_user_interface_panels}

   \itempar{Detune Type}{addsynth!detune type}
   Frequency Detune Type.
   This setting provides a coarse detuning.
   We would welcome an explanation of exactly is meant by the numbers and
   the "E" versus "L" designation.

   Values: \texttt{L35cents, L10cents, E100cents, E1200cents}

   \itempar{Coarse det}{addsynth!coarse detune}
   Coarse Detune, "C.detune".
   The one-arrow buttons change the value by one.
   The two-arrow buttons change the value by ten.
   Again, we need a way to explain the interactions of the slider, the
   octave setting, the detune type, and the coarse detune settings.

   Values: \texttt{-64 to 63, 0*}

   \itempar{Show Voice Parameters}{addsynth!voice parameters}
   ADDsynth Show Voice Parameters.
   This button brings up the "voice parameters" dialog discussed in the next
   section.

\subsection{ADDsynth / Voice Parameters}
\label{subsec:addsynth_voice_parameters}

   Again, this dialog is built from some stock sections and stock
   sub-panels, plus additional elements.

   Each \textsl{Yoshimi} ADDsynth instrument consists of up to 8 voices
   that can all interact with each other to some degree. Each of those voices
   has a waveform oscillator, and, optionally a modulation one with five
   different modulation types. It also has "Unison" capability (effectively
   sub-voices).

   The oscillators themselves have a wide range of waveform shaping controls,
   almost all of which can be changed in real time.

    The image shows your entry point with all the global controls. Much of this
    actually consists of standardised inserts that are used across all engines.
    Detailed descriptions of these are provided in the following sections.
    This dialog provides a way to define each of the 8 voices in great detail.
    By default, an ADDsynth instrument consists of one voice, voice 1.

\begin{figure}[H]
   \centering
%  \includegraphics[scale=0.75]{2.0/AddVoice.png}
   \includegraphics[scale=0.65]{2.1.2/voice.png}
   \caption{ADDsynth Voice Parameters Dialog}
   \label{fig:addsynth_voice_parameters_dialog}
\end{figure}

   Note the 8 \textbf{VOICE} tabs at the top of the window.
   These make it easy to switch to another voice, and without opening up yet
   another editing window.
   Non-active voices are shown with an inactive tab/number appearance.
   This is all in synchrony with the \textbf{Voice List} window.
   The tabs select the particular voice to manage, and its modulator section.
   The tab colour changes to cyan to indicate it is the selected one.
   Active voices show their tab \textsl{number} in black to show
   at a glance which ones have been enabled.

   This dialog consists of a few extra settings, plus a number of
   stock dialog sections.  Take some time to compare
   \figureref{fig:addsynth_edit_dialog},
   which covers the overall instrument, with
   \figureref{fig:addsynth_voice_parameters_dialog},
   which covers each of the voices.
   The stock sections in the former cover the whole instrument as one,
   while the very similar stock sections in the latter cover only the
   voice they configure.
   Obviously, the combinations of settings are essentially endless.

   Each voice can be amplitude-controlled, filter-controlled, and
   frequency-controlled.  Each voice can also be modulated by a
   modulator.

  Another property of the voice is that one can tell \textsl{Yoshimi} to
  import a given lower-numbered voice, either as an oscillator, or as a
  modulator.

   \begin{enumber}
      \item \textbf{Voice Number Tab}
      \item \textbf{On}
      \item \textbf{Delay}
      \item \textbf{Resonance}
      \item \textbf{AMPLITUDE} (see the stock-panel section below)
      \item \textbf{FILTER} (see the stock-panel section below)
      \item \textbf{MODULATOR} (see the stock-panel section below)
      \item \textbf{FREQUENCY} (see the stock-panel section below)
   \end{enumber}

   \setcounter{ItemCounter}{0}      % Reset the ItemCounter for this list.

   \itempar{Voice Number Tab}{voice!number}
   ADDsynth Voice Number.
   When highlighted this indicates the voice currently being viewed.  Each
   \textsl{Yoshimi} part/instrument can consist of up to eight voices. The
   voice being worked on can be selected using the \textbf{Current Voice} tab.

   Values: \texttt{1* to 8}

   \itempar{On}{voice!on/off}
   ADDsynth Voice On/Off.
   Enables this voice in the part/instrument.
   It enables or disables the entire voice, including it's modulator.
   Also, with the exception of \textbf{Volume/Panning},
   all inserts can be switched on or off.

   Values: \texttt{Off, On}

   \itempar{Delay}{voice!delay}
   ADDsynth Voice Delay.
   Individual voices can be delayed against the overall note on point.
   However, if one is delayed beyond the note release point, it will never sound.

   Values: \texttt{0* to 4.90 s}

   \itempar{Resonance}{voice!resonance}
   ADDsynth Voice Resonance On/Off.
   This determines whether overall resonance is applied to a particular voice.

   The rest of the GUI elements
   (AMPLITUDE, FILTER, MODULATOR, FREQUENCY, and Voice Oscillator)
   are more detailed, and discussed in the sections that follow.

   Values: \texttt{Off, On*}

\subsubsection{ADDsynth / Voice Parameters / AMPLITUDE}
\label{subsubsec:addsynth_voice_parameters_amplitude}

   This section of the voice parameters dialog also includes a couple of
   stock sub-panels that have an additional "Enable" control.

   \begin{enumber}
      \item \textbf{Minus}
      \item \textbf{Volume}
      \item \textbf{Vel Sens}
      \item \textbf{Pan}
      \item \textbf{Rand}
      \item \textbf{Width}

      The controls above are discussed in detail in
       \sectionref{subsec:volume_panning}. However their values for ADDsynth
       Voice are as below.

      \item \textbf{Amplitude Env, Stock + Enable}
      \item \textbf{Amplitude LFO, Stock + Enable}
   \end{enumber}

   \setcounter{ItemCounter}{0}      % Reset the ItemCounter for this list.

   \itempar{Minus}{voice par amp!invert phase}
   ADDsynth Amplitude Minus.
   This control inverts the phase of the waveform relative to
   all the other voices.
   With only one voice enabled, this control will seem to do nothing.
   With two voices enabled with \textsl{identical} waveforms, the
   \textbf{Minus} control will indeed seem to reverse the effect of the volume
   control. But if the waveforms are different, then it can provide some
   interesting harmonic change effects.

   Values: \texttt{Off*, On}

   \itempar{Volume}{voice par amp!volume}

   Values: \texttt{off, -57.6dB to 0dB, -12.8dB*}

   \itempar{Vel Sens}{voice par amp!vel sens}

   Values: \texttt{-48dB to -0.8dB, disabled*}

   \itempar{Pan}{voice par amp!pan}

   Values: \texttt{100\% left to 100\% right, centered*}

   \itempar{Rand}{voice par amp!random pan}

   Values: \texttt{off*, on}

   \itempar{Width}{voice par amp!random width}

   Values: \texttt{0 to 100\%* }

   \itempar{Amplitude Env, Stock + Enable}{voice par amp!amp env}
   ADDsynth Amplitude Envelope Sub-panel.
   See \sectionref{subsubsec:amplitude_envelope_subpanel}.
   Additionally, the \textbf{Enable} checkbox allows the enabling of this
   component.

   \itempar{Amplitude LFO, Stock + Enable}{voice par amp!amp lfo}
   ADDsynth Amplitude LFO Sub-panel.
   See \sectionref{subsubsec:lfo_user_interface_panels}.
   Additionally, the \textbf{Enable} checkbox allows the enabling of this
   component.

\subsubsection{ADDsynth / Voice Parameters / FILTER}
\label{subsubsec:addsynth_voice_parameters_filter}

   This section of the voice parameters dialog also includes a couple of
   stock sub-panels that have an additional "Enable" control.

   \begin{enumber}
      \item \textbf{Enable}
      \item \textbf{Bypass Global F.}
      \item \textbf{Filter Params, Stock}
      \item \textbf{Filter Env, Stock + Enable}
      \item \textbf{Filter LFO, Stock + Enable}
   \end{enumber}

   \setcounter{ItemCounter}{0}      % Reset the ItemCounter for this list.

   \itempar{Enable}{voice par filter!enable}
   ADDsynth Voice Enable Filter.
   This value enables the whole FILTER dialog section.

   Values: \texttt{Off*, On}

   \itempar{Bypass Global F}{voice par filter!bypass}
   ADDsynth Voice Bypass Global Filter.
   The voice signal bypasses the global filter,
   and only uses the filter block for the voice;
   otherwise it is applied, but before the global one.

   Values: \texttt{Off*, On}

   \itempar{Filter Params, Stock}{voice par filter!parameters}
   See \sectionref{subsubsec:filter_parameters_user_interface}.

   \itempar{Filter Env, Stock + Enable}{voice par filter!env}
   See \sectionref{subsubsec:envelope_settings_for_filter}.

   \itempar{Filter LFO, Stock + Enable}{voice par filter!lfo}
   See \sectionref{subsubsec:filter_lfo_sub_panel}.

\subsubsection{ADDsynth / Voice Parameters / FREQUENCY}
\label{subsubsec:addsynth_voice_parameters_frequency}

   This frequency section is almost a stock part.
   It is similar to the ADDsynth Edit's \textbf{FREQUENCY} section.

   \begin{enumber}
      \item \textbf{Detune}
      \item \textbf{FREQUENCY slider}
      \item \textbf{Bend}
      \item \textbf{Offset}
      \item \textbf{440Hz}          % addition to stock
      \item \textbf{Eq.T}          % addition to stock
      \item \textbf{Octave}
      \item \textbf{Detune Type}
      \item \textbf{Coarse det}
      \item \textbf{Frequency Env, Stock + Enable}
      \item \textbf{Frequency LFO, Stock + Enable}
      \item \textbf{Voice Oscillator}
   \end{enumber}

   \setcounter{ItemCounter}{0}      % Reset the ItemCounter for this list.

   \itempar{Detune}{voice parameters!detune}
   Voice Parameters Detune.
   Shows the value selected by the frequency slider.

   \itempar{FREQUENCY slider}{voice parameters!freq slider}
   Frequency Slider.
   Provides fine detune, in cents.
   Note that 35 cents is roughly one-third of a semitone.

   Values: \texttt{-35.00 to 35.00, 0*}

   \itempar{Bend}{voice parameters!bend}
   Bend.
   It modifies the pitch bend control.  It is possible to make the pitch
   bend control work in the opposite direction.

   \itempar{Offset}{voice parameters!offset}
   Offset.
   It shifts the overal pitch of the engine (up or down) relative to the rest
   of the engines.

   \itempar{440Hz}{voice parameters!440 hz}
   440 Hz Selection.
   Fixes the voice base frequency to 440 Hz.
   One can adjust this with the detune settings.
   No matter what key is played on the keyboard, this voice will emit only
   440 Hz.  This is useful for defining a constant frequency to use as a
   modulator for the other voices in the part.
   For example, one can define voice 1 to be a tone, then
   define voice 2 to be 440 Hz.  The two voices will mix, but only voice 1
   will change frequencies as different keys are played.

   Values: \texttt{Off*, On}

   \itempar{Eq.T}{voice parameters!eq type}
   Equal Temperament
   This item is enabled only if the \textbf{440Hz} check-box is enabled.
   If this is greater than zero it modifies the effect of the 440Hz checkbox. The A4
   key remains at 440Hz, but the frequency of the other keys vary according to the
   key pressed. When set to the middle of the value range (64), the step size is
   exactly like the classical equal temperament, i.e. one note step for one semitone
   and 12 steps will double the frequency.

   Values: \texttt{0 to 127}

   \itempar{Octave}{voice parameters!octave}
   Voice Parameters Octave.

   Values: \texttt{-8 to 7, 0*}

% Where did I get this setting?  An older version of the dialog?
%
%  \itempar{RelBW}{voice parameters!relbw}
%  Relative bandwidth.

   \itempar{Detune Type}{voice parameters!fine detune}
   Detune Type.

\begin{figure}[H]
   \centering
   \includegraphics[scale=1.0]{bottom-panel/instrument-edit/ADD/frequency-detune-type.jpg}
   \caption{Frequency Detune Type}
   \label{fig:frequency_detune_tYpe}
\end{figure}

   Values: \texttt{Default*, L35cents, L10cents, E100cents, E1200cents}

   \itempar{Coarse det}{voice parameters!coarse detune}
   Coarse Detune.
   Is this setting in units of semitones?
   \index{todo!coarse detune units}

   Values: \texttt{-64 to 63, 0*}

   \itempar{Frequency Env, Stock + Enable}{voice parameters!freq env}
   Frequency Envelope.
   See \sectionref{subsubsec:envelope_settings_for_frequency}.

   \itempar{Frequency LFO, Stock + Enable}{voice parameters!freq lfo}
   Frequency LFO.
   See \sectionref{subsubsec:frequency_lfo_sub_panel}.

   \itempar{Voice Oscillator}{voice parameters!oscillator}
   Voice Parameters Oscillator.
   See the next section.

\subsubsection{ADDsynth / Voice Parameters / UNISON}
\label{subsubsec:addsynth_voice_parameters_unison}
   Enabling this item causes the Unison-related items to become
   activated.

   \begin{enumber}
      \item \textbf{On}
      \item \textbf{Size}
      \item \textbf{Frequency Spread}
      \item \textbf{Ph.rnd}
      \item \textbf{Stereo}
      \item \textbf{Vib.}
      \item \textbf{V.speed}
      \item \textbf{Invert}
   \end{enumber}

   \setcounter{ItemCounter}{0}      % Reset the ItemCounter for this list.

   \itempar{On}{unison!enable}
   Enables or disables unison. When disabled the size is always set at 2.

   Values: \texttt{Off*, On}

   \itempar{Unison Size}{unison!size}
   Sets the number of unison sub-voices.

   Values: \texttt{2* to 50}

   \itempar{Unison Frequency Spread}{unison!freq spread}
   Frequency spread of the unison (cents).

   Values: \texttt{0 to 200, 44.6*}

   \itempar{Phase Randomness}{unison!Ph.rnd}
   Unison Phase Randomness.

   Values: \texttt{0 to 127*}

   \itempar{Stereo Spread}{unison!Stereo}
   Unison Stereo Spread.

   Values: \texttt{0 to 127, 64*}

   \itempar{Unison Vibrato}{unison!Vibrato}
   Unison Vibrato.

   Values: \texttt{0 to 127, 64*}

   \itempar{Vibrato Speed}{unison!V.speed}
   Unison Vibrato Average Speed.

   Values: \texttt{0 to 127, 64*}

   \itempar{Phase Invert}{unison!Invert}
   Unison Phase Invert.
   Values: \texttt{None*, Random, 50\%, 33\%, 25\%, 20\%}

\begin{figure}[H]
   \centering
   \includegraphics[scale=1.0]{bottom-panel/instrument-edit/ADD/voice-oscillator-phase-invert-dropdown.jpg}
   \caption{Unison Phase Invert Dropdown}
   \label{fig:phase_invert_dropdown}
\end{figure}


\subsubsection{ADDsynth / Voice Parameters / Voice Oscillator}
\label{subsubsec:addsynth_voice_parameters_oscillator}

   The ADDsynth Voice Oscillator panel is tucked in the lower left side of the
   ADDSynth Voice Parameters editor.

   \begin{enumber}
     \item \textbf{Voice}
      \item \textbf{Sound}
      \item \textbf{Waveform graph}
      \item \textbf{Use}
      \item \textbf{Waveform} (was Change)
      \item \textbf{Phase}
      \item \textbf{C}
      \item \textbf{P}
      \item \textbf{Close Window}
   \end{enumber}


   \setcounter{ItemCounter}{0}      % Reset the ItemCounter for this list.

  \itempar{Source}{voice oscillator!Source}
  ADDSynth Voice Import. Selects whether to import a lower-numbered voice as
  oscillator for this voice, or to generate a local voice. All parameters from
  the imported voice remain in effect, except for volume, panning, base
  frequency and pitch bend scaling factor. The voice is also converted to mono.
  Parameters in the current voice will then tweak the signal further.

   \itempar{Sound}{voice oscillator!sound}
   ADDSynth Oscillator Type (sound/noise).
   Sound/Noise choice.
   Select the mode of the oscillator (sound versus white noise).

   Values: \texttt{Sound* (green), Noise (black), Noise (pink), Noise (cyan)}

   The noise types are, respectively, white noise, pink noise, and spot noise, as
   explained below.

\begin{figure}[H]
   \centering
   \includegraphics[scale=1.0]{1.6.0/voice_oscillator_sound_dropdown.png}
   \caption{Voice Oscillator Choices}
   \label{fig:voice_oscillator_choices}
\end{figure}

   \itempar{Waveform graph}{voice oscillator!waveform}
   Waveform Graph.
   Shows a period of the currently configured oscillator.

\begin{figure}[H]
   \centering
   \includegraphics[scale=0.75]{1.5.7/sound_banner.png}
   \caption{Oscillator in ADDSynth Voice}
   \label{fig:voice_oscillator_oscillator}
\end{figure}

   If white \textbf{Noise} is selected, then the waveform graph simply announces
   "White Noise".  Also, the \textbf{Unison, Frequency, Modulator} controls are
   all disabled.

\begin{figure}[H]
   \centering
   \includegraphics[scale=0.75]{1.5.7/white_noise_banner.png}
   \caption{White Noise in ADDSynth Voice}
   \label{fig:voice_oscillator_white_noise}
\end{figure}

   If the pink \textbf{Noise} entry is selected,
   then the waveform graph simply announces "Pink Noise".
   Also, the \textbf{Unison, Frequency, Modulator} controls are
   all disabled.

\begin{figure}[H]
   \centering
   \includegraphics[scale=0.75]{1.5.7/pink_noise_banner.png}
   \caption{Pink Noise in ADDSynth Voice}
   \label{fig:voice_oscillator_pink_noise}
\end{figure}

\index{new!spot noise}
   If the cyan (spot) \textbf{Noise} entry is selected,
   then the waveform graph simply announces
   "Spot Noise".  Also, the \textbf{Unison, Frequency, Modulator} controls are
   all disabled.
   Spot noise is based on white noise but has a very broken sound. Combined
   with envelope shaping this is useful for adding 'grit' or sizzle,
   particularly for percussion.

\begin{figure}[H]
   \centering
   \includegraphics[scale=0.75]{1.6.0/spot_noise_banner.png}
   \caption{Spot Noise in ADDSynth Voice}
   \label{fig:voice_oscillator_spot_noise}
\end{figure}

   \itempar{Use (oscillator)}{voice oscillator!use}
   Use Oscillator.
   If the \textbf{Current Voice} is set to a value greater than 1, meaning
   that one is editing additional voices, then this dropdown item also
   includes the values of all oscillators less then this one, marked as
   "Voice n", where "n" is the voice number.
   For example, if one is currently editing current voice 3,
   then the dropdown list includes \textbf{Internal}, \textbf{Voice 1}, and
   \textbf{Voice 2}.

  Unlike Source, using an oscillator from a different voice
  only imports the waveform, not any other parameters.

   Values: \texttt{Internal*, Other oscillators ("Voice n")}

   \itempar{Waveform}{voice oscillator!Waveform}
   ADDSynth Voice Oscillator Waveform.
   This button brings up the ADDsynth Oscillator Editor dialog.
   This dialog is described elsewhere; it used to be called
   \textbf{Change}.

   \itempar{Phase}{voice oscillator!phase}
   Voice Oscillator Phase.

   Values: \texttt{-90, 0*, 88.6 (degrees)}

\subsubsection{ADDsynth / Voice Parameters / MODULATOR}
\label{subsubsec:addsynth_voice_parameters_modulator}

   \begin{enumber}
      \item \textbf{Type:}
      \item \textbf{Modulator Source}
      \item \textbf{Mod AMPLITUDE}
      \item \textbf{Mod FREQUENCY}
      \item \textbf{Local Oscillator}
   \end{enumber}

   \setcounter{ItemCounter}{0}      % Reset the ItemCounter for this list.

   \itempar{Type:}{modulator!type}
   ADDsynth Modulator Type.

\begin{figure}[H]
   \centering
   \includegraphics[scale=0.75]{bottom-panel/instrument-edit/ADD/modulator-type.jpg}
   \caption{Voice Modulator Type}
   \label{fig:voice_modulator_type}
\end{figure}

   \begin{enumber}
      \item \textbf{OFF}.
         This setting turns off the modulator.
      \item \textbf{MORPH}
         \index{MORPH}
         \index{morph modulator}
         \index{modulator!morph}
         The morph modulator works by combining the output of the voice oscillator
         and the modulator opscillator into one sonud, with the amplitude envelope
         translating between one waveform and the other. It is important that the
         amplitude envelope is enabled, otherwise no change will take place.

      \item \textbf{RING}
         \index{RING}
         \index{ring modulator}
         \index{modulator!ring}
         The ring modulator is useful for making bell-like sounds and some
         weird effects.  The ring modulator works by multiplying two
         waveforms together, producing a signal that possesses the sum and
         difference of the frequencies present in the waveforms.  The
         ins-and-outs of the ring modulator are explained in detail in
         \paragraphref{paragraph:tips_using_the_ring_modulator}.
      \item \textbf{PM}
         \index{PM}
         \index{phase modulator}
         \index{modulator!phase}
         The PM (phase modulation) modulator works by using a modulator
         envelope to change the phase of the voice oscillator. It produces a
         similar effect to frequency modulation.
         Generally, set \textbf{F.Damp} to zero, so that the modulation amount
         doesn't depend on the note number.
      \item \textbf{FM}
         \index{FM}
         \index{frequency modulator}
         \index{modulator!frequency}
         The (frequency modulation) morph modulator works by modulating the
         frequency.  Examples can be heard in the "Ethereal" and "Steel Wire"
         instruments.
      \item \textbf{PWM}
         \index{PWM}
         \index{PWM modulator}
         \index{modulator!PWM}
         The pulse width modulator works by pulse-width modulation.
   \end{enumber}

   Values: \texttt{OFF, MORPH, RING, PM, FM, PWM}

   \itempar{Modulator Source}.
   AddSynth Modulator Source.
   \index{modulator!source}
   Use another voice as a modulator instead of the modulator of the internal
   voice. One can make a "modulation stack". All parameters from the imported
   voice remain in effect, except for volume, panning, base frequency and pitch
   bend scaling factor. The voice is also converted to mono. Parameters in the
   current voice will then tweak the modulator further.  One can only select
   Local (i.e. Internal), or one from a lower numbered voice.

   This feature allows one of the voices (of the up to 8 allowed in a single
   ADDsynth instrument) to be used as a modulator or external oscillator for
   another voice in the instrument.
   Note that the voice must be one with a number \textsl{below} the current
   voice.
   It's important to understand that oscillators always exist even if not
   used.
   \index{oscillator!local}
   This option specifies to use the oscillator of another voice or
   the \textsl{local} oscillator.

   Values: \texttt{Local*, "Mod n"}

   The parameters must be lower than the voice index; one cannot use the
   oscillator from a voice with a bigger index (e.g. one can't use the
   oscillator of voice 8 for voice 4). This is very useful because, if
   one uses many voices with the same oscillator settings, one can use only
   one oscillator and select other voices to use this, and if one changes a
   parameter of this oscillator, all voices using this oscillator will be
   affected.

   If one sets up voice 2 as a square wave, and voice 1 as a triangle wave,
   then sets voice 3 to voice 2, voice 3 will get a square wave.

   If one then sets voice 2 to voice 1, voice 2 will get a triangle wave but
   voice 3 will still get a square wave.

   Voice 3 can use the oscillator from voice 1,
   \textsl{even if voice 1 is switched off}.

   Modulator 3 can use the oscillator from modulator 1, even if modulator 1 is
   switched off, but modulator 3 can't use voice 1 if voice 1 is
   switched off.

   However, if voice 2 is using the oscillator from voice 1, and modulator 3 is
   using voice 2, it will still get voice 2 oscillator.

   When a voice or modulator is pointed to another voice/modulator, the
   oscillator window will show the waveform of the actual source, and all the
   controls will change this, not the internal oscillator.

   \textbf{Local}.
   Uses the local (internal) oscillator as the modulator of another voice.

   Values: \texttt{Local*, Other voice numbers}

   \itempar{Mod AMPLITUDE}{modulator!amplitude}
   Modulator Amplitude.

   \begin{enumber}
      \item \textbf{Vol}
         Volume.
         Values: \texttt{0 to 127, 90*}
      \item \textbf{V.Sns}
         Velocity Sensing Function; set to rightmost/max to disable.
         Values: \texttt{0 to 127, 64*}
      \item \textbf{F.Damp}
         Modulator Damp at higher frequency.
         How the modulator intensity is lowered according to lower/higher
         note frequencies.
         Values: \texttt{0 to 127, 90*}
      \item \textbf{Amplitude Env, Stock + Enable}
         See \sectionref{subsubsec:amplitude_envelope_subpanel}.
   \end{enumber}

   \itempar{Mod FREQUENCY}{modulator!frequency}
   Modulator Frequency.

% TODO FOR 1.6.0:  Uncomment the commented text.

   \begin{enumber}
     \item \textbf{Follow voice}
        Applies all detuning in the main voice oscillator to the modulator as
        well. If turned off, the modulator will be completely unaffected by all
        detuning in the main voice oscillator, including detuned unison voices.
     \item \textbf{440Hz}
        Use 440Hz as base frequency for the modulator.
      \item \textbf{Detune slider}
         Fine Detune (cents).
         Values: \texttt{-35.00 to 35.00, 0*}
      \item \textbf{Detune Type}
         Fine Detune (cents).
         Values: \texttt{L35cents, L10cents, E100cents, E1200cents}
         See \figureref{fig:addsynth_freq_detune_type}.
      \item \textbf{Octave}
         Octave.
         Values: \texttt{-8 to 7, 0*}
      \item \textbf{Coarse Det.}
         Coarse Detune.
         Values: \texttt{-64 to 63, 0*}
      \item \textbf{Filter Env, Stock + Enable}
         See \sectionref{subsubsec:envelope_settings_for_filter}.
   \end{enumber}

   \itempar{Local Oscillator}{modulator!oscillator}
   Local Oscillator.
   Provides the modulator for the oscillator.
   This value must indicate a modulator from a voice with a number less than
   the current voice. Entries are greyed out for Voice 1, as there is no
   lower-numbered voice available for modulation.

   Values: \texttt{Internal*, "Mod n"}

   \begin{enumber}
      \item \textbf{Waveform}
         ADDsynth Oscillator Editor.
      \item \textbf{Use}
         Oscillator to Use.
         See the paragraph below.
         Values: \texttt{Internal*, Other oscillators?}
      \item \textbf{Phase}
         Oscillator Phase.

         Values: \texttt{-90, 0*, 88.6 (degrees)}
      \item \textbf{Waveform graph}
         Waveform graph.
   \end{enumber}

   One has the choice between \textbf{Internal}, which in this case means a
   completely independent modulator oscillator per voice (extra waveform button),
   or \textbf{Mod. (n)}, which refers to the modulation oscillators one has
   already defined for the voices with a lower index.
   The voice of lower index doesn't need to be enabled in order to be used as a
   modulator. This means one can make one modulation oscillator for voice 1, and
   reuse it in voices 2 and 3.  This is the same system used for the normal (voice)
   oscillators.

\paragraph{Tip: Using the Ring Modulator}
\label{paragraph:tips_using_the_ring_modulator}
\index{tips!ring modulator}
\index{tips!internal modulator}

   This section is derived from one of the short text files in the
   \textsl{Yoshimi} source-code bundle (\cite{yoshimi} or \cite{yoshimi2}).
   It notes that "Some people have
   been confused about how to use an 'external' Mod Oscillator", and
   provides usage notes that we will elaborate on here.  Here is the way to
   use the ring modulator:

   \begin{enumber}
      \item Open the ADDsynth editing window.  Then open
         \textbf{Show Voice Parameters}.
      \item For \textbf{Type}, select the \textbf{RING} value.  This
         selection will activate the \textbf{Mod Oscillator}.
      \item In the \textbf{Local Oscillator}, click on \textbf{Waveform} to open
         the \textbf{ADDsynth Oscillator Editor}.
      \item Set the wave-shape to \textbf{Triangle}.
      \item Switch to voice number 2 and enable it.
      \item Again, for \textbf{Type}, select the \textbf{RING} value.
         However, feel free to select one of the other modulators, if one
         wishes.
      \item One can now use \textbf{Internal} for voice 2, or select
         \textbf{Voice} 1, to use the first voice as in internal modulator.
      \item Change the internal voice to, for example, \textbf{Square}.
      \item Do the same setup for voice 3.
         One will find that one can use its \textbf{Internal} or
         either of the two previous ones.
   \end{enumber}

   Now the joker in the pack is that one can disable both the previous
   voices but \textsl{still} use their Mod Oscillators.

   In a newsgroup (\cite{ringmodulator}, the following note is found.

   \begin{quotation}
      Say I want the A tone ring-modulated by 880Hz. A is 440 Hz, the ring
      modulation setting lets me choose the modulation frequency relative
      to the frequency of the tone. So I choose octave 1 and let the
      detune at zero. If I move the detune, it'll shift the modulation
      frequency a bit, which will make a disharmonic effect.

      Wet/dry setting is controlled by volume in "modulation amplitude".
      The modulation frequency can further be multiplied or several
      modulations can be simulated by changing the oscillator waveform.

      One huge letdown is that it is only available for Adsynth. PadSynth
      does not seem to have ring modulation option, so the coolest sounds
      stay out of question for massive lead tones. :-(
   \end{quotation}

   We have provided a more useful "tutorial" on using the ring modulator in the
   \textsl{Yoshimi Cookbook} \cite{cookbook} document.

   Finally, at the bottom of the ADDsynth Voice Part dialog, (under the Modulator
   section, we find the last few controls.

   \setcounter{ItemCounter}{0}      % Reset the ItemCounter for this list.

   \itempar{C}{voice oscillator!copy}
   Copy D note parameters ("DnoteParameters").

   \itempar{P}{voice oscillator!paste}
   Paste D note parameters ("DnoteParameters").

   \itempar{Close Window}{voice oscillator!close}
   Close.


\subsection{ADDsynth / Voice List}
\label{subsec:addsynth_voice_list}

   The ADDsynth Voices List shows a summary of voices 1 to 8, and allows
   some overall control of them, almost like a simple mixer.
   It is brought on-screen via the \textbf{Show Voice List} button
   of the ADDsynth global part editor.
   It is fully in sync with the voice windows.

\begin{figure}[H]
   \centering
   \includegraphics[scale=0.75]{2.0/AddSynthVoiceList.png}
   \caption{ADDsynth Voices List}
   \label{fig:addsynth_voices_list}
\end{figure}

   \begin{enumber}
      \item \textbf{No. (1 to 8)}
      \item \textbf{Edit}
      \item \textbf{Wave}
      \item \textbf{Mod}
      \item \textbf{Vol}
      \item \textbf{Pan}
      \item \textbf{Res}
      \item \textbf{Detune}
      \item \textbf{Vibrato Depth}
      \item \textbf{Hide Voice List}
   \end{enumber}

   \setcounter{ItemCounter}{0}      % Reset the ItemCounter for this list.

   \itempar{No. (1 to 8)}{voice list!number}
   Voice List Number.
   This check-box enables or disables a given voice in the current part.

   Values: \texttt{Off, On}

   \itempar{Edit}{voice list!edit}
   ADDSynth Voice List Edit Button.
   This button brings up the appropriate ADDSynth Voice dialog so that the
   waveform can be easily brought up for modification.

   \itempar{Wave Icon}{voice list!waveform icon}
   Waveform Icon.
   The waveform icon shows a rough rendering of the actual shape of the voice
   waveform, or the letter \textbf{N} is the voice is constructed from white
   noise, or the letters \textbf{O} or \textbf{V} followed by a number if the
   voice is using an external oscillator or voice input, respectively.
   Note that this picture isn't updated if the voice is edited, until the voice
   list is closed and reopened.

% TODO FOR 1.6.0:  Uncomment the commented text.

  \itempar{Modulation Icon}{voice list!modulation icon}
  Modulation Icon.
  The modulation icon shows a rough rendering of the actual shape of the voice
  modulation, or the letters \textbf{M} or \textbf{V} followed by a number if
  the voice is using an external modulator oscillator or voice input,
  respectively.
  Note that this picture isn't updated if the voice is edited, until the voice
  list is closed and reopened.

   \itempar{Vol}{voice list!volume}
   Voice Volume.
   This slider controls the relative volume of a given voice in the current
   part.

   Values: \texttt{0 to 127, 100*}

   \itempar{Pan}{voice list!panning}
   Voice Panning (0/leftmost is Random).
   This slider controls the panning of a given voice in the current part.

   Values: \texttt{0 to 127, 64*}

   \itempar{Res}{voice list!resonance}
   Resonance On/Off.
   Enable/disable the resonance effect of a voice.
   Note that the resonance is configured in by the Resonance dialog brought
   up by the \textbf{Resonance} button at the bottom of the ADDsynth main
   dialog.  The resonance dialog has its own Enable button, as well.
   This is provided so individual voices can bypass the resonance filter.

   Values: \texttt{Off, On*}

   \itempar{Detune Value}{voice list!detune value}
   This read-only text-box shows the current value of detune as selected by
   its slider.

   \itempar{Detune}{voice list!detune}
   Fine Detune (cents).

   Values: \texttt{-35 to 35, 0*}

   \itempar{Vibrato Depth}{voice list!vibrato depth}
   Frequency LFO Amount/Depth.
   This setting can be very useful because, with the detune settings, one can
   create very good sounding instruments.

   Values: \texttt{0 to 127, 40*}

   \itempar{Hide Voice List}{voice list!hide}
   Hide Voice List.  A Close button, really, and that is what it is in the
   latest version of \textsl{Yoshimi}.

\subsection{ADDsynth / Oscillator}
\label{subsec:addsynth_oscillator}

   Pressing the \textbf{Waveform} button in the ADDsynth
   voice-editor dialog brings up a very complex dialog for modifying the
   harmonics of the voice.

\begin{figure}[H]
   \centering
   \includegraphics[scale=0.75]{2.0/Oscillator.png}
   \caption{ADDsynth Oscillator Editor}
   \label{fig:addsynth_oscillator_editor}
\end{figure}

   This item is nearly identical to the PADsynth harmonic editor depicted in
   \figureref{fig:padsynth_harmonic_content_editor},
   except for the items noted below.
   Obviously, it is a topic unto itself!

   \begin{enumber}
      \item \textbf{Oscillator Spectrum Graph}
      \item \textbf{Oscillator Waveform Graph}
      \item \textbf{Mag.Type}
      \item \textbf{Phase Randomness} (ADDsynth Oscillator Editor only)
      \item \textbf{H.rnd} (ADDsynth Oscillator Editor only)
      \item \textbf{H.rnd knob} (ADDsynth Oscillator Editor only)
   \end{enumber}

   \setcounter{ItemCounter}{0}      % Reset the ItemCounter for this list.

   \itempar{Oscillator Spectrum Graph}{addsynth!oscillator graph}
   Oscillator Spectrum Graph.
   This graph shows the spectrum of the oscillator as a series of vertical
   lines, a kind of frequency histogram.

   \itempar{Oscillator Waveform Graph}{addsynth!waveform graph}
   Oscillator Waveform Graph.
   This graph shows the temporal waveform  of the oscillator.

   \itempar{Mag.Type}{addsynth!mag type}
   Oscillator Magnitude Type.
   Sets how the magnitudes from the user interface behave.  See the values
   below.

   \itempar{Phase Randomness}{addsynth!rnd}
   Phase Randomness. Sets the randomness of the oscillator phase which is
   settable from -64 (-90 deg) to 63 (+90 deg).

   \itempar{H.rnd}{addsynth!H.rnd}
   Harmonic Amplitude Randomness.
   Enables the feature and sets the 'law' for the way randomness is applied.

\begin{figure}[H]
   \centering
   \includegraphics[scale=1.0]{bottom-panel/instrument-edit/ADD/ADDsynth-hrnd.png}
   \caption{ADDsynth Oscillator Harmonic Randomness Selections}
   \label{fig:addsynth_hrnd}
\end{figure}

   Values: \texttt{None, Pow, and Sin}

   \itempar{H.rnd knob}{addsynth!H.rnd knob}
   Harmonic Amplitude Randomness Knob.
   Adjusts the amount of randomness applied to the amplitudes of individual harmonics.

   Values: \texttt{0 to 127, 64*}

\subsection{ADDsynth / Voice Warning}
\label{subsec:addsynth_voice-warning}
   As of V 1.5.11 there is a warning about editing waveforms.

   When editing an AddSynth voice that is taking it's waveform from another one
   the 'Waveform' text turns blue as a warning. However when actually entering
   the waveform window iteself, until now there was no indication that one was
   likely to change a shared oscillator. This is put right since V 1.5.11 and
   it also applies to modulator waveforms.

\begin{figure}[H]
   \centering
   \includegraphics[scale=0.75]{1.5.11/oscillator_warning.png}
   \caption{Warning in ADDSynth Oscillator}
   \label{fig:voice_oscillator_warning}
\end{figure}

   As can be seen, not only does one see the warning, but also exactly which
   voice (or modulator) is being edited.

\subsection{ADDsynth / Resonance}
\label{subsec:addsynth_resonance}

   This dialog, shared in common with the PADsynth editor, is a stock
   user-interface element described in
   \sectionref{subsec:stock_resonance_settings}.

%-------------------------------------------------------------------------------
% vim: ts=3 sw=3 et ft=tex
%-------------------------------------------------------------------------------


% PADsynth User-Interface items

%-------------------------------------------------------------------------------
% yum_padsynth
%-------------------------------------------------------------------------------
%
% \file        yum_padsynth.tex
% \library     Documents
% \author      Chris Ahlstrom
% \date        2015-06-07
% \update      2015-06-27
% \version     $Revision$
% \license     $XPC_GPL_LICENSE$
%
%     Provides the PADsynth section of yoshimi-user-manual.tex.
%
%-------------------------------------------------------------------------------

\section{PADsynth}
\label{sec:padsynth}

   The \textsl{Yoshimi} PADsynth dialog is a complex dialog for creating a
   pad instrument,  "PADsynth" or "PADnote" is engine that makes very
   beautiful pads and other instruments. (These instruments can be exported
   for use with other programs too).

   The PADsynth dialog consists of two major tabs, "Harmonic Structure" and
   "EnvelopesLFOs".  Each of these tabs is fairly complex, so the discussion
   will break the tabs down by sub-sections.

\subsection{PADsynth / Algorithm}
\label{subsec:padsynth_algorithm}

\subsubsection{PADsynth / Algorithm / General}
\label{subsubsec:padsynth_algorithm_general}

   This algorithm generates very beautiful sounds, even if its idea is much
   simpler than other algorithms. It generates a perfectly looped wave-table
   sample which can be used in instruments. It easily generates sounds of
   ensembles, choirs, metallic sounds (bells) and many other types of sound.
   Paul Nasca wanted to make this algorithm known, and everyone is welcome to
   learn and use this algorithm into one's projects or products (non-commercial
   or commercial).
   
   Quote \cite{zyndoc}:

   \begin{quotation}
      You will not be disappointed by this algorithm.

      I hope that this algorithm will be implemented in many software/hardware
      synthesizers. Use it, spread it, write about it, create beautiful
      instruments with it. If your synthesizer uses plenty of samples, you can
      use this algorithm to generate many ready-to-use samples.

      This algorithm, this page, the images, the implementations from this
      page, the audio examples, the parameter files from this page
      are released under Public Domain by Nasca Octavian Paul.
      e-mail: zynaddsubfx AT yahoo DOT com
   \end{quotation}

   In order to understand how this algorithm works, one needs to be familiar
   with how to think about musical instruments. Please read an introduction
   for the description of the meaning and the importance of bandwidth of each
   harmonic and randomness.

   This algorithm generates some large wave-tables that can be played at
   different speeds to get the desired sound. This algorithm describes only
   how these wave-tables are generated. The result is a perfectly looped
   wave-table.  Unlike other synthesis methods, which use the
   Inverse Fast Fourier Transform, this one does not use overlap/add methods
   and there is only one IFFT for the whole sample.

   The basic steps are:

   \begin{enumber}
      \item Make a very large array that represents the amplitude spectrum of
         the sound (all default values are zero).
      \item Generate the distribution of each harmonic in the frequency
         spectrum and add it to the array.
      \item Put random phases to each frequency of the spectrum.
      \item Do a single Inverse Fourier Transform of the whole spectrum. There
         is no need of any overlapping windows, because there is only one
         single IFFT for the whole sample. 
   \end{enumber}

   The output is a sample which can be used as a wave-table In the next image,
   the steps are represented graphically:

      TODO:  A GRAPHIC

\subsubsection{PADsynth / Algorithm / Harmonic Bandwidth}
\label{subsubsec:padsynth_algorithm_harmonic_bandwidth}

   We consider one harmonic (overtone) as being composed of many frequencies.
   These sine components of one harmonic are spread over a certain band of
   frequencies.  Higher harmonics have a wider bandwidth. In natural
   choirs/ensembles the bandwidth is proportional to the frequency of the
   harmonic.
   
   Here is an example of a spectrum of an instrument generated by ZynAddSubFX:

      TODO:  A GRAPHIC, full spectrum, closeup of the spectrum

   The harmonics becomes wider and wider, until a certain frequency, where
   they may merge into a noise band (as in the full spectrum image from above
   shows). This is a normal thing and we recommend to not avoid this by
   limiting the bandwidth of the harmonics.

   The frequency distribution of one harmonic/overtone (or the harmonic
   profile).

   This describes the function of the spread of the harmonic.
   Here are some examples of how they can be spread:

   a) A special case is where there is only a single sine component inside the
   harmonic In this case, the harmonic and the "sine component" are the same
   thing.

   b) Detuned. In this case there are two sine components which are detuned.

   c) Evenly spread inside the harmonic (all components has the same amplitude)

   d) Normal (Gaussian) distribution. The sine components amplitude are bell
   shaped. The largest amplitude is in the center of the band. This distribution
   gives the most natural sounds (it simulates a very, very large ensemble).

   Of course, one can use many other profiles of the harmonic. ZynAddSubFX's
   PADsynth module offers many ways to generate the harmonic profile.  Also, it's
   very important that the harmonic must have the same amplitude, regardless of
   the profile functions/parameters and the bandwidth.

   For many more details of this algorithm, see Paul Nasca's document
   \cite{zyndoc}.

\paragraph{Tip: Using the PADsynth}
\label{tips_using_the_padsynth}
\index{tips!padsynth usage}

   Keep in mind that the resulting wave-tables are perfectly looped.
   When using the wave-tables for instruments, on each NoteOn, start from a
   random position and not from the start. This avoids hearing the same sound
   on each keystroke.

   One can use the same wave-table for generating stereo sounds, by playing
   the same wave-table at different positions for left and right. The best is
   to create a difference between left right of N/2.

   Generate different wave-tables for different pitches and use the one that
   is closest to the desired pitch.

   Upsample or downsample the amplitude array of the harmonic before running
   the algorithm, according to the fundamental frequency. In this case we need
   to set a parameter "base\_frequency" which represents the frequency where
   the array is left unchanged. 

   Example:
   We have A\_orig[]={1,2,1,3,0,0,1,0} and base\_frequency is equal to 440 Hz
   Here are some cases:

A[] for 440 Hz: is the same as A\_orig[] 

A[] for 220 Hz: is the A\_orig[] upsampled by factor of 2

so: A[]={1, 1, 1.5, 2, 1.5, 1, 2, 3, 1.5, 0, 0, 0, 0.5, 1, 0.5, 0}

(the original A\_orig amplitudes are shown as bold) 

A[] for 880 Hz: the A\_orig[] is downsampled by a factor of 2

so: A[]={1.5, 2, 0, 0.5} 

A[] for F Hz: the A\_orig[] is scaled by a factor of 440/F.

   Even if this idea is very simple, the resulting sounds are very natural,
   because it keeps the spectrum constant according to the frequency of the
   harmonic and not to the number of the harmonic. This follows the point 4
   from the document where I described some principles regarding synthesis.

\subsection{PADsynth / Harmonic Structure}
\label{subsec:padsynth_harmonic_structure}

\begin{figure}[H]
   \centering 
   \includegraphics[scale=1.0]{bottom-panel/instrument-edit/PAD/PADsynth-parameters-harmonic-structure.jpg}
   \caption{PADsynth Edit Dialog}
   \label{fig:padsynth_edit_dialog}
\end{figure}

   \begin{enumber}
      \item \textbf{Basics} (section)
      \item \textbf{Harmonic} (section)
      \item \textbf{Resonance} (section)
      \item \textbf{Change} (section)
      \item \textbf{Bandwidth and Position} (section)
      \item \textbf{Export} (section)
      \item \textbf{C}
      \item \textbf{P}
      \item \textbf{Apply Changes}
      \item \textbf{Close}
   \end{enumber}

\subsubsection{PADsynth / Harmonic Structure / Basics}
\label{subsubsec:padsynth_harmonic_structure_basics}

   \begin{enumber}
      \item \textbf{BaseType}
      \item \textbf{Width}
      \item \textbf{FreqMlt}
      \item \textbf{Str}
      \item \textbf{SFreq}
      \item \textbf{Size}
      \item \textbf{Full/Upper/Lower}
      \item \textbf{AmpMultiplier}
      \item \textbf{AmpMode}
      \item \textbf{Par1}
      \item \textbf{Par2}
   \end{enumber}

   \setcounter{ItemCounter}{0}      % Reset the ItemCounter for this list.

   \itempar{BaseType}{padsynth!harmonic type}
   Base Type of Harmonic.

\begin{figure}[H]
   \centering 
   \includegraphics[scale=1.0]{bottom-panel/instrument-edit/PAD/base-type.jpg}
   \caption{Base Type of Harmonic}
   \label{fig:padsynth_base_type_of_harmonic}
\end{figure}

   Values: \texttt{Gauss*, Square, DoubleExp}

   \itempar{Width}{padsynth!harmonic width}
   Width of Harmonic.

   Values: \texttt{1 to 127?}

   \itempar{FreqMlt}{padsynth!freq mult}
   Frequency Multiplier.

   Values: \texttt{1 to 127?}

   \itempar{Str}{padsynth!harmonic stretch}
   Stretch.

   \itempar{SFreq}{padsynth!harmonic sfreq}
   Harmonic Sfreq?

   \itempar{Size}{padsynth!harmonic size}
   Harmonic Size.

   \itempar{Full/Upper/Lower}{padsynth!harmonic fup}
   Harmonic Spread???

\begin{figure}[H]
   \centering 
   \includegraphics[scale=1.0]{bottom-panel/instrument-edit/PAD/full-upper-lower.jpg}
   \caption{PADsynth Full/Upper/Lower Harmonics}
   \label{fig:padsynth_full_upper_lower}
\end{figure}

   Values: \texttt{Full*, Upper Half, Lower Half}

   \itempar{AmpMultiplier}{padsynth!amp mult}
   Amplitude Multiplier.

\begin{figure}[H]
   \centering 
   \includegraphics[scale=1.0]{bottom-panel/instrument-edit/PAD/amp-multiplier.jpg}
   \caption{PADsynth Amplitude Multiplier}
   \label{fig:padsynth_amplitude_multiplier}
\end{figure}

   Values: \texttt{OFF*, Gauss, Sine, Flat}

   \itempar{AmpMode}{padsynth!amp mode}
   Amplitude Mode.

\begin{figure}[H]
   \centering 
   \includegraphics[scale=1.0]{bottom-panel/instrument-edit/PAD/amp-mode.jpg}
   \caption{PADsynth Amplitude Mode}
   \label{fig:padsynth_amplitude_mode}
\end{figure}

   Values: \texttt{Sum*, Mult, Div1, Div2}

   \itempar{Par1}{padsynth!harmonic par1}
   Harmonic Parameter 1?

   Values: \texttt{0 to 127?}

   \itempar{Par2}{padsynth!harmonic par2}
   Harmonic Parameter 2?

   Values: \texttt{0 to 127?}

\subsubsection{PADsynth / Harmonic Structure / Harmonic}
\label{subsubsec:padsynth_harmonic_structure_harmonic}

   \begin{enumber}
      \item \textbf{Profile of One Harmonic}
      \item \textbf{Harmonic Content Window}
      \item \textbf{base}
      \item \textbf{smp/oct}
      \item \textbf{no.oct}
      \item \textbf{Sample Size}
      \item \textbf{Resonance} (section)
      \item \textbf{Change} (section)
   \end{enumber}

   \setcounter{ItemCounter}{0}      % Reset the ItemCounter for this list.

   \itempar{Profile of One Harmonic}{padsynth!harmonic profile}
   Profile of One Harmonic (Frequency Distribution).

   \itempar{Harmonic Content Window}{padsynth!harmonic content}
   Harmonic Content Window.

   \itempar{base}{padsynth!harmonic base}

\begin{figure}[H]
   \centering 
   \includegraphics[scale=1.0]{bottom-panel/instrument-edit/PAD/base.jpg}
   \caption{Harmonic Base Dropdown}
   \label{fig:padsynth_harmonic_base_dropdown}
\end{figure}

   Values: \texttt{C-2, G-2, C-3, G-3, C-4*, G-4, C-5, G-5, G-6}

   \itempar{smp/oct}{padsynth!harmonic samples per oct}
   Harmonic Samples Per Octave?

\begin{figure}[H]
   \centering 
   \includegraphics[scale=1.0]{bottom-panel/instrument-edit/PAD/smp-per-octave.jpg}
   \caption{Harmonic Samples Per Octave}
   \label{fig:padsynth_harmonic_samples_per_octave}
\end{figure}

   \itempar{no.oct}{padsynth!harmonic no. of octaves}
   Number of Octaves of Harmonic.

\begin{figure}[H]
   \centering 
   \includegraphics[scale=1.0]{bottom-panel/instrument-edit/PAD/number-of-octaves.jpg}
   \caption{Harmonic Number of Octaves}
   \label{fig:padsynth_harmonic_number_of_octaves}
\end{figure}

   Values: \texttt{1, 2, 3, 4*, 5, 6, 7, 8}

   \itempar{Sample Size}{padsynth!harmonic sample size}
   Harmonic Sample Size.

\begin{figure}[H]
   \centering 
   \includegraphics[scale=1.0]{bottom-panel/instrument-edit/PAD/sample-size.jpg}
   \caption{Harmonic Sample Size Dropdown}
   \label{fig:padsynth_harmonic_sample_size_dropdown}
\end{figure}

   Values: \texttt{16k (Tiny), 32k, 64k (Small), 128k*, 256k (Normal), 512k, 1M (Big)}

\subsubsection{PADsynth / Harmonic Structure / Bandwidth and Position}
\label{subsubsec:padsynth_harmonic_structure_bw_and_pos}

   \begin{enumber}
      \item \textbf{BandWidth}
      \item \textbf{cents}
      \item \textbf{Bandwidth Scale}
      \item \textbf{Spectrum Mode}
      \item \textbf{OvertonesPosition}
      \item \textbf{Par1}
      \item \textbf{Par2}
      \item \textbf{ForceH}
      \item \textbf{Harmonics Plot}
   \end{enumber}

   \setcounter{ItemCounter}{0}      % Reset the ItemCounter for this list.

   \itempar{BandWidth}{padsynth!bandwidth}
   Harmonics Bandwidth.

   Values: \texttt{0 to 127?}

   \itempar{cents}{padsynth!bandwidth reading}
   Bandwidth Reading (cents).

   \itempar{Bandwidth Scale}{padsynth!bandwidth scale}
   Bandwidth Scale.

\begin{figure}[H]
   \centering 
   \includegraphics[scale=1.0]{bottom-panel/instrument-edit/PAD/bandwidth-scale.jpg}
   \caption{Harmonics Bandwidth Scale.}
   \label{fig:padsynth_harmonics_bandwidth_scale}
\end{figure}

   Values: \texttt{Normal, EqualHz, Quater, Half, 75\%, 150\%, Double, Inv.  Half}

   \itempar{Spectrum Mode}{padsynth!spectrum mode}
   Harmonics Spectrum Mode.

\begin{figure}[H]
   \centering 
   \includegraphics[scale=1.0]{bottom-panel/instrument-edit/PAD/spectrum-mode.jpg}
   \caption{PADsynth Harmonics Spectrum Mode}
   \label{fig:padsynth_harmonics_spectrum mode}
\end{figure}

   Values: \texttt{Bandwidth*, Discrete, Continuous}

   \itempar{OvertonesPosition}{padsynth!overtones}
   Overtones Position.

\begin{figure}[H]
   \centering 
   \includegraphics[scale=1.0]{bottom-panel/instrument-edit/PAD/overtones-position.jpg}
   \caption{PADsynth Overtones Position}
   \label{fig:padsynth_overtones_position}
\end{figure}

   Values: \texttt{Harmonic*, ShiftU, ShiftL, PowerU, PowerL, Sine, Power}

   \itempar{Par1}{padsynth!par1}
   PADSynth Bandwidth Parameters 1?

   \itempar{Par2}{padsynth!par2}
   PADSynth Bandwidth Parameters 2?

   \itempar{ForceH}{padsynth!forceh}
   PADSynth Bandwidth ForceH.

   \itempar{Harmonics Plot}{padsynth!harmonics plot}
   PADSynth Harmonics Plot.

\subsubsection{PADsynth / Harmonic Structure / Export}
\label{subsubsec:padsynth_harmonic_structure_export}

\begin{figure}[H]
   \centering 
   \includegraphics[scale=1.0]{bottom-panel/instrument-edit/PAD/export-dialog.jpg}
   \caption{Harmonics Structure Export Dialog}
   \label{fig:harmonics_structure_export_dialog}
\end{figure}

   This export dialog is a file dialog similar to other file dialogs, such as
   that shown in section .........

\subsubsection{PADsynth / Harmonic Structure / Resonance}
\label{subsubsec:padsynth_harmonic_structure_resonance}

   The PADsynth Harmonics resonance dialog is identical to the resonance
   dialog described in section ........

   Also see this image file:
   images/bottom-panel/instrument-edit/PAD/resonance.jpg.
   It shows something that the ADDsynth version doesn't... an "Apply"
   button.

\subsubsection{PADsynth / Harmonic Structure / Change}
\label{subsubsec:padsynth_harmonic_structure_change}

   Harmonic Content Editor.  Another complex dialog.
   Like the ADDsynth Oscillator Editor, it allows one to create an unlimited
   number of oscillators. 

\begin{figure}[H]
   \centering 
   \includegraphics[scale=0.75]{bottom-panel/instrument-edit/PAD/harmonic-content-editor.jpg}
   \caption{Harmonic Content Editor}
   \label{fig:padsynth_harmonic_content_editor}
\end{figure}

   This dialog is complex enough that it makes sense to break it down into
   sub-sections.

   \begin{enumber}
      \item \textbf{Oscillator} (section)
      \item \textbf{Base Function} (section)
      \item \textbf{Middle} (section)
      \item \textbf{Harmonic} (section)
   \end{enumber}

\paragraph{PADsynth / Harmonic Structure / Change / Oscillator}
\label{paragraph:padsynth_harmonic_structure_change_oscillator}

   \begin{enumber}
      \item \textbf{Oscillator Spectrum Graph}
      \item \textbf{Oscillator Waveform Graph}
      \item \textbf{Mag.Type}
      \item \textbf{rnd} (ADDsynth Oscillator Editor only)
      \item \textbf{H.rnd} (ADDsynth Oscillator Editor only)
      \item \textbf{H.rnd knob} (ADDsynth Oscillator Editor only)
      \item \textbf{Apply} (not present in ADDsynth Oscillator Editor)
   \end{enumber}

   TODO:  Describe the 3 ADDsynth elements noted above.

   rnd - Set the randomness of the oscillator output. There are 2 types of
   randomnesses, first is group randomness(the oscillator starts at random
   position), second is from -64(max) to -1 (min) and each harmonic
   (the oscillator is phase distorted) is from 1(min) to 63 (max). 0 is no
   randomness. One could use this parameter to make warm sounds like
   analogue synthesizers.

   \setcounter{ItemCounter}{0}      % Reset the ItemCounter for this list.

   \itempar{Oscillator Spectrum Graph}{padsynth!oscillator graph}
   Oscillator Spectrum Graph.

   \itempar{Oscillator Waveform Graph}{padsynth!waveform graph}
   Oscillator Waveform Graph.

   \itempar{Mag.Type}{padsynth!mag type}
   Oscillator Magnitude Type.
   Sets how the magnitudes from the user interface behave.  See the values
   below.

\begin{figure}[H]
   \centering 
   \includegraphics[scale=1.0]{bottom-panel/instrument-edit/PAD/harmonic-content-mag-type.jpg}
   \caption{PADsynth Harmonic Content Mag Type}
   \label{fig:padsynth_harmonic_content_mag_type}
\end{figure}

   Values: \texttt{Linear*, -40dB, -60db, -80dB, -100dB}

   \itempar{Apply}{padsynth!apply button}
   PADsynth Harmonic Content Editor Apply Button.

\paragraph{PADsynth / Harmonic Structure / Change / Base Function}
\label{paragraph:padsynth_harmonic_structure_change_base_function}

   \begin{enumber}
      \item \textbf{Base Func. Spectrum Graph}
      \item \textbf{Base Func. Waveform Graph}
      \item \textbf{Base F..}
      \item \textbf{Par. Value}
      \item \textbf{Par. Wheel}
      \item \textbf{B.F.Mod.}
      \item \textbf{Wheel 1}
      \item \textbf{Wheel 2}
      \item \textbf{Wheel 3}
   \end{enumber}

   \setcounter{ItemCounter}{0}      % Reset the ItemCounter for this list.

   \itempar{Base Func. Spectrum Graph}{padsynth!base function spectrum}
   Harmonic Base Function Spectrum Graph.

   \itempar{Base Func. Waveform Graph}{padsynth!base function waveform}
   Harmonic Base Function Waveform Graph.

   \itempar{Base F..}{padsynth!base function}
   Harmonic Base Function.
   Sets what function to use as the harmonics base function.
   One can use any base function as harmonics. 

\begin{figure}[H]
   \centering 
   \includegraphics[scale=1.0]{bottom-panel/instrument-edit/PAD/harmonic-content-base-function.jpg}
   \caption{PADsynth Harmonic Content Base Function}
   \label{fig:padsynth_harmonic_content_base_function}
\end{figure}

   Values: \texttt{Sine*, Triangle, Pulse, Saw, Power, Gauss, Diode, AbsSine,
           PulseSine, StrchSine, Chirp, AbsStrSine, Chebyshev,
           Sqr, Spike, Circle}

   \itempar{Par. Value}{padsynth!par value}
   PADsynth Parameter Value.

   \itempar{Par. Wheel}{padsynth!par wheel}
   PADsynth Parameter Wheel.
   Change the parameter of the base function. 

   \itempar{B.F.Mod.}{padsynth!bf mod}
   PADSynth Base Frequency Mod.

   \itempar{Wheel 1}{padsynth!wheel 1}
   PADsynth Wheel 1.

   \itempar{Wheel 2}{padsynth!wheel 2}
   PADsynth Wheel 2.

   \itempar{Wheel 3}{padsynth!wheel 3}
   PADsynth Wheel 3.


\paragraph{PADsynth / Harmonic Structure / Change / Middle}
\label{paragraph:padsynth_harmonic_structure_change_middle}

   \begin{enumber}
      \item \textbf{Use as base}
      \item \textbf{Clr.}
      \item \textbf{Wsh.}
      \item \textbf{Wsh Value}
      \item \textbf{Wsh Wheel}
      \item \textbf{Filter}
      \item \textbf{Filter Wheel 1}
      \item \textbf{Filter Wheel 2}
      \item \textbf{Filter p}
      \item \textbf{Mod.}
      \item \textbf{Mod. Wheel 1}
      \item \textbf{Mod. Wheel 2}
      \item \textbf{Mod. Wheel 3}
      \item \textbf{Sp.adj.}
      \item \textbf{Sp.adj. Wheel}
   \end{enumber}

   \setcounter{ItemCounter}{0}      % Reset the ItemCounter for this list.

   \itempar{Use as base}{padsynth!harm editor use-as-base}
   Use as Base.
   Convert the oscillator output to a base function. Changing the Base
   function or its parameter will erase the converted base function. 

   \itempar{Clr.}{padsynth!harm editor clr}
   Clear.
   Clear the settings and make the oscillator equal to a base function. If
   this is cleared, one can click the \textbf{Use as base} button to make
   multiple conversions to base functions. 

   \itempar{Wsh.}{padsynth!harm editor wsh}
   Harmonic Editor Wave-shaping, "W.sh".

   Wave shaping function that applies to the oscillator.
   It has one parameter that fine-tunes the wave-shaping function. 

   \itempar{Wsh Value}{padsynth!harm editor wsh value}
   Harmonic Editor Wave-shaping Value.

\begin{figure}[H]
   \centering 
   \includegraphics[scale=1.0]{bottom-panel/instrument-edit/PAD/harmonic-content-waveshaping-function.jpg}
   \caption{PADsynth Harmonic Content Editor Wave-Shaping Function}
   \label{fig:padsynth_harmonic_content_editor_waveshaping_function}
\end{figure}

   Values: \texttt{None*, Atan, Asym1, Pow, Sine, Qnts, Zigzg, Lmt,
              LmtU, LmtL, ILmt, Clip, Asym2, Pow2, Sgm}

   The type of wave-shaping distortion has much influence on how the
   overtones are being placed. Sometimes, one gets a "fat" bass, and
   sometimes, high frequencies are added, making the sound "crystal clear".

   \textbf{Atan \& Sigmoid}.
   This is the default setting. It is an easy way to apply loudness to a wave
   without getting undesired high overtones. Thus, it can be used both for
   making instruments that sound like "real" ones, but also for electronic
   music. The transformation turns, roughly said, every amplitude into a
   square amplitude. Thus, sine, power, pulse and triangle turn into a usual
   square wave, while a saw turns into a phased square wave. A chirp wave
   turns into a kind of phase modulated square wave.

   \textbf{Quants} ("Qnts")
   Quantization adds high overtones early. It can be seen as an unnatural
   effect, which is often used for electronic music.  The transformation is a
   bit similar to building the lower sum of a wave, mathematically said. This
   means that the transformation effect turns an "endless high" sampled
   wave into only a few samples. The more distortion one applies, the fewer
   samples will be used. Indeed, this is equivalent to say that more input
   amplification is used. To see this, here is a small sample of code, where
   "ws" is the (correctly scaled amount of input amplification, and "n" the
   number of original samples.

   If one turns on quantisation very high, one might be confused that,
   especially high notes, make no sound. The reason: High frequencies are
   "forgotten" if one samples with only few samples. Also, the sign of an
   amplitude can be forgotten. This behaviour might make some quantisations a
   bit unexpected.

   \textbf{Limiting} ("Lmt*" and "Clip")
   Limiting usually means that for a signal, the amplitude is modified
   because it exceeds its maximum value. Overdrive, as often used for
   guitars, is often achieved by limiting: It happens because an amplifier
   "overdrives" the maximum amplitude it can deliver.

   ZynAddSubFX has two types of limiting. Soft limiting, here as Lmt, means
   that the sound may not exceed a certain value. If the amplitude does so,
   it will simply be reduced to the limiting value. The overtones are
   generated in the lower frequencies first.

   Hard limiting, is also called clipping and abbreviated Clip. This means
   that if the maximum is exceeded, instead of being constant at the limiting
   value, the original signal still has some influence on the output signal.
   Still, it does not exceed the limiting value. For ZynAddSubFX, a signal
   exceeding the limiting value will continue to grow "in the negative". This
   leads to overtones being generated on the full frequency band.

   \itempar{Wsh Wheel}{padsynth!harm editor wsh wheel}
   Harmonic Editor Wave-shaping Wheel?

   \itempar{Filter}{padsynth!harm editor filter}
   Harmonic Editor Filter.
   Sets the type of the harmonic filter.

\begin{figure}[H]
   \centering 
   \includegraphics[scale=1.0]{bottom-panel/instrument-edit/PAD/harmonic-content-filter.jpg}
   \caption{PADsynth Harmonic Content Filter}
   \label{fig:}
\end{figure}

   Values: \texttt{None*, LP1, HP1a, HP1b, BP1, BS1, LP2, HP2, BP2,
              BS2, Cos, Sin, LSh, S}

   \itempar{Filter Wheel 1}{padsynth!harm editor filter wheel}
   Harmonic Editor Filter, Wheel 1.

   \itempar{Filter Wheel 2}{padsynth!harm editor filter wheel}
   Harmonic Editor Filter, Wheel 2.
   The knob in the right sets the filter parameter (frequency).

   \itempar{Filter p}{padsynth!harm editor filter p}
   Harmonic Editor Filter, p?

   \itempar{Mod.}{padsynth!harm editor mod}
   Harmonic Editor Modulation.

\begin{figure}[H]
   \centering 
   \includegraphics[scale=1.0]{bottom-panel/instrument-edit/PAD/harmonic-content-modulation.jpg}
   \caption{PADsynth Harmonic Content Editor Modulation}
   \label{fig:padsynth_harmonic_content_editor_modulation}
\end{figure}

   Values: \texttt{None*, Rev, Sine, Pow}

   \itempar{Mod. Wheel 1}{padsynth!harm editor mod wheel}
   Harmonic Editor Modulation Wheel 1?

   \itempar{Mod. Wheel 2}{padsynth!harm editor mod wheel}
   Harmonic Editor Modulation Wheel 2?

   \itempar{Mod. Wheel 3}{padsynth!harm editor mod wheel}
   Harmonic Editor Modulation Wheel 3?

   \itempar{Sp.adj.}{padsynth!harm editor spadj}
   Harmonic Editor Spectrum Adjust.
   Adjust the spectrum of the waveform.

   MORE FROM ZYN:

   RMS normalize. Enables the RMS normalization method (recommended); this
   keeps the same loudness regardless the harmonic content.

   Below are the harmonics and their phases. One can use them to add to
   oscillator harmonics that has the waveform of the base function.
   Increasing the number of harmonics has virtually no effect on CPU usage. 
   Right click to set a harmonic/phase to the default value. 

\begin{figure}[H]
   \centering 
   \includegraphics[scale=1.0]{bottom-panel/instrument-edit/PAD/harmonic-content-osc-spectrum-adjust.jpg}
   \caption{PADsynth Harmonic Content Editor Spectrum Adjust}
   \label{fig:padsynth_harmonic_content_editor_spectrum_adjust}
\end{figure}

   Values: \texttt{None*, Pow, ThrsD, ThrsU}

   \itempar{Sp.adj. Wheel}{padsynth!harm editor spadj wheel}
   Harmonic Editor Spectrum Adjust Wheel?

\paragraph{PADsynth / Harmonic Structure / Change / Harmonic}
\label{paragraph:padsynth_harmonic_structure_change_harmonic}

   \begin{enumber}
      \item \textbf{Harmonics Amplitude}
      \item \textbf{Harmonics Bandwidth}
      \item \textbf{Harmonics Scrollbar}
      \item \textbf{Harmonic Shift}
      \item \textbf{Harmonic Shift R} (dialog?)
      \item \textbf{Harmonic Shift preH}
      \item \textbf{Adpt.Harm.}
      \item \textbf{Adpt.Harm. Slider}
      \item \textbf{Adpt.Harm. baseF}
      \item \textbf{Adpt.Harm. pow}
      \item \textbf{Clear}
      \item \textbf{Sine}
      \item \textbf{C}
      \item \textbf{P}
      \item \textbf{Close}
   \end{enumber}

   \itempar{Harmonics Amplitude}{padsynth!harmonics amplitude}
   Harmonics Amplitude.
   Provides 128? sliders for the amplitude of harmonics.

   \itempar{Harmonics Bandwidth}{padsynth!harmonics bandwidth}
   Harmonics Bandwidth.
   Provides 128? sliders for the bandwidth of harmonics.

   \itempar{Harmonics Scrollbar}{padsynth!harmonics scrollbar}
   Harmonics Scrollbar.

   \itempar{Harmonic Shift}{padsynth!harmonics shift}
   Harmonics Shift.

   Values: \texttt{-x to 0 to x?}

   \itempar{Harmonic Shift R}{padsynth!harmonics shift r}
   Harmonics Shift R?.

   \itempar{Harmonic Shift preH}{padsynth!harmonics shift preh}
   Harmonics Shift preH?
   preF in Zyn?

   preF. Set the order of doing the filter and wave-shaper (uncheck to filter
   after wave-shaping, check to wave-shape after filtering).

   OKAY?

   \itempar{Adpt.Harm.}{padsynth!harmonics}
   Adaptive Harmonics?

\begin{figure}[H]
   \centering 
   \includegraphics[scale=1.0]{bottom-panel/instrument-edit/PAD/harmonic-content-adaptive-harmonic-type.jpg}
   \caption{PADsynth Adaptive Harmonic Type}
   \label{fig:padsynth_adaptive_harmonic_type}
\end{figure}

   Values: \texttt{OFF*, ON, Square, 2xSub, 2xAdd, 3xSub, 3xAdd, 4xSub, 4xAdd}

   \itempar{Adpt.Harm. Slider}{padsynth!harmonics slider}
   Adaptive Harmonics Slider?

   \itempar{Adpt.Harm. baseF}{padsynth!harmonics basef}
   Adaptive Harmonics Base Frequency?

   \itempar{Adpt.Harm. pow}{padsynth!harmonics pow}
   Adaptive Harmonics Power?

   \itempar{Clear}{padsynth!harmonics clear}
   Harmonics Clear.
   Clears the harmonics settings.

   \itempar{Sine}{padsynth!harmonics sine}
   Harmonics Sine.  Is this a dialog?

   \itempar{C}{padsynth!harmonics copy}
   Harmonics Copy.

   \itempar{P}{padsynth!harmonics paste}
   Harmonics Paste.

   \itempar{Close}{padsynth!harmonics close}
   Harmonics Close.

\subsection{PADsynth / Envelopes and LFOs}
\label{subsec:padsynth_envelopes_lfos}

\begin{figure}[H]
   \centering 
   \includegraphics[scale=1.0]{bottom-panel/instrument-edit/PAD/PADsynth-parameters-envelopes-LFOs.jpg}
   \caption{PADSynth Parameters, Envelopes and LFOs}
   \label{fig:padsynth_parameters_envelopes_and_lfos}
\end{figure}

   \begin{enumber}
      \item \textbf{AMPLITUDE}
      \item \textbf{FILTER} (section)
      \item \textbf{FREQUENCY} (section)
      \item \textbf{Export}
      \item \textbf{C}
      \item \textbf{P}
      \item \textbf{Close}
   \end{enumber}

   \itempar{AMPLITUDE}{padsynth!amplitude section}
   See \sectionref{subsec:addsynth_amplitude}.
   This stock dialog section provide volume, velocity sensing, panning, an
   amplitude envelope sub-panel, and an amplitude LFO sub-panel.

   \itempar{FILTER}{padsynth!amplitude section}
   See \sectionref{subsec:addsynth_filter}.

   \itempar{FREQUENCY}{padsynth!amplitude section}
   See \sectionref{subsec:addsynth_frequency}.

   \itempar{Export}{padsynth!export}
   Very similar to 
   \figureref{fig:harmonics_structure_export_dialog}.

   \itempar{C}{padsynth!copy}
   The stock copy dialog.

   \itempar{P}{padsynth!paste}
   The stock paste dialog.

   \itempar{Close}{padsynth!close}
   Close.

%-------------------------------------------------------------------------------
% vim: ts=3 sw=3 et ft=tex
%-------------------------------------------------------------------------------


% SUBsynth User-Interface items

%-------------------------------------------------------------------------------
% yum_subsynth
%-------------------------------------------------------------------------------
%
% \file        yum_subsynth.tex
% \library     Documents
% \author      Chris Ahlstrom
% \date        2015-06-07
% \update      2021-11-28
% \version     $Revision$
% \license     $XPC_GPL_LICENSE$
%
%     Provides the SUBsynth section of the manual.
%
%-------------------------------------------------------------------------------

\section{SUBsynth}
\label{sec:subsynth}

   The \textsl{Yoshimi} SUBsynth dialog is yet another complex dialog, this time
   for creating a subtractive-synthesis instrument,
   "SUBsynth" or "SUBnote" is a simple engine which makes sounds through
   subtraction of harmonics from white noise.  \cite{zyndoc}

\begin{figure}[H]
   \centering
   \includegraphics[scale=0.5]{2.3.3/sub.png}
   \caption{SUBsynth Edit Dialog}
   \label{fig:subsynth_edit_dialog}
\end{figure}

   This dialog, though very complex, consists of a number of stock sections
   that are described elsewhere in this manual.
   Some descriptions are repeated here, though.

   \begin{enumber}
      \item \textbf{AMPLITUDE} (section)
      \item \textbf{BANDWIDTH} (section)
      \item \textbf{FREQUENCY} (section)
      \item \textbf{OVERTONES} (section)
      \item \textbf{FILTER} (section)
      \item \textbf{Harmonics} (section)
      \item \textbf{Clear}
      \item \textbf{C}
      \item \textbf{P}
      \item \textbf{Close}
   \end{enumber}

\subsection{SUBsynth / AMPLITUDE}
\label{subsec:subsynth_amplitude}

   \begin{enumber}
      \item \textbf{Volume}
      \item \textbf{Vel Sens}
      \item \textbf{Pan}
      \item \textbf{Rand}
      \item \textbf{Width}

       The controls above are discussed in detail in
       \sectionref{subsec:volume_panning}. However their values for
       SUBsynth are as below.
      \item \textbf{Amplitude Env} (stock sub-panel)
   \end{enumber}

   \setcounter{ItemCounter}{0}      % Reset the ItemCounter for this list.

   \itempar{Volume}{subsynth!volume}

   Values: \texttt{-60dB to 19.4dB, 0dB*}

   \itempar{Vel Sens}{subsynth!vel sens}

   Values: \texttt{-48dB to -0.8dB, disabled, -2.59dB*}

   \itempar{Pan}{subsynth!pan}

   Values: \texttt{100\% left to 100\% right, centered*}

   \itempar{Rand}{subsynth!random pan}

   Values: \texttt{off*, on}

   \itempar{Width}{subsynth!random width}

   Values: \texttt{0 to 100\%* }

   \itempar{Amplitude Env}{subsynth!amplitude envelope}
   Amplitude Envelope.
   See \sectionref{subsubsec:amplitude_envelope_subpanel},
   for information on this stock sub-panel.

\subsection{SUBsynth / BANDWIDTH}
\label{subsec:subsynth_bandwidth}

   \begin{enumber}
      \item \textbf{BandWidth}
      \item \textbf{B.Width Scale}
      \item \textbf{Bandwidth Env}
   \end{enumber}

   \setcounter{ItemCounter}{0}      % Reset the ItemCounter for this list.

   \itempar{BandWidth}{subsynth!bandwidth}
   SUBsynth Bandwidth.
   Sets the bandwidth of each harmonic.
   At the default setting the \textsl{equivalent} multiplier is 8.731 (as
   indicated by the tooltip) although the control value is zero. Left of center
   reduces the bandwidth and right of center increases it.

   Values: \texttt{1 to 127, 40*}

   \itempar{B.Width Scale}{subsynth!bandwidth scale}
   SUBsynth Bandwidth Scale.
   This provides a factor to alter the ratio of the bandwidth related to the
   frequency of the harmonic. When the dial is centered the ratio is 1:1 as
   indicated by the tooltip, so effectively unchanged. Left of center reduces
   the LF bandwidth and increases the HF bandwith (relative to 1kHz) and right
   of center has the reverse action.

   Values: \texttt{-64 to 63, 0*}

   \itempar{Bandwidth Env}{subsynth!bandwidth envelope}
   SUBsynth Bandwidth.

   \begin{enumber}
      \item \textbf{Enabled}
      \item \textbf{A.val}
      \item \textbf{A.dt}
      \item \textbf{R.dt}
      \item \textbf{R.val}
      \item \textbf{Stretch}
      \item \textbf{frcR}
      \item \textbf{C}
      \item \textbf{P}
      \item \textbf{E}
   \end{enumber}

   \setcounter{ItemCounter}{0}      % Reset the ItemCounter for this list.

   \itempar{Enabled}{bandwidth!enable}
   Enable the panel.

   \itempar{A.val}{bandwidth!attack value}
   Attack value.
   The value of the multiplier is shown in the tooltip in the GUI interface.
   The default of 100 provides an equivalent multiplier of 49.351 for the
   initial bandwidth (as swown by the tooltip). The bandwidth will gradually
   resolve to its original setting at a rate set by A dt.

   Values: \texttt{0 to 127, 100*}

   \itempar{A.dt}{bandwidth!attack time}
   Attack duration. Attack time.

   Values: \texttt{0 to 127, 70*}

   \itempar{R.dt}{bandwidth!release time}
   Release time.

   Values: \texttt{0 to 127, 60*}

   \itempar{R.val}{bandwidth!release value}
   Release Value.
   Actually present only on the Frequency Env sub-panel.

   Values: \texttt{0 to 127, 64*}

   \itempar{Stretch}{bandwidth!stretch}
   Bandwidth Stretch. On lower notes make the bandwidth wider.

   Values: \texttt{0 to 127, 64*}

   \itempar{frcR}{bandwidth!forced release}
   Forced release.
   If this option is turned on, the release will go to the final value, even if
   the sustain level is not reached.

   Also present in this sub-panel are the usual \textbf{C}opy and \textbf{P}aste
   buttons that call up a copy-parameters or paste-parameters dialog, as well as
   a button to bring up the editor window.

   Values: \texttt{Off, On*}

\subsection{SUBsynth / FREQUENCY}
\label{subsec:subsynth_frequency}

   \begin{enumber}
      \item \textbf{Detune}
      \item \textbf{FREQUENCY Slider}
      \item \textbf{Bend}
      \item \textbf{Offset}
      \item \textbf{440Hz}
      \item \textbf{Eq.T}
      \item \textbf{Octave}
      \item \textbf{Detune Type}
      \item \textbf{Coarse Det.}
      \item \textbf{Frequency Env}
   \end{enumber}

   \setcounter{ItemCounter}{0}      % Reset the ItemCounter for this list.

   \itempar{Detune}{subsynth!freq detune}
   Frequency Detune Indicator

   \itempar{FREQUENCY Slider}{subsynth!freq slider}
   Frequency Slider.

   Values: \texttt{-35 to 34.99}

   \itempar{Bend}{voice parameters!bend}
   Bend.
   It modifies the pitch bend control.  It is possible to make the pitch bend
   control work in the opposite direction.

   \itempar{Offset}{voice parameters!offset}
   Offset.
   It shifts the overall pitch of the engine (up or down) relative to the rest
   of the engines.

   \itempar{440Hz}{subsynth!freq 440hz}
   Frequency 440Hz.
   Fixes the base frequency to 440Hz.
   One can adjust it with detune settings.

   \itempar{Eq.T}{subsynth!freq eq t}
   Frequency Equalise Time.
   Sets how the frequency varies according to the keyboard.
   Set to the leftmost setting for a fixed frequency.

   \itempar{Octave}{subsynth!freq octave}
   Frequency Octave.
   Octave Shift.

   \itempar{Detune Type}{subsynth!freq detune type}
   Frequency Detune Type.
   Sets the "Detune" and "Coarse Detune" behaviour

   \itempar{Coarse Det}{subsynth!freq detune coarse}
   Frequency Coarse Detune, "C.Detune".

   \itempar{Frequency Env}{subsynth!freq env}
   Frequency Envelope Stock Sub-Panel.

   \begin{enumber}
      \item \textbf{Enable}
      \item \textbf{A.value} or \textbf{A.val}
      \item \textbf{A.dt}
      \item \textbf{R.dt}
      \item \textbf{R.val}
      \item \textbf{Stretch}
      \item \textbf{frcR}
      \item \textbf{C}
      \item \textbf{P}
      \item \textbf{E}
   \end{enumber}

   See \sectionref{fig:amplitude_envelope_editor}, for more details.

\subsection{SUBsynth / OVERTONES}
\label{subsec:subsynth_overtones}
   By default harmonic overtones are exact multiples of the base frequency, but
   this section allows one to shift them in various unharmonic ways to produce
   metalic or noisy variations.
   \begin{enumber}
      \item \textbf{Overtones Position}
      \item \textbf{Par1}
      \item \textbf{Par2}
      \item \textbf{ForceH}
   \end{enumber}

   \setcounter{ItemCounter}{0}      % Reset the ItemCounter for this list.

   \itempar{Overtones Position}{subsynth!overtone position}
   Subsynth Overtones Position.
   Sets the type of overtone variation. For \textsl{Harmonic} there is no
   control, so the other parameters are inactive. Similarly \textsl{par2} does
   nothing for \textsl{Shift} so is disabled for that variation.

   Values: \texttt{Harmonic*, ShiftU, ShiftL, PowerU, PowerL, Sine, Power, Shift}

\begin{figure}[H]
   \centering
   \includegraphics[scale=1.0]{bottom-panel/instrument-edit/SUB/harmonic-type.jpg}
   \caption{Harmonic Type Dropdown}
   \label{fig:harmonic_type_dropdown}
\end{figure}

   \itempar{Par1}{subsynth!overtone par1}
   Subsynth Overtones Par1.
   Spreads the harmonics according to the 'position' type.

   Values: \texttt{0* to 127}

   \itempar{Par2}{subsynth!overtone par1}
   Subsynth Overtones Par2.
   Provides a further variation on the harmonics spread.

   Values: \texttt{0* to 127}

   \itempar{ForceH}{subsynth!overtone forceh}
   Subsynth Overtones ForceH.
   Moves the shifted harmonics by a variable amount towards to the nearest actual
   multiple of the fundamental.

   Values: \texttt{0* to 127}

\subsection{SUBsynth / FILTER}
\label{subsec:subsynth_filter}

   \begin{enumber}
      \item \textbf{Enabled}
      \item \textbf{Filter Params} (stock sub-panel)
      \item \textbf{Filter Env} (stock sub-panel)
      \item \textbf{Stereo}
      \item \textbf{Filter Stages}
      \item \textbf{Mag. Type}
      \item \textbf{Start}
   \end{enumber}

   \setcounter{ItemCounter}{0}      % Reset the ItemCounter for this list.

   \itempar{Enabled}{subsynth!filter enable}
   SUBsynth Filter Enabled.

   \itempar{Filter Params}{subsynth!filter params}
   Filter Params.  See
   \sectionref{subsubsec:filter_parameters_user_interface},
   which describes this stock sub-panel.

   \itempar{Filter Env}{subsynth!filter env}
   Filter Params.  See
   \sectionref{subsubsec:envelope_settings_for_filter},
   which describes this stock sub-panel.

   \itempar{Stereo}{subsynth!filter stereo}
   SUBsynth Stereo.
   Make the instrument stereo. The CPU usage goes up about 2 times.
   This item isn't really a \textbf{FILTER} item, it is just located in that
   same area.

   \itempar{Filter Stages}{subsynth!filter stages}
   Filter Stages.  Filter Order.
   Sets the number of filter stages applied to white noise. This parameter
   affects the CPU usage.

   Values: \texttt{0, 1, 2*, 3, 4, 5}

   \itempar{Mag. Type}{subsynth!filter mag type}
   Magnitude Type.
   Sets the type of magnitude settings (linear versus dB values)

\begin{figure}[H]
   \centering
   \includegraphics[scale=1.0]{bottom-panel/instrument-edit/SUB/mag-type.jpg}
   \caption{SUBSynth Magnitude Type Dropdown}
   \label{fig:subsynth_magnitude_type_dropdown}
\end{figure}

   Values: \texttt{Linear, -40dB, -60dB, -80dB, -100dB}

   \itempar{Start}{subsynth!filter start type}
   Start Type.
   How to start the filters.

\begin{figure}[H]
   \centering
   \includegraphics[scale=1.0]{bottom-panel/instrument-edit/SUB/start-type.jpg}
   \caption{SUBsynth Start Type}
   \label{fig:subsynth_start_type}
\end{figure}

   Values: \texttt{Zero, RND, Max.}

\subsection{SUBsynth / Harmonics}
\label{subsec:subsynth_harmonics}

   The harmonics settings controls the harmonic intensities/relative bandwidth.
   Moving the sliders upwards increases the relative bandwidth.  Please note
   that, if one increases the number of harmonics, the CPU usage increases. Right
   click to set the parameters to default values.

   This section consists of 64 sliders to control the amplitude of the narrow
   noise band at a given harmonic, and 64 sliders to control the bandwidth of
   each band.

   The top row of SUBsynth sliders sets the \textsl{relative} amplitude.  This
   use of the word "relative" is an important distinction, as the overall level
   of the output is normalised; all actual levels will be dependent on whichever
   is the highest.

   The bottom row sets the bandwidth of each harmonic. If one has just the
   fundamental, and drops the bandwidth to the minimum, one gets very nearly a
   sinewave.  Set it to maximum and it is very obviously filtered noise.


%-------------------------------------------------------------------------------
% vim: ts=3 sw=3 et ft=tex
%-------------------------------------------------------------------------------


% Kit Edit User-Interface items

%-------------------------------------------------------------------------------
% yum_kitedit
%-------------------------------------------------------------------------------
%
% \file        yum_kitedit.tex
% \library     Documents
% \author      Chris Ahlstrom
% \date        2015-06-07
% \update      2015-06-27
% \version     $Revision$
% \license     $XPC_GPL_LICENSE$
%
%     Provides the Kit section of yoshimi-user-manual.tex.
%
%-------------------------------------------------------------------------------

\section{Kit Edit}
\label{sec:kit_edit}

   The \textsl{Yoshimi} Kit dialog is a dialog for creating a
   set of drums or layered instruments.
   It provides a way to use individual voices and synth blocks to create
   drumlike sounds, or complex layered sounds.
   Within this window one can create drum kits, layered instruments, or one
   can combine more instruments into one instrument.  

   Is this true of \textsl{Yoshimi}?:

   Item 0 is a special type: it cannot be disabled (but it can be muted), to
   edit it one must use "ADs edit" or "SUBs edit" from the part window.

\begin{figure}[H]
   \centering 
   \includegraphics[scale=1.0]{bottom-panel/instrument-edit/Kit/instrument-kit-edit.jpg}
   \caption{Kit Edit Dialog}
   \label{fig:kit_edit_dialog}
\end{figure}

   \begin{enumber}
      \item \textbf{Rows 1 to 16.}
         This dialog contains 16 identical rows containing the following
         elements, in the order given:
         \begin{enumber}
            \item \textbf{No.}
            \item \textbf{Enable}
            \item \textbf{M.}
            \item \textbf{Instrument Name}
            \item \textbf{Min.k}
            \item \textbf{m} (set minimum note)
            \item \textbf{R} (reset default note range)
            \item \textbf{M} (set maximum note)
            \item \textbf{Max.k}
            \item \textbf{ADsynth}
               \begin{enumber}
                  \item \textbf{Enable}
                  \item \textbf{edit}
               \end{enumber}
            \item \textbf{SUBsynth}
               \begin{enumber}
                  \item \textbf{Enable}
                  \item \textbf{edit}
               \end{enumber}
            \item \textbf{PADsynth}
               \begin{enumber}
                  \item \textbf{Enable}
                  \item \textbf{edit}
               \end{enumber}
            \item \textbf{FX.r}
         \end{enumber}
      \item \textbf{Mode}
      \item \textbf{Drum mode}
      \item \textbf{Close Window}
   \end{enumber}

   Some items described in ZynAddSubFX that aren't seen in any diagrams:

   \begin{enumber}
      \item Kit Mode. Enable the kit mode.
      \item Protect the kit. when loading an instrument, only item 0 will be
      changed,
      Other items will remain untouched. This allows one to combine more
      instruments. If one wants to add more instruments to the kit, one must
      copy the item 0 to another item, because the item 0 will be replaced. If
      one loads master settings or clearx the instrument/master setting,
      the kit is cleared .
      \item Swap/Copy. Swap two items or copy a item to other item.
   \end{enumber}

   \setcounter{ItemCounter}{0}      % Reset the ItemCounter for this list.

   \itempar{No.}{kit!row number}
   Kit Row Number.
   Kit Item Number.
   A simple label to indicate the instrument number in the kit.

   \itempar{Enable}{kit!enable}
   Kit Row Enable.

   Value: \texttt{Off*, On}

   \itempar{M.}{kit!M.}
   Kit Row "M".
   Mute an item of the kit.

   \itempar{Instrument Name}{kit!name}
   Kit Instrument Name.

   \itempar{Min.k}{kit!minimum key}
   Kit Instrument Minimum Key.
   Sets the minimum key of the item of the kit.

   \itempar{m}{kit!m}
   Sets the minimum note of this instrument to value of the last note
   pressed.

   \itempar{R}{kit!R}
   Resets the minimum and maximum notes to their default values.

   \itempar{M}{kit!M}
   Sets the maximum note of this instrument to value of the last note
   pressed.

   \itempar{Max.k}{kit!maximum key}
   Kit Instrument Maximum Key.
   Sets the maximum key of the item of the kit.

   \itempar{ADsynth}{kit!addsynth}
   Kit ADDsynth.
   A checkbox is provided to enable/disable this synth component, and
   an edit button is provided to edit the component.

   \itempar{SUBsynth}{kit!subsynth}
   Kit SUBsynth.
   A checkbox is provided to enable/disable this synth component, and
   an edit button is provided to edit the component.

   \itempar{PADsynth}{kit!padsynth}
   Kit PADsynth.
   A checkbox is provided to enable/disable this synth component, and
   an edit button is provided to edit the component.

   \itempar{FX.r}{kit!fx.r}
   Kit Effect.
   Chooses the Part Effect (PartFX) to process the item (OFF means that is
   unprocessed). 

   Values: \texttt{OFF, FX1, FX2, FX3}

   \itempar{Mode}{kit!mode}
   Kit Mode.

   \begin{itemize}
      \item \textbf{OFF} means no kit is enabled, so one only has the Add,
         Sub, and Pad sounds in the Instrument Edit window.
      \item \textbf{MULTI} means all the kit items will sound together
         regardless of their note ranges.
      \item \textbf{SINGLE} means only the lowest numbered item will sound
         in a given note range. There will be no overlap.
   \end{itemize}

   For example:
   Item 0 has \textbf{Min.k} set to 0 and \textbf{Max.k} set to 60, and
   Item 1 has \textbf{Min.k} set to 40 and \textbf{Max.k} set to 127.

   In \textbf{SINGLE mode}, only Item 0 will sound in the note range 0 to
   60, and Item 1 will sound in the range 61 to 127.

   In \textbf{MULTI} mode, only Item 0 will sound in the range 0 to 40, both
   items will sound from 41 to 60, and only Item 1 will sound from 61 to
   127.

   Values: \texttt{OFF*, MULTI, SINGLE}.

   \itempar{Drum mode}{kit!drum mode}
   Kit Drum Mode.
   If drum-mode  is set, then microtonal tuning is ignored for this kit,
   otherwise it could make drum sounds very unpredictable!

   \itempar{Close Window}{kit!close}
   Close.

%-------------------------------------------------------------------------------
% vim: ts=3 sw=3 et ft=tex
%-------------------------------------------------------------------------------


% Banks

%-------------------------------------------------------------------------------
% yum_bank_collection
%-------------------------------------------------------------------------------
%
% \file        yum_bank_collection.tex
% \library     Documents
% \author      Chris Ahlstrom
% \date        2015-06-20
% \update      2015-07-18
% \version     $Revision$
% \license     $XPC_GPL_LICENSE$
%
%     Provides the banks_collection section of yoshimi-user-manual.tex.
%
%-------------------------------------------------------------------------------

\section{Banks Collection}
\label{sec:banks_collection}

   In this section, we attempt to collect and summarise all of the existing
   banks for \textsl{Yoshimi} and \textsl{ZynAddSubFX}
   that we can find.  Many of them
   are supplied by the two projects.

   Between all of the collections, there is a large amount of duplication.
   There is also semi-duplication, with slight variations on the same basic
   instrument.
   Various Linux distributions which package \textsl{ZynAddSubFX}
   and \textsl{Yoshimi}
   might add some banks to their versions of these packages.
   Thus, there are far more sound settings than we can discuss and categorise.

   One thing we're looking for is a good General MIDI (GM) bank for
   \textsl{Yoshimi}.  As part of our \textsl{Yoshimi Cookbook}
   \cite{cookbook}, we include a basic General MIDI bank for.
   However, there are number of patches with no good implementation in it.

\subsection{Yoshimi Banks}
\label{subsec:banks_collection_yoshimi}

   \textsl{Yoshimi} comes with the following banks, which may be found in
   \texttt{/usr/share/yoshimi/banks} as installed by the installer.  In this
   case, it is the Debian installer.

   \begin{enumber}
      \item \textbf{Arpeggios}.
         Also in ZynAddSubFX.
      \item \textbf{Bass}.
         Also in ZynAddSubFX.
      \item \textbf{Brass}.
         Also in ZynAddSubFX.
      \item \textbf{chip}.
      \item \textbf{Choir\_and\_Voice}
         Also in ZynAddSubFX, slightly different bank name.
      \item \textbf{Drums}.
         Also in ZynAddSubFX, but with only one drum kit included.
      \item \textbf{Dual}.
         Also in ZynAddSubFX.
      \item \textbf{Fantasy}.
         Also in ZynAddSubFX.
      \item \textbf{Guitar}.
         Also in ZynAddSubFX.
      \item \textbf{Misc}.
         Also in ZynAddSubFX.
      \item \textbf{Noises}.
         Also in ZynAddSubFX.
      \item \textbf{Organ}.
         Also in ZynAddSubFX.
      \item \textbf{Pads}.
         Also in ZynAddSubFX.
      \item \textbf{Plucked}.
         Also in ZynAddSubFX.
      \item \textbf{Reed\_and\_Wind}.
         Also in ZynAddSubFX, slightly different bank name.
      \item \textbf{Rhodes}.
         Also in ZynAddSubFX.
      \item \textbf{Splited}.
         Also in ZynAddSubFX, slightly different bank name.
      \item \textbf{Strings}.
         Also in ZynAddSubFX.
      \item \textbf{Synth}.
         Also in ZynAddSubFX.
      \item \textbf{SynthPiano}.
         Also in ZynAddSubFX.
      \item \textbf{The\_Mysterious\_Bank}.
         Also in ZynAddSubFX, slightly different bank name.
         ZynAddSubFx has three more mysterious banks (see next section).
      \item \textbf{Will\_Godfrey\_Collection}.
      \item \textbf{Will\_Godfrey\_Companion}.
   \end{enumber}

\subsection{Additional ZynAddSubFX Banks}
\label{subsec:banks_collection_zynaddsubfx}

   \textsl{ZynAddSubFX} comes with the following banks, which may be found in
   the source code \cite{zynsource} or installation packages of this project.
   \textsl{ZynAddSubFX} has some of the same
   banks (as far as we can tell) as \textsl{Yoshimi}, but with the following
   additions:

   \begin{enumber}
      \item \textbf{Companion}.
      \item \textbf{Cormi\_Noise and Cormi\_Sound} \cite{cormi}.
      \item \textbf{Laba170bank}.
      \item \textbf{olivers-100}.
         Some very good instruments, including sitar
         and steel drums.
      \item \textbf{the\_mysterious\_bank}.
      \item \textbf{the\_mysterious\_bank\_2}.
      \item \textbf{the\_mysterious\_bank\_3}.
      \item \textbf{the\_mysterious\_bank\_4}.
   \end{enumber}

\subsection{Additional Banks}
\label{subsec:banks_collection_additional}

   Here are some additional banks we have found, or have built ourselves.
   It often happens that, later on, a site is no longer available.
   Or we forget from whence we got the banks.
   In these cases, the banks are stored in the \texttt{contrib/banks}
   directory of this project.

   \begin{enumber}
      \item \textbf{Alex\_J}
         The site seems to be gone/expired. So one will find these in the
         "contrib/banks" directory for safekeeping.
      \item \textbf{Bells}
         *
      \item \textbf{C\_Ahlstrom}
         These are mine, but not yet made into a systematic bank.
         They are included with this document.
      \item \textbf{Chromatic Percussion}
         *
      \item \textbf{Drums\_DS}
         *
      \item \textbf{Electric Piano}
         *
      \item \textbf{Flute}
         *
      \item \textbf{folderol collection}
         \cite{folderol}, also found at \cite{zyndemos}.
      \item \textbf{Internet Collection}
         *
      \item \textbf{Leads}
         *
      \item \textbf{Louigi\_Verona\_Workshop}
         The site seems to be gone/expired. So one will find these in the
         "contrib/banks" directory for safekeeping.
      \item \textbf{Misc Keys}
         *
      \item \textbf{mmxgn Collection} \cite{mmxgn}
      \item \textbf{Piano}
         *
      \item \textbf{RB Zyn Presets}
         *
      \item \textbf{Vanilla}
         See \cite{zyndemos} for this bank, and for some demonstration files
         of \textsl{ZynAddSubFX} sounds, and some other nice links.
      \item \textbf{VDX}
         *
      \item \textbf{x31eq.com} \cite{x31eq}
      \item \textbf{XAdriano Petrosillo}
         *
      \item \textbf{Zen Collection}
         *
   \end{enumber}
   * The source of these banks is currently unknown.


   To find out how to integrate these into one's installation, see
   \sectionref{subsec:banks_and_roots_roots}, and
   \sectionref{subsubsec:menu_patch_sets_show_patch_banks}

%-------------------------------------------------------------------------------
% vim: ts=3 sw=3 et ft=tex
%-------------------------------------------------------------------------------


% NRPNS and effects in more detail

%-------------------------------------------------------------------------------
% yum_nrpns
%-------------------------------------------------------------------------------
%
% \file        yum_nrpns.tex
% \library     Documents
% \author      Chris Ahlstrom
% \date        2015-06-26
% \update      2015-09-07
% \version     $Revision$
% \license     $XPC_GPL_LICENSE$
%
%     Provides the concepts NRPNs and vector control.  And effects.
%
%-------------------------------------------------------------------------------

\section{Non-Registered Parameter Numbers}
\label{sec:nrpns}

   This section comes from the source-code documentation file
   \texttt{Zyn nrpn.txt} or the \textsl{ZynAddSubFx}
   online manual \cite{zyndoc} and the \textsl{Using\_NRPNS.txt} document that
   accompanies the \textsl{Yoshimi} source code.

   \textsl{Yoshimi} implements System and Insertion effects control in a
   manner compatible with \textsl{ZynAddSubFX}. As with all
   \textsl{Yoshimi}'s NRPNs, the controls can be sent on any MIDI channel.

\subsection{NRPN / Basics}
\label{subsection:nrpns_midi_nrpn_basics}

   NRPN stands for "Non Registered Parameters Number".
   NRPNs can control all System and Insertion effect parameters.
   Using NRPNs, \textsl{Yoshimi} can now directly set some part values
   regardless of what channel that part is connected to.  For example, one
   may change the reverb time when playing to keyboard, or
   change the flanger's LFO frequency.
%  One can disable the NRPN receiving by deselecting the "NRPN" checkbox
%  from the main window (near "Master Keyshift" counter).
   The controls can be sent on any MIDI channel 
   (the MIDI channels numbers are ignored).

   The parameters are:

   \begin{itemize}
      \item \textbf{NRPN MSB}
      (coarse) (99 or 0x63) sets the system/insertion effects
      (4 for system effects or 8 for insertion effects).
      We abbreviate this value as \texttt{Nhigh}.
      \item \textbf{NRPN LSB}
      (fine) (98 or 0x62) sets the number of the effect (first
      effect is 0).
      We abbreviate this value as \texttt{Nlow}.
      \item \textbf{Data entry MSB}
      (coarse) (6) sets the parameter number of effect to
      change (see below).
      We abbreviate this value as \texttt{Dhigh}.
      \item \textbf{Data entry LSB}
      (fine) (26) sets the parameter of the effect.
      We abbreviate this value as \texttt{Dlow}.
   \end{itemize}

   If the effect/parameter doesn't exists or is set to none, then the NRPN is
   ignored.

   One must send NRPN coarse/fine before sending Data entry coarse/fine.  If
   the effect/parameter doesn't exists or is set to none, then the NRPN is
   ignored.

   It's generally advisable to set NRPN MSB before LSB However, once MSB has
   been set one can set a chain of LSBs if they share the same MSB.

   The data CCs associated with these are 6 for MSB and 38 for LSB.

   Only when an NRPN has been established can the data values be entered
   (they will be ignored otherwise).

   If a supported control is identified, these data values will be stored
   locally (if needed) so that other NRPNs can be set.

   Whenever either byte of the NRPN is changed, the data values will be
   cleared (but stored settings will not be affected).

   If either NRPN byte is set to 127, all data values are ignored again.

   In \textsl{Yoshimi} NRPNs are not themselves channel-sensitive, but the
   final results will often be sent to whichever is the current channel.

   \textsl{Yoshimi} also supports the curious 14-bit NRPNs, but this shouldn't
   be noticeable to the user. In order to deal with this, and also some
   variations in the way sequencers present NRPNs generally, if a complete
   NRPN is set
   (i.e. \texttt{Nhigh}, \texttt{Nlow}, \texttt{Dhigh}, \texttt{Dlow}),
   then the data bytes can be in
   either order, but must follow \texttt{Nhigh} and \texttt{Nlow}.

   (In these notes, where practical we also list the 14 bit values in square
   brackets.)

   After this, for running values, once
   \texttt{Dhigh} and \texttt{Dlow} have been set if one
   changes either of these, the other will be assumed.
   For example, starting with \texttt{Dhigh} = 6 and \texttt{Dlow} = 20:

   Change \texttt{Dlow} to 15 and \textsl{Yoshimi} will regard this as a
   command \texttt{Dhigh} 6 + \texttt{Dlow} 16 Alternatively change
   \texttt{Dhigh} to 2 and \textsl{Yoshimi} will regard this as a
   command \texttt{Dhigh} 2 + \texttt{Dlow} 20.
   This can be useful but may have unintended consequences!
   If in doubt change either of the NRPN bytes and both data bytes will be
   cleared.

   Additionally there is 96 for data increment and 97 for decrement.

   Data increment and decrement operation enables one to directly change the
   data LSB by between 0 and 63. To change the MSB add 64 to cover the same
   range. Setting 0 might seem pointless, but it gives an alternative way
   to make an initial setting if one's sequencer doesn't play nice.

   Although data increment and decrement are only active if a valid NRPN has
   been set, they are otherwise quite independent single CCs.  For example:

   \begin{verbatim}
      Start Value       Command value   Result
      ----- -----       --------------  ------
      LSB     5           inc 20      25
      MSB     7           inc 68      11
      LSB     128(off)    inc 1       1
      MSB     126         dec 74      116
      MSB     128(off)    dec 65      127
   \end{verbatim}

   A small example (all values in this example are hex):

   \begin{verbatim}
       B0 63 08 // Select the insertion effects
       B0 62 01 // Select the second effect (remember: the first is 00 and not 01)
       B0 06 00 // Select the effect parameter 00
       B0 26 7F // Change the parameter of effect to the value 7F (127)
   \end{verbatim}

   \textbf{WARNING}:
   Changing of some of the effect parameters produces clicks when sounds
   passes thru these effects.  We advise one to change only when the sound
   volume that passes through the effect is very low (or silence).  Some
   parameters produce clicks when they are changed rapidly.

   Here are the effects parameter number (for Data entry, coarse).
   The parameters that produces clicks are written in \textcolor{red}{red}
   and have (AC) after their entry (always clicks).
   The parameter that produces clicks only when they are changed fast are
   written in \textcolor{blue}{blue} and have a (FC) after the entry (Fast
   Clicks).
   Most parameters have the range from 0 to 127.
   When parameters have another range, it is written as "(low...high)".

   Here are the basic formats:

   \begin{enumerate}
      \item Send NRPN: 
      \begin{itemize}
         \item MSB = 64 (same as for vectors)
         \item LSB = 0
      \end{itemize}
      \item Send Data MSB (6); all value ranges start from zero, not 1.
      \begin{itemize}
         \item 0 : data LSB = part number
         \item 1 : data LSB = program number
         \item 2 : data LSB = controller number
         \item 3 : data LSB = controller value
         \item 4 : data LSB = part's channel number (15 to 127 disconnects
            the part from any channel)
         \item 5 : data LSB = part's audio destination, one of
                    1 = main L\&R;
                    2 = direct L\&R;
                    3 = both;
                    all other values are ignored
         \item 7 : data LSB = main volume (not yet implemented)
         \item 35 (0x23) : data LSB = controller LSB value (not yet implemented)
         \item 39 (0x27) : data LSB = main volume LSB (not yet implemented)
         \item Other values are currently ignored.
      \end{itemize}
   \end{enumerate}

   Other values are currently ignored by \textsl{Yoshimi}.

   NOTE:  THE PARAGRAPHS THAT FOLLOW COULD BE MOVED TO THE EFFECTS SECTION.

\subsection{NRPN / Vector Control}
\label{subsection:nrpns_midi_nrpn_vector_control}

   Vector control is a way to control more than one part with the
   controllers.  It is a little bit reminiscent of the "vector" control knob
   on the Yamaha PSS-790 consumer MIDI synthesizer.  Vector control is only
   possible if one has 32 or 64 parts active.

   Vector control has been extended so that there are four independent
   'features' that each axis can control, One is fixed as \textsl{volume} (if
   enabled) but the other three can be any valid CC, and can also be
   reversed. The vector 'sweep' CCs are split out very early in the MIDI
   chain, and the new CCs created are fed back in before any other
   processing. The result of this is that once we eventually get MIDI-learn
   implemented, the control possibilities will expand dramatically.
   \textsl{(Will notes: "sorry about the extreme delay :(")}

   In vector mode parts will still play together but the vector controls can
   change their volume, pan, filter cutoff in pairs, controlled by
   user-defined CCs set up with NRPNs.

   One must set the X axis CC before the Y axis, but if one doesn't set the
   Y axis at all, one can run just a single axis.
   If one has only 32 parts active, Y settings are ignored.

   For example:
   parts 1 and 17 can be set as x1 \& x2 (volume only) while parts 33 and 49
   can be y1 \& y2 (pan only).

   Independently of this Parts 2 \& 18 could use filter and pan from another
   CC.

   Setting up vector control is currently done as follows.

   In the required channel send:

   \begin{itemize}
      \item NRPN MSB (99) set to 64
      \item NRPN LSB (98) set to 1 [8192]
      \item Data MSB (6) set mode:
      \begin{itemize}
         \item 0 = X CC
         \item 1 = Y CC
         \item 2 = X features
         \item 3 = Y features
         \item 4 = x1 instrument (optional)
         \item 5 = x2 instrument (optional)
         \item 6 = y1 instrument (optional)
         \item 7 = y2 instrument (optional)
      \end{itemize}
   \end{itemize}

   Setting CC for X enables vector control; any value outside the above list
   disables it.

   Data LSB (38) value to set features:

   \begin{itemize}
       \item 1 = Volume
       \item 2 = Pan
       \item 4 = Filter Cutoff (Brightness)
       \item 8 = default is Mod Wheel
       \item 0x12 = 18 = Reversed Pan
       \item 0x24 = 36 = Reversed Filter Cutoff
       \item 0x48 = 72 = Reversed Mod Wheel
   \end{itemize}

   The feature numbers are chosen so they can be combined. So, 5 would be
   Volume + Brightness and 19 would be Volume + Reversed Pan.

   Setting the sweep CC for the X axis enables vector control. It also sets,
   but doesn't enable the default X axis features.  Setting the sweep CC for
   the Y axis sets, but doesn't enable the default Y axis features.  If you
   don't enable any features not a lot will happen.

   Optional settings (MSB value):

   \begin{itemize}
      \item 4 = x1 instrument
      \item 5 = x2 instrument
      \item 6 = y1 instrument
      \item 7 = y2 instrument
      \item 8 = set CC for X feature 2
      \item 9 = set CC for X feature 4
      \item 10 = set CC for X feature 8
      \item 11 = set CC for Y feature 2
      \item 12 = set CC for Y feature 4
      \item 13 = set CC for Y feature 8
   \end{itemize}
              
Any data MSB value outside the above list disables vector control.

Sweep CCs and feature CCs are sanity checked.
    
   An Example: From channel 1, send the following CCs

   \begin{verbatim}
      CC      Value
      99       64
      98        1
       6        0
      38       14
      98        1 *
       6        1
      38       15
      98        1 *
       6        2
      38        1
      98        1 *
       6        3
      38        2
   \end{verbatim}

   This sequence will set up CC 14 as the X axis incoming controller,
   and CC 15 as the Y axis incoming controller, with X set to volume control
   and Y set to pan control.

   One can either go on with the NRPNs to set the instruments (this will load
   and enable instruments from the current bank), or enable and load
   them by hand.  For channel 1 this would be part 1 and 17 for X and part 33
   and 49 for Y.

   The (*) CCs ensure that the data bytes are reset each time. This is not
   really necessary for the earlier commands, but should be done if one sets
   the instruments with NRPNs as well, otherwise one will try to set them
   twice.

\subsection{NRPN / Effects Control}
\label{subsection:nrpns_midi_nrpn_effects_control}

\paragraph{Reverb}

   \begin{itemize}
      \item \textcolor{blue}{00 - Volume or Dry/Wet (FC)}
      \item \textcolor{blue}{01 - Pan (FC)}
      \item 02 - Reverb Time
      \item \textcolor{blue}{03 - Initial Delay (FC)}
      \item 04 - Initial Delay Feedback
      \item \textcolor{magenta}{05 - reserved}
      \item \textcolor{magenta}{06 - reserved}
      \item 07 - Low Pass
      \item 08 - High Pass
      \item 09 - High Frequency Damping (64..127) 64=no damping
      \item \textcolor{red}{10 - Reverb Type (0..1) 0-Random, 1-Freeverb (AC)}
      \item \textcolor{red}{11 - Room Size (AC)}
   \end{itemize}

\paragraph{Echo}

   \begin{itemize}
      \item \textcolor{blue}{00 - Volume or Dry/Wet (FC)}
      \item \textcolor{blue}{01 - Pan (FC)}
      \item \textcolor{red}{02 - Delay (AC)}
      \item \textcolor{red}{03 - Delay between left and right (AC)}
      \item \textcolor{blue}{04 - Left/Right Crossing (FC)}
      \item 05 - Feedback
      \item 06 - High Frequency Damp
   \end{itemize}

\paragraph{Chorus}

   \begin{itemize}
      \item \textcolor{blue}{00 - Volume or Dry/Wet (FC)}
      \item \textcolor{blue}{01 - Pan (FC)}
      \item 02 - LFO Frequency
      \item 03 - LFO Randomness
      \item 04 - LFO Type (0..1)
      \item 05 - LFO Stereo Difference
      \item 06 - LFO Depth
      \item 07 - Delay
      \item 08 - Feedback
      \item \textcolor{blue}{09 - Left/Right Crossing (FC)}
      \item \textcolor{magenta}{10 - reserved}
      \item \textcolor{red}{11 - Mode (0..1) (0=add, 1=subtract) (AC)}
   \end{itemize}

\paragraph{Phaser}

   \begin{itemize}
      \item \textcolor{blue}{00 - Volume or Dry/Wet (FC)}
      \item \textcolor{blue}{01 - Pan (FC)}
      \item 02 - LFO Frequency
      \item 03 - LFO Randomness
      \item 04 - LFO Type (0..1)
      \item 05 - LFO Stereo Difference
      \item 06 - LFO Depth
      \item 07 - Feedback
      \item \textcolor{red}{08 - Number of stages (0..11) (AC)}
      \item \textcolor{blue}{09 - Let/Right Crossing (FC)}
      \item \textcolor{red}{10 - Mode (0..1) (0=add, 1=subtract) (AC)}
      \item 11 - Phase
   \end{itemize}

\paragraph{AlienWah}

   \begin{itemize}
      \item \textcolor{blue}{00 - Volume or Dry/Wet (FC)}
      \item \textcolor{blue}{01 - Pan (FC)}
      \item 02 - LFO Frequency
      \item 03 - LFO Randomness
      \item 04 - LFO Type (0..1)
      \item 05 - LFO Stereo Difference
      \item 06 - LFO Depth
      \item 07 - Feedback
      \item 08 - Delay (0..100)
      \item \textcolor{blue}{09 - Left/Right Crossing (FC)}
      \item 10 - Phase
   \end{itemize}

\paragraph{Distortion}

   \begin{itemize}
      \item \textcolor{blue}{00 - Volume or Dry/Wet (FC)}
      \item \textcolor{blue}{01 - Pan (FC)}
      \item 02 - Left/Right Crossing
      \item \textcolor{blue}{03 - Drive (FC)}
      \item \textcolor{blue}{04 - Level (FC)}
      \item 05 - Type (0..11)
      \item 06 - Invert the signal (negate) (0..1)
      \item 07 - Low Pass
      \item 08 - High Pass
      \item 09 - Mode (0..1) (0=mono,1=stereo)
   \end{itemize}

\paragraph{EQ}

   \begin{itemize}
      \item \textcolor{blue}{00 - Gain (FC)}
   \end{itemize}

   All other settings of the EQ are shown in a different way.
   The N represent the band ("B." setting in the UI) and the first band is 0
   (and not 1), like it is shown in the UI.
   Change the "N" with the band one likes.
   If one wants to change a band that doesn't exist, the NRPN will be ignored.

   \begin{itemize}
      \item \textcolor{red}{10+N*5 - Change the mode of the filter (0..9) (AC)}
      \item 11+N*5 - Band's filter frequency
      \item 12+N*5 - Band's filter gain
      \item 13+N*5 - Band's filter Q (bandwidth or resonance)
      \item \textcolor{magenta}{14+N*5 - reserved}
   \end{itemize}

   Example of setting the gain on the second band in the EQ module:

   \begin{itemize}
      \item The bands start counting from 0, so the second band is
         1 =\textless N=1.
      \item The formula is 12+N*5 =\textless 12+1*5=17, so the number of effect
         parameter (for Data entry coarse) is 17.
   \end{itemize}

\paragraph{DynFilter}

   \begin{itemize}
      \item 0 - Volume
      \item 1 - Pan
      \item 2 - LFO Frequency
      \item 3 - LFO Randomness
      \item 4 - LFO Type
      \item 5 - LFO Stereo Difference
      \item 6 - LFO Depth
      \item 7 - Filter Amplitude
      \item 8 - Fitler Amplitude Rate Change
      \item 9 - Invert the signal (negate) (0..1)
   \end{itemize}

   Click behaviour of DynFilter has not yet been tested.

\paragraph{Yoshimi Extensions}

   If the Data MSB bit 6 is set (64) then Data LSB sets the effect type
   instead of a parameter number.  This must be set before making a parameter
   change.

   \begin{itemize}
      \item 0 - Reverb
      \item 1 - Echo
      \item 2 - Chorus
      \item 3 - Phaser
      \item 4 - AlienWah
      \item 5 - Distortion
      \item 6 - EQ
      \item 7 - DynFilter
   \end{itemize}

   For Insert effects, if the Data MSB bit 5 is set (32) then Data LSB sets
   the destination part number. 127 is off and 126 is the Master Output.

   A complete example:

   \begin{itemize}
      \item 99 -   8 ~ insert effects
      \item 98 -   3 ~ number 4 (as displayed)
      \item 6 -  32 ~ set destination
      \item 38 - 126 ~ Master Out
      \item 99 -   8  *
      \item 98 -   3  *
      \item 6 -  64 ~ change effect
      \item 38 -   4 ~ Alienwah
      \item 99 -   8  *
      \item 98 -   3  *
      \item 6 -   0 ~ Dry/Wet
      \item 38 -  30 ~ value
   \end{itemize}

   Notes (*): these repeats are not needed as far as \textsl{Yoshimi}
   is concerned, but some sequencers get unhappy without them.

   Change just a parameter on an exisiting system effect:

   \begin{itemize}
      \item 99 -   4 ~ system effects
      \item 98 -   0 ~ the first effect
      \item 6 -   1 ~ Pan
      \item 38 -  75 ~ value
   \end{itemize}

\subsection{NRPN / Dynamic System Settings}
\label{subsection:nrpns_dynamic_system_settings}

   Almost all dynamic setup (i.e. that doesn't require a restart) can now be
   done via NRPNs, so a MIDI file can manage \textsl{Yoshimi} starting from a
   pretty random state, and set up important features like Bank and Program
   change behavior and the number of available parts.

   In parallel with this setup, there is a command to list all of these
   settings. One can also list the available bank roots, the banks in any
   root, and instruments in any bank, along with their numeric IDs. These IDs
   can then be used with normal MIDI CCs to get exactly the instrument you
   want at any time.

   This arrangment looks positively steam-punk, but is actually very easy to
   use, requiring only a command line interface and any utility that can send
   MIDI CCs. NRPNs aren't special. They are simply a specific pattern of CCs.
   \textsl{Yoshimi}'s implementation is very forgiving, doesn't mind if you
   stop halfway through (will just get on with other things while it waits),
   and will report exactly what it is doing.  So ...

   ... If \textsl{Yoshimi} has been started from the command line (but not
   necessarily in the no-GUI mode), all of the system settings that don't
   require a restart can now be viewed by sending the appropriate NRPN. Most
   of them can also be changed in this way.

   To access this functionality, set NRPN MSB (CC 99) to 64 and NRPN LSB (CC
   98) to 2 (8130).

   After that send the following DATA values. Commands with LSB x don't
   actually use DATA LSB, but one still needs to send it (unless it has
   already been set by a previous command in this control group).

   \begin{table}[H]
      \centering
      \caption{Dynamic System Commands}
      \label{table:dynamic_system_commands}
      \begin{tabular}{r l l}

\textbf{DATA MSB} & \textbf{DATA LSB} & \textbf{Setting} \\

  2 & LSB key       & Set master key shift, 52\textless=key\textless=76 (-12 to + 12) \\
  7 & LSB volume    & Set master Volume 'volume' \\
100 & LSB \textgreater 63 & Send reports to Reports window, otherwise to stderr. \\
109 & LSB x [13843] & List the following: \\
    & &             Reports destination \\
    & &             Current Root path \\
    & &             Current Bank \\
    & &             CC to control Root path change \\
    & &             CC to control Bank change \\
    & &             Accept Program change (enabled/disabled) \\
    & &             Activate part when program changed (enabled/disabled) \\
    & &             CC to control Extended Program change (instruments 128-159) \\
    & &             Number of available parts \\
110 & LSB x [13970]      & List all root paths \\
111 & LSB path [14224]   & List all banks in Root ID 'path';
                           path=127 for current root) \\
112 & LSB bank [14351]   & List all instruments, current root, bank 'bank' \\
    & &                       (bank = 127 for current bank in current root) \\
113 & LSB root      & Set CC to control Root path change (root\textgreater 119 disables) \\
114 & LSB bank      & Set CC to control Bank change (bank\textgreater 119 disables) \\
115 & LSB \textgreater 63      & Enable Program change otherwise disable \\
116 & LSB \textgreater 63      & Enable activation of part when program changed \\
117 & LSB extprog   & Set CC control Extended program change
                        (extprog\textgreater 119 disables) \\
118 & LSB parts     & Set number of available parts (parts = 16, 32 or 64) \\
119 & LSB x  [15113]  &  Save all dynamic settings \\

      \end{tabular}
   \end{table}

%-------------------------------------------------------------------------------
% vim: ts=3 sw=3 et ft=tex
%-------------------------------------------------------------------------------


% "Vector" control

%-------------------------------------------------------------------------------
% yum_vector_control
%-------------------------------------------------------------------------------
%
% \file        yum_vector_control.tex
% \library     Documents
% \author      Chris Ahlstrom
% \date        2015-10-24
% \update      2017-03-03
% \version     $Revision$
% \license     $XPC_GPL_LICENSE$
%
%     Provides the concepts NRPNs and vector control.  And effects.
%
%-------------------------------------------------------------------------------

\section{Vector Control}
\label{sec:vector}

   This section comes from the source-code documentation file
   \texttt{doc/Vector\_Control.txt}.
   Also see 
   \sectionref{subsubsec:stock_settings_elements_automation}, for a discussion
   of vector automation, and
   \sectionref{subsubsec:menu_yoshimi_vectors}, for a discusson of the vector
   configuration dialog.

   Vector load and save also work from the command-line, for a complete vector
   set, with all mappings, instruments, etc.
   One can independently decide which
   channel to load and save from, so one can actually build up a vector set in
   (say) channel 3, then later decide to use it in channel 7.
   The vector settings file
   \index{vector! settings file}
   has the extension \texttt{.xvy} standing for 
   \textsl{Xml / Vector / Yoshimi}.

   Vector controls can be set on any and all channels,
   stored in both patch sets and saved state.

\subsection{Vector / Basics}
\label{subsection:vector_basics}

   Vector control is a way to control more than one part with the controllers.
   It is a little bit reminiscent of the "vector" control knob on the Yamaha
   PSS-790 consumer MIDI synthesizer.  Vector control is only possible if one
   has 32 or 64 parts active.  Setup is per MIDI channel, so one can have
   totally different vector behaviour on, say, channel 1 and channel 5.
   \index{vector!base channel} The term \textsl{base channel} refers to the
   incoming MIDI channel that a particular vector setup will respond to, and
   the base channel directly relates to the 1 to 16 range of parts in the mixer
   panel.

   Vector control has been extended so that there are four independent
   'features' that each axis can control. One is fixed as \textsl{Volume} (if
   enabled) but the other three can be any valid CC, and can also be reversed.
   The vector 'sweep' CCs are split out very early in the MIDI chain, and the
   new CCs created are fed back in before any other processing. The result of
   this is that once we eventually get MIDI-learn implemented, the control
   possibilities will expand dramatically.

   In vector mode parts will still play together but the vector controls can
   change their volume, pan, filter cutoff in pairs, controlled by user-defined
   CCs set up with NRPNs.

   One must set the X axis CC before the Y axis, but if one doesn't set the Y
   axis at all, one can run just a single axis.  If one has only 32 parts
   active, Y settings are ignored.  One cannot make any Y axis settings until,
   at the very least, the X CC has be set, and if one sets that back to zero,
   the Y axis is again disabled.  Setting an X axis control CC will immediately
   enable the base channel part and the part number + 16, as well as setting
   \textsl{Yoshimi} for 32 parts, if it was less than that. If it was the same,
   or was set to 64 parts, then nothing changes. Setting the CC will also
   ensure that both parts are actually responding to that MIDI channel (they
   might have been set to something else, or even disabled).

   Setting a Y axis control CC will immediately enable the part (base channel +
   32) and the part number + 48, as well as setting for 64 parts, if it was
   less than that. If it was the same, again nothing changes, and again the
   parts are set to the correct MIDI channel.

   The instruments that are loaded into the respective parts are always shown,
   regardless of whether there is a configured vector or not. They are a direct
   analogue of the main part instrument selector and behave in the same way
   (i.e.  click on them to open the instrument selection window).  There are
   tooltips for these items, along with the base channel and controller.

   The features are pretty self descriptive as soon and you click on them. They
   apply inversely to the \textsl{pair} of instruments on each axis.  One could
   have all four if One wanted to, but it would probably sound messmessy

   Options are pretty obvious, and follow a familiar pattern for load, save,
   etc.  Loading or saving a vector will put the leafname in the bottom text
   field.

   Disabling or clearing vectors will \textsl{not} change the number of parts
   because they may have already been set to increased numbers for some other
   purpose.  Similarly disabling or clearing vectors will  \textsl{not} clear
   any instrument patches that have been loaded.
   Of course making any changes to the parts outside vector control will likely
   mess them up. It won't do any harm, just be puzzling.

   For example:
   parts 1 and 17 can be set as x1 \& x2 (volume only) while parts 33 and 49
   can be y1 \& y2 (pan only).

   Independently of this Parts 2 \& 18 could use filter and pan from another
   CC.

\subsection{Vector / Vector Control}
\label{subsection:vector_control}

   Setting up vector control is currently done as follows.
   In the required channel send:

   \begin{itemize}
      \item NRPN MSB (99) set to 64
      \item NRPN LSB (98) set to 1 [8192]
      \item Data MSB (6) set mode:
      \begin{itemize}
         \item 0 = X sweep CC
         \item 1 = Y sweep CC
         \item 2 = enable X features
         \item 3 = enable Y features
         \item 4 = x1 instrument (optional)
         \item 5 = x2 instrument (optional)
         \item 6 = y1 instrument (optional)
         \item 7 = y2 instrument (optional)
      \end{itemize}
   \end{itemize}

   Setting CC for X enables vector control; any value outside the above list
   disables it.

   Data LSB (38) value to set features:

   \begin{itemize}
       \item 1 = Volume (fixed)
       \item 2 = Pan (the default)
       \item 4 = Filter Cutoff (Brightness, it is the default)
       \item 8 = Mod Wheel (the default)
       \item 0x12 = 18 = Reversed Pan
       \item 0x24 = 36 = Reversed Filter Cutoff
       \item 0x48 = 72 = Reversed Mod Wheel
   \end{itemize}

   The feature numbers are chosen so they can be combined. So, 5 would be
   Volume + Brightness and 19 would be Volume + Reversed Pan.

   Setting the sweep CC for the X axis enables vector control. It also sets,
   but doesn't enable the default X axis features.  Setting the sweep CC for
   the Y axis sets, but doesn't enable the default Y axis features.  If one
   doesn't enable any features, not a lot will happen.

   The feature numbers are chosen so they can be combined. So, 5 would be
   Volume + Brightness and 19 would be Volume + Reversed Pan.

   Optional settings.  The first part, the number, is the MSB value.
   The second part is the LSB, the parameter value to set.  Note that the
   instrument IDs are for instruments in the current bank.

   \begin{itemize}
      \item 4 = x1 instrument ID
      \item 5 = x2 instrument ID
      \item 6 = y1 instrument ID
      \item 7 = y2 instrument ID
      \item 8 = set CC for X feature 2
      \item 9 = set CC for X feature 4
      \item 10 = set CC for X feature 8
      \item 11 = set CC for Y feature 2
      \item 12 = set CC for Y feature 4
      \item 13 = set CC for Y feature 8
   \end{itemize}
              
   The IDs are for instruments in the current bank.
   Any data MSB value outside the above list disables vector control.
   Sweep CCs and feature CCs are sanity-checked.
    
   An Example. From channel 1, send the following CCs:

   \begin{verbatim}
      CC      Value
      99       64
      98        1
       6        0
      38       14
      98        1 *
       6        1
      38       15
      98        1 *
       6        2
      38        1
      98        1 *
       6        3
      38        2
   \end{verbatim}

   This sequence will set up CC 14 as the X axis incoming controller,
   and CC 15 as the Y axis incoming controller, with X set to volume control
   and Y set to pan control.

   One can either go on with the NRPNs to set the instruments (this will load
   and enable instruments from the current bank), or enable and load
   them by hand.  For channel 1 this would be part 1 and 17 for X and part 33
   and 49 for Y.

   The (*) CCs ensure that the data bytes are reset each time. This is not
   really necessary for the earlier commands, but should be done if one sets
   the instruments with NRPNs as well, otherwise one will try to set them
   twice.

\subsection{Vector / Command Line}
\label{subsection:vector_command_line}

   This section covers material that could be in the command-line section
   (see \sectionref{sec:command_line}), but is really too detailed to cover
   there.  The examples here, to set up vectors from the command line,
   are provided by Will.

   Assuming we want just a single axis on channel 1 (which is channel
   2 in the GUI), first we need to make sure we have enough parts available:

   \begin{verbatim}
      yoshimi> set available 32
      Available parts set to 32
   \end{verbatim}

   The next command must aways be the first command, as everthing else depends
   on it. It's the command that \textsl{enables} vector control.  The
   \texttt{x} token denotes the "x axis", and the \texttt{cc} token, followed
   by 14, is the incoming sweep CC (control change) that will vary the
   features one sets.

   \begin{verbatim}
      yoshimi > set vector 1 x cc 14
      Vector channel set to 1
   \end{verbatim}

   Note that, according to the list of MIDI CC's at
   \url{http://nickfever.com/music/midi-cc-list}, CC 14 is undefined,
   normally.  It is thus available for \textsl{Yoshimi} to assign
   for its own purpose.

   \index{vector!features}
   There are four vector features currently available:

   \index{vector!1 volume}
   \index{vector!2 pan}
   \index{vector!3 brightness}
   \index{vector!4 modulation}
   \begin{itemize}
      \item \textbf{1} is fixed as \textsl{volume}.
      \item \textbf{2} is \textsl{pan} by default.
      \item \textbf{3} is \textsl{brightness} by default.
      \item \textbf{4} is \textsl{modulation} by default.
   \end{itemize}

   We will select \textsl{volume} for this example.  Let's enable this
   feature:

   \begin{verbatim}
      yoshimi Vect Ch 1 X > set features 1 enable
      Set X features 1 en
   \end{verbatim}

   Next, one needs to set the instruments that will be used.
   The instruments can only be selected from the instruments
   in the current bank. Therefore, assuming the current bank is
   the "\textsl{Will Godfrey Companion}", let's set up two instruments:

   \begin{verbatim}
      yoshimi Vect Ch 1 X > set program left 20
      Loaded 20 "Bubbles" to Part 1

      yoshimi Vect Ch 1 X > set program right 120
      Loaded 120 "Ghost Ensemble" to Part 16
   \end{verbatim}

   The \texttt{left} token merely assigns instrument 20 to a "virtual"
   left side of the X axis,
   and the \texttt{right} token assigns instrument 20 to a "virtual"
   right side of the X axis.

   \textsl{(Chris asks:  How did the part numbers 1 and 16 come about? What
   are the rules?  Why are 32 parts needed, if only 1 to 16 are involved?  Why
   do we need to have 64 parts when we add the Y axis below?  Can we set
   intermediate values between "left" and "right" and "up" and "down" to
   get some really weird morphs?)}

   \index{vector!morph}
   If one now sweeps the the controller assigned to CC 14,
   the sound will morph between these two instruments.

   To continue on to using the other axis as well, one needs to have 64 parts
   available:

   \begin{verbatim}
      yoshimi Vect Ch 1 X > /set available 64
      Available parts set to 64
   \end{verbatim}

   \index{command level}
   Note the slash, which lets the user immediately access the topmost command
   level, where the "available parts" setting can be performed.
   Then:

   \begin{verbatim}
      yoshimi > set vector y cc 15
      Vector 1 Y CC set to 15
   \end{verbatim}

   This command sets up the Y axis to be controlled by MIDI CC 15, which is,
   again, a CC that is normally undefined.
   We will use \textsl{panning} (feature 2) for this vector, which is defined
   on the Y axis:

   \begin{verbatim}
      yoshimi Vect Ch 1 Y > set features 2 enable
      Set Y features 2 en
   \end{verbatim}

   Analogous to the "left" and "right" virtual directions used above for the X
   axis, the Y axis used the "up" and "down" virtual directions:

   \begin{verbatim}
      yoshimi Vect Ch 1 Y > set program down 107
      Loaded 107 "Angel Harp" to Part 32

      yoshimi Vect Ch 1 Y > set program up 78
      Loaded 78 "Brassy Flutter" to Part 48
   \end{verbatim}

   Notice that the directions left, right, up, and down match the directions
   provided by a traditional joystick.

   So we have now set up a vector sound where MIDI CC 14 morphs the sound
   through a continuous linear combination of two different instruments,
   and MIDI CC 15 morphs the sound between two other instruments.
   One can then save one's cool vector sound to a file:

   \begin{verbatim}
      yoshimi Vect Ch 1 Y > save vector CoolSound
      Saved channel 1 Vector to CoolSound
   \end{verbatim}

   The file extension for the save vector sound file is \texttt{.xvy}, and
   this extension is added automatically.  The final name of the file is
   \texttt{CoolSound.xvy}.

   \textsl{(Chris asks:  Where is this file saved?  Is there a way to modify
   this location?)}

   At any time one can reload this vector sound file from the command-line:

   \begin{verbatim}
      yoshimi> load vector channel 0 CoolSound
      Loaded Vector CoolSound to channel 0
   \end{verbatim}

   If there is no channel number provided, then the vector sound
   will be loaded to the same channel as it was saved from:

   \begin{verbatim}
      yoshimi> load vector CoolSound
      Loaded Vector CoolSound to source channel
   \end{verbatim}

%-------------------------------------------------------------------------------
% vim: ts=3 sw=3 et ft=tex
%-------------------------------------------------------------------------------


% "MIDI Learn"

%-------------------------------------------------------------------------------
% yum_midi_learn
%-------------------------------------------------------------------------------
%
% \file        yum_midi_learn.tex
% \library     Documents
% \author      Chris Ahlstrom
% \date        2016-12-22
% \update      2018-05-16
% \version     $Revision$
% \license     $XPC_GPL_LICENSE$
%
%     Provides the midi_learn section of yoshimi-user-manual.tex.
%
%-------------------------------------------------------------------------------

\section{MIDI Learn}
\label{sec:midi_learn}

   \index{MIDI Learn}
   \index{midi!learn}
   In this section, we show how to use the new (with 1.5.0)
   \textbf{MIDI Learn} feature of \textsl{Yoshimi}.
   MIDI Learn is a method to remotely control many parameters in an audio/MIDI
   application via a MIDI controller.  Each parameter that is "learned" can be
   controlled, and the setting changes recorded.

\subsection{MIDI Learn / Basics}
\label{subsec:midi_learn_basics}

%   In \textsl{Yoshimi}'s direct-access system, MIDI Learn is available, to
%   the extent that it can handle controls that have a range of 0 to 127.

   One can have multiple controls on the same CC.  But, although they all work,
   only the last one updates the user interface.  They are channel specific. So
   if one is rich enough to have two MIDI keyboards, one can set them up to do
   quite different jobs.
   Some of the controls, like volume and pan, are immediate, but most are "next
   note".

   In order to unset a learned value, simply delete the line.
%  One cannot get them to block others on the same
%  CC/channel pair; that will come later.

   Some external controllers use the pitch wheel control per-channel for up to 16
   high-resolution faders.
   Some synthesizers send a number of high resolution controllers as NRPNs.
   \textsl{Yoshimi} MIDI Learn can handle these.
   The controls that can actually benefit from better resolution are
   most of the volume and detune ones.
   They are learned in exactly the same way as ordinary CCs, but instead of
   presenting a line that includes an editable CC field, they show a non-editable
   hexadecimal number with a space between the bytes and followed by an "h",
   such as \texttt{0a 2c h}.
   Also, these lines default to having \textbf{Block} set.
   See that item's discussion below.

   Now \textsl{Yoshimi} can respond to aftertouch.
   One might wonder why one would want to MIDI-learn modulation when there is
   already a dedicated CC for it; and the same for "brightness". The answer is
   that there is currently no way to link these to aftertouch, and this is
   especially relevant for people using wind controllers.
   MIDI Learn sees aftertouch as CC 129 (via a sneaky conversion).
   This has another nice result.
   If one does not have an aftertouch device currently
   in hand, use any other controller; then, in the editing window, just change
   the controller number to 129. Save the learned set, and next time one
   \textsl{does} have such a device, just load the file and off you go!  MIDI
   Learn can emulate the MOD wheel, as it is an accessible control in the
   little panel brought up when one right-clicks the \textbf{Controllers}
   button in the bottom panel of the \textsl{Yoshimi} main window: One can use
   it for any volume or pan, and can have a lot of fun with things like the
   \textbf{Phaser} effect as these are all "instant".  The \textbf{Mod} wheel
   is seen as CC 130.

\begin{figure}[H]
   \centering
%  \includegraphics[scale=1.0]{1.5.3/midi-controls-panel.png}
   \includegraphics[scale=1.0]{1.5.4/MIDIcontrols.png}
   \caption{MIDI Controls Panel}
   \label{fig:midi_controls_panel}
\end{figure}

   The emulated \index{MIDI controls} MIDI controls are:
   \index{Modulation} \textbf{Modulation},
   \index{Expression} \textbf{Expression},
   \index{Filter Q} \textbf{Filter Q},
   \index{Filter Cutoff} \textbf{Filter Cutoff}, and
   \index{Master Bandwidth} \textbf{Master Bandwidth}.

\subsection{MIDI Learn / User Interface}
\label{subsec:midi_learn_user_interface}

   To activate MIDI Learn, \texttt{Ctrl-right-click}
   on any user interface control.
   A pop-up window will detail the control selected, or indicate that the
   control is not learnable.
   A message will also appear in the console window or command-line interface
   (if active).

\begin{figure}[H]
   \centering
   \includegraphics[scale=0.75]{1.5.0/Learning-30amp.png}
   \caption{MIDI Learn Prompt Example 1}
   \label{fig:midi_learn_ex_1}
\end{figure}

   Here is another example:

\begin{figure}[H]
   \centering
   \includegraphics[scale=0.75]{1.5.0/Learning-Mods.png}
   \caption{MIDI Learn Prompt Example 2}
   \label{fig:midi_learn_ex_2}
\end{figure}

   If a \texttt{Yoshimi} control is not MIDI-learnable, a message pop-up
   will indicate that it is not
   learnable:

\begin{figure}[H]
   \centering
   \includegraphics[scale=0.75]{1.5.0/Learning-No.png}
   \caption{MIDI Learn Prompt Unsupported Example}
   \label{fig:midi_learn_unsupported}
\end{figure}

   \index{midi!learn, slider quirk}
   Note that the majority of controls, including the sliders, are MIDI
   learnable.  However, also note that, for sliders to be learned, one must
   click on the \textsl{track} of the slider, not the \textsl{thumb} of the
   slider.

   If the \textsl{Yoshimi / Midi Learn} button is pressed, and there are no
   MIDI-learn entries available yet, then the following empty dialog appears:

\begin{figure}[H]
   \centering
   \includegraphics[scale=0.50]{1.5.0/MidiLearn-empty.png}
   \caption{Empty Midi Learn Dialog}
   \label{fig:empty_midi_learn_dialog}
\end{figure}

   A message will also appear in the console window/CLI.

   After turning on learn, the first physical controller moved, or CC message
   sent, will be locked in, and one will see the user-interface knob or slider
   move in synchrony with the physical control. The pop-up window will
   disappear, and the console message \texttt{Learned} appears, with a line
   underneath with exactly what control was caught.

   There is also an activity "LED" between \textbf{Chan} and
   \textbf{Min} indicators that flickers when the associated CC or channel
   is received, provided the line is not muted, or blocked by an earlier one.

   One can stack up message lines that have the same CC and channel, so that a
   single incoming can change a part volume and at the same time change the
   filter cutoff and with a third line change the panning of a
   \textsl{different} part.

   Multiple lines will always be displayed (and actioned) in ascending order of
   CC, then channel number

   For a quick example, on our system we're running only ALSA.  So we
   plug a \textsl{Korg NanoKEY2} mini-USB keyboard in.
   We determine the existing ports using

   \begin{verbatim}
      $ aconnect -i -o
   \end{verbatim}

   We see that we have (among other ports) client 24:0 is the nanoKEY2 MIDI
   keyboard, and \textsl{Yoshimi} at client 129:0.
   We connect it to \textsl{Yoshimi} using

   \begin{verbatim}
      $ aconnect 24:0 129:0
   \end{verbatim}

   We ctrl-right-click on \textsl{Yoshimi}'s \textbf{Velocity Sense} knob in
   the main window, then we click on the \textbf{MOD} button on the
   \textsl{nanoKEY2}, and we see an entry CC=1, Chan=1, Min=0, Max=127, and
   control function name of "Part 1 Vel Sens".  Every time we press the
   \textbf{MOD} button on the \textsl{nanoKEY2}, we see the "LED" appear, and
   movement in \textsl{Yoshimi}'s \textbf{Velocity Sense} knob.

   Once entries have been added, a fully-fleshed list of learned items is
   presented if one then uses the \textsl{Midi Learn} button; one will
   see a new window displaying the recently-learned controller. Along with a
   number of settings, one sees text with precise details of this complete
   action.

\begin{figure}[H]
   \centering
   \includegraphics[scale=0.75]{1.5.8/MidiLearn.png}
   \caption{MIDI Learn Dialog}
   \label{fig:midi_learn_dialog}
\end{figure}

   If the controller learned was an NRPN, this dialog will show a hexadecimal
   number in the CC field, and this item will not be editable. Notice that a small indented button will appear between the mute status and the NRPN value. If this it clicked on it will turn red indicating that it is now a 7 bit value. This is type of range is sent by some hardware synths and controllers.

   Adding or deleting rows in this dialog, or changing the CC or channel
   number, will cause the rows to be sorted again.
   \index{MIDI Learn!200 lines}
   The maximum number of MIDI-Learned lines per session is 200.

   The major items of this dialog are the editor settings available:

   \begin{enumber}
      \item \textbf{Mute}
      \item \textbf{CC}
      \item \textbf{Chan}
      \item \textbf{Min}
      \item \textbf{Max}
      \item \textbf{Limit}
      \item \textbf{Block}
      \item \textbf{Control Function}
      \item \textbf{Load}
      \item \textbf{Save}
      \item \textbf{Recent}
      \item \textbf{Clear}
   \end{enumber}

   Now click on the \texttt{Midi Learn} button, to see a new window displaying
   the recently-learned controller. Along with a number of settings, it shows
   text with precise details of this complete action.

   Also shown is an \texttt{activity} LED that flickers when the associated
   CC/channel is received.

   \setcounter{ItemCounter}{0}      % Reset the ItemCounter for this list.

   \itempar{Mute}{MIDI Learn!Mute}
   Mute.
   Disables the MIDI Learn control specified by the corresponding line of
   settings.  The control is still available, but will not be in effect.

   Values: \texttt{Checked, Unchecked}

   \itempar{CC}{MIDI Learn!CC}
   CC.
   Incoming CC.
   Provides the value of the controller that is learned.
   For example, a value of 7 indicates the control value that would normally
   affect the main volume of \texttt{Yoshimi}.
   Note that NRPNs are \textsl{not included}.
   Also note that CC has no default values; the values are whatever the
   incoming learned values are.

   Values: \texttt{1 to 127}

   \itempar{Chan}{MIDI Learn!Chan}
   Chan.
   Incoming channel number.
   Note that, in the \textbf{MIDI Learn} window, the channel numbers start
   from 1,  as do all the other numbers in that window except controllers, which
   start from 0, following MIDI convention.
   Also note that the channel has no default values; the values are whatever the
   incoming learned values are.

   Values: \texttt{1 to 16, and All}

   \itempar{Min/Max}{MIDI Learn!Min/Max}
   Min and Max.
   Provides the minimum and maximum incoming values for the controller value.
   Since V 1.5.2 this is shown as a percentage with a resolution of 0.5
   If \textbf{Min} is greater than \textbf{Max}, this reverses the control
   direction.
   If \textbf{Min} is equal to \textbf{Max}, it becomes a threshold setting.
   Any value lower than this threshold will be passed on as 0, and any
   value higher will be passed on as 127.
   These values will then be translated to the
   \textbf{Min} and \textbf{Max} of whatever
   controller has been linked. Thus, for a simple switch those values
   will be 0 (off) and 1 (on). For most controls it will be 0 and 127.
   For an Addsynth modulator it will be OFF and PWM.

   Bear in mind that \textbf{Min} and \textbf{Max} are percentage values, not
   0-to-127 MIDI values.  Divide the incoming MIDI value by 1.27 to get the
   percentage value.  The resolution is to the nearest 0.5.
   To summarize:

   \begin{enumerate}
      \item In MIDI Learn, set both \textbf{Min} and \textbf{Max} to the same
         percentage value.  Call it "M\%".
      \item Multiply \texttt{M\%} by 1.27, and increase the result up to the
         next integer value.
   \end{enumerate}

   For example, if \texttt{M\% = 90.0}, any MIDI value $\leq$ 115 will turn a
   switch off, and any value $>$ 115 will turn it on.

   Values: \texttt{0 * (min) to 100 * (max)}

   \itempar{Limit}{MIDI Learn!Limit}
   Limit/Compression switch.
   Limiter versus compression.
   The Min/Max range can either be in the style of a limiter or a compression.
   Set to \textbf{limit}, the \textbf{Min} and \textbf{Max} will be hard cutoffs.
   For example, if \textbf{Min} is 20 and the incoming value is 5, then the
   result is 20.

   Set to \textbf{compress}, the incoming value will be converted to fit
   the range. For example, if \textbf{Min} is 32, \textbf{Max} is 95, then
   an incoming value of 0 will be 32, an incoming value of 2
   will be 33, etc.

   Values: \texttt{Checked (Limit), Unchecked (Compress)}

   \itempar{Block}{MIDI Learn!Block}
   Block.
   Specifies blocking of all later actions on the same CC/channel pair
   (including system ones).
   If a loaded set refers to \textsl{Yoshimi} controls that are disabled, or
   don't exist, such controls will be ignored.
   The \textsl{block} feature will be active unless the line is muted.

   Values: \texttt{Checked, Unchecked}

   For devices that send high resolution controllers as NRPNs.
   Also, these lines default to having \textbf{Block} set.
   This is so that the NRPN is
   not passed on to \textsl{Yoshimi}, which would result in "go away" messages
   or obscure actions. However, like ordinary CCs, they will stack and one can
   set several lines with the same NRPN performing mutiple actions, and then
   unblock all but the last one.

   The way this fits in with the rest is that the incoming data values are
   combined as a 14 bit number, then (as a floating point number) divided by
   128 so the overall range is exactly the same as normal CCs.  However, when
   decoded for various controls *all* CCs are converted by a second (hidden)
   set of limits to get the maximum possible resolution:

   \begin{itemize}
      \item \textbf{Humanise}: 0 to 50
      \item \textbf{Engine fine detunes}: -8192 to 8191
      \item \textbf{Engine coarse detunes}: -64 to 63
      \item \textbf{LFO frequency}: 0.0 to 1.0 (float)
   \end{itemize}

   The actual resolution is determined by the physical control
   source. Most controls seem to be 10 bit, but if generated within an
   automation source, one will get the full 14 bits.

   \itempar{Control Function}{MIDI Learn!Control Function}
   Control function.
   Provides text describing what control is affected, or if the
   part is disabled or not.

   One can delete any existing MIDI Learn via
   \texttt{Ctrl-right-click}
   on the
   \texttt{Control Function} text for that line.
   One is then presented with a confirmation message giving the line number and
   the text as a reminder.
   Adding lines, or changing either CC or channel numbers, will
   re-order the lines.
   Deleting lines will cause a redraw, but not a re-sort.
   Changing the CC or channel will only do a re-sort when necessary, as when the
   new number is now higher or lower than the adjacent ones.

   The same CC/controller can be used to change several different internal
   \texttt{Yoshimi} controls.  For example, one can have a part's volume being
   changed while another part is having an effect level changed.
   This is done by selecting one part, and making a setting with the desired
   controller, and then selecting another part, and making a setting with the
   same controller.
   This single controller will then affect both parts at once.

   \itempar{Load}{MIDI Learn!Load}
   Load.
   Loads a set of MIDI Learn values from a file.
   The extension of the file is \texttt{.xly}.
   If a loaded set refers to \texttt{Yoshimi}
   controls that are disabled, or don't exist, those controls will be ignored.
   However the Block feature will still be active, unless the line is muted.

   \itempar{Save}{MIDI Learn!Save}
   Save.
   A complete list of MIDI Learn values
   will be saved by clicking on the Save button; one then sees
   the usual file-chooser window.
   The file is saved where desired, with the extension \texttt{.xly},

   \itempar{Recent}{MIDI Learn!Recent}
   Recent.
   This button is used for loading a set of
   MIDI Learn values from the recent history.

   \itempar{Clear}{MIDI Learn!Clear}
   Clear.
   This button clears the entire learned list from the
   \textbf{MIDI Learn} dialog.

% To come:
% Paging of the display to avoid scrolling through a massive list.

\subsection{MIDI Learn / Tutorial}
\label{subsec:midi_learn_tutorial}

   \textsl{This mini-tutorial is courtesy of Will.}

   Say one has a foot pedal that outputs CC values on the standard volume, CC 7.
   Now this is per channel, so only instruments on the first channel will pick
   it up.  This presents a problem if one has automation/backing tracks on
   other channels and one wants to keep everything together. So here is what to
   do:

   While holding down \texttt{Ctrl}, right-click on the
   \textbf{Volume} knob at the top of the
   main window. A window will open with the message "Learning Main Volume".
   If one now operates the foot pedal, the window will disappear and one will
   see that the main volume control is now responding to the foot pedal.

   \textsl{However}, this means one is changing both the main volume
   \textsl{and} the part 1 volume at the same time.  So now open the
   \textbf{MIDI learn} window via the \textbf{Yoshimi} button. One will
   see that it now has a line detailing the incoming CC and channel, along with
   other controls and the control function named \textbf{Main Volume}.  Click
   on the \textbf{Block} check box, and one will see that the part 1 volume
   control no longer responds.

   Now the foot pedal will control \textsl{only} the master volume, not the
   individual part volumes. This setup will survive loading new patch sets, and
   also a main reset (while still running).

   It's quite likely that the foot pedal will go from 0 to 127, when one actually
   wants a much smaller control range.  In that case, one can change the
   \textbf{Min} and \textbf{Max}
   values to (for example) 40 and 90.
   In this way, the entire range of the pedal control will be reduced linearly to
   40-90.

   If one sets the \textbf{Limit} checkbox, then these values will instead be
   cutoff points so anything from the pedal between 0 and 40 will be 40, and
   anything between 90 and 127 will be 90

   To temporarily disable this controller line, use the \textbf{Mute} checkbox.
   The entire line will be greyed, and as the \textbf{Block} is no longer
   active normal part volume control will be restored.

   A point that is not obvious is that although incoming CCs are per-channel,
   the actions are per-\textsl{part}, so if one sets controller 94 for part 1
   volume and then set it again for part 2 volume, one gets \textsl{two} lines,
   each controlling a different part but \textsl{acting together}.  Change the
   \textbf{Min} and \textbf{Max} of one of them to 127 and 0 respectively and
   one will increase in volume while the other reduces.

   Volume, pan, and most of the effects are \textsl{immediate}, while most of
   the other controls start on the \textsl{next note}.  Eventually it would be
   nice to get filters etc. to be immediate, but that's for another release!

%-------------------------------------------------------------------------------
% vim: ts=3 sw=3 et ft=tex
%-------------------------------------------------------------------------------


% Command-line mode

% %-------------------------------------------------------------------------------
% yum_command_line
%-------------------------------------------------------------------------------
%
% \file        yum_command_line.tex
% \library     Documents
% \author      Chris Ahlstrom
% \date        2015-09-06
% \update      2015-10-25
% \version     $Revision$
% \license     $XPC_GPL_LICENSE$
%
%     Provides the description of the no-gui mode of Yoshimi.
%
%-------------------------------------------------------------------------------

\section{The Yoshimi Command Line Interface 1.3.6M}
\label{sec:command_line}

   \textsl{Yoshimi} provides a "no GUI" or "command-line" mode of operation
   where some aspects of the application can be controlled via textual commands.
   This mode is useful for blind people, for example.  To access this mode, add
   the \texttt{-i} or \texttt{--no-gui} command-line option when starting
   \textsl{Yoshimi} on the command-line.  But note that, when starting
   \textsl{Yoshimi} on the command-line, the "command-line" mode of operation is
   available at the same time as the GUI.

   One of the main features of the 1.3.6 release is improved non-GUI
   accessibility.  In a command line environment, almost all the 'running'
   commands are available, but none of the instrument editing ones are... yet!
   One can decide what MIDI/audio setup is wanted, list and set roots and banks,
   load instruments into any part, change a part's channel, set main volume and
   key shift, and set up vector control.  A number of first-time defaults have
   been changed to make this feataure easier.

   When starting from the command line, an argument can be included for a new
   root path to be defined to point to a set of banks fetched from elsewhere.
   This will be given the next free ID. A future upgrade will allow the ID to
   be set to any valid one when it is created, mirroring the GUI behaviour.

   Once running, almost all dynamic setup (i.e. doesn't require a restart) can now
   be done within the terminal window. There is also extensive control of roots,
   banks, parts and instruments including the ability to list and set all of
   these. One can now do things like:

   \begin{verbatim}
      yoshimi> path add /home/music/yoshimi/banks
      yoshimi> set part 4 program 130
   \end{verbatim}

   Additional controls that are frequently taken for granted in the GUI, but
   otherwise get forgotten, are \textsl{master key shift} and \textsl{master
   volume}.  The most important parts of vector control are exposed to the
   command line.

   The command-line mode provides extensive error checking and feedback.
   Note the change in nomenclature from "Parameters" to "Patch Set", which is
   visible in the main screen, and also reflected in the command line.

   Here is an example startup session, using ALSA for audio and MIDI support:

   \begin{verbatim}
      Yoshimi is starting
      March little endian = 1
      Format = Signed Little Endian 32 Bit 2 Channel
      Didn't find alsa MIDI source 'default'
      Yoshimi 1.3.6 M
      Clientname: yoshimi
      Config: Audio: alsa -> 'default'
      Midi: alsa -> 'default'
      Oscilsize: 1024
      Samplerate: 48000
      Period size: 1024

      Yay! We're up and running :-)
      yoshimi> 
   \end{verbatim}

\subsection{Command Table}
\label{subsec:command_line_command_table}

   When running from the command line, these commands
   (see \tableref{table:yoshimi_text_commands})
   can be entered after the 'up and running' message.
   The commands are not case-sensitive.
   The commands can be abbreviated to the first three letters of each command.

   The brief descriptions in the following table can be obtained
   using the "help" command in the \textsl{Yoshimi} command-line mode.
   More detailed descriptions are given in the section following the table.

   \begin{table}[H]
      \centering
      \caption{Yoshimi Text Commands}
      \label{table:yoshimi_text_commands}
      \begin{tabular}{l l}

         \texttt{load patchset [s]} &
            Load a complete patch set from the named file. \\

         \texttt{save patchset [s]} &
            Save the patch set to the named file. \\

         \texttt{save setup} &
            Save the current dynamic system settings. \\

         \texttt{paths} &
            Display all defined bank root paths and their IDs. \\

         \texttt{path add [s]} &
            Define a new bank root path and returns its ID. \\

         \texttt{path remove [n]} &
            Remove the path-entry ID \textsl{n} from bank roots. \\

         \texttt{list banks [n]} &
            List instruments and IDs in bank \textsl{n} or
            current bank/root. \\

         \texttt{list instruments [n]} &
            List all instruments and IDs in bank \textsl{n}
            or current bank/root. \\

         \texttt{list current} &
            List number of parts available, and more. \\

         \texttt{list setup} &
            Displays the current dynamic system settings. \\

         \texttt{list vector [n]} &
            Lists the settings for vector on channel \textsl{n}. \\

         \texttt{set reports [n]} &
            Set report destination (1=GUI, anything else sets stderr). \\

         \texttt{set root [n]} &
            Set current root path to ID \textsl{n}. \\

         \texttt{set bank [n]} &
            Set current bank to ID \textsl{n}. \\

         \texttt{set part [n1] program [n2]} &
            Load instrument \textsl{n2} into part \textsl{n1}. \\

         \texttt{set part [n1] channel [n2]} &
            Set the MIDI channel \textsl{n2} for part \textsl{n1}. \\

         \texttt{set part [n1] destination [n2]} &
            Set audio destination of part \textsl{n1}
            to main (\textsl{1}), part (\textsl{2}), both (\textsl{3}). \\

         \texttt{set ccroot [n]} &
            Set the MIDI CC for root path changes (128 disables). \\

         \texttt{set ccbank [n]} &
            Set the MIDI CC for bank changes (non-0 or non-32 disables). \\

         \texttt{set program [n]} &
            Set MIDI program change (0 disables, anything else enables). \\

         \texttt{set activate [n]} &
            Set part-activate on program change (\textsl{n}=0 disables
            1 enables). \\

         \texttt{set extend [n]} &
            Set CC value for extended prog. change (above 119 disables). \\

         \texttt{set available [n]} &
            Set the number of available parts (16, 32, 64). \\

         \texttt{set volume [n]} &
            Set the master volume. \\

         \texttt{set shift [n]} &
            Set master key shift for notes, semitones (+- octave,
            64=no shift). \\

         \texttt{set alsa midi [s]} &
            * Sets the name of the MIDI device ALSA looks for. \\

         \texttt{set alsa audio [s]} &
            * Sets the name of the audio hardware device ALSA looks for. \\

         \texttt{set jack server [s]} &
            * Sets the name of the JACK server Yoshimi tries to connect to. \\

         \texttt{set vector [n1] x/y cc [n2]} &
            CC \textsl{n2} is for ch. \textsl{n1} X/Y axis sweep.
            For X, enables vector. \\

         \texttt{set vector [n1] x/y features [n2]} &
            Sets channel \textsl{n1} X or Y features to \textsl{n2}. \\

         \texttt{set vector [n1] x/y program [l/r] [n2]} &
            Loads program \textsl{n2} to ch. \textsl{n1} X or Y
            \textsl{left} or \textsl{right} part. \\

         \texttt{set vector [n1] x/y control [n2] [n3]} &
            Sets \texttt{n3} CC to use for X or Y feature \texttt{n2}
            (2, 4, 8). \\

         \texttt{set vector [n] [off]} &
            Disables vector control for channel \textsl{n}. \\

         \texttt{stop} &
            Cease all sound immediately! \\

         \texttt{mode [s]} &
            Change to different menus: addsynth, subsynth, or padsynth. \\

         \texttt{?} or \texttt{help} &
            List commands for current mode. \\

         \texttt{exit} &
            Tidy up and close Yoshimi down. \\

      \end{tabular}
   \end{table}

   Commands are not case sensitive and an invalid one will print a reminder.
   usually, one only needs the first 3 letters of the names, provided that is
   unambiguous.

   Commands with '*' in the description need the setup to be saved, and Yoshimi
   restarted to be activated. 

   More commands will be added, and the organisation of the commands
   may change slightly.

\subsection{Command Descriptions}
\label{subsec:command_line_command_descriptions}

   This section describes the command-line commands in more detail.

   \setcounter{ItemCounter}{0}      % Reset the ItemCounter for this list.

   \itempar{load patchset [s]}{cmd!load patchset}
      Loads a complete patch set from the named file.

   \itempar{save patchset [s]}{cmd!save patchset}
      Saves the patch set to the named file.

   \itempar{save setup}{cmd!save setup}
      Saves the current dynamic system settings.

   \itempar{paths}{cmd!paths}
      Displays all the currently defined bank root paths and their IDs.

   \itempar{path add [s]}{cmd!path add}
      Defines a new bank root path and returns its ID.
      Example: \texttt{path add /home/music/yoshimi/banks}

   \itempar{path remove [n]}{cmd!path remove}
      Removes the path-entry ID \textsl{n} from the bank roots. 
      It does not delete anything.

   \itempar{list banks [n]}{cmd!list banks}
      Lists all the instruments and their IDs in bank \textsl{n} (or the
      current bank) of the current root.

   \itempar{list instruments [n]}{cmd!list instruments}
      Lists all the instruments and their IDs in bank \textsl{n}
      (or the current bank) of the current root.

   \itempar{list current}{cmd!list current}
      Lists the number of parts available and parts with instruments
      currently installed along with any enabled with the default sound.
      Also shows their audio destination:
      \textsl{M} = main L/R, \textsl{P} = part L/R, \textsl{B} = both, and
      \textsl{-} = disabled or unavailable.
      This way one can tell if an instrument patch is installed even if it is
      not currently usable.
      To avoid unnecessary list length, the default "Simple Sound" is not shown
      unless it is enabled.

   \itempar{list setup}{cmd!list setup}
      Displays the current dynamic system settings.

   \itempar{list vector [n]}{cmd!list vector}
      Lists the settings for vector on channel \textsl{n}.

   \itempar{set reports [n]}{cmd!set reports}
      Set report destination (1 = GUI, anything else sets stderr).

   \itempar{set root [n]}{cmd!set root}
      Set current root path to ID \textsl{n}.

   \itempar{set bank [n]}{cmd!set bank}
      Set current bank to ID \textsl{n}.

   \itempar{set part [n1] program [n2]}{cmd!set part program}
      Load instrument \textsl{n2} into part \textsl{n1}.
      Example: \texttt{set part 4 program 130}

   \itempar{set part [n1] channel [n2]}{cmd!set part channel}
      Set the MIDI channel \textsl{n2} for part \textsl{n1}.
      If the channel number is greater than 15, no further MIDI
      messages will be accepted by that part.

   \itempar{set part [n1] destination [n2]}{cmd!set part destination}
      Set the audio destination of part \textsl{n1}
      to main (\textsl{1}), part (\textsl{2}), both (\textsl{3}).
      Also enables the part if not already enabled.

   \itempar{set ccroot [n]}{cmd!set ccroot}
      Set the MIDI CC for root path changes (128 disables).

   \itempar{set ccbank [n]}{cmd!set ccbank}
      Set the MIDI CC for bank changes (anything other than 0 or 32
      disables it).

   \itempar{set program [n]}{cmd!set program}
      Set MIDI program change (0 disables, anything else enables).

   \itempar{set activate [n]}{cmd!set activate}
      Set part-activate on program change (\textsl{n} = 0 disables
      part activation, anything else enables it). This features
      applies to command line program change as well.

   \itempar{set extend [n]}{cmd!set extend}
      Set the CC value for extended program change (anything greater
      than 119 disables it).

   \itempar{set available [n]}{cmd!set available}
      Set the number of available parts (16, 32, 64).

   \itempar{set volume [n]}{cmd!set volume}
      Set the master volume.

   \itempar{set shift [n]}{cmd!set shift}
      Set the master key shift for following notes in semitones (+-
      octave, 64 for no shift).

   \itempar{set alsa midi [s]}{cmd!set alsa midi}
      * Sets the name of the MIDI device ALSA looks for.

   \itempar{set alsa audio [s]}{cmd!set alsa audio}
      * Sets the name of the audio hardware device ALSA looks for.

   \itempar{set jack server [s]}{cmd!set jack server}
      * Sets the name of the JACK server Yoshimi tries to connect to.

   \itempar{set vector [n1] x/y cc [n2]}{cmd!set vector cc}
      CC \textsl{n2} is used for channel \textsl{n1} X or Y axis sweep.
      For X, this also enables vector control for the channel.

   \itempar{set vector [n1] x/y features [n2]}{cmd!set vector features}
      Sets channel \textsl{n1} X or Y features to \textsl{n2}.

   \itempar{set vector [n1] x/y program [l/r] [n2]}{cmd!set vector program}
      Loads program \textsl{n2} to channel \textsl{n1} X or Y
      \textsl{left} or \textsl{right} part.

   \itempar{set vector [n1] x/y control [n2] [n3]}{cmd!set vector control}
      Sets \texttt{n3} CC to use for X or Y feature \texttt{n2} (2, 4, 8).
      \textsl{n3} is the CC to be used for feature number \textsl{n2} on X
      vector channel \textsl{n1}. The \textsl{x} is a sort of hidden parameter
      as the code uses an offset dependent on whether it is \textsl{x} or
      \textsl{y}. Also \textsl{n1} can be omitted in which case it will use the
      last defined channel number. Using alternate words and numbers gives a
      great deal of flexibility like this.

   \itempar{set vector [n] [off]}{cmd!set vector}
      Disables vector control for channel \textsl{n}.

   \itempar{stop}{cmd!stop}
      Cease all sound immediately!

   \itempar{mode [s]}{cmd!mode}
      Change to different menus: addsynth, subsynth, or padsynth.

   \itempar{? or help}{cmd!help}
      List commands for current mode.

   \itempar{exit}{cmd!exit}
         Tidy up and close Yoshimi down.

%-------------------------------------------------------------------------------
% vim: ts=3 sw=3 et ft=tex
%-------------------------------------------------------------------------------

%-------------------------------------------------------------------------------
% yum_command_line_v2
%-------------------------------------------------------------------------------
%
% \file        yum_command_line_v2.tex
% \library     Documents
% \author      Chris Ahlstrom
% \date        2015-09-06
% \update      2018-10-26
% \version     $Revision$
% \license     $XPC_GPL_LICENSE$
%
%     Provides the description of the no-gui mode of Yoshimi.
%
%-------------------------------------------------------------------------------

\section{The Yoshimi Command-Line Interface}
\label{sec:command_line}

   \index{CLI}
   \index{cmd}
   \index{command line}
   \textsl{Yoshimi} provides a command-line mode of operation where almost
   all aspects of the application can be controlled via text commands. This
   mode is useful for blind people, those with motor control problems and for
   programmers, for example. These text commands can also be put into a
   script file, and that script can then be run with full error checking.

   By default, starting \textsl{Yoshimi} from a terminal window will enable
   both the command line and the GUI at the same time. However you can
   specifically disable the GUI at the terminal prompt as you start with:
   yoshimi -i or yoshimi --no-gui

   One of the main features of recent \textsl{Yoshimi} releases
   is improved non-GUI accessibility.  In fact,
   \textsl{Yoshimi} can run with neither GUI nor CLI input access. Working
   purely as a hidden MIDI device, a daemon of sorts. To enable a tidy close,
   there is a new short-form NRPN. Just send 68 to both MSB and LSB (CC99 and
   CC98). If you make the LSB 69, it will return an exit value of 16. This is
   the 'forced exit' and can be used by the system for other cleanups and a
   shut down.

   For full details of NRPNs see
   \sectionref{subsection:nrpns_midi_nrpn_basics}

   In a command line environment, all the 'running' commands are available,
   and now all of the instrument editing ones are too. Also, the whole of
   vector control, and MIDI-learn, is exposed to the command line.

   One can decide what MIDI/audio setup is wanted, list and set roots and
   banks, load instruments into any part, change a part's channel, set main
   volume and key shift, and set up vector control.  A number of first-time
   defaults have been changed to make this feature easier.

   When starting from the command line, an argument can be included for a new
   root path to be defined to point to a set of banks fetched from elsewhere.
   This will be given the next free ID. A future upgrade will allow the ID to
   be set to any valid one when it is created, mirroring the GUI behaviour.

   Once running, all configuration can be done within the terminal
   window.  There is also extensive control of roots, banks, parts and
   instruments including the ability to list and set all of these.
   Additional controls that are frequently taken for granted in the GUI, but
   otherwise get forgotten, are \textsl{master key shift} and \textsl{master
   volume}.

   The command-line mode provides extensive error checking and feedback.
   \index{command level}
   The prompt will always show what \textsl{command level}
   one is on, along with relevant information.

   \subsection{Command Depth}
\label{subsec:command_line_command_depth}

   Recent developments in \textsl{Yoshimi} have made it possible to greatly
   (one could say dramatically) extend command-line access deep into the
   synth structures. This creates a problem where the command line itself
   could become unmanageably long.
   Thus, now only the current context level is printed in full. The levels
   higher up the tree are minimised:

   \begin{verbatim}
      yoshimi Part 1+>
      yoshimi P1+, Sub>
      yoshimi P1+, Sub+>
      yoshimi P1+, S+, Filter analog>
   \end{verbatim}

   Rather than stating that a switch is on or off, there is now just a
   \texttt{+} sign for "on", and nothing for "off". This is clearer than
   using a \textsl{-}, and the slight shift in the line gives the user
   another visual clue.

   There is a new command in the "config" context that controls where this is
   displayed, or whether it is shown at all.
   This command is:

   \begin{verbatim}
      EXPose {OFF, ON, PRompt}
   \end{verbatim}

   \textbf{Off} will give the bare prompt with no other information.
   \textbf{On} shows it as a separate line above the prompt:

   \begin{verbatim}
           @ P1+, S+, Filter analog
           yoshimi>
   \end{verbatim}

   \textbf{Prompt} shows it as a part of the prompt:

   \begin{verbatim}
           yoshimi P1+, S+, Filter analog>
   \end{verbatim}

   The default setting is \textbf{ON}.

   At the CLI prompt, when effects are being managed, the preset number is
   also shown on the prompt, so one typically sees something like:

   \begin{verbatim}
      yoshimi p2+ eff 1 reverb-1 >
   \end{verbatim}

   One will also get a confirmation message.
   Here is an example session:

   Starting from the \texttt{yoshimi} prompt:

   \begin{verbatim}
      yoshimi> s p 2 on
      Main Part Number 2
      Part 2 Enable Value 1
      yoshimi part 2+ > s pr 107
      Loaded Smooth Guitar to Part 2
      yoshimi part 2+ >
   \end{verbatim}

   This command \textbf{s}ets \textbf{p}art number 2 to \textbf{pr}ogram
   number 107 from the \textsl{current} instrument bank.
   \textsl{Yoshimi} is now on part 2 as the current part (indicated by the
   prompt), and all subsequent commands will relate to this "level". At this
   level, one can change the current part simply with:

   \begin{verbatim}
      yoshimi part 2+ > s 4 on
      yoshimi part 4+ >
   \end{verbatim}

   For clarity we omit the confirmation messages from here on.

   \textsl{Yoshimi} is now on part number 4. Now set an effect:

   \begin{verbatim}
      yoshimi part 4+ > s ef re
      yoshimi p4+ eff 1 reverb-1 >
   \end{verbatim}

   This command \textbf{s}ets the part's \textbf{ef}fect 1 (implicit) to the
   \textbf{re}verb type.

   Note that many settings parameters are optional, and if omitted, either a
   default or last-used value will be assumed. Also, names are truncated to
   6 characters so the prompt line doesn't get unmanageably long. From here
   one can set a preset for this effect:

   \begin{verbatim}
      yoshimi 4+ eff 1 reverb-1 > s pre 3
   \end{verbatim}

   Since V 1.5.0 the \textbf{pre}sets have been shown in the prompt, and one
   will still get a confirmation message. Also, it used to be necessary to
   enter \textbf{ty}pe for an effect but since V 1.5.10 the name is entered
   directly. Note that when entering effects, there will always be a preset.
   It will be number 1 unless it is changed.

   Many settings that follow in a direct command "path" through several
   levels can be made all at once, and one will be left at the appropriate
   level. Thus, summarising some of the above commands:

   \begin{verbatim}
      yoshimi part 4+ > s ef 2 re
      yoshimi p4+ eff 2 reverb-1>
   \end{verbatim}

   A new feature in \textsl{Yoshimi} V 1.6.0 is a warning when an effect
   isn't at the default settings for the given preset. This is in the form
   of a question mark on the end of the line.
   \begin{verbatim}
      yoshimi p4+ eff 2 reverb-1? >
   \end{verbatim}
   This warning will be there as soon as one changes any of the effect
   controls until either another preset or effect is selected. It will also
   show if a patchset or instrument is loaded that was previously saved with
   altered effects controls.

   One cannot combine \texttt{type} and \texttt{preset} as they
   are both at the same level.

   To go back one level, use the ".." command (reminiscent of the
   \texttt{cd ..} operation in an OS command shell):

   \begin{verbatim}
      yoshimi p4+ eff 2 reverb-5 > ..
      yoshimi part 4+ >
   \end{verbatim}

   To go back to the top command level, use the "/" command:

   \begin{verbatim}
      yoshimi part 4 > /
      yoshimi >
   \end{verbatim}

   These two special level-movement commands can also be put on the front of
   any other command.  Starting where we were before:

   \begin{verbatim}
      yoshimi p4+ eff 2 reverb-1 > .. s vol 70
      yoshimi part 4+ >
   \end{verbatim}

   Part 4 volume is now at 70, and \textsl{Yoshimi} is once again at the
   "part level", not the "part FX level".
%  Also note that the space after the ".." is optional.

   The help menus and lists are also partially context sensitive. This
   feature should help avoid clutter and confusion.

   As well as an immediate history, \textsl{Yoshimi} maintains a single
   command history file for all instances of \textsl{Yoshimi} that records
   any non-duplicated loads or save.  Thus, provided one makes a normal
   command-line exit, the last commands will be available on the next run of
   \textsl{Yoshimi}.

   The command-line now has formal methods of opening, selecting and closing
   additional instances.

   When loading external files from the command line, there is an alternative
   to entering the full name if \textsl{Yoshimi} has already seen this file
   and it is in the history list. In this situation one enters '@' followed
   by the list number.

   \begin{verbatim}
      yoshimi> l h v

      Recent Vectors:
      1  /home/will/another.xvy
      2  /home/will/Subtle.xvy
      3  /home/will/excellent.xvy
      4  /home/will/yoshimi-code/examples/CoolSound.xvy

      yoshimi> lo ve @4
      Main Vector Loaded /home/will/yoshimi-code/examples/CoolSound.xvy to chan 1
   \end{verbatim}

   The loading of externally-saved instruments is also done, by default,
   relative to one's \textsl{Yoshimi} home directory.  However, saving an
   external instrument from the command-line still requires a full pathname.

   The 'recent history' lists can load MIDI-learned files, patchsets, or
   vector files numerically from the associated list, instead of having to
   type the names out.
   \index{cmd!at-sign}
   This uses the '@' (list number) operator.

   Commands with "*" in the description need the setup to be saved, and
   \textsl{Yoshimi} restarted to be activated.

   Note that \textsl{Yoshimi}'s command-line can also load and save states,
   patchsets, and scales, and can list recent histories. Vector load and save
   is also supported from the command-line. That's a complete vector set,
   with all mappings, instruments, etc.
   One can independently decide which channel to load and save from, so that
   one can actually build up a vector set in (say) channel 3, then later
   decide to use it in channel 7. It has the extension \texttt{.xvy},
   standing for "Xml/Vector/Yoshimi".
   Since V 1.5.9 this has also been integrated with the saved states.

   Another small detail is that all of the minimum command-line abbreviations
   are now \textbf{C}apitalised in the help lists.

   The organisation of these features may be adjusted slightly, based on
   comments from users.

\subsection{Command Level}
\label{subsec:command_line_command_level}

   \index{cli!command level}
   \index{cli!context level}
   A command level (also known as a "context level")
   is simply a position in the hierarchy of commands that cover
   some aspect of \textsl{Yoshimi} functionality.
   The major levels are:

   \begin{itemize}
      \item \textbf{Top Level}
      \item \textbf{System Effects}
      \item \textbf{Insertion Effects}
      \item \textbf{Part}
      \item \textbf{Part Effects}
      \item \textbf{Bank}
      \item \textbf{Scales (microtonal)}
      \item \textbf{Vector}
      \item \textbf{Config}
      \item \textbf{Synth Engines}
      \begin{itemize}
         \item \textbf{Addsynth}
         \item \textbf{Addsynth Voice}
         \item \textbf{Subsynth}
         \item \textbf{Padsynth Harmonics}
         \item \textbf{Padsynth Envelopes}
      \end{itemize}
   \end{itemize}

   Any level that has direct numerical content can be changed with "set (n)"
   once at that level.  The level is indicated by the text in the
   \textsl{Yoshimi} prompt.
   For example, one can set 1 to 16 vector channels, so, from the
   \textsl{Top} level, the following command will set the default (1, or the
   last-used number). The second command will, given this level (the
   \textsl{Vector} level), switch to vector channel 5. However, at the start,
   one could have gone straight to 5 with the third command.

   \begin{verbatim}
      set vector                          # or "s ve"; sets the context
      set 5                               # or "s 5"
      set vector 5                        # or "s ve 5"; quicker!
   \end{verbatim}

   A detailed discussion of command-line vector control is presented in
   \sectionref{subsection:vector_command_line}.

\subsection{Command Scripts}
\label{subsec:command_line_command_scripts}

   \textsl{Yoshimi} command-line users can run plain-text scripts that behave
   in exactly the same way as if the commands had been entered from the
   command-line directly.
   The actual script command can be initiated from any context/level and is
   simply:

   \begin{verbatim}
      RUN {filepath-of-script}
   \end{verbatim}

   To avoid confusion, the script routine first sets the context to the top
   level, then performs all the commands, following context level changes. If
   there is a fault in the script, it will be reported along with the number
   of the line where the error occurred. Due to the buffering used, the
   script will return before many of the actions have actually taken place.
   Therefore an error report is likely to be some way up the responses.
   Typically it will be something like:

   \begin{verbatim}
    *** Error: Which Operation? @ line 13 ***
   \end{verbatim}

   Here is a simple example:

   \begin{verbatim}
      # A script test
      set part on
          # These two lines are spaced in a bit
          set add on
      set voice on
      set volume 45
   \end{verbatim}

   This script makes sure the part is on, that the relative addsynth and
   voice are on, and finally sets its volume to 45.

   Although this process starts from the top level, it will use the
   parameters that were last set. Thus, if one had been working on part 7,
   addsynth voice 2, then that is the one that will have its volume adjusted.
   This means one can set up generic preferences, then apply them to any
   part, engine, etc.

   The script routine honours any normal abbreviations. Blank lines are
   ignored. A '\#' at the start of a line marks it as a comment so will also
   be ignored. However, both of these kinds of lines will be in the line
   count if an error is reported.

   Buffering can also cause commands to go out of sequence if a buffered one
   is immediately followed by a related direct one. For this reason there is
   an extra command specific to scripts:

   \begin{verbatim}
      WAIT [n]
   \end{verbatim}

   The command must be entered in full, and the range is 1 to 30,000
   milliseconds.
   In the example below, switching the kit mode is buffered but setting a kit
   item isn't, so without a delay the setting would be attempted before it
   was available. A 20ms wait seems to be enough of a delay in this case.
   \begin{verbatim}
      # Using delays
      s p 1 on
      s add off
      ..
      s mul
      wait 20
      s 5
      s on
      s sub on
   \end{verbatim}

   Since \textsl{Yoshimi} V 2.0.2 there has been an alternative command:
   \begin{verbatim}
      RUNLocal {filepath-of-script}
   \end{verbatim}
   As it's name suggests, this will run from the current context level.

\subsection{Other Command Tables}
\label{subsec:command_line_other_command_tables}

   When running from the command line, commands can be entered after the
   'up and running' message. Commands are \textsl{not} case-sensitive.
   Commands can be abbreviated to the first three letters of each command,
   or, in some cases, just one letter.  This is indicated by uppercase
   letters in command descriptions. The commands available depend on the
   current "context" of the command line. However, there is a group of
   commands always available:

   \begin{itemize}
      \item \texttt{\textbf{?}} or \texttt{\textbf{help}}
      \item \texttt{\textbf{L}ist}
      \item \texttt{\textbf{RES}et}
      \item \texttt{\textbf{EX}it}
   \end{itemize}

   Apart from these commands, the command line works on a system of context
   levels, and normally only the commands relevant to that "level"
   will be available.

   We describe the command lists here. These lists are relative to a
   particular context, and what one sees if one enters \textbf{?} while at
   that level, to get help. However the command lists can all be called
   \textsl{specifically} from any level and the normal abbreviations are
   accepted. For example, to list just the part commands regardless of what
   context it is called from:
   \begin{verbatim}
      yoshimi> ? p
   \end{verbatim}
   \noindent With all of the following commands \newline
   Optional parameters are shown as: [ n ] for numbers or [ s ] for text strings \newline
   Compulsory parameters are shown as: \begin{math} \langle \end{math} n \begin{math} \rangle \end{math} for numbers or \begin{math} \langle \end{math} s \begin{math} \rangle \end{math} for text
   strings.

   \noindent Also, from any level, \textbf{? ?} will show the top level one.


\subsubsection{Top Commands}
\label{subsec:command_line_top_command_list}

   These commands are part of the Top context/command level. First, one gets
   the default options, always available. Then there are several options that
   have ellipsis (...); these are the context submenus. After that come all
   the actual top level controls; there are still a lot!

   \begin{verbatim}
      yoshimi> ?
   \end{verbatim}

   Note that there are a number of commands common to all command levels.
   We describe them here.

\begin{center}
\begin{longtable}{p{6cm} p{10cm}}
\caption[Yoshimi Top-Level Commands]{Yoshimi Top-Level Commands} \\
\texttt{'...'} &
   Indicates a context switch as well as a \textbf{Help} sub-menu. \\
\texttt{Part [n] ...} &
   Enter context level. \\
\texttt{VEctor [n] ...} &
   Enter context level. \\
\texttt{SCale ...} &
   Enter context level. \\
\texttt{MLearn ...} &
   Edit learned lines. \\
\texttt{COnfig ...} &
   Enter context level. \\
\texttt{BAnk ...} &
   Enter context level. \\
\texttt{LIst ...} &
   Show various settings and available parameters. \\
\texttt{LOad ...} &
   Load various files. \\
\texttt{SAve ...} &
   Save various files. \\
\texttt{ADD} &
   Add paths and files. \\
\texttt{- Root <s>} &
   Add a root path named s to the root list. \\
\texttt{- Bank <s>} &
   Add a bank named s to the current root. \\
\texttt{- YOshimi [n]} &
   Start a new instance optionally with ID n. \\
\texttt{IMport [s <n1>] <n2> <s>} &
   Imports a named directory to slot n2 of current root, (or 'Root' n1). \\
\texttt{EXPort [s <n1>] <n2> <s>} &
   Exports a bank at slot n2 of current root, (or 'Root' n1) to named directory
 \\
\texttt{REMove} &
   Remove paths, files, and entries. \\
\texttt{- Root <n>} &
   De-list root path ID n. \\
\texttt{- Bank <n1> [s <n2>]} &
   Delete bank ID n1 (and all instruments) from current root (or 'Root' n2). \\
\texttt{- Instrument <n>} &
   Delete instrument from slot (n) in current bank. \\
\texttt{- YOshimi <n>} &
   Close instance n. \\
\texttt{- MLearn <s> [n]} &
   Delete midi learned 'ALL' whole list, or '@' line n \\

\texttt{Set/Read/MLearn} &
   Set, read or learn all the following main parameters. \\
\texttt{MINimum/MAXimum/DEFault} &
   Or find ranges (mostly in the part context). \\
   \texttt{Part [n] ...} &
   Enter context level. Optionally to part n\\
\texttt{VEctor [n] ...} &
   Enter context level. Optionally to vector n\\
\texttt{SCale ...} &
   Enter context level. \\
\texttt{MLearn ...} &
   Edit learned lines. \\
\texttt{COnfig ...} &
   Enter context level. \\
\texttt{BAnk ...} &
   Enter context level. \\


\texttt{YOshimi [n]} &
   Read the current instance or change to [n]. \\
\texttt{MONo [s]} &
   Switches the main audio output to mono (ON / other). \\

\texttt{SYStem effects [n]} &
   Enter system effect context. Optionally to n. \\
\texttt{- SEnd <n1> <n2>} &
   Send this effect to effect n1 at volume n2. \\
\texttt{...} &
   Effect dependent controls. \\
\texttt{INSert effects [n]} &
   Enter insertion effect context. Optionally to n. \\
\texttt{- SEnd <s>/<n>} &
   Send effect destination to (Master, Off or part number). \\
\texttt{...} &
   Effect dependent controls. \\

\texttt{AVailable <n>} &
   Read available number of parts or set to n = 16, 32, 64. \\
\texttt{Volume <n>} &
   Read Master volume or set to n. \\
\texttt{SHift <n>} &
   Read Master key shift in semitones or set to n (0 no shift). \\
\texttt{DEtune <n>} &
   Read Master fine detune or set to n to match other sound sources. \\
\texttt{SOlo [s]} &
   Read 'solo' switcher or set to (OFF, Row, Column, Loop) \\
\texttt{SOlo CC [n]} &
   Read incoming 'solo' channel number or set to n. \\

\end{longtable}
\end{center}

   Some of the commands in the table above have more extensive descriptions
   in the sections that follow.

\paragraph{List}
\label{paragraph:command_line_list}
   With no suffix this displays all the lists that are available. We show
   this here as the entries are extremely useful.
\begin{center}
\begin{longtable}{p{6cm} p{10cm}}
\caption[Yoshimi Lists]{Yoshimi Lists} \\
\texttt{Roots} &
   all available root paths \\
\texttt{Banks [n]} &
   banks in root ID or current \\
\texttt{Instruments [n]} &
   instruments in bank ID or current \\
\texttt{Group <s1> [s2]} &
   instruments by s1 type grouping (s2 'Location' for extra details) \\
\texttt{Parts [s]} &
   parts with instruments installed ('More' for extra details) \\
\texttt{Vectors} &
   settings for all enabled vectors \\
\texttt{Tuning} &
   microtonal scale tunings \\
\texttt{Keymap} &
   microtonal scale keyboard map \\
\texttt{Config} &
   current configuration \\
\texttt{MLearn [s <n>]} &
   midi learned controls ('@' n for full details on one line) \\
\texttt{SECtion ...} &
   Copy/Paste section presets available for the current context. \\
\texttt{History [s]} &
   recent files (Patchsets, SCales, STates, Vectors, MLearn) \\
\texttt{Effects [s]} &
   effect types ('all' include preset numbers and names) \\
\texttt{PREsets } &
  all the presets for the currently selected effect \\
\texttt{SECtion } &
  all the internal C/P presets for the current section \\
\end{longtable}
\end{center}
\paragraph{MONo}
\label{paragraph:command_line_mono}
    The \textbf{MONo [s]} command is provided for switching between mono and
    stereo on the fly while \textsl{Yoshimi} is actually playing. This helps
    one make sure there is a good balance between these two.

\paragraph{SOlo}
\label{paragraph:command_line_solo}
   The \textbf{SOlo [s]} and \textbf{SOlo CC [n]}
   commands enable and set \textsl{Yoshimi}'s 'Solo' feature,
   whereby one can silently switch MIDI input to different parts.
   The 'Row' and 'Loop' types use the first 16 parts, while
   'Column' type can use all possible 64 parts.

   The type setting has to be decided before setting 'CC', which then
   determines which MIDI controller to listen to for performing the actual
   switch. See \sectionref{subsec:mixer_panel_window}; it goes into more
   details about this setting, at a graphical user interface level.

\paragraph{Set / Read / MLearn Context Levels}
\label{paragraph:command_line_context_levels}

   The Set / Read commands set or read all main parameters and the MLearn one
   initiates a MIDI learn with exactly the same parameters.
   In fact there are three more commands that follow this pattern:

   \begin{itemize}
   \item \texttt{MINimum}. Show the minimum value a command may set.
   \item \texttt{MAXimum}. Show the maximum value a command may set.
   \item \texttt{DEFault}. Show the default value of a command.
   \end{itemize}

   There are a few commands that set the context or command level, where
   additional commands peculiar to the "context" are provided.  Here are the
   command/context levels
   (also see \sectionref{subsec:command_line_command_level}.)
   Note that we also list commands for the effects levels.

   \begin{itemize}
      \item \texttt{Part}. Enter context level for part operations.
      \item \texttt{VEctor}. Enter context level for vector operations.
      \item \texttt{SCale}. Enter context level for scale (microtonal)
      operations.
      \item \texttt{MLearn}. Enter context level for MIDI Learn line editing.
      \item \texttt{COnfig}. Enter context level for configuration settings.
      \item \texttt{SYStem effects [n]}. Enter the effects context level.
      \begin{itemize}
         \item \texttt{[s]}.  Set the effect type directly by name.
         \item \texttt{PREset [n]}. Set the numbered effect preset to n.
         \item \texttt{SEnd [n1] [n2]}. Send the current system effect to
         effect n1 at volume n2.
      \end{itemize}
      \item \texttt{INSert effects [n]}. Enter effects context level.
      \begin{itemize}
         \item \texttt{[s]}. Set the effect type directly by name.
         \item \texttt{PREset [n]}. Set numbered effect preset to n.
         \item \texttt{SEnd [s]/[n]}. Set where to send the effect
            ('Master', 'Off', or a part number).
      \end{itemize}
   \end{itemize}

\paragraph{Part Command Level}
\label{paragraph:command_line_context_level_part}

   This command switches to the part context level and makes all its commands
   accessible.  If no number '[n]' is entered it will be on the default part
   (1) or whatever was the previous part in use.

   \textsl{Yoshimi} has a number of commands for controlling and configuring
   the synth engines from the command-line. First of all there is the part
   kit structure.
   There are three modes (i.e. settings) that the kits of the engines
   can take:

   \begin{enumerate}
      \item MUlti
      \item SIngle
      \item CRossfade
   \end{enumerate}

   These forms are exactly the same as the graphical controls,
   and can be set once in the part context.
   Starting at the part level prompt, this command will return the setting
   on the line after the command, and show a new context level prompt:

   \begin{verbatim}
      yoshimi Part 1+> set multi
      Part 1 Kit Mode multi
      yoshimi p1+, Multi 1+>
   \end{verbatim}

   This setting is at the kit item 1 (which is always enabled).
   Let's change to kit item 4, and, since it hasn't yet been enabled,
   let's enable it and turn on the
   SubSynth engine so that it will sound:

   \begin{verbatim}
      yoshimi p1+, Multi 1+> set 4
      yoshimi p1+, Multi 4> set on
      yoshimi p1+, Multi 4+> set sub on
      yoshimi p1+, M4+ Sub+>
   \end{verbatim}

   Note how the prompt line is more compact, and indicates via the
   plus-signs that the part, kit, and subsynth are all on.

   These new controls are shown in the part context help list.
   The synth engines, AddSynth, Voice, SubSynth and PadSynth also have
   their own contexts with appropriate help lists.
   Also, LFO, Filter, Envelope and Resonance have their own contexts
   above whichever engine they are sitting on, so again have their own
   help lists.
   A fairly deep context is:

   \begin{verbatim}
      yoshimi P1+, M4+, A+, Voice 5+> set lfo frequency on
   \end{verbatim}

   From the voice context, this command would set the context shown below.

   \begin{verbatim}
      yoshimi P1+, M4+, A+, V5+, LFO freq+>
   \end{verbatim}

   The need for compression is obvious, and once at this level, the
   \texttt{?} command will list the LFO controls.

\subsubsection{Part Common Commands}
\label{subsec:command_line_part_common_commands}
The idea of having a list of common commands is now deprecated. This is
because although originally intended to keep the individual command lists to
manageable proportions, it actually made it more confusing when trying
to see what controls were available in each context.
\newpage
\subsubsection{Part Commands}
\label{subsec:command_line_part_commands}

   Note that the table below assumes one is already at part level.

\begin{center}
\begin{longtable}[l]{ ll}
\caption[Part Commands]{Part Commands} \\
   \label{table:yoshimi_part_commands}
\texttt{(part)~[n]} &
   Change part number.  \\
\texttt{[ON/OFF]} &
   Enables / Disables the part. \\
\texttt{Volume~[n]} &
   Volume.  \\
\texttt{Pan~[n]} &
   Panning.\\
\texttt{VElocity~[n]} &
   Velocity sensing sensitivity. \\
\texttt{LEvel~[n]} &
   Velocity sense offset level.  \\
\texttt{MIn [[s][n]]} &
   Minimum MIDI last seen, or note value.   \\
\texttt{MAx [[s][n]]} &
   Maximum MIDI last seen, or note value.   \\
\texttt{FUll} &
   Restore full key range.   \\
\texttt{POrtamento [s]} &
   (ON / other).   \\
\texttt{Mode [s]} &
   Mode (Poly, Mono, Legato). \\
\texttt{Note [n]} &
   Polyphony.  \\
\texttt{SHift [n]} &
   Shift semitones (0 no shift). \\
\texttt{BYpass [n] [s]} &
   bypass part effect number n, (ON / other).  \\
\texttt{EFfects [n]} &
   Enter effects context level and optionally change number.  \\
 \texttt{(effect) [s]} &
   Effect type.   \\
 \texttt{(effect) PREset [n]} &
   Numbered effect preset to n. \\
 \texttt{(effect)~Send~[n1]~[n2]} &
   Current part to system effect n1 at volume n2. \\
\texttt{PRogram~[s]/[[s]~[n]]} &
   Instrument ID / Group (n) loads from search list. \\
\texttt{LAtest} &
   The most recent bank instrument loaded or saved. \\
\texttt{TYPe [s]} &
   The instrument type, i.e. Piano, Guitar etc. \\
\texttt{NAme [s]} &
   The display name the part can be saved with. \\
\texttt{COPyright [s]} &
   The copyright message. \\
\texttt{INFo [s]} &
   Additional information, use suggestions. \\
\texttt{Humanise Pitch [n]} &
   A small random detune. \\
\texttt{Humanise Velocity [n]} &
   A small random velocity reduction. \\
\texttt{CLear} &
  Resets the part to the default 'Simple Sound'. \\
\texttt{Channel [n]} &
   CHannel (32 disables, 16 note off only).   \\
\texttt{AFtertouch~Chan~<s1>~[s2]} &
   Off, Filter (Down) + Peak (Down) + Bend (Down) + Modulation + Volume. \\
\texttt{AFtertouch~Key~<s1>~[s2]} &
   Off, Filter (Down) + Peak (Down) + Bend (Down) + Modulation. \\
\texttt{Destination [s]} &
   JACK audio destination (Main, Part, Both).  \\
\texttt{MUlti} &
   Set kit mode and allow kit item overlaps. \\
\texttt{SIngle} &
   Set kit mode and only lowest numbered item in key range. \\
\texttt{CRoss} &
   Set kit mode and cross fade item pairs. \\
\texttt{KIT} &
   Re-enter kit mode editing if it is enabled. \\
\texttt{NORmal} &
   Return to normal (not kit) mode. \\
   \texttt{(kit mode) [n]} &
      Kit item number (1-16). \\
   \texttt{(kit mode) [ON/OFF]} &
      Enables / Disables and removes the kit item. \\
   \texttt{(kit mode) QUiet [s]} &
      Mute this kit item without changing it's features (ON, {other}). \\
   \texttt{(kit item) MIn [[s][n]]} &
      Minimum MIDI last seen, or note value   \\
   \texttt{(kit item) MAx [[s][n]]} &
      Maximum MIDI last seen, or note value.   \\
   \texttt{(kit item) FUll} &
      Restore full key range.   \\
   \texttt{(kit item) EFfect [n]} &
      Effect for this item (0-none, 1-3). \\
   \texttt{(kit item) NAme [s]} &
      The name of this item. \\
    \texttt{DRum [s]} &
      Kit to drum mode (ON, {other}). \\
\texttt{ADDsynth ...} &
   Enter AddSynth context. \\
\texttt{SUBsynth ...} &
   Enter SubSynth context. \\
\texttt{PADsynth ...} &
   Enter PadSynth context. \\
\texttt{MCOntrol ...} &
   Enter MIDI controllers context. \\
\end{longtable}
\end{center}

   Kit mode is an unusual form of context level as it either modifies some
   part level controls or adds new ones. At the same time it still allows
   all the other part controls. The examples below show how kit mode
   interacts with part, and how one is informed of the exact status.

   \begin{verbatim}
      yoshimi Part 1+> set multi
      yoshimi Part 1+, Multi 1+> set 2
      yoshimi Part 1+, Multi 2+> ..
      yoshimi Part 1+, (Multi)> set 3
      yoshimi Part 3> set 1
      yoshimi Part 1+, (Multi) set kit
      yoshimi Part 1+, Multi 1+>
   \end{verbatim}

   Note how the kit mode is shown in parenthesis when not actually in the kit
   editing context. Also, to disable kit mode the specific \textbf{NOR}mal
   command must be used, as the \textbf{ON/OFF} commands refer to individual
   kit items.

\paragraph{Part MIDI Controllers}
\label{paragraph:command_line_part_midi_commands}
   \begin{table}[H]
      \centering
      \caption{Part MIDI Controllers}
      \label{table:yoshimi_part_midi_commands}
      \begin{tabular}{l l}
\texttt{VOlume [s]} &
   enables/disables volume control (OFF / other). \\
\texttt{VRange [n]} &
   relative degree of volume control. \\
\texttt{PAn [n]} &
   relative range (width) of panning control. \\
\texttt{MOdwheel [s]} &
   enables/disables exponential modulation (ON / other). \\
\texttt{MRange [n]} &
   relative modulation control range. \\
\texttt{EXpression [s]} &
   enables/disables expression control (OFF / other). \\
\texttt{SUstain [s]} &
   enables/disables sustain control (OFF / other). \\
\texttt{PWheel [n]} &
   relative pitch wheel control range. \\
\texttt{BReath [s]} &
   enables/disables breath control (OFF / other). \\
\texttt{CUtoff [n]} &
   relative filter cutoff control range. \\
\texttt{Q [n]} &
   relative filter Q control range. \\
\texttt{BANdwidth [s]} &
   enables/disables exponential bandwidth (ON / other). \\
\texttt{BARange [n]} &
   relative bandwidth control range. \\
\texttt{FMamplitude [s]} &
   enables/disables FM amplitude control (OFF / other). \\
\texttt{RCenter [n]} &
   resonance center frequency. \\
\texttt{RBand [n]} &
   resonance bandwidth. \\
\texttt{POrtamento [s]} &
   enables/disables portamento control (OFF / other). \\
\texttt{PGate [n]} &
   point where portamento starts or ends. \\
\texttt{PForm [s]} &
   whether portamento is from or to (Start, End). \\
\texttt{PTime [n]} &
   portamento sweep speed. \\
\texttt{PDownup [n]} &
   portamento up/down speed ratio. \\
\texttt{PProportional [s]} &
   enables/disables proportional portamento (ON / other). \\
\texttt{PExtent [n]} &
   distance to double change. \\
\texttt{PRange [n]} &
   difference from non proportional. \\
\texttt{CLear} &
   set all controllers to defaults. \\
      \end{tabular}
   \end{table}

   It is also possible to emulate the action of some incoming MIDI CCs. This
   was initially needed so that they can be MIDI-learned, but also provides
   a means of checking and modifying these settings directly.

   \begin{table}[H]
      \centering
      \caption{MIDI Emulators}
      \label{table:yoshimi_part_midi_emulators}
      \begin{tabular}{l l}
\texttt{E Modulation [n]} &
   emulate modulation controller. \\
\texttt{E Expression [n]} &
   emulate expression controller. \\
\texttt{E BReath [n]} &
   emulate breath controller (MIDI and CLI only). \\
\texttt{E Cutoff [n]} &
   emulate filter cutoff controller. \\
\texttt{E Q [n]} &
   emulate filter Q controller. \\
\texttt{E BAndwidth [n]} &
   emulate bandwidth controller. \\
      \end{tabular}
   \end{table}

   The following commands are available once in the context for the various
   types of synth engines:

\paragraph{Part AddSynth Commands}
\label{paragraph:command_line_part_addsynth_commands}
   \begin{table}[H]
      \centering
      \caption{Part AddSynth Commands}
      \label{table:yoshimi_part_addsynth_commands}
      \begin{tabular}{l l}
\texttt{[ON/OFF]} &
   Enables / Disables the AddSynth engine. \\
\texttt{Volume [n]} &
   AddSynth Volume.  \\
\texttt{Pan [n]} &
   Panning position.\\
\texttt{PRandom [s]} &
   Enable random panning (ON / other).\\
\texttt{PWidth [n]} &
   Random panning range.\\
\texttt{VElocity [n]} &
   Velocity sensing sensitivity. \\
\texttt{STEreo [s]} &
   Sets this engine as stereo or mono (ON / other). \\
\texttt{DEPop [n]} &
   Initial attack slope.   \\
\texttt{PUnch Power [n]} &
   Attack boost amplitude. \\
\texttt{PUnch Duration [n]} &
   Attack boost time.   \\
\texttt{PUnch Stretch [n]} &
   Attack boost extend. \\
\texttt{PUnch Velocity [n]} &
   Attack boost velocity sensitivity.  \\
\texttt{DETune Fine [n]} &
   Fine frequency.   \\
\texttt{DETune Coarse [n]} &
   Coarse stepped frequency.  \\
\texttt{DETune Type [s]} &
   Type of coarse stepping. (DEFault, L35, L10, E100, E1200)  \\
\texttt{OCTave [n]} &
   Shift octaves up or down.  \\
\texttt{GRoup [s]} &
   Disables harmonic amplitude randomness of voices \\
\texttt{ } &
             with a common oscillator (ON / other). \\
\texttt{Bandwidth [n]} &
   Modifies the relative fine detune of voices. \\
\texttt{VOice ...} &
   Enter the Addsynth voice context. \\
\texttt{LFO ...} &
   Enter LFO insert context.  \\
\texttt{FILter ...} &
   Enter Filter insert context.  \\
\texttt{ENVelope ...} &
   Enter Envelope insert context.   \\
\texttt{REsonance ...} &
   Enter the AddSynth resonance context. \\
      \end{tabular}
   \end{table}

   \begin{table}[H]
      \caption{AddSynth Voice Commands}
      \label{table:yoshimi_part_addsynth_voice_commands}
      \begin{tabular}{l l}
\texttt{(voice) [n]} &
   Change voice number.  \\
\texttt{[ON/OFF]} &
   Enables / Disables the Voice. \\
\texttt{Volume [n]} &
   Voice Volume.  \\
\texttt{Pan [n]} &
   Panning position.\\
\texttt{PRandom [s]} &
   Enable random panning (ON / other).\\
\texttt{PWidth [n]} &
   Random panning range.\\
\texttt{VElocity [n]} &
   Velocity sensing sensitivity. \\
\texttt{BENd Adjust [n]} &
   Pitch bend range. \\
\texttt{BENd Offset [n]} &
   Pitch bend shift. \\
\texttt{DETune Fine [n]} &
   Fine frequency.   \\
\texttt{DETune Coarse [n]} &
   Coarse stepped frequency.  \\
\texttt{DETune Type [n]} &
   Type of coarse stepping.   \\
\texttt{OCTave [n]} &
   Shift octaves up or down.  \\
\texttt{FIXed [s]} &
   Set base frequency to 440Hz (ON / other).  \\
\texttt{EQUal [n]} &
   Equal temper variation. \\
\texttt{Type [s]} &
   Sound type (Oscillator, White noise, Pink noise, Spot noise). \\
\texttt{SOurce [n]} &
   Voice source number (Local for self). \\
\texttt{OSCillator [n]} &
   Oscillator source number (Internal for self). \\
\texttt{Phase [n]} &
   Relative voice phase. \\
\texttt{Minus [s]} &
   Invert entire voice (ON / other). \\
\texttt{DELay [n]} &
   Delay before this voice starts. \\
\texttt{Resonance [s]} &
   Enable resonance for this voice (ON / other). \\
\texttt{Bypass [s]} &
   Bypass global filter for this voice (ON / other). \\
\texttt{Unison [s]} &
   (ON, OFF). \\
\texttt{Unison Size[n]} &
   Number of unison elements. \\
\texttt{Unison Frequency[n]} &
   Frequency spread of elements. \\
\texttt{Unison Phase[n]} &
   Phase randomness of elements. \\
\texttt{Unison Width[n]} &
   Stereo width. \\
\texttt{Unison Vibrato[n]} &
   Vibrato depth. \\
\texttt{Unison Rate[n]} &
   Vibrato speed. \\
\texttt{Unison Invert [s]} &
   Phase inversion type (None, Random, Half, Third, Quarter, Fifth). \\
\texttt{MOdulator ...} &
   Enter the modulator context. \\
\texttt{WAveform ...} &
   Enter the oscillator waveform context. \\
\texttt{LFO ...} &
   Enter LFO insert context.  \\
\texttt{FILter ...} &
   Enter Filter insert context.  \\
\texttt{ENVelope ...} &
   Enter Envelope insert context.   \\
      \end{tabular}
   \end{table}

   \begin{table}[h]
      \caption{Voice Modulator Commands}
      \label{table:yoshimi_part_addsynth_voice_modulator_commands}
      \begin{tabular}{l l}
\texttt{[s]} &
   Directly set the type (OFF, Morph, Ring, Phase, Frequency, Pulse width). \\
\texttt{SOurce [[s]/[n]]} &
   Oscillator source (Local, voice number). \\
\texttt{Volume [n]} &
   Modulator depth.  \\
\texttt{VElocity [n]} &
   Velocity sensing sensitivity. \\
\texttt{Damping [n]} &
   Higher frequency relative damping. \\
\texttt{OSCillator [[s]/[n]]} &
   Modulation oscillator (Internal, modulator number). \\
\texttt{FOLlow [s]} &
   Use source oscillator detune (ON / other). \\
\texttt{SHift [n]} &
   Oscillator relative phase. \\
\texttt{WAveform ...} &
   Enter the modulator waveform context. \\
      \end{tabular}
   \end{table}

\paragraph{Part PadSynth Commands}
\label{paragraph:command_line_part_padsynth_commands}

   \begin{table}[H]
      \centering
      \caption{Part PadSynth Commands}
      \label{table:yoshimi_part_padsynth_commands}
      \begin{tabular}{l l}
\texttt{[ON/OFF]} &
   Enables / Disables the PadSynth engine. \\
\texttt{Volume [n]} &
   PadSynth Volume. \\
\texttt{Pan [n]} &
   Panning position.\\
\texttt{PRandom [s]} &
   Enable random panning (ON / other).\\
\texttt{PWidth [n]} &
   Random panning range.\\
\texttt{VElocity [n]} &
   Velocity sensing sensitivity. \\
\texttt{STEreo [s]} &
   Sets this engine as stereo or mono (ON / other). \\
\texttt{DEPop [n]} &
   Initial attack slope.   \\
\texttt{PUnch Power [n]} &
   Attack boost amplitude. \\
\texttt{PUnch Duration [n]} &
   Attack boost time. \\
\texttt{PUnch Stretch [n]} &
   Attack boost extend. \\
\texttt{PUnch Velocity [n]} &
   Attack boost velocity sensitivity. \\
\texttt{BENd Adjust [n]} &
   Pitch bend range. \\
\texttt{BENd Offset [n]} &
   Pitch bend shift. \\
\texttt{DETune Fine [n]} &
   Fine frequency. \\
\texttt{DETune Coarse [n]} &
   Coarse stepped frequency. \\
\texttt{DETune Type [n]} &
   Type of coarse stepping. \\
\texttt{OCTave [n]} &
   Shift octaves up or down. \\
\texttt{FIXed [s]} &
   Set base frequency to 440Hz (ON / other). \\
\texttt{EQUal [n]} &
   Equal temper variation. \\
\texttt{PRofile [s]} &
   The shape of harmonic profile (Gauss, Square Double exponent). \\
\texttt{WIdth [n]} &
   Width of the harmonic profile. \\
\texttt{COunt [n]} &
   Number of profile repetitions. \\
\texttt{EXpand [n]} &
   Adds harmonics and changes the distribution. \\
\texttt{FRequency [n]} &
   Further modifies distribution (dependent on stretch). \\
\texttt{SIze [n]} &
   Changes harmonic width while retaining shape. \\
\texttt{CRoss [s]} &
   Cross section of profile (Full, Upper, Lower). \\
\texttt{MUltiplier [s]} &
   amplitude multiplier (Off, Gauss, Sine, Flat). \\
\texttt{MOde [s]} &
   Amplitude mode (Sum, Mult, D1, D2). \\
\texttt{CEnter [n]} &
   Changes the central harmonic component width. \\
\texttt{RELative [n]} &
   Changes central component relative amplitude. \\
\texttt{AUto [s]} &
   Autoscaling (ON / other). \\
\texttt{BASe [s]} &
   Base profile distribution (C2, G2, C3, G3, C4, G4, C5, G5, G6). \\
\texttt{SAmples [n]} &
   Samples per octave (0.5, 1, 2, 3, 4, 6, 12). \\
\texttt{RAnge [n]} &
   Number of octaves (1 to 8). \\
\texttt{LEngth [n]} &
   Length of one sample in k (16, 32, 64, 128, 256, 512, 1024). \\
\texttt{BAndwidth [n]} &
   Overall bandwidth. \\
\texttt{SCale [s]} &
   Bandwidth scale. (Normal, Equalhz, Quarter, Half, Three Quarter,\\
\texttt{} &
   One and a half, Double, Inverse Half).\\
\texttt{SPectrum [s]} &
   Spectrum mode (Bandwidth, Discrete, Continuous). \\
\texttt{OVertone Position [s]} &
   Relationship to fundamental (HArmonic, SIne,\\
\texttt{} &
   POwer, SHift, UShift, LShift, UPower, LPower). \\
\texttt{OVertone First [n]} &
   Degree of first parameter. \\
\texttt{OVertone Second [n]} &
   Degree of second parameter.   \\
\texttt{OVertone Harmonic [n]} &
   Amount harmonics are forced.  \\
      \end{tabular}
   \end{table}

   \begin{table}
      \begin{tabular}{l l}
\texttt{XFadeupdate [n]} &
   Cross fade (millisec) after building new wavetable. \\
\texttt{BUildtrigger [n]} &
   Re-trigger wavetable build after n millisec. \\
\texttt{RWDetune [n]} &
   Random walk spread of voice detune on re-triggered build (0 off 96 factor 2). \\
\texttt{RWBandwidth [n]} &
   Random walk spread of line bandwidth. \\
\texttt{RWFilterFreq [n]} &
   Random walk spread of filter cutoff frequency. \\
\texttt{RWWidthProfile [n]} &
   Random walk spread of profile line width. \\
\texttt{RWStretchProfile [n]} &
   Random walk spread of profile modulation stretch. \\
\texttt{APply} &
   Puts the latest changes into the wavetable. \\
\texttt{XPort} &
   Export the current sample set to named file. \\
\texttt{WAveform ...} &
   Enter the oscillator waveform context. \\
\texttt{LFO ...} &
   Enter LFO insert context.  \\
\texttt{FILter ...} &
   Enter Filter insert context.  \\
\texttt{ENVelope ...} &
   Enter Envelope insert context.   \\
\texttt{REsonance ...} &
   Enter the PadSynth resonance context. \\
      \end{tabular}
   \end{table}


\paragraph{Part SubSynth Commands}
\label{paragraph:command_line_part_subsynth_commands}

   \begin{table}[H]
      \centering
      \caption{Part SubSynth Commands}
      \label{table:yoshimi_part_subsynth_commands}
      \begin{tabular}{l l}
\texttt{[ON/OFF]} &
   Enables / Disables the SubSynth engine. \\
\texttt{Volume [n]} &
   SubSynth Volume.  \\
\texttt{Pan [n]} &
   Panning position.\\
\texttt{PRandom [s]} &
   Enable random panning (ON / other).\\
\texttt{PWidth [n]} &
   Random panning range.\\
\texttt{VElocity [n]} &
   Velocity sensing sensitivity. \\
\texttt{STEreo [s]} &
   Sets this engine as stereo or mono (ON / other). \\
\texttt{BENd Adjust [n]} &
   Pitch bend range. \\
\texttt{BENd Offset [n]} &
   Pitch bend shift. \\
\texttt{DETune Fine [n]} &
   Fine frequency.   \\
\texttt{DETune Coarse [n]} &
   Coarse stepped frequency.  \\
\texttt{DETune Type [n]} &
   Type of coarse stepping. (DEFault, L35, L10, E100, E1200)  \\
\texttt{OCTave [n]} &
   Shift octaves up or down.  \\
\texttt{FIXed [s]} &
   Set base frequency to 440Hz (ON / other).  \\
\texttt{EQUal [n]} &
   Equal temper variation. \\
\texttt{HArmonic [n1] Amp [n2]} &
   Set harmonic n1 to n2 intensity. \\
\texttt{HArmonic [n1] Band [n2]} &
   Set harmonic n1 to n2 width. \\
\texttt{HArmonic Stages [n]} &
   Number of stages. \\
\texttt{HArmonic Mag [s]} &
   Harmonics filtering type. (Linear, 40dB, 60dB, 80dB, 100dB)\\
\texttt{HArmonic Position [s]} &
   Start position. (Zero, Random, Maximum)\\
\texttt{BAnd Width [n]} &
   Common bandwidth. \\
\texttt{BAnd Scale [n]} &
   Bandwidth slope versus frequency. \\
\texttt{OVertone Position [s]} &
   Relationship to fundamental (HArmonic, SIne,\\
\texttt{} &
   POwer, SHift, UShift, LShift, UPower, LPower).\\
\texttt{OVertone First [n]} &
   Degree of first parameter. \\
\texttt{OVertone Second [n]} &
   Degree of second parameter.   \\
\texttt{OVertone Harmonic [n]} &
   Amount harmonics are forced.  \\
\texttt{FILter ...} &
   Enter Filter insert context.  \\
\texttt{ENVelope ...} &
   Enter Envelope insert context.   \\
      \end{tabular}
   \end{table}

\paragraph{Part Resonance Commands}
\label{paragraph:command_line_part_resonance_commands}

   \begin{table}[H]
      \centering
      \caption{Resonance Commands}
      \label{table:part_resonance_commands}
      \begin{tabular}{l l}
   \texttt{(enable) [s]} &
      Activate resonance (ON / other). \\
    \texttt{PRotect [s]} &
      Leave fundamental unchanged (ON / other). \\
   \texttt{Maxdb [n]} &
      Set the maximum attenuation of points. \\
   \texttt{Random [s]} &
      Set a random distribution (Coarse, Medium, Fine). \\
   \texttt{CEnter [n]} &
      Set the center frequency of the resonance range. \\
   \texttt{Octaves [n]} &
      The number of octaves covered. \\
   \texttt{Interpolate [s]} &
      Turn isolated peaks into lines or curves (Linear, Smooth). \\
   \texttt{Smooth} &
      Reduce range and sharpness of peaks. \\
   \texttt{CLear} &
      Set all points to the mid level. \\
   \texttt{POints [[n1] [n2]]} &
      Show all or read n1 or set n1 to n2 \\
      \end{tabular}
   \end{table}

\paragraph{Part Waveform Commands}
\label{paragraph:command_line_part_waveform_commands}

   \begin{table}[H]
      \centering
      \caption{Part Waveform Commands}
      \label{table:yoshimi_part_waveform_commands}
      \begin{tabular}{l l}
   \texttt{[s]} &
      Directly set the basic waveform type by name. \\
   \texttt{HArmonic [n1] Amp [n2]} &
      Set harmonic n1 to n2 intensity. \\
   \texttt{HArmonic [n1] Phase [n2]} &
      Set harmonic n1 to n2 phase. \\
   \texttt{Harmonic Shift [n]} &
      Amount harmonics are moved. \\
   \texttt{Harmonic Before [s]} &
      Shift before waveshaping and filtering (ON other). \\
   \texttt{COnvert} &
       Change resultant wave to groups of sine waves. \\
   \texttt{CLear} &
      Clear harmonic settings. \\
   \texttt{Base Par [n]} &
      Basic wave parameter. \\
   \texttt{Base Mod Type [s]} &
      Basic modulation type (OFF, Rev, Sine Power). \\
   \texttt{Base Mod Par [n1] [n2]} &
       Parameter number n1 (1 - 3) set to n2 value. \\
   \texttt{Base Convert [s]} &
       Use resultant wave as basic waveform type. \\
   \texttt{} &
      Also clear modifiers and harmonics (OFF other). \\
   \texttt{SHape Type [s] } &
      Wave shape modifier type. (OFF, ATAn, ASYm1, POWer, SINe,\\
   \texttt{} &
       QNTs, ZIGzag, LMT, ULMt, LLMt, ILMt, CLIp, AS2, PO2, SGM) \\
   \texttt{SHape Par [n]} &
      Wave shape modifier amount. \\
   \texttt{Filter Type [s]} &
      (OFF, LP1, HPA1, HPB1, BP1, BS1, LP2, HP2, BP2, BS2, \\
      \texttt{} &
      COS, SIN, LSH, SGM)\\
   \texttt{Filter Par [n1] [n2]} &
      Filter parameters  n1 (1/2) set to n2 value. \\
   \texttt{Filter Before [s]} &
      Do filtering before waveshaping (ON other). \\
   \texttt{Modulation Par [n1] [n2]} &
      Overall modulation parameter n1 (1 - 3) set to n2 value. \\
   \texttt{SPectrum Type [s]} &
       Spectrum adjust type (OFF, Power, Down/Up threshold). \\
   \texttt{SPectrum Par } &
      Spectrum adjust amount. \\
   \texttt{ADdaptive Type [s]} &
      Adaptive harmonics (OFF, ON, SQUare, 2XSub, 2XAdd, \\
   \texttt{} &
       3XSub, 3XAdd, 4XSub, 4XAdd) \\
   \texttt{ADdaptive Base [n]} &
      Adaptive base frequency. \\
   \texttt{ADdaptive Level [n]} &
      Adaptive power. \\
   \texttt{ADdaptive Par [n]} &
      Adaptive parameter amount. \\

   \texttt{APply} &
      Fix settings (only for PadSynth). \\
      \end{tabular}
   \end{table}

   This list shows the "basic waveform type" settings available.

   \begin{itemize}
      \item SINe
      \item TRIangle
      \item PULse
      \item SAW
      \item POWer
      \item GAUss
      \item DIOde
      \item ABSsine
      \item PULsesine
      \item STRetchsine
      \item CHIrp
      \item ASIne
      \item CHEbyshev
      \item SQUare
      \item SPIke
      \item CIRcle
   \end{itemize}

\subsubsection{Engine Envelopes}
\label{subsec:command_line_engine_envelopes}

   \begin{table}[H]
      \centering
      \caption{Engine Envelopes, Type}
      \label{table:yoshimi_engine_envelopes_types}
      \begin{tabular}{l l}
\texttt{AMplitude} &
   Amplitude type. \\
\texttt{FRequency} &
   Frequency type. \\
\texttt{FIlter} &
   Filter type. \\
\texttt{BAndwidth} &
   Bandwidth type (SubSynth only). \\
      \end{tabular}
   \end{table}

   \begin{table}[H]
      \centering
      \caption{Engine Envelopes, Controls}
      \label{table:yoshimi_engine_envelopes_controls}
      \begin{tabular}{l l}
\texttt{Expand [n]} &
   Envelope time on lower notes. \\
\texttt{Force [s]} &
   Force release (ON / other). \\
\texttt{Linear [s]} &
   Linear slopes (ON / other). \\
\texttt{FMode [s]} &
   Set as Freemode (ON / other). \\
      \end{tabular}
   \end{table}

   \begin{table}[H]
      \centering
      \caption{Engine Envelopes, Fixed}
      \label{table:yoshimi_engine_envelopes_fixed}
      \begin{tabular}{l l}
\texttt{Attack Level [n]} &
   Initial attack level. \\
\texttt{Attack Time [n]} &
   Time before decay point. \\
\texttt{Decay Level [n]} &
   Initial decay level. \\
\texttt{Decay Time [n]} &
   Time before sustain point. \\
\texttt{Sustain [n]} &
   Sustain level. \\
\texttt{Release Time [n]} &
   Time to actual release. \\
\texttt{Release Level [n]} &
   Level at envelope end. \\
      \end{tabular}
   \end{table}

Example: "S FR D T 40" is "set frequency decay time 40".

Note: Some envelopes have limited controls.

   \begin{table}[H]
      \centering
      \caption{Engine Envelopes, Freemode}
      \label{table:yoshimi_engine_envelopes_freemode}
      \begin{tabular}{l l}
\texttt{Points} &
   Number of defined points (read only). \\
\texttt{Sustain [n]} &
   Point number where sustain starts. \\
\texttt{Insert [n1] [n2] [n3]} &
   Insert point at n1 with X increment n2, Y value n3. \\
\texttt{Delete [n]} &
   Remove point n. \\
\texttt{Change [n1] [n2] [n3]} &
   Change point n1 to X increment n2, Y value n3. \\
      \end{tabular}
   \end{table}

\subsubsection{Engine Filters}
\label{subsec:command_line_engine_filters}

   \begin{table}[H]
      \centering
      \caption{Engine Filters}
      \label{table:yoshimi_engine_filters}
      \begin{tabular}{l l}
\texttt{CEnter [n]} &
   Center frequency. \\
\texttt{Q [n]} &
   Q factor.   \\
\texttt{Velocity [n]} &
   Velocity sensitivity.   \\
\texttt{SLope [n]} &
   Velocity curve.   \\
\texttt{Gain [n]} &
   Overall amplitude.   \\
\texttt{TRacking [n]} &
   Frequency tracking.  \\
\texttt{Range [s]} &
   Extended tracking (ON, {other}) - goes from 0 to 198. \\
\texttt{STages [n]} &
   filter stages (1 to 5). \\
\texttt{CAtegory [s]} &
   Analog, Formant, State variable. \\
\texttt{TYpe [s]} &
   Category dependent, and not formant filters.  See the filter types below.   \\
\texttt{EDit ...} &
   Formant filter only. Enter the format editor. See below.   \\
      \end{tabular}
   \end{table}

   The list of filter types:

   \begin{itemize}
      \item Analog filters:
      \begin{itemize}
          \item \textbf{l1}. One stage low pass.
          \item \textbf{h1}. One stage high pass.
          \item \textbf{l2}. Two stage low pass.
          \item \textbf{h2}. Two stage high pass.
          \item \textbf{band}. Two stage band pass.
          \item \textbf{stop}. Two stage band stop.
          \item \textbf{peak}. Two stage peak.
          \item \textbf{lshelf}. Two stage low shelf.
          \item \textbf{hshelf}. Two stage high shelf.
      \end{itemize}
      \item State variable filters:
      \begin{itemize}
         \item \textbf{low}. Low pass.
         \item \textbf{high}. High pass.
         \item \textbf{band}. Band pass.
         \item \textbf{stop}. Band stop.
      \end{itemize}
   \end{itemize}

   \begin{table}[H]
      \centering
      \caption{Engine Filters, Formant Editor}
      \label{table:yoshimi_engine_filters_formant_editor}
      \begin{tabular}{l l}

\texttt{Invert [s]} &
   Invert effect of LFOs, envelopes (ON, OFF). \\
\texttt{CEnter [n]} &
   Center frequency of sequence. \\
\texttt{Range [n]} &
   Octave range of formant. \\
\texttt{Expand [n]} &
   Stretch overall sequence time. \\
\texttt{Lucidity [n]} &
   Clarity of vowels. \\
\texttt{Morph [n]} &
   Speed of change between formants. \\
\texttt{SIze [n]} &
   Number of vowels in sequence. \\
\texttt{COunt [n]} &
   Number of formants in vowels. \\
\texttt{VOwel [n]} &
   Vowel being processed. \\
\texttt{Point [n1] [n2]} &
   Sequence position n1 vowel n2. \\
\texttt{Formant [n]} &
   Formant being processed. \\
\texttt{per formant} &
    \\
\texttt{-  FRequency [n]} &
   Center frequency of formant. \\
\texttt{-  Q [n]} &
   Bandwidth of formant. \\
\texttt{-  Gain} &
   Amplitude of formant. \\
      \end{tabular}
   \end{table}

\subsubsection{Engine LFOs}
\label{subsec:command_line_engine_lfos}

   \begin{table}[H]
      \centering
      \caption{Engine LFOs}
      \label{table:yoshimi_engine_lfos}
      \begin{tabular}{l l}

\texttt{AMplitude} &
   Amplitude type. \\
\texttt{FRequency} &
   Frequency type. \\
\texttt{FIlter} &
   Filter type. \\
\texttt{ - Controls - } & \\
\texttt{Rate [n]} &
   Frequency. \\
\texttt{Intensity [n]} &
   Amount of effect. \\
\texttt{Start [n]} &
   Start position in cycle. \\
\texttt{Delay [n]} &
   Time before effect. \\
\texttt{Expand [n]} &
   Rate / note pitch. \\
\texttt{BPM [s]} &
   Sync frequency to MIDI clock (ON / other). \\
\texttt{Continuous [s]} &
   Free running - ignore start position (ON / other). \\
\texttt{Type [s]} &
   LFO oscillator shape. See the list below. \\
\texttt{AR [n]} &
   Amplitude randomness. \\
\texttt{RR [n]} &
   Rate (frequency) randomness. \\

      \end{tabular}
   \end{table}

Example: "S FI T RU" sets the filter type, ramp up.
Filter types (s parameter, LFO oscillator shape):

   \begin{itemize}
      \item SIne
      \item Triangle
      \item SQuare
      \item RUp (ramp up)
      \item RDown (ramp down)
      \item E1dn
      \item E2dn
      \item SH (sample / hold)
      \item RSU (random square up)
      \item RSD(random square down)
   \end{itemize}

\subsection{MLearn}
\label{subsec:command_line_mlearn}
   There are actually two entry points for MLearn. The first takes the same form
   as set/read and initiates the MIDI learning process for a given control.
   The second is within 'set' and is for editing current lines.

   To enter the context for the second form start out with:
   \begin{verbatim}
      set mlearn
   \end{verbatim}
   and to find out what lines have been set:
   \begin{verbatim}
      list mlearn
   \end{verbatim}
   This will show all the current learned lines with their list numbers.
   In the same way as for the other numbered contexts, one can set the number
   of the line directly to edit it.

   The available commands are:

   \begin{itemize}
      \item \texttt{MUte [s]}, Set to ON disables or, for any other provided token
         restores this line.
      \item \texttt{SEven} will interpret a learned NRPN as a 7-bit value.
      \item \texttt{CC [n]} will set the incoming controller value that will
         be recognised.  This command might re-order the list.
      \item \texttt{CHan [n]} will set the incoming channel number that will
         be recognised.  This command might re-order the list.
      \item \texttt{MIn [n]} will set the conversion for the incoming
         value to a minimum percentage.
      \item \texttt{MAx [n]} will set the conversion for the incoming
         value to a maximum percentage.
      \item \texttt{LImit [s]} set to Enable will use limiting instead of
         compression.  The conversion uses the minimum and maximum limits.
      \item \texttt{BLock [s]} set to ON, prevents this CC/channel pair from
        being passed on to any other lines or system controls.  It has no effect
        if the line has been muted.
   \end{itemize}

\subsection{EFfects Commands}
\label{subsec:command_line_effects}
   Since V 1.5.10 all effects controls have been exposed. However those
   available change dependent on the specific effect that is being managed.
   These controls are the same for System, Insertion and individual part
   effects. However, the exception is the 'Level' control that most of these
   have. This refers to the 'Dry/Wet' proportion of the effect apart from in
   System effects where it behaves as a volume control, without changing the
   'through' content.

   While part effects are 'immediate', in that one hears the results as soon as
   an effect type is selected, System effects need the part's \textbf{send} level
   to be increased, as it defaults to zero. Insertion effects however, need to
   have a \textbf{send} route selected as well as the effect type.

   There is some extra information given in V 1.5.11.
   When one chooses a new effect the first preset is always selected, which can
   then be edited to taste. Up to now, there was no indication that it had
   been altered apart from checking each control against its default. One
   could easily forget having edited this - especially if saving everything and
   coming back to it some days later.

   In more recent \textsl{Yoshimi} versions there is a '?' on the end of the line if any of the
   controls on that preset are not the default values.
   So:
   \begin{verbatim}
      yoshimi> s sys 2 rev
      yoshimi Sys eff 2 REverb-1> s pre 4
      yoshimi Sys eff 2 REverb-4> s pan 36
   \end{verbatim}
   This has now changed the preset.
   \begin{verbatim}
      yoshimi Sys eff 2 REverb-4?> s pre 4
      yoshimi Sys eff 2 REverb-4>
   \end{verbatim}
   Notice how even re-selecting the same preset will return it to its default
   settings, and would be an easy mistake if there was no warning.

   Just leaving the effects to work on another part of \textsl{Yoshimi}, then
   returning to them won't change anything of course, and will again correctly
   show whether the current one has been altered.

\subsubsection{Effects List}
\label{subsubsec:command_line_effects_list}
   Below is the list of the effects themselves:
   \begin{itemize}
      \item OFF
      \item REverb
      \item ECho
      \item CHorus
      \item PHaser
      \item ALienwah
      \item DIstortion
      \item EQ
      \item DYnfilter
   \end{itemize}

   Here, in detail are the controls available for each effect:
% Reverb
   \begin{table}[H]
      \centering
      \caption{Reverb}
      \begin{tabular}{l l}
\texttt{LEVel [n]} &
   Amount of the effect. \\
\texttt{PANning [n]} &
   Left-right panning. \\
\texttt{TIMe [n]} &
   Reverb time. \\
\texttt{DELay [n]} &
   Initial delay. \\
\texttt{FEEdback [n]} &
   Delay feedback. \\
\texttt{LOW [n]} &
   Low pass filter. \\
\texttt{HIGh [n]} &
   High pass filter. \\
\texttt{DAMp [n]} &
   Feedback damping. \\
\texttt{TYPe [s]} &
   Reverb type (Random, Freeverb, Bandwidth). \\
\texttt{ROOm [n]} &
   Room size. \\
\texttt{BANdwidth [n]} &
   Actual bandwidth (only for bandwidth type). \\
\texttt{PREset [n]} &
   Select numbered preset (sets all above controls). \\
      \end{tabular}
   \end{table}
   Reverb presets:
   \begin{enumerate}
      \item cathedral 1
      \item cathedral 2
      \item cathedral 3
      \item hall 1
      \item hall 2
      \item room 1
      \item room 2
      \item basement
      \item tunnel
      \item echoed 1
      \item echoed 2
      \item very long 1
      \item very long 2
    \end{enumerate}

% Echo
   \begin{table}[H]
      \centering
      \caption{Echo}
      \begin{tabular}{l l}
\texttt{LEVel [n]} &
   Amount of the effect. \\
\texttt{PANning [n]} &
   Left-right panning. \\
\texttt{DELay [n]} &
   Initial delay. \\
\texttt{LRDelay [n]} &
   Left-right delay. \\
\texttt{CROssover [n]} &
   Left-right crossover. \\
\texttt{FEEdback [n]} &
   Echo feedback. \\
\texttt{DAMp [n]} &
   Feedback damping. \\
\texttt{SEPerate} &
   Change DELay to left delay and LRDelay to right delay. \\
\texttt{BPM [s>]} &
   delay BPM sync (ON / other) \\
\texttt{PREset [n]} &
   Select numbered preset (sets all above controls). \\
      \end{tabular}
   \end{table}
Echo presets:
   \begin{enumerate}
      \item echo 1
      \item echo 2
      \item simple echo
      \item canyon
      \item panning echo 1
      \item panning echo 2
      \item panning echo 3
      \item feedback echo
    \end{enumerate}

% Chorus
   \begin{table}[H]
      \centering
      \caption{Chorus}
      \begin{tabular}{l l}
\texttt{LEVel [n]} &
   Amount of the effect. \\
\texttt{PANning [n]} &
   Left-right panning. \\
\texttt{FREquency [n]} &
   LFO frequency. \\
\texttt{RANdom [n]} &
   LFO randomness. \\
\texttt{WAVe [s]} &
   LFO waveshape (sine, triangle). \\
\texttt{SHIft [n]} &
   Left-right phase shift. \\
\texttt{DEPth [n]} &
   LFO depth. \\
\texttt{DELay [n]} &
   LFO delay. \\
\texttt{FEEdback [n]} &
   Chorus feedback. \\
\texttt{CROssover [n]} &
   Left-right routing. \\
\texttt{SUBtract [s]} &
   Invert output (ON / other). \\
\texttt{BPM [s]} &
   LFO BPM sync (ON / other). \\
\texttt{STArt [n]} &
   LFO BPM phase start. \\
\texttt{PREset [n]} &
   Select numbered preset (sets all above controls). \\
      \end{tabular}
   \end{table}
Chorus presets:
   \begin{enumerate}
      \item chorus 1
      \item chorus 2
      \item chorus 3
      \item celeste 1
      \item celeste 2
      \item flange 1
      \item flange 2
      \item flange 3
      \item flange 4
      \item flange 5
   \end{enumerate}

% Phaser
   \begin{table}[H]
      \centering
      \caption{Phaser}
      \begin{tabular}{l l}
\texttt{LEVel [n]} &
   Amount of the effect. \\
\texttt{PANning [n]} &
   Left-right panning. \\
\texttt{FREquency [n]} &
   LFO frequency. \\
\texttt{RANdom [n]} &
   LFO randomness. \\
\texttt{WAVe [s]} &
   LFO waveshape (sine, triangle). \\
\texttt{SHIft [n]} &
   Left-right phase shift. \\
\texttt{DEPth [n]} &
   LFO depth. \\
\texttt{FEEdback [n]} &
   Phaser feedback. \\
\texttt{STAges [n]} &
   The number of filter stages. \\
\texttt{CROssover [n]} &
   Left-right routing. \\
\texttt{SUBtract [s]} &
   Invert output against source (ON / other). \\
\texttt{HYPer [s]} &
   Hyper-sine (ON / other). \\
\texttt{OVErdrive [n]} &
   Distortion. \\
\texttt{ANAlog [s]} &
   Analog emulation  (ON / other). \\
\texttt{BPM [s]} &
   LFO BPM sync (ON / other). \\
\texttt{STArt [n]} &
   LFO BPM phase start. \\
\texttt{PREset [n]} &
   Select numbered preset (sets all above controls). \\
      \end{tabular}
   \end{table}
Phaser presets:
   \begin{enumerate}
   \item phaser 1
   \item phaser 2
   \item phaser 3
   \item phaser 4
   \item phaser 5
   \item phaser 6
   \item aphaser 1
   \item aphaser 2
   \item aphaser 3
   \item aphaser 4
   \item aphaser 5
   \item aphaser 6
   \end{enumerate}

% Alienwah
   \begin{table}[H]
      \centering
      \caption{Alienwah}
      \begin{tabular}{l l}
\texttt{LEVel [n]} &
   Amount of the effect. \\
\texttt{PANning [n]} &
   Left-right panning. \\
\texttt{FREquency [n]} &
   LFO frequency. \\
\texttt{WAVe [s]} &
   LFO waveshape (sine, triangle). \\
\texttt{SHIft [n]} &
   Left-right phase shift. \\
\texttt{DEPth [n]} &
   LFO depth. \\
\texttt{FEEdback [n]} &
   Filter feedback. \\
\texttt{DELay [n]} &
   LFO delay. \\
\texttt{CROssover [n]} &
   Left-right routing. \\
\texttt{RELative [n]} &
   Relative phase. \\
\texttt{BPM [s]} &
   LFO BPM sync (ON / other). \\
\texttt{STArt [n]} &
   LFO BPM phase start. \\
\texttt{PREset [n]} &
   Select numbered preset (sets all above controls). \\
      \end{tabular}
   \end{table}
AlienWah presets:
   \begin{enumerate}
   \item alienwah 1
   \item alienwah 2
   \item alienwah 3
   \item alienwah 4
   \end{enumerate}

% Distortion
   \begin{table}[H]
      \centering
      \caption{Distortion}
      \begin{tabular}{l l}
\texttt{LEVel [n]} &
   Amount of the effect. \\
\texttt{PANning [n]} &
   Left-right panning. \\
\texttt{MIX [n]} &
   Left-right mix. \\
\texttt{DRIve [n]} &
   Input level. \\
\texttt{OUTput [n]} &
   Output balance. \\
\texttt{WAVe [s]} &
   Function waveshape (ATAn, ASYm1, POWer, SINe, QNTs,\\
\texttt{ } &
    ZIGzag, LMT, ULMt, LLMt, ILMt, CLIp, AS2, PO2, SGM) \\
\texttt{INvert [s]} &
   Invert ?  (ON / other). \\
\texttt{LOW [n]} &
   Low pass filter. \\
\texttt{HIGh [n]} &
   High pass filter. \\
\texttt{STEreo [s]} &
   Stereo (ON / other). \\
\texttt{FILter [s]} &
   Filter before distortion (ON / other). \\
\texttt{PREset [n]} &
   Select numbered preset (sets all above controls). \\
      \end{tabular}
   \end{table}
Distortion presets:
   \begin{enumerate}
   \item overdrive 1
   \item overdrive 2
   \item exciter 1
   \item exciter 2
   \item guitar amp
   \item quantisize
   \end{enumerate}

% EQ
   \begin{table}[H]
      \centering
      \caption{EQ}
      \begin{tabular}{l l}
\texttt{LEVel [n]} &
   Amount of the effect. \\
\texttt{BANd [n]} &
   EQ band number for following controls. \\
\texttt{- FILter [s]} &
   Filter type (LP1, HP1, LP2, HP2, NOT, PEA, LOW, HIG). \\
\texttt{- FREquency [n]} &
   Cutoff/band frequency. \\
\texttt{- GAIn [n]} &
   Makeup gain. \\
\texttt{- Q [n]} &
   Filter Q. \\
\texttt{- STAges [n]} &
   The number of \textsl{extra} filter stages. \\
      \end{tabular}
   \end{table}
EQ has no presets.

% DynFilter
   \begin{table}[H]
      \centering
      \caption{Dynfilter}
      \begin{tabular}{l l}
\texttt{LEVel [n]} &
   Amount of the effect. \\
\texttt{PANning [n]} &
   Left-right panning. \\
\texttt{FREquency [n]} &
   LFO frequency. \\
\texttt{RANdom [n]} &
   LFO randomness. \\
\texttt{WAVe [s]} &
   LFO waveshape (sine, triangle). \\
\texttt{SHIft [n]} &
   Left-right phase shift. \\
\texttt{DEPth [n]} &
   LFO depth. \\
\texttt{INVert [s]} &
   Reverse effect of sensitivity (ON / other). \\
\texttt{RATe [n]} &
   Speed of filter change with amplitude. \\
\texttt{FILter ...} &
   Enter the dynamic filter context. (a stock filter)\\
\texttt{BPM [s]} &
   LFO BPM sync (ON / other). \\
\texttt{STArt [n]} &
   LFO BPM phase start. \\
\texttt{PREset [n]} &
   Select numbered preset (sets all above controls). \\
      \end{tabular}
   \end{table}
DynFilter presets:
   \begin{enumerate}
   \item wahwah
   \item autowah
   \item vocal morph 1
   \item vocal morph 2
   \end{enumerate}

\subsection{Bank Commands}
\label{subsec:command_line_bank_command_list}

   During the development of \textsl{Yoshimi} V 1.6 it was realised that
   the command line had little control of instrument bank management.
   This was expanded in V 1.6.1. One can read the list of
   \textbf{Bank} commands from any level with '? b'.

   \begin{verbatim}
      yoshimi> ? b
   \end{verbatim}

   In the table that follows, we leave off the commands already noted
   above in the first table
   (see \sectionref{subsec:command_line_top_command_list}).

   \begin{table}[H]
      \centering
      \caption{Yoshimi Bank Commands}
      \label{table:yoshimi_text_bank_commands}
      \begin{tabular}{l l}

   \texttt{Bank} &
   Enters the Bank context. \\
   \texttt{[n]} &
   Sets the current bank to the one at slot n of the current bank root \\
   \texttt{  } &
   or reads the current bank's slot and name. \\
  \texttt{Name [s]} &
  Changes or reads just the name of the current bank. \\
  \texttt{Root [n]} &
  Sets the current bank root to n or reads the current full path. \\
  \texttt{Root ID [n]} &
  Changes the current bank root ID to n.  \\
  \texttt{INstrument Rename [n] [s]} &
  Changes the name of the instrument in slot n of the current bank.  \\
  \texttt{INstrument SAve [n]} &
  Saves the current part's instrument to bank slot n.  \\

        \end{tabular}
   \end{table}

\subsection{Vector Commands}
\label{subsec:command_line_vector_command_list}

   Although \textbf{Vector} is a top-level control, one can read the list of
   commands from any level with '? ve'.

   \begin{verbatim}
      yoshimi part 1+> ? ve
   \end{verbatim}
   A vector can be set on any standard MIDI channel, and allows a two or four
   part column to be managed as pairs for the given controls.
   The commands at this level deal with control of an X axis and a Y axis.
   The CC for the X axis must be set before everything else.  Then the CC for
   the Y axis must be set.  Finally, the other Y controls can be set.

   In the table that follows, we leave off the commands already noted
   above in the first table
   (see \sectionref{subsec:command_line_top_command_list}).

   Note that for vectors, n by itself is the base channel.

   \begin{table}[H]
      \centering
      \caption{Yoshimi Vector Commands}
      \label{table:yoshimi_text_vector_commands}
      \begin{tabular}{l l}

\texttt{VEctor} &
   Enters the Vector context. \\
   \texttt{[n]} &
   Changes the channel number for setting/editing.\\
\texttt{[X/Y] CC [n]} &
   CC n is used for the X or Y axis sweep. \\
\texttt{[X/Y] Features [n] [s]} &
   Sets X or Y features n = 1 to 4 (s = "Enable", "Reverse", other = off).  \\
\texttt{[X] PRogram [l/r] [n]} &
   Sets X program change ID n for left or right part. \\
\texttt{[Y] PRogram [d/u] [n]} &
   Sets Y program change ID n forL DOWN or UP part. \\
\texttt{[X/Y] Control [n1] [n2]} &
   Sets n2 CC to use for X or Y feature n1 = 2 to 4. \\
\texttt{Off} &
   Disables vector control for this channel.  Parts are unchanged. \\
\texttt{Name [s]} &
   Sets the internal name to s for this complete vector.  \\

      \end{tabular}
   \end{table}

   The \textbf{X/Y Features} command sets the features for the
   selected axis, and also if they are to be off or reversed.

   The \textbf{X/Y Control [n1] [n2]} command
   sets the n2 CC to use for the X or Y feature n1 = 2 to 4.
   This allows a change of the actual CC associated with features 2 through 4.
   They can be any CC that \textsl{Yoshimi} recognises.

   Feature 1 is always fixed as Volume, and cannot be reversed. It is arranged
   such that it will try to maintian a constant overall volume from the paired
   instruments.

\subsection{Scales Commands}
\label{subsec:command_line_scales_command_list}

   A fairly new one is the scales list:

   \begin{verbatim}
      yoshimi> ? sc
   \end{verbatim}

   In the table that follows, we leave off the following commands, already noted
   above in the first table
   (see \sectionref{subsec:command_line_top_command_list}):
   \textbf{?}, \textbf{Help}, \textbf{STop}, \textbf{RESet},
   \textbf{EXit}, \textbf{..}, and \textbf{/}.

\begin{center}
\begin{longtable}{p{4cm} p{10cm}}
\caption[Yoshimi Scales Commands]{Yoshimi Scales Commands} \\

\texttt{[ON/OFF]} & Enable/disable microtonal tuning. \\
\texttt{IMPort [s1] [s2]} &
   Import Scala file s2 to s1 TUNing or KEYmap.  \\
\texttt{EXPort [s1] [s2]} &
   Export s1 TUNing or KEYmap to Scala file s2. \\

\texttt{FRequency [n]} &
   Set the reference note's actual frequency to n, usually 69 (A4) to 440 Hz. \\
\texttt{NOte [n]} &
   Set the reference note's number to n. \\
\texttt{Invert [s]} &
   Invert the entire scale (ON / other) \\
\texttt{CEnter [n]} &
   Set the note number of the key's center to n. \\
\texttt{SHift [n]} &
   Shift the entire scale up or down by n. \\
\texttt{SCale [s]} &
   Activate microtonal scale (ON / other). \\
\texttt{MApping [s]} &
   Activate keyboard mapping (ON / other). \\
\texttt{FIrst [n]} &
   Set the first note number to be mapped to n. \\
\texttt{MIddle [n]} &
   Set the middle note number to be mapped to n. \\
\texttt{Last [n]} &
   Set the last note number to be mapped to n. \\
\texttt{Tuning [s1]} &
   Set the CSV tuning values.
   Tuning sets the CSV tuning values, which are decimal numbers or ratios
   (n1.n1 or n1/n1, n2.n2 or n2/n2, etc.).\\
\texttt{Keymap [s]} &
   Set the CSV keymap (n1, n2, n3, etc.)
   Keymap sets the keyboard mapping values as a comma separated list. \\
\texttt{SIze [n]} &
   Sets the actual keymap size. \\
\texttt{NAme [s]} &
   Set the internal name for this scale. \\
\texttt{DEscription [s]} &
   Sets the description of this scale. \\
\texttt{CLEar} &
   Clear all settings and revert to the standard scale. \\
\end{longtable}
\end{center}


\subsection{Help List}
\label{subsec:command_line_help_list}

   One can now clearly see which items can be listed with:

   \begin{verbatim}
      yoshimi> list
      yoshimi> help
   \end{verbatim}

   In general \textbf{?} and \textbf{help} indicate what can be changed, while
   \textbf{list} reports what the current status is. In fact in most contexts just 'L' by itself works!

   In the table that follows, we leave off commands noted above
   (see \sectionref{subsec:command_line_top_command_list}).

\begin{center}
\begin{longtable}{p{4cm} p{10cm}}
\caption[Yoshimi Help Commands]{Yoshimi Help Commands} \\

\texttt{Roots} &
   List all available root paths. \\
\texttt{Banks [n]} &
   List the banks in root ID [n] or the current root.
   This command shows all of the banks present in either the numbered ([n])
   bank root, or in the current one (if no number is provided).  \\
\texttt{Instruments [n]} &
   List instruments in bank ID [n] or current bank.
   This command shows all of the instruments present in either the numbered
   (n) bank root, or in the current one (if no number is provided).  \\
\texttt{Group [s1] [s2]} &
   List instruments types, s1 (name) list all of that type with list number, \\
\texttt{ } &
   s2 'Location' include root, bank, instrument numbers. \\
\texttt{Parts} &
   List parts with instruments installed. ('More' for extra details)\\
\texttt{Vectors} &
   List settings for all enabled vectors. \\
\texttt{Config} &
   List dynamic configuration. \\
\texttt{Tuning} &
   Microtonal scale tunings. See the Scales section. \\
\texttt{Keymap} &
   Microtonal scale keyboard map.  See the Scales section. \\
\texttt{Config} &
   Show the current configuration. See the Config section. \\
\texttt{MLearn [s[n]]} &
   MIDI learned controls ('@ n' for full details on one line). \\
\texttt{History [s]} &
   Show recent files. See the extensive description below. \\
\texttt{Effects [s]} &
   List the effect types ([s] = 'all' includes preset numbers and names).
   If this command is called from the Effects level, then one see only the name
   of the current effect and the number of presets. \\
\texttt{PREsets} &
   Show all the presets for the currently selected effect. \\

\end{longtable}
\end{center}

A few more detailed descriptions occur in the following sections, where there
is not enough room in the table above.

\subsubsection{List Group [s]}
\label{subsec:command_line_instrument_group_list}
   By itself, \textbf{List Group} or \textbf{L G} will show what instrument types
   are available. Repeating the command with one of these names will then search
   the entire bank structure listing the ones of that type, along with an index
   number. Assuming one is at part level, load this instrument into the current
   part with:
   \begin{verbatim}
      set instrument group (n)
   \end{verbatim}

   Unfortunately many instrument authors didn't identify a type, so the
   \textbf{undefined} list is the longest (although with revision
   of banks in V 2.0, this has reduced somewhat).

\subsection{List History [s]}
\label{subsec:command_line_list_history}

   Show the recent history of the following items [s]:
   Instruments, Patchsets, SCales, STates, Vectors, and MLearn).
   If no parameter is provided, show them all.

   The last-used file in any section is now always at the top of its history list,
   so it's easier to pick up where one left off.
   Instruments, patch sets, vectors, scales, MIDI-learn and state all offer the
   most recent entry whenever one wants to load or save.  On first-time use (when
   there is no history) the home directory will be offered as a location,
   regardless of where \textsl{Yoshimi} was called from.

   In the specific case of instruments, when \textsl{saving},
   one is offered the instrument in the currently-selected part to the home
   directory, otherwise, when saving these 'managed' files,
   one won't be offered the previous last-used entry unless it was seen on that
   session, either by being loaded, or saved by name. This is to give some
   protection against accidental overwrites.

   For example:
   You have been working on the 'foo' instrument
   for a whole day, saving the whole patch set as you
   go. Then, the following day, you start up \textsl{Yoshimi}
   and immediately have a completely
   new idea 'bar' and start working on it. Without thinking, you save and hit
   Enter. Oops, you just wiped out 'foo'.
   Only now you haven't, because loading \textsl{Yoshimi}
   afresh would not have seen the older file, so now, saving
   will offer the home directory to put a new name in.

\subsubsection{List Load/Save}
\label{subsec:command_line_loadsave_list}

   And the same for load and save:

   \begin{verbatim}
      yoshimi> ? lo
      yoshimi> ? sa
   \end{verbatim}

   In the tables that follow, we leave off the commands noted above
   (see \sectionref{subsec:command_line_top_command_list}).

   \begin{table}[H]
      \centering
      \caption{Yoshimi Load Commands}
      \label{table:yoshimi_text_load_commands}
      \begin{tabular}{l l}

\texttt{Instrument [s]} &
   Load instrument to current part from a named file [s]. \\
\texttt{SCale [s]} &
   Load and activate scale settings from named file [s]. \\
\texttt{VEctor [n] [s]} &
   Load and activate vector to channel n (or saved) from named file [s]. \\
\texttt{Patchset [s]} &
   Load and activate a complete patch set from named file [s]. \\
\texttt{MLearn [s]} &
   Load the full MIDI learned list from named file [s]. \\
\texttt{STate [s]} &
   Load all system settings and patch sets from named file [s]. \\

      \end{tabular}
   \end{table}

   For the \textbf{Load Instrument} command, the instrument is enabled if it is
   configured to be enabled.  For the \textbf{Load Vector} command, if there is
   no number parameter, the vector is loaded to the channel it was originally
   saved from.
   For the \textbf{Load Patchset} command, all instruments, scales, and vectors
   are loaded from the named file.
   For the \textbf{Load STate} command, all configuration, system settings,
   patch sets, and MIDI-learned lines are loaded from the named file.
   These notes also apply to the \textbf{Save} version of these commands.

   \begin{table}[H]
      \centering
      \caption{Yoshimi Save Commands}
      \label{table:yoshimi_text_loadsave_commands}
      \begin{tabular}{l l}

\texttt{Instrument [s]} &
   Save current part to named file [s]. \\
\texttt{SCale [s]} &
   Save current scale settings to named file [s]. \\
\texttt{VEctor [n] [s]} &
   Save vector on channel n to named file [s]. \\
\texttt{Patchset [s]} &
   Save complete set of instruments to named file [s]. \\
\texttt{MLearn [s]} &
   Save midi learned list to named file [s]. \\
\texttt{STate [s]} &
   Save all system settings etc. to named file [s].  See above. \\
\texttt{Config} &
   Save current configuration. \\

      \end{tabular}
   \end{table}

\subsubsection{Config Commands}
\label{subsec:command_line_config_list}

   Finally there is the (quite big) COnfig command level:

   \begin{verbatim}
      yoshimi> ? con
   \end{verbatim}

   In the table that follows, we leave off the commands noted above
   (see \sectionref{subsec:command_line_top_command_list}).  Also
   note that more complete descriptions follow this table.

\begin{center}
\begin{longtable}{p{4cm} p{10cm}}
\caption[Yoshimi Config Commands]{Yoshimi Config Commands} \\

\texttt{Oscillator [n]} &
   * Add/Pad size (power 2 256-16384).
   This sets the size of the buffer used for both AddSynth and PadSynth
   oscillators, and is always a power of 2.  \\
\texttt{BUffer [n]} &
   * Internal size (power 2 16-4096).
   This is the size of the audio buffer that \textsl{Yoshimi} uses.  For ALSA
   audio, it will always be the same size as ALSA's buffering, but for JACK it
   can be the same, bigger, or smaller. It is always a power of 2. \\
\texttt{PAdsynth [s]} &
   Interpolation type (Linear, other = cubic).
   Sets the quality of the interpolation that PadSynth uses on its wavetables.
   'Linear' is faster, but 'Cubic' is (potentially) very slightly better
   quality. \\
\texttt{BUIldpad [s]} &
   PADSynth wavetable build mode (Muted, Background, Autoapply)\\
\texttt{Virtual [n]} &
   Keyboard layout (0 = QWERTY, 1 = Dvorak, 2 = QWERTZ, 3 = AZERTY).
   This setting controls the layout of the virtual keyboard, and can match the
   commonest computer keyboards. \\
\texttt{Xml [n]} &
   Set the XML compression level to [n] (0-9).
   This is the amount of compression used on all
   \textsl{Yoshimi}'s data files. 9 is the most-compressed setting.
   0 is no compression, so that the configuration file
   can be read in an ordinary text editor. \\
\texttt{REports [s]} &
   Destination for reporting (Stdout, other = console).
   Determines where almost all information and error messages will be sent. A
   few will always go to \texttt{stderr} (such as the ones reporting a GUI
   problem). \\
\texttt{SAved [s]} &
   Saved instrument type. (Legacy - .xiz, Yoshimi - .xiy, Both).\\
\texttt{ENGine [s]} &
   Enable instrument engines and types info (OFF, {other})
   Slower initial start, but bank instruments can be selected by type and
   engines will be identified in lists.\\
\texttt{EXPose [s]} &
   Show current context level (ON, OFF, PRompt).\\
\texttt{STate [s]} &
   * Autoload default at start (Enable; other = disable).
   Sets whether a pre-saved default state file will be loaded on start-up. \\
\texttt{SIngle [s]} &
   * Force 2nd startup to open new instance instead (ON / other)\\
\texttt{Hide [s]} &
   Hide non-fatal errors (Enable; other = disable).
   Sets to ignore non-fatal system errors, or verbose messages. \\
\texttt{Display [s]} &
   GUI splash screen (Enable; other = disable).
   Enables \textsl{Yoshimi}'s start-up splash screen (which is
   enabled at first time start). \\
\texttt{Time [s]} &
   Add to instrument load message (Enable; other = disable).
   Attaches the time an instrument took to load and initialise to the loading
   message. \\
\texttt{Include [s]} &
   Include XML headers on file load (Enable; other = disable). \\
\texttt{Keep [s]} &
   Include inactive data on all file saves (Enable; other = disable).
   Sets up to include all data on file saves, including data for
   inactive and random elements. \\
\texttt{Gui [s]} &
   * Run with GUI (Enable, Disable).
   Run with the command-line interface enabled or disabled.  \\
\texttt{Cli [s]} &
   * run with CLI (Enable, Disable).
   Run with the command-line interface enabled. \\
\texttt{LOCk [s1][s2]} &
   Lock history of group s1 (ON, OFF).
   Groups - INstrument, PAtchset, SCale, STate, VEctor, MLearn. \\
\texttt{MIdi [s]} &
   * connection type (Jack, Alsa).
   Sets whether MIDI input comes from JACK or from ALSA.
   If not specified, JACK is the default, if it is present.  Otherwise,
   \textsl{Yoshimi} falls back to ALSA. \\
\texttt{AUdio [s]} &
   * connection type (Jack, Alsa).
   Sets whether audio is passed out to JACK or ALSA.  Again, JACK is the
   default, and ALSA is the fall-back.  \\
\texttt{ALsa Type [s]} &
   * midi connection type (Fixed, Search, External)
   Sets whether to use the list below, search for sources, or none at all. \\
\texttt{ALsa Midi [s]} &
   * comma separated source name list.
   Sets the entries of a list of ALSA MIDI sources to which
   \textsl{Yoshimi}i will try to connect.  \\
\texttt{ALsa Audio [s]} &
   * Name of ALSA audio hardware device.
   Sets the name of a hardware (or software)
   audio device to which ALSA will try to connect.  \\
\texttt{ALsa Sample [n]} &
   * ALSA sampling rate (0 = 192000, 1 = 96000, 2 = 48000, 3 = 44100).
   Sets the sampling rate when using ALSA audio. \\
\texttt{Jack Midi [s]} &
   * Name of JACK MIDI source.
   Sets the name of a JACK MIDI source to which
   \textsl{Yoshimi} will try to connect.  \\
\texttt{Jack Server [s]} &
   * Name of JACK server.
   Sets the name of an audio server to which JACK will try to connect. \\
\texttt{Jack Auto [s]} &
   * Connect JACK on start (Enable; other = disable).
   Determines whether JACK will try to connect the main L=R audio outputs at
   start-up time. \\
\texttt{ROot [s]} &
   Root CC (Msb, Lsb, Off).
   Provides the MIDI CC that \textsl{Yoshimi} expects bank root changes to come
   from.  \\
\texttt{BAnk [s]} &
   Bank CC (Msb, Lsb, Off).
   Provides the MIDI CC that \textsl{Yoshimi} expects
   bank changes to come from.  \\
\texttt{PRogram [s]} &
   MIDI program change enabling (OFF / other).
   Determines whether MIDI program changes are honoured or ignored.  \\
\texttt{ACtivate [s]} &
   MIDI program change activates part (ON / other).
   Enables a part when it gets a MIDI program change message,
   if it was disabled.  \\
\texttt{EXtended [n]} &
   Extended program change.
   Sets a MIDI CC for receiving program changes for the top (extra)
   32 instruments in a bank, or disable if above 119  \\
\texttt{Quiet [s]} &
   Ignore 'reset all controllers' (ON / other).
   Sets up to ignore MIDI 'reset all controllers' messages. \\
\texttt{Nrpn [s]} &
   enable incoming NRPNs (ON / other).
   Displays the value of received MIDI CCs. \\
\texttt{Log [s]} &
   Log incoming MIDI CCs (ON / other).
   Displays the value of received MIDI CCs. \\
\texttt{SHow [s]} &
   GUI MIDI learn editor (ON / other).
   A setting for the GUI MIDI learn editor, where s is "Enable", or some
   other token to disable the feature.
   This setting indicates to automatically open the MIDI-learn editor window
   when a successful "learn" has been made.  \\

\end{longtable}
\end{center}

   '*' marks entries that need to be saved, and \textsl{Yoshimi}
   restarted, to activate them.

\subsection{Command Descriptions}
\label{subsec:command_line_command_descriptions}

   This section describes the command-line commands in more detail.
   Obviously, some more needs to be written about some of the commands.
   Note that all the parameters for the \texttt{load} and \texttt{save}
   parameters are strings, and the parameters are compulsory, not optional.

   \setcounter{ItemCounter}{0}      % Reset the ItemCounter for this list.

   \itempar{".."}{cmd!up one level}
      \index{cmd!..}
      Step back up one command context level.
      This command can immediately precede another command, so that the second
      command takes places at the context above the current context.
      Note that it is like the OS's "cd .." command to change to the parent
      directory.

   \itempar{/}{cmd!to top level}
      \index{cmd!/}
      Step back up to the top command context level.
      This command can immediately precede another command, so that the second
      command takes places at the top context.
      Note that it is like the OS's "cd /" command to change to the root
      directory.

   \itempar{add bank [s]}{cmd!add bank}
      Define a new bank, \textsl{s}, where \textsl{s} is a bank name,
      and add it to the current root.

   \itempar{remove bank [s]}{cmd!remove bank}
      Delete the bank named \textsl{s}, and all its contents,
      from the current root path.

   \itempar{export bank [s [n1]] [n2] [s]}{cmd!export bank}
      The command line now has two commands to provide access to the new bank
      export and import controls. These are top level controls and are used as
      below.  The command above is used to export a bank. The square bracket term
      is optional, and enables one to select a different root to export from and
      would be in the form:

\begin{verbatim}
   EXport Root (root ID number) (bank ID number) (full path name to export to)
\end{verbatim}

   If one is happy to export from the currently selected root, then this
   simplifies to:

\begin{verbatim}
   EXport (bank ID number) (full path name to export to)
\end{verbatim}

   \itempar{import bank [s [n1]] [n2] [s]}{cmd!import bank}
      Import of a bank uses the identical syntax of the export command.
      A full example using the normal abbreviations is:

\begin{verbatim}
   im r 5 25 /home/will/downloads/some new bank
\end{verbatim}

      This will look for the directory "some new bank" (spaces are accepted) in
      the download directory of user "will". It will then copy it into bank number
      25 of root number 5. It first checks to ensure that the new named bank
      exists, root 5 exists, and bank 25 is empty.

   \itempar{add root [s]}{cmd!add root}
      Define a new root path, \textsl{s}, and add it to the list of root paths.

   \itempar{remove root [s]}{cmd!remove root}
      De-list the root path named \textsl{s}, but doesn't alter the contents.

   \itempar{list banks [n]}{cmd!list banks}
      List the instruments and IDs in bank \textsl{n} or the
      current bank/root.

   \itempar{list effects [s]}{cmd!list effects}
      List effect types for \textsl{s}.
      If the parameter is the word \textsl{all},
      then list every effect and all its
      presets along with the preset number.

   \itempar{list history [s]}{cmd!list history}
      Displays the recently-used files, including patchsets, scales, and
      states.  If no parameter is given, then this command lists all three
      files in sequence.  The shortest version of this command is
      \texttt{l h p} (for patchsets, which returns the last-seen patchset list).

      \index{at-sign}
      \index{cmd!at-sign}
      \index{cmd!list operator}
      Once that list is displayed,
      the \texttt{@} operator can be used to access
      the item by number.  For example, to load the patch set at location 4 in
      the list:

      \begin{verbatim}
         yoshimi> lo p @4
      \end{verbatim}

   \itempar{list instruments [n]}{cmd!list instruments}
      List all instruments and IDs in bank \textsl{n}
      or the current bank/root.
      Listing instruments will identify the current one with an asterisk, and
      shows the current bank and root one is listing from, and adds a suffix to
      the entry with \textbf{A}, \textbf{S}, or \textbf{P}
      depending on the combination of AddSynth, SubSynth, and PadSynth.

   \itempar{list parts}{cmd!list parts}
      Lists the number of parts available and parts with instruments
      currently installed along with any enabled with the default sound.
      Also shows their audio destination:
      \textsl{M} = main L/R, \textsl{P} = part L/R, \textsl{B} = both, and
      \textsl{-} = disabled or unavailable.
      This way one can tell if an instrument patch is installed even if it is
      not currently usable.
      To avoid unnecessary list length, the default "Simple Sound" is not shown
      unless it is enabled.

   \itempar{list roots}{cmd!list roots}
      Displays all defined root paths.
      Listing roots will identify the current ones with an asterisk.

   \itempar{list setup}{cmd!list setup}
      Displays the current dynamic system settings.

   \itempar{list vector [n]}{cmd!list vector}
      Lists the settings for vector on channel \textsl{n}.

   \itempar{load instrument [s]}{cmd!load instrument}
      Loads an instrument into the current part from the named file.
%     The file-name parameter \textsl{s} is mandatory.

   \itempar{load patchset [s]}{cmd!load instrument}
      Load a complete patch set from a named file, \textsl{s}.
      A variation on this command is \texttt{load patchset @4}, which
      loads the patchset at location 4, the 4th item in the list.

   \itempar{load vector [s]}{cmd!load vector}
      Loads an vector setup from the named file.
      The file-name parameter \textsl{s} is mandatory.

   \itempar{save patchset [s]}{cmd!save patchset}
      Saves the current patchset to the file named \textsl{s}.

   \itempar{save instrument [s]}{cmd!save instrument}
      Saves the instrument of the current part to the named file.
      The file-name parameter \textsl{s} is mandatory.

   \itempar{save setup}{cmd!save setup}
      Save the current dynamic system settings.
      Most settings get saved to the state file for the current instance,
      but a few are saved in the master config file.

   \itempar{save vector [s]}{cmd!save vector}
      Saves the vector setup to the named file.
      The file-name parameter \textsl{s} is mandatory.

   \itempar{set activate [n]}{cmd!set activate}
      Set part-activate on MIDI program change.
      \textsl{n}=0 disables this feature, and
      1 or any non-zero value enables this feature.
      This feature applies to command line program change as well.

   \itempar{set alsa audio [s]}{cmd!set alsa audio}
      Sets the name of the audio hardware device ALSA looks for.
      Requires a restart of \textsl{Yoshimi}.

   \itempar{set alsa midi [s]}{cmd!set alsa midi}
      Sets the name of the MIDI source ALSA looks for.
      Requires a restart of \textsl{Yoshimi}.

   \itempar{set available [n]}{cmd!set available}
      Set the number of available parts (16, 32, 64).
      Note that 32 and 64 are supported in the newest versions of
      \textsl{Yoshimi}.  Also note that a single two-part vector setup (the
      \textbf{X} vector) requires 32 parts, while the dual two-part vector
      setup (both \textbf{X} and \textbf{Y}) requires 64 parts.

   \itempar{set bank [n]}{cmd!set ccbank}
      Set the MIDI CC for bank changes ((Msb, Lsb, Off)).

   \itempar{set root [n]}{cmd!set ccroot}
      Set the MIDI CC for root path changes ((Msb, Lsb, Off)).

   \itempar{set extend [n]}{cmd!set extend}
      Set CC value for extended program change (values above 119 disables this
      feature).

   \itempar{set insert effect [n]}{cmd!set insert effects}
      Set insertion effect for editing. Where \textsl{n} is from 1 to 8.

   \itempar{set jack midi [s]}{cmd!set jack midi}
      Sets the name of the JACK MIDI source for \textsl{Yoshimi}.
      Requires a restart of \textsl{Yoshimi}.

   \itempar{set jack server [s]}{cmd!set jack server}
      Sets the name of the JACK server \textsl{Yoshimi} tries to connect to.
      Requires a restart of \textsl{Yoshimi}.

   \itempar{set part [n1] program [n2]}{cmd!set part program}
      Load instrument \textsl{n2} into part \textsl{n1}.
      Example: \texttt{set part 4 program 130}

   \itempar{set part [n1] channel [n2]}{cmd!set part channel}
      Set the MIDI channel \textsl{n2} for part \textsl{n1}.
      If the channel number is greater than 15, no further MIDI
      messages will be accepted by that part.

   \itempar{set part [n1] destination [n2]}{cmd!set part destination}
      Set the audio destination of part \textsl{n1}
      to main (\textsl{1}), part (\textsl{2}), both (\textsl{3}).
      Also enables the part if not already enabled.

   \itempar{set preferred audio [s]}{cmd!set preferred audio}
      Set the audio connection type.
      The parameter should be either "jack" or "alsa".

   \itempar{set preferred midi [s]}{cmd!set preferred midi}
      Set the MIDI connection type.
      The parameter should be either "jack" or "alsa".

   \itempar{set - preset [n]}{cmd!set effect preset}
      Set effect preset.
      Set numbered effect preset.

   \itempar{set program [n]}{cmd!set program}
      Set MIDI program change (0 disables, anything else enables).

   \itempar{set reports [n]}{cmd!set reports}
      Sets the report destination or where messages are displayed, and, to some
      extent, which messages are displayed.  Here are the variations on this
      command that are supported:

      \begin{itemize}
         \item \texttt{set reports gui} or \texttt{s r g}.
            All reports are sent to the GUI console window.
         \item \texttt{set reports stderr} or \texttt{s r s}.
            All reports are sent to stderr.
         \item \texttt{set reports show} or \texttt{s r sh}.
            All messages are displayed.
         \item \texttt{set reports hide} or \texttt{s r h}.
            Non fatal low-level messages are discarded.
         \item \texttt{set reports (any other word or nothing at all)} or
            \texttt{s r (other)}.
            This sets the default condition of sending reports to the CLI and
            displaying all of them.
      \end{itemize}

   \itempar{set root [n]}{cmd!set root}
      Set current root path to ID \textsl{n}.

   \itempar{set shift [n]}{cmd!set shift}
      Set the master key shift for following notes in semitones (+-
      octave, 64 for no shift).

   \itempar{set system effects [n]}{cmd!set system effects}
      Set System Effects for editing.

   \itempar{set vector [n1] x/y cc [n2]}{cmd!set vector cc}
      CC \textsl{n2} is used for channel \textsl{n1} X or Y axis sweep.
      For X, this also enables vector control for the channel.

      \index{vector!features}
      The individual features are now numbered 1-4 and can be
      \index{vector!enable} \texttt{enabled} or
      \index{vector!reverse} \texttt{reversed} (any
      other word disables the feature).
      "Reversed" means that, instead of the X left rising in
      value with increasing CC value, it decreases.
      X right does the opposite of course.

      Feature 1 is always fixed as 7 (volume) and is not reversible.
      Features 2 to 4 can also have the outgoing CC changed to any valid one.
      The vector is just about the only command-line
      entry that starts from 1.

      The original system where bits were ORred together was done to make NRPN
      control as efficient as possible. That hasn't changed, but log messages
      refer to the command-line numbering.

      A more detailed discussion of command-line vector control is presented in
      \sectionref{subsection:vector_command_line}.

   \itempar{set vector [n1] x/y features [n2]}{cmd!set vector features}
      Sets channel \textsl{n1} X or Y features to \textsl{n2}.

   \itempar{set vector [n1] x/y program [l/r] [n2]}{cmd!set vector program}
      Loads program \textsl{n2} to channel \textsl{n1} X or Y
      \textsl{left} or \textsl{right} part.

   \itempar{set vector [n1] x/y control [n2] [n3]}{cmd!set vector control}
      Sets \texttt{n3} CC to use for X or Y feature \texttt{n2} (2, 4, 8).
      \textsl{n3} is the CC to be used for feature number \textsl{n2} on X
      vector channel \textsl{n1}. The \textsl{x} is a sort of hidden parameter
      as the code uses an offset dependent on whether it is \textsl{x} or
      \textsl{y}. Also \textsl{n1} can be omitted in which case it will use the
      last defined channel number. Using alternate words and numbers gives a
      great deal of flexibility like this.

   \itempar{set vector [n] [off]}{cmd!set vector}
      Disables vector control for channel \textsl{n}.

   \itempar{set volume [n]}{cmd!set volume}
      Set the master volume.

   \itempar{reset}{cmd!reset}
      Return to the start-up conditions, if 'y' selected.

   \itempar{stop}{cmd!stop}
      Cease all sound immediately!

   \itempar{? or help}{cmd!help}
      List commands for current mode.  All of the minimum command-line
      abbreviations are capitalised in the help listing.

   \itempar{exit}{cmd!exit}
         Tidy up and close \textsl{Yoshimi} down.

\subsection{Direct Access}
\label{subsec:command_line_direct_access}

   \index{Direct Access}
   \index{cmd!direct access}
   \textsl{Direct Access} is a very low-level access method for most of the
   controls in \textsl{Yoshimi} control.
   It is a test feature accessible only from the command line.
   There are virtually no error checks on direct-access, so one can easily
   crash \textsl{Yoshimi} with out of range values.
   It mostly updates the GUI too.

%   Since V1.5.11 users are strongly discouraged from using it as numerical
%   values will be changing and all available controls are now provided in a
%   fully protected manner by the above listed commands.

    Since V 1.6.0 direct access has been disabled as it no longer produces
    valid data. Also, all available controls are now provided in a fully
    protected manner.

   For developers there is relevant information within the source code.

%-------------------------------------------------------------------------------
% vim: ts=3 sw=3 et ft=tex
%-------------------------------------------------------------------------------


% Yoshimi as an LV2 plugin

%-------------------------------------------------------------------------------
% yum_LV2
%-------------------------------------------------------------------------------
%
% \file        yum_LV2.tex
% \library     Documents
% \author      Chris Ahlstrom
% \date        2015-10-25
% \update      2016-05-24
% \version     $Revision$
% \license     $XPC_GPL_LICENSE$
%
%     Provides a very incomplete description and discusson of Yoshimi LV2
%     support.
%
%-------------------------------------------------------------------------------

\section{LV2 Plug-in Support}
\label{sec:lv2_plugin}

   \textsl{Yoshimi} now runs as an LV2 plugin.

Supported features:

   \begin{enumber}
      \item Sample-accurate midi timing.
      \item State save/restore support via LV2\_State\_Interface.
      \item Working UI support via LV2\_External\_UI\_Widget.
      \item Programs interface support via LV2\_Programs\_Interface.
      \item Multi channel audio output. 'outl' and 'outr' have LV2 index 2
         and 3. All individual ports numbers start at 4.
   \end{enumber}

   Planned feature: Controls automation support. This will be a part of a
   common controls interface.

   Download and build the source code found at the
   \textsl{Yoshimi} site \cite{yoshimi},
   and one will find a file named
   \texttt{LV2\_Plugin/yoshimi\_lv2.so}

% Just list discussion, and never intended to be a permanent record. Also it is
% now completely untrue! It has been untrue since Version 1.3.6 was released at
% the end of September 2015.
%
%  The LV2 \textsl{Yoshimi} interface can be run in hosts such as
%  \textsl{Ardour 3}, \textsl{Carla}, and \textsl{QTractor}.  We have tested
%  LV2 using the latest versions of \textsl{Ardour}, \textsl{Muse} and
%  \textsl{Qtractor} as LV2 hosts. We also tried \textsl{Carla} but couldn't
%  get anywhere with it at all... if someone is familiar with it we'd be
%  interested in what results they get.  Slightly to our surprise the one that
%  performed the best in every respect was \textsl{Qtractor}. \textsl{Muse} got
%  everything correct, but was prone to Xruns on program changes.
%  \textsl{Ardour} was very problematical.  There were no Xruns but it seemed
%  to have odd timing issues. Also, on two occasions it managed to shorten the
%  decay times of two of the instruments.  We don't understand how it managed
%  that.  Our reference was the original MIDI file played into a stand-alone
%  \textsl{Yoshimi} via \texttt{aplaymidi}. This also behaves identically to
%  the file being sequenced by \textsl{Rosegarden}.

   \textsl{Yoshimi}'s LV2 impementation is frequently tested using the latest
   versions of \textsl{Ardour}, \textsl{Muse}, and \textsl{Qtractor} as LV2
   hosts. Like \textsl{Yoshimi}, these are also in continuous development. So
   far we've not been able to get anywhere with \textsl{Carla}; if someone is
   familiar with it, we'd be interested in what results they get.

   At some point we hope to document the process of setting up and using
   the \textsl{Yoshimi} LV2 plugin.

%  In the meantime, we include some notes.

%  If \textsl{Yoshimi}'s internal buffer is \textsl{smaller} than the JACK
%  buffer, it sounds quite horrible. If it's the same or greater there's no
%  problem. The reason we didn't find this before is that it only affects
%  \textsl{Yoshimi} LV2.  Also it doesn't apply to any of the released
%  versions.  The cause of the problem is that the LV2 code doesn't have the
%  same looping structure that was added to the standalone routine to deal with
%  exactly this situation (re-entering the audio 'construction' function until
%  the JACK buffer is filled). We hope Andrew can deal with this fairly soon as
%  we don't understand the LV2 code very well.  There are valid reasons for
%  wanting different sizes for these buffers. The internal buffer size as well
%  as affecting latency and CPU load also alters the sound in quite subtle
%  ways. It particularly affects the behaviour of filters.  One day we may be
%  able to stop this happening, but in the mean time we have to live with it.
%  For our purposes, we find a buffer size of 128 or 256 is best.

%-------------------------------------------------------------------------------
% vim: ts=3 sw=3 et ft=tex
%-------------------------------------------------------------------------------


% Man page

%-------------------------------------------------------------------------------
% yum_manpage
%-------------------------------------------------------------------------------
%
% \file        yum_manpage.tex
% \library     Documents
% \author      Chris Ahlstrom
% \date        2015-06-22
% \update      2015-10-24
% \version     $Revision$
% \license     $XPC_GPL_LICENSE$
%
%     Provides the man page section of yoshimi-user-manual.tex.
%
%-------------------------------------------------------------------------------

\section{Yoshimi Man Page}
\label{sec:yoshimi_man_page}

   The \textsl{Yoshimi} man page is actually the output of the
   \texttt{yoshimi --help} command, which prints out the command-line that
   are discussed in this section.

Yoshimi 1.3.6, a derivative of ZynAddSubFX - Copyright 2002-2009
Nasca Octavian Paul and others, Copyright 2009-2011 Alan Calvert,
Copyright 20012-2013 Jeremy Jongepier and others,
Copyright 20014-2015 Will Godfrey and others.

   \setcounter{ItemCounter}{0}      % Reset the ItemCounter for this list.

  \optionpar{-a}{--alsa-midi[="device"]}
      Use ALSA MIDI input.
      From the command line, as well as autoconnecting the main L \& R
      outputs to JACK, with ALSA MIDI one can now auto-connect to a known source.

   \begin{verbatim}
      ./yoshimi -K --alsa-midi="Virtual Keyboard"
   \end{verbatim}

      ALSA can often manage with just the client name.  This command is case
      sensitive, and quite fussy about spaces, etc., so it's wise to use
      quotes for the source name, even if they don't seem to be needed.

  \optionpar{-A}{--alsa-audio[=device]}
      Use ALSA audio output.

  \optionpar{-b}{--buffersize=size}
      Set ALSA internal audio buffer size.

  \optionpar{-c}{--show-console}
      Show the console on startup.

  \optionpar{-D}{--define-root}
      Define the path to a new bank root.
      \textsl{Yoshimi} will then immediately scan this path for new banks,
      but won't make the root (or any of its banks) current. The final
      directory doesn't in fact have to be 'banks' but traditionally
      \textsl{Yoshimi} has always done this.
      Also, when running from the command line there is now access to many of
      the system, root, bank, etc. settings.
      See \sectionref{sec:command_line}.

  \optionpar{-i}{--no-gui}
      Do not show the GUI.  See \sectionref{sec:command_line} for more
      information about this mode of operation.  Note that the command and the
      GUI can be available simultaneously.

  \optionpar{-j}{--jack-midi[=device]} 
      Use JACK MIDI input.
      From the command line, as well as autoconnecting the main L \& R
      outputs to JACK, with JACK MIDI one can now auto-connect to a known source.

   \begin{verbatim}
      ./yoshimi -K --jack-midi="jack-keyboard:midi_out"
   \end{verbatim}
   
   JACK needs the port as well as the name.
   This command is case sensitive, and quite fussy about spaces, etc.,
   so it's wise to use quotes for the source name, even if they don't seem to
   be needed.

  \optionpar{-J}{--jack-audio[=server]}
      Use JACK audio output.
      Connect to the given JACK server if given.

  \optionpar{-k}{--autostart-jack}
      Auto-start the JACK server.

  \optionpar{-K}{--auto-connect}
      Auto-connect JACK audio.

  \optionpar{-l}{--load=file}
      Load a \texttt{.xmz} file.

  \optionpar{-L}{--load-instrument=file}
      Load an \texttt{.xiz} file

  \optionpar{-N}{--name-tag=tag}
      Add tag to client-name.

  \optionpar{-o}{--oscilsize=size}
      Set ADDSynth oscillator size (OscilSize).

  \optionpar{-R}{--samplerate=rate}
      Set ALSA audio sample rate.

  \optionpar{-S}{--state[=file]}
      Load saved state from file, where the file defaults to
      \texttt{\$HOME/.config/yoshimi/yoshimi.state}

  \optionpar{-u}{--jack-session-file[=file]}
      Load the named JACK session file.

  \optionpar{-U}{--jack-session-uuid[=uuid]}
      Load the named JACK session by UUID.

  \optionpar{-?}{--help}
      Give this help list.

   \optionpar{--usage}
      Provide a short usage message.

  \optionpar{-V}{--version}
      Print program version.

   Mandatory or optional arguments to long options are also mandatory or
   optional for any corresponding short options.

   From the command line, as well as autoconnecting the main L \& R outputs
   to JACK, with either JACK or ALSA MIDI one can now auto-connect to a
   known source.

   ALSA can often manage with just the client name, but JACK needs the port
   as well. These commands are case sensitive, and quite fussy about spaces
   etc. so it's wise to use quotes for the source name, even if they don't
   seem to be needed.

%-------------------------------------------------------------------------------
% vim: ts=3 sw=3 et ft=tex
%-------------------------------------------------------------------------------


% Important Concepts

%-------------------------------------------------------------------------------
% yum_concepts
%-------------------------------------------------------------------------------
%
% \file        yum_concepts.tex
% \library     Documents
% \author      Chris Ahlstrom
% \date        2015-06-15
% \update      2016-03-07
% \version     $Revision$
% \license     $XPC_GPL_LICENSE$
%
%     Provides the concepts.
%
%-------------------------------------------------------------------------------

\section{Concepts}
\label{sec:concepts}

   Before we start with the user-interface, let's cover some concepts and terms.
   \textsl{Yoshimi} requires the user to understand many concepts and terms.
   Understanding them makes it easier to configure \textsl{Yoshimi}
   and to drive it from a sequencer application.
   
   Significant portions of this section are shamelessly copied (and tweaked)
   from Paul Nasca's original \textsl{ZynAddSubFX}
   manual \cite{zynodt} or \cite{zynpdf}.
   One can discern such sections by the usage of the term
   \textsl{ZynAddSubFX} instead of \textsl{Yoshimi}.
   However, even the \textsl{Yoshimi} developers sometimes
   refer to \textsl{ZynAddSubFX} or \textsl{Zyn}.

   Note that there are some audio/electrical concepts discussed in greater
   detail in \sectionref{sec:stock_settings_elements}.
   Perhaps they belong in this "concepts" section, but
   they are directly tied to user-interface items.

\subsection{Concepts / Terms}
\label{subsec:concepts_terms}

   This section doesn't provide comprehensive coverage of terms.  It
   covers mainly terms that might puzzle one at first, or have a
   special meaning in \textsl{Yoshimi}.

\subsubsection{Concepts / Terms / Cent}
\label{subsubsec:concepts_terms_cent}

   The \textbf{cent}
   \index{cent}
   is a logarithmic unit of measure for musical intervals.
   Twelve-tone equal temperament divides the octave into 12
   semitones of 100 cents each. Typically, cents are used to measure
   extremely small pitch intervals, or to compare the sizes of comparable
   intervals in different tuning systems.
   The interval of one cent is much too small to be heard between
   successive notes.

\subsubsection{Concepts / Terms / Frame}
\label{subsubsec:concepts_terms_frame}

   The audio \textbf{frame}
   \index{frame}
   is a single sample of however many channels an application is handling.
   If one is using JACK, a mono signal will have frames of 1 float, 2 floats
   for stereo, etc.  A six-channel device will have six samples in a single
   frame.

   An audio or JACK buffer will contain more than one frame of data.
   Buffers generally range in size from 16 to 1024 frames.
   Low values provide less latency, but make the CPU work harder.

\subsubsection{Concepts / Terms / Instrument}
\label{subsubsec:concepts_terms_instrument}

   \index{instrument}
   In \textsl{Yoshimi}, an \textsl{instrument} is a complex sound that can
   be constructed using ADDsynth, SUBsynth, PADsynth, and kits.
   Each instrument is loaded into a \textsl{part}
   (see \sectionref{subsubsec:concepts_terms_part}).

   In our documentation, we will sometimes use the terms "instrument",
   "patch", and even "program" interchangeably and loosely.
   However, "part" now has a different meaning, as seen in the next term.

\subsubsection{Concepts / Terms / Part}
\label{subsubsec:concepts_terms_part}

   \index{part}
   In \textsl{Yoshimi}, a \textsl{part} is one of 16, 32, or
   64 "slots" into which one can load an instrument (see
   \sectionref{subsubsec:concepts_terms_instrument}).  Each part can be
   enabled or disabled, and assigned to a particular MIDI channel, one of
   the 16 MIDI channels.
   Note that the previous \textsl{Yoshimi} limit on parts was 16.
   Since 1.3.5, this limit has been raised to 64.

\subsubsection{Concepts / Terms / Patch}
\label{subsubsec:concepts_terms_patch}

   \index{patch}
   In MIDI jargon, a \textsl{patch} is a sound played on
   one of 16 channels in a MIDI device. Many synthesizers
   can handle several waveforms per patch, mixing different
   instruments together to create synthetic sounds. Each waveform counts as
   a MIDI voice. Some sound cards can support two or more waveforms per
   patch.  \textsl{Yoshimi} has some ability to combine waveforms ("voices")
   into one instrument (\sectionref{subsubsec:concepts_terms_instrument}),
   which can then be loaded into a
   \textsl{Yoshimi} part (\sectionref{subsubsec:concepts_terms_part}).

   Before General MIDI, which standardized patches, MIDI vendors assigned
   patch numbers to their synthesizer products in an arbitrary manner.

\subsubsection{Concepts / Terms / Patch Set}
\label{subsubsec:concepts_terms_patch_set}

   \index{patchset}
   \index{patch set}
   A patch set (also known as "patchset")
   is basically a group of instruments related simply by the user
   wanting to have them all loaded at once into \textsl{Yoshimi}.
   A patch set is stored in a \texttt{.xmz} file.
   A patch set is akin to a preset, in that it stores a combination of
   items, that took awhile to set up, for easy retrieval later.

   As with most applications, \textsl{Yoshimi} and \textsl{ZynAddSubFX}
   allow for one to save one's work and reload it.
   \textsl{Yoshimi} has a number of different files that make up the current
   configuration.
   Together, they make up the concept of a \textsl{patch set} (also called a
   \textsl{patchset}).
   See \sectionref{sec:configuration}.

\subsubsection{Concepts / Terms / Presets}
\label{subsubsec:concepts_terms_preset}

   \index{preset}
   Presets allow one to save the
   settings for any of the components which support copy/paste operations.
   This is done with preset files (\texttt{.xpz}), which get stored in the
   folders indicated by \textbf{Paths / Preset Dirs...}.
   Note that the number of preset directories that can be set is limited to 128
   (the same as for roots and banks).

   In MIDI jargon, a \textsl{preset} is an instrument that can be easily
   loaded.  It is also called a \textsl{program} or a \textsl{patch}.  A
   program is selected via a "program-change" message.  A
   \textsl{preset} is any collection of settings that can be saved to the
   clipboard or to a file, for later loading elsewhere.

\subsubsection{Concepts / Terms / Program}
\label{subsubsec:concepts_terms_program}

   \index{program}
   In MIDI jargon, a \textsl{program} is the same as a \textsl{preset}
   (\ref{subsubsec:concepts_terms_preset}).

\subsubsection{Concepts / Terms / Voice}
\label{subsubsec:concepts_terms_voice}

   \index{voice}
   In MIDI jargon, a \textsl{voice} is the same as
   a \textsl{preset} or a \textsl{program}.
   In \textsl{Yoshimi}, a \textsl{voice} is a single configurable waveform
   that is just one of up to eight waveforms in an ADDsynth setting.
   Such voices can also be used as modulators for other voices.

\subsection{Concepts / ALSA Versus JACK}
\label{subsec:concepts_alsa_versus_jack}

   Some discusson from the \textsl{Yoshimi} wiki.  Here for eventual
   clarification.

   A bit of a question mark was raised over ALSA MIDI support. A lot of
   people seem to be giving this up and relying on bridges like
   \textsl{a2jmidi} for legacy software and hardware inputs. JACK MIDI is
   already synchronous so should be jitter-free whereas ALSA MIDI runs on a
   'best effort' basis. Added to which JACK is available for OS X and
   Windows so concentrating on this could make a possible port to other
   platforms more attractive -- not to me I (Will J. Godfrey) hasten to add!

   \textsl{Seq24} (a nice, if old, sequencer) uses ALSA MIDI. To connect
   applications that exclusively support JACK MIDI, \textsl{a2jmidid} will
   do the translation.  (Jack v. 1 has this integrated in recent versions,
   apparently).

   A frame is a single sample of however many channels one is handling, so if
   one is using JACK, a mono signal will have frames of 1 float, 2 for stereo,
   etc.

   ALSA is more complex as it handles the sound card's format, commonly integers
   (16 bit), 24 bit integers (low byte ignored), and short integers. Less
   commonly it may be floats or the weird 24 L ints. We're still not sure if
   these are packed or low-aligned (top byte ignored). We've assumed they are
   low aligned, but we don't know anyone who has such a card, in order to prove
   it.  As a matter of interest the only ALSA format \textsl{Yoshimi} doesn't
   support is float.

   Something that's not obvious is the way that ALSA audio is controlled and who
   takes command.  If one sets a specific destination, then \textsl{Yoshimi}
   says what it wants. It's often a negotiation on bit depth and channel count,
   but \textsl{Yoshimi} nearly always gets to decide the buffer size (it will be
   set to the internal buffer size).  However, if the destination is "default",
   then ALSA decides on the sound card, bit depth, number of channels and the
   buffer size, and \textsl{Yoshimi} will set it's internal buffer size to
   match.  On most machines this always seems to be 1024.

\subsection{Concepts / Banks and Roots}
\label{subsec:concepts_banks_and_roots}

   In \textsl{Yoshimi}, a \textsl{root} is a location in which banks can be
   stored.  It is basically a directory, though it ultimately is assigned a
   number by \textsl{Yoshimi}, to be able to access it in an
   automated way.  By choosing a root, one can hone in on a smaller
   collection of banks.

   Sometimes, one will see the term \textsl{path}.
   In \textsl{Yoshimi}, a \textsl{path} is simply the directory location of a
   root.  This change is reflected in the user-interfaces, both graphical and
   command-line.
   Note that there are other file categories, such as presets, that are located
   via paths.
   
   Another important concept in \textsl{Yoshimi} is \textsl{banks}.
   Instruments can be stored in banks. These are loaded and saved
   automatically by the program.  On program start, the last used bank is
   loaded.
   \index{extended program}
   A single bank can store up to 128 instruments normally, and 160
   using extended programs.
   A bank isn't a file... it is a directory, managed by \textsl{Yoshimi},
   which contains instrument (\texttt{.xiz}) files.

   These concepts are discussed in great detail
   in \sectionref{sec:banks_and_roots}.

\subsection{Concepts / Basic Synthesis}
\label{subsec:concepts_basics}

   This section describes some of the basic principles of synthesis,
   and contains suggestions on
   how to make instruments that sound like they have been made with
   professional equipment. This applies to \textsl{Yoshimi} or to any
   synthesizer (even if one wrote it oneself with a few lines of code). All
   the ideas from \textsl{Yoshimi} are derived from from the principles
   outlined below.

\begin{figure}[H]
   \centering 
   \includegraphics[scale=0.65]{zyn/zyn-diagram1.png}
   \caption{ZynAddSubFX/Yoshimi Main Structure}
   \label{fig:zynaddsubfx_main_structure}
\end{figure}

   For a given part, the synthesizer first creates a note.  Each note's
   waveform (for example, in a chord) is summed (mixed).  This complex
   waveform is then send to the series of Insertion effects (if any) that
   are defined.  Each part is then sent to a System effect and (depending on
   the wetness of the mix) directly to a mixer.  Additional Insertion
   effects (if any) are then applied.  The result is the final output of the
   synthesizer.

   The synthesizer has three major types of parameters: 

   \begin{enumber}
      \item \textbf{Master settings/parameters}.
         Contains all parameters (including effects and instruments).
      \item \textbf{Instrument parameters}.
         Contains ADDnote/SUBnote/PADnote parameters for a part.
      \item \textbf{Scale settings}.
         Contains the settings of scales (\textsl{Yoshimi}
         is a micro-tonal synth) and few other parameters related to
         tunings.
   \end{enumber}

\subsubsection{Concepts / Basic Synthesis / Panning}
\label{subsubsec:concepts_basics_panning}

   Pan lets one apply panning, which means that the sound source can move to
   the right or left. Set it to 0.0 to only hear output on the right side, or
   to the maximum value to only hear output on the left side.

\subsubsection{Concepts / Basic Synthesis / Wetness}
\label{subsubsec:concepts_basics_wetness}

   Wetness determines the mix of the results of the effect and its input.
   This mix is made the effects output. If an effect is wet, it means that
   none of the input signal is bypassing the effect. If it is dry, then
   the effect is bypassed completely, and has no effect.

\subsubsection{Concepts / Basic Synthesis / Single Note}
\label{subsubsec:concepts_basics_single_note}

   The idea of this synthesis model is from another synthesizer Paul Nasca
   wrote years ago, released on the Internet as "Paul's Sound
   Designer".  The new model is more advanced than that project
   (adding SUBsynth, more LFO's/Envelopes, etc.), but the idea is
   the same.

\begin{figure}[H]
   \centering 
   \includegraphics[scale=0.4]{zyn/zyn-adnote-diagram2.png}
   \caption{ZynAddSubFX/Yoshimi Note Generation}
   \label{fig:zynaddsubfx_note_generation}
\end{figure}
   
   The figure represents the synthesizer module components. The continuous
   lines are the signal routing, and the dotted lines are frequency
   controlling signals (they control the frequency of something).  The
   dashed lines control the bandwidths of bandpass filters. "Env" are the
   envelopes, "LFO" the Low Frequency Oscillators, "BPF" are band pass
   filters, "bw" are the bandwidth of the BPF's.  If one uses instrument kits,
   the "note out" represents the output of the kit item.

\subsubsection{Concepts / Basic Synthesis / Harmonics}
\label{subsubsec:concepts_basics_harmonics}

   Harmonics are sine waves that are multiple of the base frequency of a
   note.  \textsl{Yoshimi} and \textsl{ZynAddSUbFX} introduce the concept of
   increasing the bandwidth of a harmonic so that it is not quite a sine
   wave.

\paragraph{Harmonic Bandwidth}
\label{paragraph:concepts_basics_harmonic_bandwidth}

   "Harmonic bandwidth" does not refer to sample-rate, it refers to the
   frequency "spread" of each harmonic. This is the most important principle
   of making instruments that sound good. Unfortunately there is very little
   documentation about it.
    
   Often it is believed that the pitched sounds (like piano, organ, choir,
   etc.) for a single note have a frequency, but it's actually
   harmonics and nothing more. Many people try to synthesize a sound using an
   exact frequency plus the harmonics, and observe that the result sounds too
   "artificial".  They might try to modify the harmonic content, add a
   vibrato, tremolo, but even that doesn't sound "warm" enough. The reason is
   that the natural sounds don't produce an exact period; their sounds are
   quasi-periodic.  Please note that not all quasi-periodic sounds are
   "warm" or pleasant.
   (Nasca's discussion of periodic vs. quasi-periodic,
   and the figures he shows, are not included here.)
   Basically, by slightly increasing the bandwidth of a periodic sound, it
   is possible to make it quasi-periodic.

   A very important thing about bandwidth and natural sounds is that the
   bandwidth has to be increased if one increases the frequency of the
   harmonic.  If the fundamental frequency is 440 Hz and the bandwidth is 10
   Hz (that means that the frequencies are spread from 435 to 445 Hz), the
   bandwidth of the second harmonics (880Hz) must be 20 Hz. A simple formula
   to compute the bandwidth of each harmonic if one knows the bandwidth of the
   fundamental frequency is \(BWn = n⋅bw1\), where \texttt{n} is the
   order of the harmonic, \texttt{bw1} is the bandwidth of fundamental
   frequency and \texttt{BWn} is the bandwidth of the n'th harmonic. If one
   does not increase the bandwidth according the frequency, the resulting
   instrument will (usually) sound too 'artificial' or 'ugly'.  There
   are at least three methods of making good sounds with the above
   considerations: 

   \begin{enumber}
      \item \textbf{Detuning}.
      By adding slightly detuned sounds (in \textsl{Yoshimi}
      it is called "ADDsynth"). The idea is not new: it has been used
      for thousands of years in choirs and ensembles. That's why choirs
      sound so beautiful.
      \item \textbf{Noise sculpting}.
      By generating white noise, subtracting all harmonics with band-pass
      filters and adding the results (in \textsl{Yoshimi}
      it is called "SUBsynth").
      \item \textbf{Generation by spectrum}.
      By "drawing" the above graph that represents the frequency
      amplitudes on a large array, put random phases and do a single
      IFFT for the whole sample.
   \end{enumber}

\paragraph{Harmonic Amplitude}
\label{paragraph:concepts_basics_harmonic_amplitude}

   An important principle of natural harmonics is to decrease the amplitude
   of higher harmonics on low-velocity notes.

   All natural notes have this property, because on low-velocity notes there
   is not enough energy to spread to higher harmonics. On artificial
   synthesis one can do this by using a low-pass filter that lowers the
   cutoff frequency on notes with low velocities or, if one uses FM
   (frequency modulation), by lowering the modulator index. 
   The spectrum of the sound should be almost the same according to
   the frequencies and not the harmonics.

   This means that, for example, the higher the pitch is, the smaller the
   number of harmonics it will contain. This happens in a natural instrument
   because of the resonance. 
   In this case there are many instruments that don't obey this, but sound
   quite good (example: synth organ). 
   If one records the C-2 note from a piano and one plays it at a very high
   speed (8 times), the result will not sound like the C-5 key from the
   piano. The pitch is C-5, but the timbre is very different. This is because
   the harmonic content is preserved (the n-th harmonic will have the
   same amplitude in both cases) and not the spectrum (eg. the
   amplitudes of the harmonics around 1000 Hz are too different from
   one case to another). 

   In artificial synthesis one can use filters to add resonance or FM
   synthesis that varies the index according to the frequency.  In
   \textsl{Yoshimi} one can add the resonance:

   \begin{enumber}
      \item \textbf{ADDsynth}:
      Use the Resonance, a high harmonics sound content, and filters or FM.
      \item \textbf{SUBsynth}:
      Add some harmonics and use the Global Filter.
   \end{enumber}

\subsubsection{Concepts / Basic Synthesis / Randomness}
\label{subsubsec:concepts_basics_randomness}

   The main reason why the digital synthesis sounds too "cold" is because the
   same recorded sample is played over and over on each key-press.  There is
   no difference between a note played the first time and second time.
   Exceptions may be the filtering and some effects, but these are not
   enough. In natural or analog instruments this doesn't happen because
   it is impossible to reproduce exactly the same conditions for each
   note. To make a warm instrument one must make sure that it sounds
   slightly different each time. In \textsl{Yoshimi}
   one can do this:

   \begin{enumber}
      \item \textbf{ADDsynth}:
      Set the "Randomness" function from Oscillator Editor
      to a value different than 0, or change the start phase of the LFO to
      the leftmost value.
      \item \textbf{SUBsynth}:
      All notes already have randomness because the
      starting sound is white noise.
      \item \textbf{PADSynth}:
      The engine starts the sample from random positions
      on each keystroke.
   \end{enumber}

   In setting the randomness of the oscillator output, there are two types of
   randomness. The first is \textsl{group randomness}, where the oscillator
   starts at a random position. The second is \textsl{phase randomness}:
   from -64 (max) to -1 (min) and each harmonic (the oscillator is phase
   distorted) is from 1 (min) to 63 (max). 0 is no randomness. One
   could use this parameter for making warm sounds like analog
   synthesizers.

   See the ADDSynth oscillator editor,
   \sectionref{subsubsec:addsynth_voice_parameters_oscillator},
   for this kind of control, named \textbf{Ph.rnd} or \textbf{rnd}.

   There is now the possibility to add a 'naturalising' random pitch element
   to a part. This is found in the part-edit window. The settings are not
   currently saved, but will be once the control values are settled, and
   there has been enough experience to decide whether it should be a part or
   voice setting.  (In the newer versions of \textsl{Yoshimi}, see the
   \textbf{Humanise} setting in the part-edit window.

\subsubsection{Concepts / Basic Synthesis / Components}
\label{subsubsec:concepts_basics_components}

   \textsl{Important}:
   All indexes of MIDI Channels, Parts, Effects starts from 0, so, for
   example, the first Part is 0.  However, in other discussions of MIDI,
   part numbers, programs, or channels are often described as starting from 1.

   \textsl{Yoshimi} components:

   \begin{enumber}
      \item \textbf{Parts}.
         They receive the note messages from MIDI
         Channels. One may assign a part to any channel. A part can store
         only one instrument.  "Add.S" represents ADDsynth and "Sub.S" is
         SUBsynth.  In recent versions of \textsl{Yoshimi}, the number of
         parts available has been increased from 16 to 64.
      \item \textbf{Insertion Effect}.
         This effect applies only to one part; one can have any number of
         insertion effects for one part, but the number of these cannot be
         bigger than NUM.INS.EFX.
      \item \textbf{Part Mixer}.
         Mixes all parts.
      \item \textbf{System Effects}.
         Applied to all parts, one can set how much signal
         is routed through a system effect.
      \item \textbf{Master mixer}.
         Mixes all outputs of Parts Mixers and System Effects.
   \end{enumber}

\subsubsection{Concepts / Basic Synthesis / Filters}
\label{subsubsec:concepts_basics_filters}

   \textsl{Yoshimi}
   offers several different types of filters, which can be used
   to shape the spectrum of a signal. The primary parameters that affect the
   characteristics of the filter are the cutoff, resonance, filter stages,
   and the filter type.

   \textbf{Cutoff}:
   This value determines which frequency marks the changing point for
   the filter. In a low pass filter, this value marks the point where higher
   frequencies are attenuated.

   \textbf{Resonance}:
   The resonance of a filter determines how much excess energy is
   present at the cutoff frequency. In \textsl{Yoshimi},
   this is represented by the Q-factor, which is defined to be the cutoff
   frequency divided by the bandwidth. In other words higher Q values result
   in a much more narrow resonant spike.

   \textbf{Stages}:
   The number of stages in a given filter describes how sharply it is
   able to make changes in the frequency response.
   The affect of the order of the filter is roughly synonymous with the
   number of stages of the filter. For more complex patches it is important
   to realize that the extra sharpness in the filter does not come for free
   as it requires many more calculations being performed. This phenomena is
   the most visible in SUBsynth, where it is easy to need several hundred
   filter stages to produce a given note.

   The \textbf{Q}:
   value of a filter affects how concentrated the signal’s energy is at
   the cutoff frequency.
   For many classical analog sounds, high Q values were used on sweeping
   filters. A simple high Q low pass filter modulated by a strong envelope is
   usually sufficient to get a good sound.

   \textbf{Filter Type}:
   There are different types of filters. The number of poles define what will
   happen at a given frequency. Mathematically, the filters are functions
   which have poles that correspond to that frequency. Usually, two poles
   mean that the function has more "steepness", and that one can set the
   exact value of the function at the poles by defining the "resonance
   value". Filters with two poles are also often referenced as Butterworth
   filters.

   For the interested reader, functions having \textsl{poles}
   means that we are given a quotient of polynomials. The denominator has
   degree 1 or 2, depending on the filter having one or two poles. In the
   file \texttt{DSP/AnalogFilter.cpp},
   \texttt{AnalogFilter::computefiltercoefs()} sets the coefficients
   (depending on the filter type), and
   \texttt{AnalogFilter::singlefilterout()} shows the whole polynomial (in a
   formula where no quotient is needed).

   Filters are thoroughly described in
   \sectionref{subsec:filter_settings}.

\subsubsection{Concepts / Basic Synthesis / Envelopes}
\label{subsubsec:concepts_basics_envelopes}

   Envelopes are long-period wave forms that are applied to frequency,
   amplitude, or filters.  Envelopes generate effects such as tremolo and
   vibrato, as well as effects that occur when a sound-generating physical
   component changes shape.
   Envelopes are thoroughly described in
   \sectionref{subsec:envelope_settings}.

\subsection{Concepts / MIDI}
\label{subsec:concepts_midi}

   It is useful to discuss some of the details of MIDI in order
   to understand \textsl{Yoshimi}.  Obviously, we assume
   some knowledge already, or one wouldn't be running
   \textsl{Yoshimi}.

\subsubsection{Concepts / MIDI / Messages}
\label{subsubsec:concepts_midi_messages}

   \textsl{Yoshimi} responds to the following MIDI controller messages
   (\ref{table:zynaddsubfx_midi_messages}).
   This list seems to be more extensive than that presented in the
   online \textsl{ZynAddSubFX} documentation.

   \begin{table}
      \centering
      \caption{ZynAddSubFX/Yoshimi MIDI Messages}
      \label{table:zynaddsubfx_midi_messages}
      \begin{tabular}{r l}
         \textbf{0} or \textbf{32} &
            Bank Change (user selectable, does \textsl{not} force a program
            change) \\
         \textbf{1} & Modulation Wheel \\
         \textbf{2} & Breath Control \\
         \textbf{6} & Data MSB \\
         \textbf{7} & Volume \\
         \textbf{10} & Panning \\
         \textbf{11} & Expression \\
         \textbf{38} & Data LSB \\
         \textbf{64} & Sustain pedal \\
         \textbf{65} & Portamento \\
         \textbf{71} & Filter Q (Sound Timbre) \\
         \textbf{74} & Filter Cutoff (Brightness) \\
         \textbf{75} & BandWidth (different from GM spec) \\
         \textbf{76} & FM amplitude (different from GM spec) \\
         \textbf{77} & Resonance Center Frequency (different from GM spec) \\
         \textbf{78} & Resonance Bandwith (different from GM spec) \\
         \textbf{96} & Data Increment \\
         \textbf{97} & Data Decrement \\
         \textbf{98} & NRPN LSB \\
         \textbf{99} & NRPN MSB \\
         \textbf{120} & All Sounds OFF \\
         \textbf{121} & Reset All Controllers \\
         \textbf{123} & All Notes OFF \\
         \textbf{192} & Program Change (voices 1-128) \\
         \textbf{224} & Pitch Bend \\
      \end{tabular}
   \end{table}

   For the controllers (numbers 75 to 78) that are not defined in GM:

   \begin{itemize}
      \item \textbf{Bandwidth} control (75) increases or decreases the bandwidth
      of instruments. The default value of this parameter is 64. 
      \item \textbf{Modulation amplitude} (76) decreases the amplitude of
      modulators on ADDsynth. The default value of this parameter is 127. 
      \item \textbf{Resonance Center Frequency} control (77) changes the center
      frequency of the resonance. 
      \item \textbf{Resonance Bandwidth} control (78) changes the bandwidth of the
      resonance. 
   \end{itemize}

   Program Change (192) also provides user selectable CC for voices
   128-160.  There is now an option to make Program Change enable a part if
   it's currently disabled.

   Key pressure (aftertouch) is internally translated as CC 900.

   Channel Pressure is internally translated as CC 901.

   Pitch Bend is internally translated as CC 1000.

   The modulation wheel only affects AddSynth and PadSynth, and then only the
   frequency LFO depth. Just to make it more confusing, it changes the level
   from 0 up to it's current (GUI) setting only. Therefore, if the LFO depth is
   set to zero, the Mod Wheel will have no effect.

   User selectable CC for Bank Root Path change.
   For more details of bank changes see
   \sectionref{subsec:concepts_banks_and_roots}.

   Instruments inside banks should \textsl{always} have file-names that
   begin with four digits,  followed
   by a hyphen. Otherwise the results can be rather unpredictable.

\subsubsection{Concepts / MIDI / NRPN}
\label{subsubsec:concepts_midi_nrpn}

   NRPN stands for "Non Registered Parameters Number".
   NRPNs can control all System and Insertion effect parameters.
   Using NRPNs, \textsl{Yoshimi} can now directly set some part values
   regardless of what channel that part is connected to.  For example, one
   may change the reverb time when playing to keyboard, or
   change the flanger's LFO frequency.

   NRPNs are described in greater detail in section
   \sectionref{sec:nrpns}.

\paragraph{Concepts / MIDI / NRPN / Vector Control}
\label{paragraph:concepts_midi_nrpn_vector_control}

   Vector control is a way to control more than one part with the
   controllers.
   Vector control is only possible if one has 32 or 64 parts active 
   in \textsl{Yoshimi}.
   In vector mode, parts will still play together but the vector controls can
   change their volume, pan, filter cutoff in pairs, controlled by user
   defined CCs set up with NRPNs.

   Vector control is described in greater detail in section
   \sectionref{subsection:vector_control}.

\paragraph{Concepts / MIDI / NRPN / Effects Control}
\label{paragraph:concepts_midi_nrpn_effects_control}

   NRPNs are very useful in modifying the parameters of the
   \textsl{Yoshimi} effects.

   Effects control is described in greater detail in section
   \sectionref{subsection:nrpns_midi_nrpn_effects_control}.

\subsection{Concepts / Command Line}
\label{subsec:concepts_command_line}

   This section covers a few terms useful in discussing the command line.

\subsubsection{Concepts / Command Line / level}
\label{subsubsec:concepts_command_line_level}

   The term \textbf{level} is used in the description of the new command-line
   facility of \textsl{Yoshimi}.
   A \textbf{level} is simply a related group of parameters or a location where
   one can go to for making changes.
   Important levels are:  the top level; system effects; part; and more.

\subsection{Concepts / LV2 Plugin}
\label{subsec:concepts_lv2_plugin}

   \textsl{Yoshimi} now runs as an LV2 plugin.

   TODO: Describe LV2 at a high-level.

%-------------------------------------------------------------------------------
% vim: ts=3 sw=3 et ft=tex
%-------------------------------------------------------------------------------


% Building and debugging Yoshimi

%-------------------------------------------------------------------------------
% yum_build
%-------------------------------------------------------------------------------
%
% \file        yum_build.tex
% \library     Documents
% \author      Chris Ahlstrom
% \date        2015-07-10
% \update      2017-03-30
% \version     $Revision$
% \license     $XPC_GPL_LICENSE$
%
%     Provides the man page section of the manual.
%
%-------------------------------------------------------------------------------

\section{Building Yoshimi}
\label{sec:yoshimi_build}

   This section describes building and debugging \textsl{Yoshimi}.
   Building \textsl{Yoshimi} requires getting the source code, making sure
   all of the necessary dependencies are installed, and using CMake to set
   up the build.

   The source-code is located at its main location (\cite{yoshimi})
   or its alternate location (\cite{yoshimi2}).

   \textsl{Yoshimi} uses CMake as its build
   system \cite{zyncmake}.  CMake is a preprocessor that can generate
   project build setups for Visual Studio, UNIX make, and Xcode.

\subsection{Yoshimi Source Code}
\label{subsec:yoshimi_source_code}

   Get the source code version you want from SourceForge
   (\url{http://sourceforge.net/projects/yoshimi/files/2.2/}).
   Download the desired tar-ball and unpack it in your work area.

   Since SourceForge has had some issues, the \textsl{Yoshimi} team
   has wisely hosted the source code at another site as well,
   \url{https://github.com/abrolag/yoshimi}.  One can grab the whole
   git repository there using the following command in your work area:

   \begin{verbatim}
      $ git clone https://github.com/abrolag/yoshimi.git
   \end{verbatim}

   Please note that this code base comes with two files,
   \texttt{INSTALL} and \texttt{INSTALL\_CUSTOM} which elaborate on
   build options well beyond what is discussed here.

\subsection{Yoshimi Dependencies}
\label{subsec:yoshimi_dependencies}

   For all versions since \textsl{Yoshimi} 1.3.9, building \textsl{Yoshimi}
   requires C++11.  For GNU builds, this requires \textsl{gcc 4.7} and above.

   To save some wasted time, make sure the \textsl{development versions}
   of the following packages have been installed using your Linux
   distribution's package manager:

   \begin{itemize}
      \item \texttt{pkg-config}
      \item \texttt{libz}
      \item \texttt{fftw3f}
      \item \texttt{mxml}
      \item \texttt{ALSA} (libasound)
      \item \texttt{JACK}
      \item \texttt{fontconfig}
      \item \texttt{libcairo}
      \item \texttt{FLTK}
      \item \texttt{lv2}
   \end{itemize}

   These package names are from \textsl{Debian Jessie}:

   \begin{itemize}
      \item \texttt{automake}
      \item \texttt{build-essential}
      \item \texttt{cmake-curses-gui}
      \item \texttt{dssi-dev}
      \item \texttt{fluid}
      \item \texttt{libboost-dev}
      \item \texttt{libcairo2-dev}
      \item \texttt{libfftw3-dev}
      \item \texttt{libfltk1.3-dev}
      \item \texttt{libglu1-mesa-dev}
      \item \texttt{libjack-jackd2-dev}
      \item \texttt{libjpeg-dev}
      \item \texttt{libmxml-dev}
      \item \texttt{libncurses5-dev} (new dependency)
      \item \texttt{libreadline-dev} (new dependency)
      \item \texttt{libxft-dev}
      \item \texttt{libxinerama-dev}
      \item \texttt{libxml2-dev}
      \item \texttt{xutils-dev}
      \item \texttt{zlib1g.dev}
   \end{itemize}

	LV2 plugin adds one more dependency:

		\texttt{lv2-dev} with version 1.0.0 or greater.

   Other distros may have slightly different names or version numbers, and may
   even have these installed by default. If in doubt, try looking for just the
   main part of the name, but with the \texttt{-dev} extension where
   appropriate.

\subsection{Build It}
\label{subsec:yoshimi_build_it}

   The following instructions are for an in-source build.  An in-source
   build is simpler if you just want to build and install \textsl{Yoshimi}.

   We will also show how to set up for an out-source-build, which keeps
   the build products out of the way.

   The location of \texttt{CMakeList.txt} does not appear to be standard.
   Basically, the build is based in the project's
   \texttt{src} directory, instead of its root directory.
   And it is recommended to use an out-of-source build.

   \begin{enumber}
      \item Enter the source directory where the code was unpacked.
      \item Generate the project build-files:
            \begin{verbatim}
               $ mkdir build
               $ cd build
               $ cmake ..
               $ cmake .
            \end{verbatim}
      \item Build the code and install it (as root):
            \begin{verbatim}
               $ make
               # make install
            \end{verbatim}
   \end{enumber}

   Here is how to make an out-of-source debug build.  Despite what
   cmake documentation (and Googling) says, using a command like the
   following \textsl{does not work} unless you have \texttt{ccmake}
   installed.

   \begin{verbatim}
      $ cmake -DCMAKE_BUILD_TYPE=Debug ..
      $ ccmake
   \end{verbatim}

   In Debian Linux, install the \texttt{cmake-curses-gui}
   package to get access to \texttt{ccmake}.  Or use the shorter
   \texttt{cmake -DBuildForDebug=on ..} command below.

   \begin{enumber}
      \item Enter the source directory where the code was unpacked.
      \item Create a "Debug" or "Release" directory for an
            out-of-source build:
            \begin{verbatim}
               $ cd src
               $ mkdir Debug
            \end{verbatim}
      \item Generate the project build-files in the \texttt{Debug}
            directory.
            \begin{verbatim}
               $ cd Debug
               $ cmake -DBuildForDebug=on ..
               $ make
            \end{verbatim}
   \end{enumber}

   The output file, and executable name \textbf{\texttt{yoshimi}}
   is now ready to run (and be debugged).

   Here is a debugging use case we used in \textsl{Yoshimi 1.3.5.1} and
   slightly earlier versions.  Here is how to verify the bug:

   \begin{enumber}
      \item Run the following command:
         \begin{verbatim}
            $ yoshimi -a -A
         \end{verbatim}
      \item Navigate the following command path:
            Menu / Instrument / Show Banks
      \item Select the \textbf{RENAME} button.
      \item Select the bank (e.g. Arpeggios).
      \item In the file prompt that comes up, click \textbf{Cancel}.
      \item Observe a "Segmentation fault".
   \end{enumber}

   To avoid a lot of debugging, let \textsl{valgrind} find the bug for you.
   Install \textsl{valgrind}.  Then, in the \texttt{src/Debug} directory,
   run:

   \begin{verbatim}
      $ valgrind --log-file=yoshvalgrind.log ./yoshimi -a -A
   \end{verbatim}

   In the log file, one sees that the last good call was in the
   \texttt{Bank :: readOnlyBank()} function.  That would be a good place to
   put a breakpoint.

   However, even without \textsl{valgrind}, this particular bug is easy to
   find under the \textsl{debugger}.  The steps are simple:

   \begin{verbatim}
      $ cd src/Debug
      $ gdb ./yoshimi
      (gdb) r -a -A
   \end{verbatim}

   Then repeat the steps above that trigger the bug.
   One sees

   \begin{verbatim}
      Program received signal SIGSEGV, Segmentation fault.
   \end{verbatim}

   Issue the command "backtrace" at the \texttt{(gdb)} prompt.  There will
   be a list of stack frames starting at 0.  Frame 1 is in \textsl{Yoshimi},
   so issue the command "frame 1", and you'll see:

   \begin{verbatim}
      if (strlen(tmp) > 2) ...
   \end{verbatim}

   \texttt{tmp} is a null pointer here; we need to add an initial check for
   the null pointer there to avoid triggering the crash.

   Not worry now though, the bug has long since been officially fixed.

\subsection{Yoshimi Code Policies}
\label{subsec:yoshimi_code_policies}

   Yes, we actually have \textsl{Yoshimi} code policies.
   Look how many there are! :)

   If the version string contains a 4th number this will always be just a
   bugfix, and will never have features added or changed from the main version.
   For example:

   \begin{itemize}
      \item yoshimi-1.3.5   Main version.
      \item yoshimi-1.3.5.1 First bugfix.
      \item yoshimi-1.3.5.2 Second bugfix. (Surely not!)
   \end{itemize}

   We won't accept fixes for spelling errors in the \textsl{code}.
   For a start, from bitter experience it is fatally easy to change two
   variables to the same name! Also, there's no point, after all they are only
   a mnemonic for memory addresses etc. 'volume' and 'LFO' could just as well
   be 'turnyfing' and 'derfingwotwiggles'.

   To avoid possible confusion, from now on 'master' will display the last
   released version number (without bugfix digits) with an 'M' suffix - unless
   it is a release candidate in which case the suffix will be rc[n].
   For example:

   \begin{itemize}
      \item Last release was yoshimi-1.3.5.2
      \item Master is shown as yoshimi-1.3.5 M
   \end{itemize}

   XML files created with this release will have:
   major version 3 and minor version 5.

   If using Fluid to edit GUI files, please close all windows and collapse all
   menus \textsl{before} the last save.  I know it's tedious, but it avoids
   storms of spurious 'changes' that make genuine ones harder to see.

%-------------------------------------------------------------------------------
% vim: ts=3 sw=3 et ft=tex
%-------------------------------------------------------------------------------


% Resource usage notes, disabled for now.
%
% %-------------------------------------------------------------------------------
% yum_resource_usage
%-------------------------------------------------------------------------------
%
% \file        yum_resource_usage.tex
% \library     Documents
% \author      Chris Ahlstrom
% \date        2017-09-24
% \update      2018-03-17
% \version     $Revision$
% \license     $XPC_GPL_LICENSE$
%
%     Provides a very incomplete description and discusson of Yoshimi
%     resource usage.
%
%-------------------------------------------------------------------------------

\section{Resource Usage}
\label{sec:resource_usage}

   This section discusses the resource usage (and hence, efficiency) of
   \textsl{Yoshimi}.

   In the meantime, we include some notes.

\subsection{Resource Usage / JACK Buffer}
\label{sec:resource_usage_jack_buffer}

   If \textsl{Yoshimi}'s internal buffer is \textsl{smaller} than the JACK
   buffer, it sounds quite horrible. If it's the same or greater there's no
   problem. The reason we didn't find this before is that it only affects
   \textsl{Yoshimi} resource_usage.  Also it doesn't apply to any of the
   released versions.  The cause of the problem is that the resource_usage code
   doesn't have the same looping structure that was added to the standalone
   routine to deal with exactly this situation (re-entering the audio
   'construction' function until the JACK buffer is filled). We hope Andrew can
   deal with this fairly soon as we don't understand the resource_usage code
   very well.  There are valid reasons for wanting different sizes for these
   buffers. The internal buffer size as well as affecting latency and CPU load
   also alters the sound in quite subtle ways. It particularly affects the
   behaviour of filters.  One day we may be able to stop this happening, but in
   the mean time we have to live with it.  For our purposes, we find a buffer
   size of 128 or 256 is best.

   Anecdote from Will:

   To get some idea of how memory was being allocated I loaded up 16 parts all
   with 'Ghost Ensemble', then using Rosegarden created a file using all 16
   parts at 5 notes each, starting the notes one at a time, then keeping all of
   them on for about half a minute. This is the maximum number of notes at any
   one time that Yoshimi supports.

   The first thing I found was that I had to go up to 256 frames per period to
   run this (I normally run at 32 these days). That wasn't really a surprise,
   what \textsl{was} a surprise was a rather crude estimation of the memory
   usage.

   Going from an empty \textsl{Yoshimi} to this 16 part monster stole an
   astonishing 100M. Now to be fair, this patch adds all three engines, so
   that's 16 lots of PADsynth samples along with 16 ADDsynth oscillators.

   Actually playing these 80 notes only took another 2M. Now this difference
   was quite consistent, even over reboots, although when the notes stop only
   1M is given back. This isn't a memory leak as playing again only re-grabs
   another 1M, so I'm guessing that the underlying OS memory management isn't
   too fussy about garbage collection until it really needs it.

   Now if we multiply that by 4 that's 8M for the default max for each part,
   which is quite a lot (and much more than the current overall limit), but
   probably quite do-able on all modern machines.

\subsection{Resource Usage / JACK and ALSA Buffer Sizes}
\label{sec:resource_usage_buffer_sizes}

   This section elaborates on
   \sectionref{paragraph:menu_yoshimi_settings_main_settings}.

   If something in the JACK connection
   graph is slow enough to need a large buffer
   size, but is not directly related to \textsl{Yoshimi},
   then having a smaller internal
   buffer size will enable \textsl{Yoshimi}
   to interlace incoming MIDI messages more accurately.

   If working ALSA MIDI with JACK Audio, a smaller buffer size will allow
   incoming MIDI to be better placed; MIDI is then buffered and it will read
   the buffer every time it loops round. This is at the cost of CPU
   availability.

   If working JACK MIDI with ALSA audio? There's no benefit having the
   two buffer sizes different.

   If entirely ALSA, \textsl{Yoshimi} \textsl{defines}
   the buffer size, and hence the latency.
   However, there is a quirk... if one accidentally sets all ALSA
   (i.e. forgets to start JACK),
   then ALSA sets the buffer size, usually a massive 1024 frames.

   In an LV2 plug-in, internal buffer size is ignored.

%-------------------------------------------------------------------------------
% vim: ts=3 sw=3 et ft=tex
%-------------------------------------------------------------------------------


\section{Summary}
\label{sec:summary}

   In summary, we can say that you will absolutely love \textsl{Yoshimi}.
%  There are some topics that this document does not yet treat:

%  \begin{itemize}
%     \item Changing the colors and style of the GUI, as seen in some
%        versions of \textsl{ZynAddSubFX}.  This does not yet seem possible,
%        because the synth part is not truly separate from the GUI, and the
%        \textsl{Yoshimi} developers have not found it easy to recode the
%        FLTK GUI.
%  \end{itemize}

   \begin{quotation}
      A car analogy:
      A sample player is a drive along a straight, wide, almost new highway
      with only 2 other cars in sight, on a lightly overcast summer's day in a
      Ford Fiesta at around 40 MPH.
      Yoshimi is a white-knuckle trip over a Swiss mountain pass in a blizzard,
      at night, facing donkeys, trucks and bandits, while driving an open-frame
      kit car doing 90 +
      In recent times we've been able to dispose of the donkeys, and the
      bandits are on the run :)
   \end{quotation}

% References

%-------------------------------------------------------------------------------
% yum_references
%-------------------------------------------------------------------------------
%
% \file        yum_references.tex
% \library     Documents
% \author      Chris Ahlstrom
% \date        2015-06-02
% \update      2017-09-24
% \version     $Revision$
% \license     $XPC_GPL_LICENSE$
%
%     Provides the References section of yoshimi-user-manual.tex.  Rather
%     than use the bibtex package, our small set of references uses a
%     simpler method.
%
%-------------------------------------------------------------------------------

\section{References}
\label{sec:yumreferences}

   This section provides references for this \textsl{Yoshimi} user's manual, as
   well as some other information.
   
   Although the \textsl{Yoshimi} project is based on SourceForge, it also has a
   mirror on GitHub, and a mailing list on FreeLists.Org.
   One can subscribe with an e-mail to:
   \url{yoshimi-request@freelists.org}
   or by visiting: \url{http://www.freelists.org/list/yoshimi}.
   To post to the list, email to: \url{yoshimi@freelists.org}
   The archive of the old SourceForge mailing list is found
   at: \url{https://www.freelists.org/archive/yoshimi}.

% \subsection{Web Sites}
% \label{subsec:references_web_sites}
%
% \iffalse
%  \begin{enumber}
%     \item \href{http://yoshimi.sourceforge.net/}{Yoshimi Download Site}
%        \url{http://yoshimi.sourceforge.net/}
%     \item \href{http://rakkarrak/}{Rakkarrak Guitar Effects}
%  \end{enumber}
% \fi

% If we wrap the URLs in \url{}, then we get web-color errors on CentOS!

\begin{thebibliography}{99}

   \bibitem{bankrootupgrades}
   Will J. Godfrey
   \emph{A discussion of making Bank/Root specifications more regular.}
   \url{http://sourceforge.net/p/yoshimi/mailman/message/33200765/}
   2014.

   \bibitem{book}
   Chris Ahlstrom
   \emph{A Yoshimi User's Manual}
   \url{https://github.com/ahlstromcj/yoshimi-doc/}
   2015.

   \bibitem{cookbook}
   Chris Ahlstrom
   \emph{A Yoshimi Cookbook}
   \url{https://github.com/ahlstromcj/yoshimi-cookbook/}
   2015.

   \bibitem{cookbookeq}
   Robert Bristow-Johnson.
   \emph{Cookbook formula for audio EQ biquad filter coefficients}
   \url{http://www.musicdsp.org/files/Audio-EQ-Cookbook.txt}
   2005.  The "files" directory has a number interesting files as wel..

   \bibitem{cormi}
   Cormi
   \emph{Cormi Collection}
   \url{https://www.freesound.org/people/cormi/}
   His cormi57.wordpress.com site has a lot of other nice sound material, as
   well.
   2015.

   \bibitem{drumsds}
   Straulino
   \emph{A collection of instruments}
   \url{http://www.straulino.ch/zynaddsubfx/}
   \url{http://www.straulino.ch/zynaddsubfx/Drums\_DS\_v2012-12-08.zip}
   \url{http://www.straulino.ch/zynaddsubfx/ZynAddSubFX\_Natural\_Drum\_Kit\_Demo\_v2012-06-22.ogg}

   \bibitem{folderol}
   "folderol"
   \emph{A collection of instruments}
   \url{http://www.kara-moon.com/forum/index.php?topic=762.0}

   \bibitem{godfrey}
   Will Godfrey
   \emph{Will Godfrey's Music}
   \url{http://www.musically.me.uk}
   See the "Bits 'n Stuff" link especially.

   \bibitem{gpl2vsgpl3}
   Will J. Godfrey
   \emph{A discussion of licensing issues with Youshimi and ZynAddSubFX
   components, and FLTK library versions.}
   \url{http://sourceforge.net/p/yoshimi/mailman/yoshimi-devel/}
   2015.

   \bibitem{kvraudio}
   caonoize
   \emph{ZynAddSubFX by paulnasca: Downloads, Banks, Patches, etc.}
   \url{http://www.kvraudio.com/product/zynaddsubfx\_by\_paulnasca/downloads}
   2015.

   \bibitem{mmxgn}
   mmxgn
   \emph{Instruments made with zynaddsubfx/yoshimi}
   \url{https://github.com/mmxgn/instruments-zyn}
   2011.

   \bibitem{rakarrack}
   Rakarrack team
   \emph{The download site for the Rakarrack software effects engine.}
   \url{http://rakarrack.sourceforge.net/}
   2015.

   \bibitem{ringmodulator}
   LinuxMusicians newsgroup
   \emph{Ring Modulation in ZynAddSubFX}
   \url{http://linuxmusicians.com/viewtopic.php?f=1&t=8178}
   2012.

   \bibitem{scala}
   Manuel Op de Coul <\url{coul@huygens-fokker.org}>
   \emph{The Scala Musical Tuning Application.}
   \url{http://www.huygens-fokker.org/scala/}
   Scala is a powerful software tool for experimentation with musical
   tunings, such as just intonation scales, equal and historical
   temperaments, microtonal and macrotonal scales, and non-Western scales.
   2014.

   \bibitem{sharphall}
   Sharphall
   \emph{How to create drum sounds in ZynAddSubFX or Yoshimi, Part 1}
   \url{http://sharphall.org/docs/zynaddsubfx\_yoshimi\_drum\_tutorial.php}
   Never got continued, unfortunately.

   \bibitem{synthsecrets}
   Gordon Reid
   \emph{Synth Secrets:  Creative Synthesis}
   \url{http://www.soundonsound.com/sos/allsynthsecrets.htm}
   1999-2004.

   \bibitem{x31eq}
   Graham
   \emph{A small musical website}
   \url{http://x31eq.com} and specifically
   \url{http://x31eq.com/zasf}

   \bibitem{yoshimi}
   Yoshimi team \url{abrolag@users.sourceforge.net}
   \emph{The download site for the Yoshimi software synthesizer.}
   \url{http://yoshimi.sourceforge.net/}
   2015.

   \bibitem{yoshimi2}
   Yoshimi team
   \emph{The alternate location for the Yoshimi source-code.}
   \url{https://github.com/abrolag/yoshimi now.}
   2015.

   \bibitem{yoshiminews}
   Yoshimi team.
   \emph{Yoshimi user/developer mailing list}
   \url{http://www.freelists.org/list/yoshimi}
   2015.

   \bibitem{yoshiminewsarchive}
   Yoshimi team.
   \emph{Yoshimi user/developer mailing list archive}
   \url{http://www.freelists.org/archive/yoshimi}
   2015.

   \bibitem{zynaddsubfx}
   Mark McCurry, Paul Nasca (ZynAddSubFX team)
   \emph{The download site for the ZynAddSubFX software synthesizer.}
   \url{http://zynaddsubfx.sourceforge.net/}
   2015.

   \bibitem{zynbanks2009}
   Paul Nasca?
   \emph{The download site for the ZynAddSubFX banks, instruments,
      parameters, and demos}
   \url{http://zynaddsubfx.sourceforge.net/doc/instruments/}
   2009.

   \bibitem{zyncmake}
   ZynAddSubFX team.
   \emph{Building ZynAddSubFX}
   \url{http://zynaddsubfx.sourceforge.net/Doc/#_appendix_b_building_zynaddsubfx}
   2009.

   \bibitem{zyndemos}
   AMSynth team.
   \emph{Provides a number of OGG demos of ZynAddSubFX sounds. It
   also includes the author's own "Vanilla" bank, and links to additional
   patch collections and demonstration videos.}
   \url{http://www.amsynth.com/zynaddsubfx.html}

   \bibitem{zyndoc}
   Mark McCurry, Paul Nasca
   \emph{ZynAddSubFX online manual.}
   \url{http://zynaddsubfx.sourceforge.net/Doc}
   2015.

   \bibitem{zynodt}
   Paul Nasca
   \emph{Original ZynAddSubFX manual, ODT format.}
   \url{http://linux.autostatic.com/docs/zynaddsubfx\_manual-v0.1.odt}
   2011.

   \bibitem{zynpdf}
   Paul Nasca
   \emph{Original ZynAddSubFX manual, PDF format.}
   \url{http://linux.autostatic.com/docs/zynaddsubfx\_manual-v0.1.pdf}
   2011.

   \bibitem{zynsource}
   Paul Nasca, Mark Murray
   \emph{The download package for the ZynAddSubFX source-code}
   \url{http://downloads.sourceforge.net/project/zynaddsubfx/zynaddsubfx/2.5.0/zynaddsubfx-2.5.0.tar.gz}

   \bibitem{zynwiki}
   Jeremy ("autostatic")
   \emph{ZynAddSubFX Manual, good overall of most Zyn settings and knobs}
   \url{http://wiki.linuxaudio.org/wiki/zynaddsubfx_manual}
   2011.

\end{thebibliography}

%-------------------------------------------------------------------------------
% vim: ts=3 sw=3 et ft=tex
%-------------------------------------------------------------------------------


\printindex

\end{document}

%-------------------------------------------------------------------------------
% vim: ts=3 sw=3 et ft=tex
%-------------------------------------------------------------------------------
