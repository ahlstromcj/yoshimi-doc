%-------------------------------------------------------------------------------
% yum_midi_learn
%-------------------------------------------------------------------------------
%
% \file        yum_midi_learn.tex
% \library     Documents
% \author      Chris Ahlstrom
% \date        2016-12-22
% \update      2016-12-22
% \version     $Revision$
% \license     $XPC_GPL_LICENSE$
%
%     Provides the midi_learn section of yoshimi-user-manual.tex.
%
%-------------------------------------------------------------------------------

\section{MIDI Learn}
\label{sec:midi_learn}

   In this section, we show how to use the new (with 1.5.0)
   \textbf{MIDI Learn} feature of \textsl{Yoshimi}.

\subsection{MIDI Learn / Basics}
\label{subsec:midi_learn_basics}

   TODO

\subsection{MIDI Learn / User Interface}
\label{subsec:midi_learn_user_interface}

   \textsl{Yoshimi} ...

   To activate MIDI Learn, \texttt{Ctrl-right-click}
   on any user interface control.
   A pop-up window will detail the control selected.

\begin{figure}[H]
   \centering 
   \includegraphics[scale=0.75]{1.5.0/Learning-30amp.png}
   \caption{MIDI Learn Prompt Example 1}
   \label{fig:midi_learn_ex_1}
\end{figure}

\begin{figure}[H]
   \centering 
   \includegraphics[scale=0.75]{1.5.0/Learning-Mods.png}
   \caption{MIDI Learn Prompt Example 2}
   \label{fig:midi_learn_ex_2}
\end{figure}

   If a \texttt{Yoshimi} control is not MIDI-learnable, a message pop-up
   will indicate that it is not learnable:

\begin{figure}[H]
   \centering 
   \includegraphics[scale=0.75]{1.5.0/Learning-No.png}
   \caption{MIDI Learn Prompt Unsupported Example}
   \label{fig:midi_learn_unsupported}
\end{figure}

   A message will also appear in the console window/CLI.

   After turning on learn, the first physical controller moved, or CC message sent
   will be locked in, and one will see the user-interface knob or slider move in
   synchrony with the physical control. The pop-up window will disappear, and the
   console message \texttt{Learned} appears,
   with a line underneath with exactly what control was caught.

\begin{figure}[H]
   \centering 
   \includegraphics[scale=0.75]{1.5.0/MidiLearn.png}
   \caption{MIDI Learn Dialog}
   \label{fig:midi_learn_dialog}
\end{figure}

   The major items of this dialog are the editor settings available:

   \begin{enumber}
      \item \textbf{Mute}
      \item \textbf{CC}
      \item \textbf{Chan}
      \item \textbf{Min}
      \item \textbf{Max}
      \item \textbf{Limit}
      \item \textbf{Block}
      \item \textbf{Control Function}
      \item \textbf{Load}
      \item \textbf{Save}
      \item \textbf{Recent}
      \item \textbf{Clear}
   \end{enumber}

   Now use the \texttt{Yoshimi} drop-down menu
   and click on \texttt{Midi Learn}, to see a new window displaying the
   recently-learned controller. Along with a number of settings, it shows text
   with precise details of this complete
   action.

   Also shown is an \texttt{activity} LED that flickers when the associated
   CC/channel is received.

   \setcounter{ItemCounter}{0}      % Reset the ItemCounter for this list.

   \itempar{Mute}{MIDI Learn!Mute}
   Mute.
   Disables the MIDI Learn control specified by the corresponding line of
   settings.

   Values: \texttt{Checked, Unchecked}

   \itempar{CC}{MIDI Learn!CC}
   CC.
   Incoming CC.
   Provides the value of the controller that is learned.
   For example, a value of 7 indicates that the control value affects the main
   volume of \texttt{Yoshimi}.

   Values: \texttt{1 to 127, 64*}

   \itempar{Chan}{MIDI Learn!Chan}
   Chan.
   Incoming channel number.
   Note that, in the \textbf{MIDI Learn} window, only the channel numbers start
   from 1.  All the other numbers in that window start from 0.  This
   follows the conventions of the \textsl{Yoshimi} command-line.

   Values: \texttt{1 to 16, 1*}

   \itempar{Min/Max}{MIDI Learn!Min/Max}
   Min and Max.
   Provides the minimum and maximum incoming values.
   If Min is greater than Max, this reverses the control direction.

   Values: \texttt{1 to 127, 64*}

   \itempar{Limit}{MIDI Learn!Limit}
   Limit.
   Limiter versus compression.
   The Min/Max range can either be in the style of a limiter or a compression.

   Values: \texttt{Checked, Unchecked}

   \itempar{Block}{MIDI Learn!Block}
   Block.
   Specifies blocking of all later actions on the same CC/channel pair
   (including system ones).
   NEEDS CLARIFICATION.

   Values: \texttt{Checked, Unchecked}

   \itempar{Control Function}{MIDI Learn!Control Function}
   Control function.
   Provides text describing what control is affected, or if the
   part is disabled or not.

   One can delete any existing MIDI Learn via
   \texttt{Ctrl-right-click}
   on the
   \texttt{Control Function} text for that line.
   One is then presented with a confirmation message giving the line number and
   the text as a reminder.
   Adding or deleting lines, or changing either CC or channel numbers, will
   re-order the lines.

   The same CC/controller can be used to change several different
   internal \texttt{Yoshimi} controls.
   For example, one can have a part's volume being changed while another part is
   having an effect level changed.
   (HOW IS THIS EFFECTED?)

   \itempar{Load}{MIDI Learn!Load}
   Load.
   Loads a set of MIDI Learn values from a file.
   The extension of the file is ...?
   If a loaded set refers to \texttt{Yoshimi}
   controls that are disabled, or don't exist, those controls will be ignored.
   However the Block feature will still be active, unless the line is muted.

   \itempar{Save}{MIDI Learn!Save}
   Save.
   A complete list of MIDI Learn values
   can be saved by clicking on the Save button; one then sees
   the usual file-chooser window.

   \itempar{Recent}{MIDI Learn!Recent}
   Recent.
   This button is used for loading a set of
   MIDI Learn values from the recent history.

   \itempar{Clear}{MIDI Learn!Clear}
   Clear.
   This button clears the entire learned list.

% To come:
% Paging of the display to avoid scrolling through a massive list.

\subsection{MIDI Learn / Tutorial}
\label{subsec:midi_learn_tutorial}

   \textsl{This mini-tutorial is courtesy of Will.}

   Say one has a foot pedal that outputs CC values on the standard volume, CC 7.
   Now this is per channel, so only instruments on the first channel will pick
   it up.  This presents a problem if one has automation/backing tracks on
   other channels and one wants to keep everything together. So here is what to
   do:

   While holding down \texttt{Ctrl}, right-click on the
   \textbf{Volume} knob at the top of the
   main window. A window will open with the message "Learning Main Volume".
   If one now operates the foot pedal, the window will disappear and one will
   see that the main volume control is now responding to the foot pedal.

   \textsl{However}, this means one is changing both the main volume
   \textsl{and} the part 0 volume at the same time.  So now open the
   \textbf{MIDI learn} window via the \textbf{Yoshimi} drop-down menu. One will
   see that it now has a line detailing the incoming CC and channel, along with
   other controls and the control function named \textbf{Main Volume}.  Click
   on the \textbf{Block} check box, and one will see that the part 0 volume
   control no longer responds.

   Now the foot pedal will control \textsl{only} the master volume, not the
   individual part volumes. This setup will survive loading new patch sets, and
   also a main reset (while still running).

   It's quite likely that the foot pedal will go from 0 to 127, when one actually
   wants a much smaller control range.  In that case, one can change the
   \textbf{Min} and \textbf{Max}
   values to (for example) 40 and 90.
   In this way, the entire range of the pedal control will be reduced linearly to
   40-90.

   If one sets the \textbf{Limit} checkbox, then these values will instead be
   cutoff points so anything from the pedal between 0 and 40 will be 40, and
   anything between 90 and 127 will be 90

   To temporarily disable this controller line, use the \textbf{Mute} checkbox.
   The entire line will be greyed, and as the \textbf{Block} is no longer
   active normal part volume control will be restored.

   A point that is not obvious is that although incoming CCs are per-channel,
   the actions are per-\textsl{part}, so if one sets controller 94 for part 1
   volume and then set it again for part 2 volume, one gets \textsl{two} lines,
   each controlling just one part but \textsl{acting together}.  Change the
   \textbf{Min} and \textbf{Max} of one of them to 127 and 0 respectively and
   one will increase in volume while the other reduces.

   Volume, pan, and most of the effects are \textsl{immediate}, while most of
   the other controls start on the \textsl{next note}.  Eventually I want to
   get LFOs, filters etc. to be immediate, but that's for another release!

%-------------------------------------------------------------------------------
% vim: ts=3 sw=3 et ft=tex
%-------------------------------------------------------------------------------
