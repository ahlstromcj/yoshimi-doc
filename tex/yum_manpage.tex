%-------------------------------------------------------------------------------
% yum_manpage
%-------------------------------------------------------------------------------
%
% \file        yum_manpage.tex
% \library     Documents
% \author      Chris Ahlstrom
% \date        2015-06-22
% \update      2015-06-25
% \version     $Revision$
% \license     $XPC_GPL_LICENSE$
%
%     Provides the man page section of yoshimi-user-manual.tex.
%
%-------------------------------------------------------------------------------

\section{Yoshimi Man Page}
\label{sec:yoshimi_man_page}

   The \textsl{Yoshimi} man page is actually the output of the
   \texttt{yoshimi --help} command, which prints out the command-line that
   are discussed in this section.

Yoshimi 1.3.4, a derivative of ZynAddSubFX - Copyright 2002-2009
Nasca Octavian Paul and others, Copyright 2009-2011 Alan Calvert

   \setcounter{ItemCounter}{0}      % Reset the ItemCounter for this list.

  \itempar{-a}{option!alsa midi}
      \textbf{--alsa-midi[=<device>]}
      Use ALSA MIDI input.
      From the command line, as well as autoconnecting the main L \& R
      outputs to JACK, with ALSA MIDI one can now auto-connect to a known source.

   \begin{verbatim}
      ./yoshimi -K --alsa-midi="Virtual Keyboard"
   \end{verbatim}

  \itempar{-A}{option!alsa audio}
      \textbf{--alsa-audio[=<device>]}
      Use ALSA audio output.

  \itempar{-b}{option!alsa buffer size}
      \textbf{--buffersize=<size>}
      Set ALSA audio buffer size.

  \itempar{-c}{option!show console}
      \textbf{--show-console}
      Show console on startup.

  \itempar{-i}{option!no gui}
      \textbf{--no-gui}
      Do not show the GUI.

  \itempar{-j}{option!jack midi}
      \textbf{--jack-midi[=<device>]} 
      Use JACK MIDI input.
      From the command line, as well as autoconnecting the main L \& R
      outputs to JACK, with JACK MIDI one can now auto-connect to a known source.

   \begin{verbatim}
      ./yoshimi -K --jack-midi="jack-keyboard:midi_out"
   \end{verbatim}

  \itempar{-J}{option!jack audio}
      \textbf{--jack-audio[=<server>]}
      Use JACK audio output.

  \itempar{-k}{option!jack autostart}
      \textbf{--autostart-jack}
      Auto-start the JACK server.

  \itempar{-K}{option!jack autoconnect}
      \textbf{--auto-connect}
      Auto-connect JACK audio.

  \itempar{-l}{option!load session}
      \textbf{--load=<file>}
      Load a \texttt{.xmz} file.

  \itempar{-L}{option!load instrument}
      \textbf{--load-instrument=<file>}
      load .xiz file

  \itempar{-N}{option!add tag}
      \textbf{--name-tag=<tag>}
      Add tag to client-name.

  \itempar{-o}{option!set oscilsize}
      \textbf{--oscilsize=<size>}
      Set OscilSize from command-line.

  \itempar{-R}{option!set alsa rate}
      \textbf{--samplerate=<rate>}
      Set ALSA audio sample rate.

  \itempar{-S}{option!load state}
      \textbf{--state[=<file>]}
      Load state from file, defaults to
      \texttt{\$HOME/.config/yoshimi/yoshimi.state}

  \itempar{-?}{option!help}
      \textbf{--help}
      Give this help list.

   \itempar{--usage}{option!usage}
      Give a short usage message.

  \itempar{-V}{option!version}
      \textbf{--version}
      Print program version.

   Mandatory or optional arguments to long options are also mandatory or optional
   for any corresponding short options.

   From the command line, as well as autoconnecting the main L \& R outputs
   to JACK, with either JACK or ALSA MIDI one can now auto-connect to a
   known source.

   ALSA can often manage with just the client name, but JACK needs the port
   as well. These commands are case sensitive, and quite fussy about spaces
   etc. so it's wise to use quotes for the source name, even if they don't
   seem to be needed.

%-------------------------------------------------------------------------------
% vim: ts=3 sw=3 et ft=tex
%-------------------------------------------------------------------------------
