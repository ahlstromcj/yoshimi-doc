%-------------------------------------------------------------------------------
% yum_manpage
%-------------------------------------------------------------------------------
%
% \file        yum_manpage.tex
% \library     Documents
% \author      Chris Ahlstrom
% \date        2015-06-22
% \update      2015-07-10
% \version     $Revision$
% \license     $XPC_GPL_LICENSE$
%
%     Provides the man page section of yoshimi-user-manual.tex.
%
%-------------------------------------------------------------------------------

\section{Yoshimi Man Page}
\label{sec:yoshimi_man_page}

   The \textsl{Yoshimi} man page is actually the output of the
   \texttt{yoshimi --help} command, which prints out the command-line that
   are discussed in this section.

Yoshimi 1.3.4, a derivative of ZynAddSubFX - Copyright 2002-2009
Nasca Octavian Paul and others, Copyright 2009-2011 Alan Calvert

   \setcounter{ItemCounter}{0}      % Reset the ItemCounter for this list.

  \optionpar{-a}{--alsa-midi[=device]}
      Use ALSA MIDI input.
      From the command line, as well as autoconnecting the main L \& R
      outputs to JACK, with ALSA MIDI one can now auto-connect to a known source.

   \begin{verbatim}
      ./yoshimi -K --alsa-midi="Virtual Keyboard"
   \end{verbatim}

  \optionpar{-A}{--alsa-audio[=device]}
      Use ALSA audio output.

  \optionpar{-b}{--buffersize=size}
      Set ALSA audio buffer size.

  \optionpar{-c}{--show-console}
      Show console on startup.

  \optionpar{-i}{--no-gui}
      Do not show the GUI.

  \optionpar{-j}{--jack-midi[=device]} 
      Use JACK MIDI input.
      From the command line, as well as autoconnecting the main L \& R
      outputs to JACK, with JACK MIDI one can now auto-connect to a known source.

   \begin{verbatim}
      ./yoshimi -K --jack-midi="jack-keyboard:midi_out"
   \end{verbatim}

  \optionpar{-J}{--jack-audio[=server]}
      Use JACK audio output.

  \optionpar{-k}{--autostart-jack}
      Auto-start the JACK server.

  \optionpar{-K}{--auto-connect}
      Auto-connect JACK audio.

  \optionpar{-l}{--load=file}
      Load a \texttt{.xmz} file.

  \optionpar{-L}{--load-instrument=file}
      Load an \texttt{.xiz} file

  \optionpar{-N}{--name-tag=tag}
      Add tag to client-name.

  \optionpar{-o}{--oscilsize=size}
      Set OscilSize from command-line.

  \optionpar{-R}{--samplerate=rate}
      Set ALSA audio sample rate.

  \optionpar{-S}{--state[=file]}
      Load state from file, where the file defaults to
      \texttt{\$HOME/.config/yoshimi/yoshimi.state}

  \optionpar{-?}{--help}
      Give this help list.

   \optionpar{--usage}
      Provide a short usage message.

  \optionpar{-V}{--version}
      Print program version.

   Mandatory or optional arguments to long options are also mandatory or optional
   for any corresponding short options.

   From the command line, as well as autoconnecting the main L \& R outputs
   to JACK, with either JACK or ALSA MIDI one can now auto-connect to a
   known source.

   ALSA can often manage with just the client name, but JACK needs the port
   as well. These commands are case sensitive, and quite fussy about spaces
   etc. so it's wise to use quotes for the source name, even if they don't
   seem to be needed.

%-------------------------------------------------------------------------------
% vim: ts=3 sw=3 et ft=tex
%-------------------------------------------------------------------------------
