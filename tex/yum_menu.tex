%-------------------------------------------------------------------------------
% yum_menu
%-------------------------------------------------------------------------------
%
% \file        yum_menu.tex
% \library     Documents
% \author      Chris Ahlstrom
% \date        2015-05-11
% \update      2016-01-30
% \version     $Revision$
% \license     $XPC_GPL_LICENSE$
%
%     Provides the Menu section of yoshimi-user-manual.tex.
%
%-------------------------------------------------------------------------------

\section{Menu}
\label{sec:menu}

   The \textsl{Yoshimi} menu, as seen at the top of
   \figureref{fig:yoshimi_main_screen},
   is fairly simple, but it is important to understand the
   structure of the menu entries.

\subsection{Menu / Yoshimi}
\label{subsec:menu_yoshimi}

   The \textsl{Yoshimi}
   menu entry contains the sub-items shown in
   \figureref{fig:yoshimi_menu_items}.
   The next few sub-sections discuss the sub-items in the 
   \textsl{Yoshimi} sub-menu.
   (Note that, in \textsl{ZynAddSubFX}, this menu is called the
   \textsl{File} menu.)

\begin{figure}[H]
   \centering 
   \includegraphics[scale=1.0]{menu/yoshimi-menu-yoshimi.jpg}
   \caption{Yoshimi Menu Items}
   \label{fig:yoshimi_menu_items}
\end{figure}

   \textbf{Bug:}
   \index{bugs!menu hot keys don't work}
   There seems to be a bug in that the expected menu hot-keys
   (Alt-Y, Alt-I, Alt-P, and Alt-S) do not work (Yoshimi 1.3.5).

\subsubsection{Menu / Yoshimi / About...}
\label{subsubsec:menu_yoshimi_about}

   There is no "Help" menu in \textsl{Yoshimi}.
   Therefore, the "About" dialog appears in the "Yoshimi" menu, as shown in
   \figureref{fig:yoshimi_about_dialog}.
   These guys need some acknowledgment for their hard work!
   And they acknowledge the massive groundwork laid by the
   \textsl{ZynAddSubFX} project.

\begin{figure}[H]
   \centering 
%  \includegraphics[scale=1.0]{menu/Yoshimi/yoshimi-about.jpg}
   \includegraphics[scale=1.0]{1.3.8/yoshimi-about.jpg}
   \caption{Yoshimi Menu, About Dialog}
   \label{fig:yoshimi_about_dialog}
\end{figure}

\subsubsection{Menu / Yoshimi / New instance}
\label{subsubsec:menu_yoshimi_new_instance}

   Creates a new instance of \textsl{Yoshimi}.
   We're not quite sure what this one does, really.
   Does it create a new run of 
   \textsl{Yoshimi}
   with a random \texttt{--name-tag} value?
   In our basic investigation, it simply finds that the previous
   \textsl{Yoshimi}
   instance has grabbed audio access, when JACK is not being used:

\begin{verbatim}
   Yay! We're up and running :-)
   failed to open alsa audio device:default: Device or resource busy
   AlsaClient audio open failed
   Failed to open MusicClient
   Yoshimi stages a strategic retreat :-(
\end{verbatim}

   Now, if JACK is running, then this feature will work.
   Start a normal (JACK-using) instance of \textsl{Yoshimi}.
   Then use this menu entry.  \textsl{Yoshimi} will start another instance
   of itself, with an ID of 1.
   This instance can be verified by running a JACK session manager such as
   QJackCtl.

   It is important to note that each instance of \textsl{Yoshimi} has its
   own configuration file.  Each also has its own MIDI and audio ports.
   Thus, these instances are independent of each other.

\subsubsection{Menu / Yoshimi / New instance with id...}
\label{subsubsec:menu_yoshimi_new_instance_with_id}

   Creates a new instance of \textsl{Yoshimi}
   with an ID that is a number.
   See \figureref{fig:yoshimi_instance_dialog}.
   It tries to open a \textsl{Yoshimi} instance based on the configuration
   found in the file
   \texttt{~/.config/\-yoshimi/\-yoshimi.configXX}, where
   \textsl{XX} is the ID one supplied.

\begin{figure}[H]
   \centering 
   \includegraphics[scale=0.75]{menu/Yoshimi/yoshimi-instance-id.jpg}
   \caption{Yoshimi Menu, Instance Dialog}
   \label{fig:yoshimi_instance_dialog}
\end{figure}

   Useful when connecting devices with JACK.
   Start a normal (JACK-using) instance of \textsl{Yoshimi}.
   Then use this menu entry, supply a number as an ID.
   \textsl{Yoshimi} will start another instance
   of itself, with an ID of whatever number one specified.
   This instance can be verified by running a JACK session manager such as
   QJackCtl.

   Again, though, in a non-JACK setup it simply fails.  

\subsubsection{Menu / Yoshimi / Settings...}
\label{subsubsec:menu_yoshimi_settings}

   The \textsl{Yoshimi Settings} dialog contains five tabs that control the
   major and overall settings of \textsl{Yoshimi}.

   Please note that the \textbf{Save and Close} and \textbf{Close Unsaved}
   buttons apply to the whole \textbf{Settings} window.  Further more, the
   saving does not refer to preserving the changes that might have been made
   in any of the tabs for the current \textsl{Yoshimi} session.  Any changes
   made in \textbf{Settings} always remain in place for the current
   \textsl{Yoshimi} session.  (So be careful!)  However, they are saved to
   the state file only if \textbf{Save and Close} is clicked.

   \setcounter{ItemCounter}{0}      % Reset the ItemCounter for this list.

   \itempar{Save and Close}{Settings!Save and Close}
   This selection saves the settings made in \textbf{all} of the tabs,
   and closes the \textsl{Yoshimi} settings dialog.

   \itempar{Close Unsaved}{Settings!Close Unsaved}
   Close Unsaved, Main Settings.

   This selection closes the \textsl{Yoshimi} settings dialog.
   However, note that any changes made in the tabs
   \textsl{are preserved}.  They are preserved for the current
   \textsl{Yoshimi} session, but are not saved to the filesystem.
   
\paragraph{Menu / Yoshimi / Settings / Main Settings}
\label{paragraph:menu_yoshimi_settings_main_settings}

   The Main Settings tab controls the main configuration items that
   follow, which apply to all patches/instruments.
   The main settings are shown in
   \figureref{fig:yoshimi_main_settings_dialog}.

   All these settings only take effect after restarting the synthesizer.
   The settings dialogs are quite different between \textsl{ZynAddSubFX} and
   \textsl{Yoshimi}.  There are some differences even between
   \textsl{Yoshimi} versions earlier than 1.3.5, and the current version
   (currently 1.3.8).

\begin{figure}[H]
   \centering 
%  \includegraphics[scale=0.75]{menu/Yoshimi/yoshimi-settings-main.jpg}
   \includegraphics[scale=0.75]{1.3.8/yoshimi-settings-main.png}
   \caption{Yoshimi Main Settings Tab}
   \label{fig:yoshimi_main_settings_dialog}
\end{figure}

   TODO:  Describe new  UI elements.

   The following settings exist in the \textsl{Main settings} tab:

   \begin{enumber}
      \item \textbf{AddSynth Oscillator Size} (was "OscilSize")
      \item \textbf{Internal Buffer Size} (new)
      \item \textbf{PADsynth interpolation}
      \item \textbf{Virtual Keyboard Layout}
      \item \textbf{XML compression level}
      \item \textbf{Send reports to}
      \item \textbf{Session state save file}
      \item \textbf{Select}
      \item \textbf{Save and Close}
      \item \textbf{Close Unsaved}
   \end{enumber}

   \setcounter{ItemCounter}{0}      % Reset the ItemCounter for this list.

   \itempar{AddSynth Oscillator Size}{Main Settings!oscillator size}
   ADDsynth Oscillator Size (in samples).  This item used to be called
   "OscilSize".

   Values: \texttt{128, 256, 512+, 1024*, 2048, 4096, 8192, 16384}

   Sets the number of the points of the ADDsynth oscillator. The
   bigger is better, but it takes more CPU time on the start of any note,
   and it may add latency to some processes.

   The default value for \textsl{Yoshimi} is shown marked with an asterisk,
   and the default value for \textsl{ZynAddSubFX} is 512.
   (This asterisk/plus-sign convention is used throughout this manual).
   See \figureref{fig:yoshimi_oscilsize_values} below for the OscilSize
   drop-down element.

\begin{figure}[H]
   \centering 
   \includegraphics[scale=1.0]{menu/Yoshimi/main-oscilsize.jpg}
   \caption[OscilSize Values]{AddSynth Oscillator Size (samples)}
   \label{fig:yoshimi_oscilsize_values}
\end{figure}

   \itempar{Internal Buffer Size}{Main Settings!buffer size}
   This is a new item for version 1.3.6.  It is actually the old
   \textbf{Period Size} field from the \textbf{Alsa} tab.
   It sets the granularity of the sound generation.
   The default value is 1024 samples.
   To find out the internal delay in milliseconds, divide the
   buffer-size value by the sample-rate, then multiply the result by 1000:
   For example, \(256 / 44100 * 1000 = 5.8 ms\).

   Values: \texttt{64, 128, 256, 512, 1024*}

   \itempar{Virtual Keyboard Layout}{Main Settings!Virtual Keyboard Layout}
   The virtual keyboard is useful, but it is difficult to move the mouse
   rapidly to the next key on the virtual keyboard.
   Therefore, \textsl{Yoshimi} supports using the computer keyboard
   to produce notes.

\begin{figure}[H]
   \centering 
   \includegraphics[scale=0.5]{top-panel/ascii-virtual-keyboard.png}
   \caption{QWERTY Virtual Keyboard}
   \label{fig:qwerty_virtual_keyboard}
\end{figure}

   See \figureref{fig:qwerty_virtual_keyboard},
   for the mapping of the computer keyboard to the
   virtual keyboard.
   Three octaves (blue, green, and red) are available, with the dark keys of
   each color representing the "black" keys.
   Note that this is a QWERTY layout.  
   \textsl{Yoshimi} also supports other keyboard layouts.
   See \figureref{fig:virtual_kbd_layout},
   for the virtual keyboard
   layout settings drop-down.

   Values: \texttt{Dvorak, QWERTY*, AZERTY}

\begin{figure}[H]
   \centering 
   \includegraphics[scale=1.0]{menu/Yoshimi/main-virtual-kbd-layout.jpg}
   \caption[Virtual Keyboard Layout]{Virtual Keyboard Layout Values}
   \label{fig:virtual_kbd_layout} 
\end{figure}

   \itempar{PADsynth interpolation}{Main Settings!PADsynth Interpolation}

   Values: \texttt{Linear(fast)*, Cubic(slow)}

   See \figureref{fig:padsynth_interpolation} below
   for the interpolation values.

\begin{figure}[H]
   \centering 
   \includegraphics[scale=1.0]{menu/Yoshimi/main-padsynth-interpolation.jpg}
   \caption[PADSynth Interpolation]{PADSynth Interpolation Values}
   \label{fig:padsynth_interpolation}
\end{figure}

   \itempar{XML compression level}{Main Settings!XML compression level}
   Compression level of \textsl{Yoshimi} XML files.

   Values: \texttt{0 to 9, 3*}

   The settings and instruments of
   \textsl{Yoshimi}
   are preserved in XML files.
   The value of 0 indicates that the XML file is uncompressed.
   In general, 0 is probably the best setting.
   Setting this option makes the XML files a bit larger, perhaps larger by a 
   factor of more than 10, making a 10K file into a 180K file.
   Using XML compression can also save file access time which may be
   beneficial if one's computer is borderline on latency.
   For a little "wasted"
   space and time, one can view the XML file in a text/programmer's editor.
   But, if one's system is tight on disk space, higher levels of compression
   can be specified.

   \itempar{Send reports to}{Main Settings!Send Reports Destination}

   Values: \texttt{stderr, Console Window*}

   Notices and error messages can be sent to the standard error log of
   the terminal in which 
   \textsl{Yoshimi} can be run, or, more usefully, to
   an output console window.
   See \figureref{fig:send_reports_to}.
   It provides a depiction of the selection drop-down.

\begin{figure}[H]
   \centering 
   \includegraphics[scale=1.0]{menu/Yoshimi/main-send-reports-to.jpg}
   \caption[Send Reports]{Send Reports To}
   \label{fig:send_reports_to}
\end{figure}

   \itempar{Session state save file}{Main Settings!Session State}
   Main Settings Session State Save File.
   Enter the name of the desired session state file here, including
   the path to it.
   Example: \texttt{/home/myself/.config/yoshimi/yoshimi.state}

   \itempar{Select}{Main Settings!Select Saved-State File}
   Select Saved-State File.
   See \figureref{fig:session_save_state}.
   It provides a depiction of this dialog, which lets one pick an existing
   file as the \textsl{Yoshimi} state file.

   Values: \texttt{~/.config/yoshimi/yoshimi.state}

\begin{figure}[H]
   \centering 
   \includegraphics[scale=0.75]{menu/Yoshimi/main-nominate-session-save-state-file.jpg}
   \caption[Session Save State]{Session Save State File}
   \label{fig:session_save_state} 
\end{figure}

\paragraph{Menu / Yoshimi / Settings / Preset dirs}
\label{paragraph:menu_yoshimi_settings_preset_dirs}

   THIS SECTION NEEDS TO BE MOVED TO
   "Menu / Paths / Preset dirs".  Probably no longer needs a restart, either.

   The \textsl{Yoshimi} preset directories are the locations where
   presets can be found.  When first installed, the system
   preset directory is

   \begin{verbatim}
      /usr/share/yoshimi/presets
   \end{verbatim}
   
   The user can provide additional directories for the presets.
   These directories are useful for containing copies of the system
   presets that one can modify safely, and for providing custom
   presets designed by the user.

   The following items are provided by the preset directory settings:

   \begin{enumber}
      \item \textbf{Preset list}
      \item \textbf{Add preset directory...}
      \item \textbf{Remove preset directory...}
      \item \textbf{Make default}
      \item \textbf{Save and Close}
      \item \textbf{Close Unsaved}
   \end{enumber}

%  \setcounter{ItemCounter}{0}      % Reset the ItemCounter for this list.

\begin{figure}[H]
   \centering 
   \includegraphics[scale=0.75]{menu/Yoshimi/yoshimi-settings-presets-dirs.jpg}
   \caption[Preset Dirs Tab]{Yoshimi Preset Dirs Dialog}
   \label{fig:yoshimi_presets_dirs_tab}
\end{figure}

   \setcounter{ItemCounter}{0}      % Reset the ItemCounter for this list.

   \itempar{Preset list}{Yoshimi Presets!Preset List}
   This interface element contains a list of preset directories.
   By default, the only directory present is the installed preset directory.
   For example, \texttt{/usr/share/yoshimi/presets}.
   Another example would be this project; let YOSHIMI-DOC be the directory
   where this project is stored.  Then one can add
   \texttt{YOSHIMI-DOC/config/yoshimi/presets} to this list, using the
   button described next.

   \itempar{Add preset directory...}{Yoshimi Presets!Add Directory}
   Use this button and dialog to add a preset directory to the list, for
   easy access.

   Press the \textbf{Add preset directory...} button, revealing the
   following dialog.

\begin{figure}[H]
   \centering 
   \includegraphics[scale=0.75]{menu/Yoshimi/presets-add-a-preset-directory.jpg}
   \caption[Add Preset Directory]{Add a Preset Directory}
   \label{fig:presets_add_a_preset_directory}
\end{figure}

   Navigate to the desired directory, select it, and press the \textbf{Ok}
   button.  (There is no need to press the \textbf{Save and Close} button;
   the directory is added as soon as \textsl{OK} is clicked.
   IMPORTANT:  Restart \textbf{Yoshimi} to used the preset directory.

   \itempar{Remove preset directory...}{Yoshimi Presets!Remove Directory}
   Select one of the preset directories in the preset list, then press this
   button to remove the preset directory from the list of preset
   directories.  It is removed immediately, no need to confirm the deletion,
   click an OK button, or click a Save button.

   \itempar{Make default presets}{Yoshimi Presets!Make Default}
   Make Default Presets Directory.
   Select one of the preset directories in the preset list, then press this
   button to make the preset directory the default preset directory.

\paragraph{Menu / Yoshimi / Settings / Jack}
\label{paragraph:menu_yoshimi_settings_jack}

   JACK is the "Jack Audio Connection Kit", useful increasing audio
   performance and configurability.

   When using the JACK audio backend, instruments can be individually routed
   and sent to the main L/R outputs. This is controlled from the
   panel window,
   \sectionref{subsec:mixer_panel_window},
   and the settings are saved with all the other parameters.

   Direct part outputs carry the Part and Insertion effects, but not the
   System ones.

\begin{figure}[H]
   \centering 
%  \includegraphics[scale=0.75]{menu/Yoshimi/yoshimi-settings-jack.jpg}
   \includegraphics[scale=0.75]{1.3.8/yoshimi-settings-jack.jpg}
   \caption[JACK Settings]{JACK Settings Dialog}
   \label{fig:yoshimi_settings_jack}
\end{figure}

   TODO:  Many new UI items to document.

   \setcounter{ItemCounter}{0}      % Reset the ItemCounter for this list.

   \itempar{Jack Server}{JACK!server name}
   Jack Server Name.
   It is possible to have more than one JACK server running.  This option
   tells this instance of \textsl{Yoshimi} which server to use.

   Values: \texttt{default*, name} name, as in "jackd --name"

\paragraph{Menu / Yoshimi / Settings / Alsa}
\label{paragraph:menu_yoshimi_settings_alsa}

   A significant improvement is to the handling of ALSA audio, which is still
   very important for some people. Up till now \textsl{Yoshimi} has insisted
   on a 2-channel 16-bit format. Tests have shown that virtually all
   motherboard sound chipsets will handle this, but many external ones don't.

   From \textsl{Yoshimi} 1.3.6 onwards, when using ALSA audio,
   \textsl{Yoshimi} first tries to connect 2 channels at 32 bit depth.  If
   that connection does not succeed, then \textsl{Yoshimi} negotiates
   whatever the soundcard will support.  For example, a card might support
   only 24 bits, and 6 channels.  So \textsl{Yoshimi} will fall back to
   24 bit, and, due to its own limits, will use only channels 1 and 2.

   With external sound modules in mind, endian swaps are also implemented.

   To be able to reliably use ALSA audio you need to set a card name, not just
   "Default".  In a terminal window enter the following command:

   \begin{verbatim}
      $ cat /proc/asound/card*/id
   \end{verbatim}

   The result of this command should be something like:

   \begin{verbatim}
      PCH
      K6
   \end{verbatim}

   Go to the ALSA settings tab illustrated below, and in 
   \textsl{Alsa Audio Device} enter, for example, "hw:PCH".
   This ensures you will always connect to this card at startup regardless of
   the order this and other ones.  Another benefit of using this hardware name
   is that ALSA will now use \textsl{Yoshimi}'s internal
   buffer size, otherwise ALSA will force \textsl{Yoshimi} to accept its
   default size.

   One can also set the sample rate, but bear in mind that not all cards can use
   all of these.  The sample rates 44100 and 48000 are almost always available.
   If you set a Midi Device as well (such as a keyboard) Yoshimi will try to
   find and connect to this device at startup.

   To find the MIDI devices available, try:

   \begin{verbatim}
      $ grep Client /proc/asound/seq/clients
   \end{verbatim}

   The result of this command should be something like:

   \begin{verbatim}
      Client info
      Client   0 : "System" [Kernel]
      Client  14 : "Midi Through" [Kernel]
      Client 128 : "TiMidity" [User]
   \end{verbatim}

   It is not obvious how ALSA audio is controlled and who takes command.  If one
   sets a specific audio destination, then \textsl{Yoshimi} makes a request.
   It's often a negotiation on bit depth and channel count, but \textsl{Yoshimi}
   nearly always gets to decide the buffer size, which is the internal buffer
   size.  However, if the destination is 'default' then ALSA decides on the
   sound card, bit depth, number of channels and the buffer size, and
   \textsl{Yoshimi} will set it's internal buffer size to match.  On most
   machines this seems to be 1024.

\begin{figure}[H]
   \centering 
%  \includegraphics[scale=0.75]{menu/Yoshimi/yoshimi-settings-alsa.jpg}
   \includegraphics[scale=0.75]{1.3.8/yoshimi-settings-alsa.png}
   \caption[ALSA Settings]{ALSA Settings Dialog}
   \label{fig:yoshimi_settings_alsa}
\end{figure}

   TODO:  Many new UI items to document.

   \setcounter{ItemCounter}{0}      % Reset the ItemCounter for this list.

   \itempar{Alsa Midi Device}{ALSA!MIDI device}
   ALSA MIDI Device.
   The purpose of this setting is the same as the command line option
   \texttt{--alsa-midi="name"}.
   It is used so that \textsl{Yoshimi} can auto connect to a MIDI source
   such as a keyboard.  For example, the one that Will has identifies itself
   as name = "Hua Xing".
   A port name, such as "128:0" (for one of the ports provided by
   \textsl{TiMidity}) should work as well.

   Values: \texttt{default*}

   \itempar{Alsa Audio Device}{ALSA!audio device}
   ALSA Audio Device.
   This specifies the sound card to which \textsl{Yoshimi} can connect.
   Normally, this will be an ALSA hardware specification such as
   "hw:0".
   ALSA audio also lets one connect to a sound card by name. For example,
   with a Komplete Audio KA 6 sound card, the device specification is
   "hw:K6". This feature is particularly useful for USB modules, as one can
   never be sure where they appear numerically.

   Values: \texttt{default*}

   \itempar{Samplerate}{ALSA!sample rate}
   Sample Rate.
   Sets the quality of the sound, higher is better, but it uses more CPU.
   One can select from a list.
   
   \textsl{ZynAddSubFX}: if one wants a sample-rate that
   is not in the list, select "Custom" and change the value from the right.
   Default is 44100.

   Values: \texttt{96000, 48000*, 44100}

   Note that, in version 1.3.6, the \textbf{Period Size} field has been
   removed from the \textbf{Alsa} tab, and is replaced by the 
   \textbf{Internal Buffer Size} field in the \textbf{Main Settings} tab.

%  \itempar{Period Size}{ALSA!period size}
%  Buffer Size.
%  Sets the granularity of the sound. The Default is 256 (TODO: 1024?)
%  samples. To find out the internal delay in milliseconds, divide the
%  Period Size value by the Sample Rate and multiply the result by 1000
%  (eg.: 256/44100*1000=5.8 ms).

   Values: \texttt{64, 128, 256, 512, 1024*}

\paragraph{Menu / Yoshimi / Settings / MIDI}
\label{paragraph:menu_yoshimi_settings_ccs}

   The CC settings tab has been renamed the "MIDI" tab.
   This tab, shown in
   \figureref{fig:yoshimi_settings_cc},
   presents MIDI bank-root, bank, program change, and extended program
   change settings, plus some new values.

   A new feature with 1.3.6 is that changes to the items in this
   tab cause a red \textbf{Pending} button to appear.  Pressing this
   button saves that particular change.

\begin{figure}[H]
   \centering 
%  \includegraphics[scale=0.75]{menu/Yoshimi/yoshimi-settings-ccs.jpg}
   \includegraphics[scale=0.75]{1.3.8/yoshimi-settings-midi.png}
   \caption[MIDI Preferences]{MIDI Preferences}
   \label{fig:yoshimi_settings_cc}
\end{figure}

   TODO:  Slight UI differences to document.

   \setcounter{ItemCounter}{0}      % Reset the ItemCounter for this list.

   The concepts of banks and roots is very useful.
   See \sectionref{subsec:concepts_banks_and_roots}.
   The settings in this tab affect the usage of banks and root changes
   controlled by MIDI messages, thereby making \textsl{Yoshimi} able to
   implement MIDI automation.

   \itempar{Enable Bank Root Change}{MIDI preferences!enable bank root change}
   Enable Bank Root Change.

   Values: \texttt{Off*, On}

   \itempar{Bank Root Change}{MIDI preferences!bank root change}

   Values: \texttt{0*, to 127}

   If enabled, a new reddish button, \textbf{Pending}, appears.
   Once the change has been made in the scroll list, click this button
   to set the change.
   \textbf{Warning:}
   The \textbf{Save and Close} button will not result in the removal of the
   \textbf{Pending} button.  This result seems counter-intuitive.

   It is not clear if this change persists after \textsl{Yoshimi} closes, or
   if the session or state must be saved.
   But apparently this change can be made without a restart being required.

   \itempar{Bank Change}{MIDI preferences!bank change}
   Bank Change.
   Defines which MIDI preferences one wants to use.
   Note that MIDI Controller 0 = CC0 = Bank Select MSB, and MIDI Controller
   32 = CC32 = Bank Select LSB.
   When combined, these Bank Select messages provide \[128*128 = 16384\]
   banks.

   But note that all a Bank Select does is selects the bank for the next
   Program Change event.  The program doesn't change after changing a bank,
   until a Program Change is sent.

   Bank changes can be completely disabled - some hardware
   synths don't play nice with banks.

   Values: \texttt{LSB, MSB*, Off}

   \itempar{Enable Program Change}{MIDI preferences!enable program change}

   Values: \texttt{Off*, On}

   Enables/disables MIDI program change.
   Program changes can be completely disabled, but some hardware synths don't
   play nice!

   \itempar{Enable Part On Program Change}{MIDI preferences!enable part change}

   Values: \texttt{Off*, On}

   The part is enabled if the MIDI program was changed on this part.

   \itempar{Enable Extended Program Change}{MIDI preferences!enable extended change}

   Values: \texttt{Off*, On}

   \itempar{Extended Program Change}{MIDI preferences!extended change}
   If enabled, a new reddish button, \textbf{Pending}, appears.
   Once the change has been made in the scroll list, click this button
   to set the change.

   Values: \texttt{0-127, 110*}

   \itempar{Set immediate - Control}{MIDI preferences!set immediate control}
   TODO.
   Sets the control number for ...

   Values: \texttt{0-127, 0*}

   \itempar{Set immediate - Value}{MIDI preferences!set immediate value}
   TODO.
   Sets the value number of the given control for ...

   Values: \texttt{0-127, 0*}

\subsubsection{Menu / Yoshimi / Exit}
\label{subsubsec:menu_yoshimi_exit}

   Simply exits from \textsl{Yoshimi}.
   The user is prompted if unsaved changes exist, as shown in
   \figureref{fig:yoshimi_change_exit}.

\begin{figure}[H]
   \centering 
   \includegraphics[scale=1.0]{menu/Yoshimi/yoshimi-menu-exit-parameters-changed.jpg}
   \caption[Yoshimi Menu, Exit]{Yoshimi Menu, Exit}
   \label{fig:yoshimi_change_exit}
\end{figure}

\subsection{Menu / Instrument}
\label{subsec:menu_instrument}

   The \textsl{Yoshimi} Instrument menu lets one select instruments and work
   with banks of instruments.
   \textsl{Yoshimi} stamps instrument XML files with its own major and minor
   version numbers so it is possible to tell which version created the files,
   or whether they were created by \textsl{ZynAddSubFX}.

   When opening an instrument bank one can now tell exactly which synth engines
   are used by each instrument. This is represented by three pale background
   colours:

   \begin{itemize}
      \item \textcolor{red}{Red}: ADDsynth
      \item \textcolor{blue}{Blue}: SUBsynth
      \item \textcolor{green}{Green}: PADsynth
   \end{itemize}

   These new colored engine backgrounds aren't just pretty. They give real
   information about expected processor load, and time to be ready when
   loaded:

   \begin{itemize}
      \item \textsl{Processor Load, low to high}: PAD, SUB, then ADD.
      \item \textsl{Time to initialize, low to high}: SUB, ADD, PAD.
   \end{itemize}

   If the instruments are kits they scanned to find out if 
   \textsl{any} member of the kit contains each engine.
   This scanning is duplicated in the current part, the mixer panel for the
   currently loaded instruments, and in the Instrument Edit window the same
   colors highlight the engine names when they are enabled with the check
   boxes. 

   The following sub-menus are provided, as shown in
   \figureref{fig:yoshimi_instrument_menu}.

\begin{figure}[H]
   \centering 
   \includegraphics[scale=1.0]{1.3.8/yoshimi-menu-instrument.jpg}
   \caption{Yoshimi Menu, Instrument}
   \label{fig:yoshimi_instrument_menu}
\end{figure}

   TODO:  Document the many differences in 1.3.8 here.

   \begin{enumber}
      \item \textbf{Clear Instrument...}
      \item \textbf{Open Instrument...}
      \item \textbf{Save Instrument...}
      \item \textbf{Show Instruments...}
      \item \textbf{Show Banks...}
      \item \textbf{Show Root Paths...}
   \end{enumber}

   These menu entries don't appear in the order in which they would normally
   be used.  For simplicity, it is better, especially for \textsl{Yoshimi}
   1.3.5 and above, to summarize how to navigate through these menu items,
   before showing each one in detail.

   \setcounter{ItemCounter}{0}      % Reset the ItemCounter for this list.

   \itempar{Set Current Root Path}{root!set current}
   \index{root!current}
   Instruments are stored in banks, and banks are stored in root directories,
   also known as "roots".  In \textsl{Yoshimi}, there can be a number of
   roots that exist in a user's directory structure, but only one root can be
   the current root.  Thus, the first step is to set up the current root
   directory to point to where we have stored all our banks of instruments.

   \begin{enumber}
      \item In \textsl{Yoshimi}, navigate to the \textbf{Instrument / Show
      Root Paths...} entry in the main menu.
      \item In the \textbf{Bank Root Paths} dialog, select the desired
      bank path by clicking on it.
      \item Click the \textbf{Make Current} button.
%     \item Then click the \textbf{Open Current} button.
      \item Then click the \textbf{Save and Close} button.
   \end{enumber}

   For example, one can make the \texttt{/usr/share/yoshimi/banks} directory
   the current root directory.  This directory is the default location for
   banks when \textsl{Yoshimi} is first installed.
   In the following figure, we set it to the author's local directory,
   \texttt{~/Audio/yoshimi/banks}.
   In general, it is recommended that one copies the default installed
   directories to a local directory, in order to be able to work with them,
   making additions and changes without needing root permissions, and
   without risking the corruption of the default installation.

   For figures and details about the \textbf{Show Root Paths...} menu entry,
   see \sectionref{subsubsec:menu_instrument_show_root_paths}.

   \itempar{Show Current Bank Set}{banks!show}
   Once the current root has been set, one can then see all of the banks
   under that root.  Navigate to the \textbf{Instrument / Show Banks...} menu
   entry and click it.  This brings up a dialog such as the one shown in
   \figureref{fig:show_ca_banks}.  That dialog shows all of the banks that
   exist in the current root directory.  Each is auto-numbered by
   \textsl{Yoshimi} the first time that root directory is accessed, and the
   current bank is highlighted in pink.

   Clicking on a bank with both make that bank the current bank, and opens an
   instruments dialog, such as that shown in
   \figureref{fig:show_alex_j_bank}.

   \itempar{Show Instruments in Current Bank}{instruments!show}
   Once the current root and current bank have been set, another way to show
   the instruments is to navigate to the
   \textbf{Instrument / Show Instruments...} menu entry and click on it.
   Again, this action opens an instruments dialog, such as that shown in
   \figureref{fig:show_alex_j_bank}.

   A left-click on a particular instrument sets that instrument into the
   current Part in force in the main \textsl{Yoshimi} window, where it can
   then be edited, if desired.

   \index{anti-auto-clutter}
   Right-clicking on an instrument causes the instruments list dialog
   to disappear, and be replaced by an instrument dialog.  While this
   behavior might be surprising, it is part of the anti-auto-clutter feature
   of \textsl{Yoshimi}.

   Now that we know how to easily navigate through roots, banks, and
   instruments, we can discuss each of the \textbf{Instrument} menu entries
   in detail.

\subsubsection{Menu / Instrument / Clear Instrument...}
\label{subsubsec:menu_instrument_clear}

   This menu entry brings up a prompt to clear the parameters of the
   instrument that is currently loaded in the current part.

\begin{figure}[H]
   \centering 
   \includegraphics[scale=0.75]{menu/Instrument/clear-instrument.jpg}
   \caption{Clear Instrument Dialog}
   \label{fig:clear_instrument_dialog}
\end{figure}

   \textbf{Bug:}
   \index{bugs!need to clear instrument?}
   Sometime it seems that one needs to clear the instrument if one is
   loading a new instrument to test it out, because some settings seem
   to remain from the previous instrument.

   \textsl{Don't quote us on that.  Maybe Will has fixed that issue by now.}

\subsubsection{Menu / Instrument / Open Instrument...}
\label{subsubsec:menu_instrument_open}

   This menu entry brings up a prompt to open a new instrument.
   This prompt is a file-dialog, and it doesn't depend at all on the settings
   of the current root or the current bank.  It does have a
   \textbf{Favorites} button to help the user get quickly to the desired
   location of instrument files.

\begin{figure}[H]
   \centering 
   \includegraphics[scale=0.75]{menu/Instrument/open-instrument.jpg}
   \caption{Open Instrument Dialog}
   \label{fig:open_instrument_dialog}
\end{figure}

   This dialog has a number of user-interface elements to discuss.

   \begin{enumber}
      \item \textbf{Show}
      \item \textbf{Favorites}
      \item \textbf{Create a new diretory}
      \item \textbf{Instrument List}
      \item \textbf{XML Preview}
      \item \textbf{Preview}
      \item \textbf{Show hidden files}
      \item \textbf{Directory Bar}
      \item \textbf{Filename}
      \item \textbf{OK}
      \item \textbf{Cancel}
   \end{enumber}

   \setcounter{ItemCounter}{0}      % Reset the ItemCounter for this list.

   \itempar{Show}{Open Instrument!show}
   Show types of files.
   This item shows a file filter for selecting instrument files.
   The types of filters are as follows (screen shot not available):

   \begin{enumber}
      \item \textbf{(\{*.xiz\})} (compressed XML files)
      \item \textbf{All Files (*)}
      \item \textbf{Custom Filter}
   \end{enumber}

   \itempar{Favorites}{Open Instrument!favorites}
   Favorite directories.
   Provides a list of options and favorite directories in which to find 
   instrument files.

\begin{figure}[H]
   \centering 
   \includegraphics[scale=0.75]{menu/Instrument/favorites-dropdown.jpg}
   \caption{Favorites Drop-down}
   \label{fig:open_instrument_favorites}
\end{figure}

   \begin{enumber}
      \item \textbf{Add to Favorites}
      \item \textbf{Manage Favorites}
      \item \textbf{File Systems}
      \item \textbf{(Additional favorite directories)}
   \end{enumber}

   \index{Add to Favorites}
   \textbf{Add to Favorites}
   simply adds the currently selected directory shown in the instrument list
   to the list of favorites.

   To add Favorites in the file dialog, navigate to the desired directory.
   Then click \textbf{Favorites}, and select \textbf{Add to Favorites}.

   Once one has a number of favorites set up,
   there is a \textbf{Manage Favorites} that can be used.
   For example, if one needs to get rid of a directory, one can use the
   \textbf{Manage Favorites}
   \index{Manage Favorites}
   dialog, shown in
   \figureref{fig:manage_instrument_favorites} below,
   to do that.

\begin{figure}[H]
   \centering 
   \includegraphics[scale=1.0]{menu/Instrument/manage-favorites.png}
   \caption{Favorites Drop-down}
   \label{fig:manage_instrument_favorites}
\end{figure}

   \textbf{File Systems} \index{File Systems}
   Provides a list of all file systems starting at root ("\texttt{/}").
   This list can be pretty confusing, with a lot of entries.
   But note that one navigates to ("\texttt{/}"), and from there to
   \texttt{/usr/share/yoshimi/banks} to get easy access to all the
   instruments that are preinstalled with
   \textsl{Yoshimi}.
   Generally, one will want to use only
   \textbf{Add to Favorites} and \textbf{Manage Favorites}.

   \itempar{Create Directory}{Open Instrument!create new directory}
   Creates a New Directory.
   This little symbol options a small "New Directory?" dialog (not shown
   here, it is very simple and stock) into which one can type a directory
   name to be added to the current directory of the instrument list.

   \itempar{Instrument List}{Open Instrument!instrument list}
   Provides a list of the instrument files available in the current
   directory.  Also shown are sub-directories (if available)
   that might contain more instruments, and a ("\texttt{../}") entry
   to navigate to the parent directory.

   \itempar{Preview}{Open Instrument!preview checkbox}
   If one thinks the preview feature is not useful, uncheck this check-box.
   so that one doesn't see the preview window.  As a bonus, one can see more
   of the instrument file-name.

   \itempar{Preview pane}{Open Instrument!preview pane}
   XML Preview.
   This box can show the beginning of the XML data of an instrument file.
   \textbf{Bug:}
   \index{bugs!compressed XML preview}
   It seems to show the XML only if the XML is not compressed.

   \itempar{Show hidden files}{Open Instrument!show hidden files}
   Shows file that are hidden.  Not sure how useful this feature is;
   who would hide a \textsl{Yoshimi} instrument file?

   \itempar{Directory Bar}{Open Instrument!directory bar}
   Provides an alternate way to move up through the directory structure.

   \itempar{Filename}{Open Instrument!filename}
   File Name.
   Provides the full path to the instrument file.

   \itempar{OK/Cancel}{Open Instrument!ok/cancel}
   We don't really need to discuss the \textbf{OK} and \textbf{Cancel}
   buttons, do we?

\subsubsection{Menu / Instrument / Save Instrument...}
\label{subsubsec:menu_instrument_save}

   This menu entry brings up a prompt to save a new instrument within the
   user's file system.
   It has all of the user-interface elements of the "Open Instrument"
   dialog shown in
   \figureref{fig:open_instrument_dialog}
   in \sectionref{subsubsec:menu_instrument_open}.
   Like that dialog, it is not dependent on the current root or current bank.
   However, if nothing has changed, then a "Nothing to Save!" prompt (not
   pictured) is shown.

   With \textsl{ZynAddsubFX} and older versions of \textsl{Yoshimi},
   it was possibly to end up with unnamed instruments. Since version
   1.3.4, \textsl{Yoshimi} will trap such an occurrence and name it
   'No Title'; it will not let one save the unedited default sound.

\subsubsection{Menu / Instrument / Show Instruments...}
\label{subsubsec:menu_instrument_show}

   Instruments are stored in banks. The banks (and current bank setting)
   are loaded/saved
   automatically by the program, so one doesn't have to worry about saving the
   banks before the program exits. On program start, the last used bank is
   loaded. A single bank can store up to 128 instruments. 
   However, there is space for a number of additional
   instruments in the bank, the extended-program section, to allow up to 160
   instruments in a bank.

   When the \textbf{Show Instruments...} button is selected, a dialog comes
   up that shows all of the instruments present in the currently-selected
   bank.
   
\begin{figure}[H]
   \centering 
   \includegraphics[scale=0.75]{1.3.6/Alex_J_bank_instruments.png}
   \caption[Instruments in Current Bank]{Instruments in Current Bank 1.3.6}
   \label{fig:show_alex_j_bank}
\end{figure}

   As \figureref{fig:show_alex_j_bank}
   shows, this is a very complex dialog with a lot of options.
   Note how \textsl{Yoshimi} now shows the color codings for the
   synth-sections used in each instrument:
   red for ADDsynth, blue for SUBsynth, and
   green for PADsynth.

   Also note how the numbers at the beginning of the filenames are used as
   an "instrument" or "program" number.  These numbers can be used in MIDI
   Program Change commands.
   
   All of the files with filenames starting with 4-digit numbers will be
   shown in the slot corresponding number.  Those without numbers will start
   with numbers at 129 or above ("extended program change").  One should give
   them numbers by renaming them outside of \textsl{Yoshimi}, then reloading
   the bank.

   \index{extended program}
   Note that MIDI CC
   (see \sectionref{paragraph:menu_yoshimi_settings_ccs})
   can be set to access voices from 129 to 160.
   All the Bank controls in the \textbf{MIDI} settings tab take immediate
   effect when set.
   Bank and program changes can be completely disabled in the settings tab;
   some hardware synths don't play nice with it.

   Learning how to use the Instruments dialog is an important way to make
   instruments easier to manage, and so this will be a long discussion.

%  An important pair of concepts in \textsl{Yoshimi} are
%  \textsl{banks} and \textsl{roots}.  These concepts are described in
%  \sectionref{subsec:concepts_banks_and_roots}.

%  A bank has 3 modes in \textsl{ZynAddSubFX}: 

%  \begin{enumber}
%     \item \textbf{READ}.
%        The instrument is loaded from the bank to the current part.
%     \item \textbf{WRITE}.
%        The instrument is written to the bank.
%     \item \textbf{CLEAR}.
%        The instrument from the bank is cleared (removed).
%  \end{enumber}

%  Pressing the left mouse button on a slot reads/writes/clears the
%  instrument from/to it (according to the current mode).
   
%  Pressing the right mouse button on a slot changes its name.

%  The setup in \textsl{Yoshimi} is a bit different than in
%  \textsl{ZynAddSubFX}.
%  Observe \figureref{fig:show_ca_bank}.
%  It shows a bank loaded from a directory containing customs
%  banks from one of the authors of this document.

   Note that this dialog has been modified in recent versions of
   \textsl{Yoshimi}.

   Here is a list of the user-interface items in the instruments/banks dialog.

   \begin{enumber}
      \item \textbf{Bank Names}
      \item \textbf{Roots}
      \item \textbf{Banks}
      \item \textbf{Instrument and Bank Matrix}
      \item \textbf{SELECT}
      \item \textbf{RENAME}
      \item \textbf{SAVE}
      \item \textbf{DELETE}
      \item \textbf{SWAP}
      \item \textbf{Show synth engines}
         (was \textbf{Show PADsynth status})
      \item \textbf{Close}
   \end{enumber}

   \setcounter{ItemCounter}{0}      % Reset the ItemCounter for this list.

   \itempar{Bank Names}{instruments!bank names}
   Instruments Bank Name.
   Basically, each bank is a directory name, with a number prepended.
   The banks are found under the current root, which is a also a directory
   name, and is the name of the parent directory of a set of banks.
   Here is the Bank Names drop-down list for "my" setup, which has the
   default banks plus a lot of banks found around the Internet:

\begin{figure}[H]
   \centering 
   \includegraphics[scale=0.75]{menu/Instrument/bank-list.jpg}
   \caption[A Sample Bank List]{A Sample Bank List}
   \label{fig:bank_list}
\end{figure}

   And here is the directory listing associated with it, in the order
   produced by the UNIX/Linux "ls -1" (list single-column) command (shown in
   two columns to save space):

   \begin{verbatim}
      Alex_J                        Noises
      Arpeggios                     Organ
      Bass                          Pads
   '  Bells                         Piano
      Brass                         Plucked
      C_Ahlstrom                    RB Zyn Presets
      chip                          README
      Choir_and_Voice               Reed_and_Wind
      Chromatic Percussion          Rhodes
      Cormi_Collection              Splited
      Drums                         Strings
      Drums_DS                      Synth
      Dual                          SynthPiano
      Electric Piano                Test
      Fantasy                       The_Mysterious_Bank
      Flute                         the_mysterious_banks
      folderol collection           Vanilla
      Guitar                        VDX
      Internet Collection           Will_Godfrey_Collection
      Laba170bank                   Will_Godfrey_Companion
      Leads                         Will_J_Godfrey_Collection
      Louigi_Verona_Workshop        x31eq.com
      Misc                          XAdriano Petrosillo
      Misc Keys                     Zen Collection
      mmxgn Collection
   \end{verbatim}

   The first thing to note is that there are only 128 \textsl{Yoshimi} banks
   supported in a \textsl{Yoshimi} root.  The list above takes up about half
   of the available slots, so it might be time to move some of those banks
   to a new root directory.

   The numbers in the drop-down list are generated by \textsl{Yoshimi} the
   first time it sees a new root path or a new bank within the root path.
   Once set, these numbers will never change unless one actually moves them
   around (using the \textbf{SWAP} button).

   The bank number is also the MIDI ID for the bank;
   one can be sure that it will always
   be there for bank changes, no matter how many banks are added later.
   \textsl{Yoshimi} always lists the banks in ID order, not alphabetical
   order, so one can group them sensibly and permanently.
   However, at first-time creation \textsl{Yoshimi} sets the IDs in
   alphabetical order and tries to space them evenly over the range to
   provide some wiggle room.                                        

   Selecting one of the items in this drop-down list selects the bank and
   loads it into the Banks dialog.

   \index{anti-auto-clutter}
   Right-clicking on a bank causes the banks list dialog
   to disappear, and be replaced by the bank dialog.  While this
   behavior might be surprising, it is part of the anti-auto-clutter feature
   of \textsl{Yoshimi}.

   \itempar{Roots}{instruments!roots}
   Instruments Roots Button.
   Shows a list of directories that can serve as "root" directories.
   The "Bank Root Paths" dialog shown in
   \figureref{fig:show_banks_roots} shows
   the system root (e.g. \texttt{/usr/share/yoshimi/banks}) and
   a user's home location for his/her banks and roots.

   \itempar{Banks}{instruments!banks}
   Banks Button.
   This item brings up a Banks dialog showing all of the banks present in the
   current root.
   It is an alternative to using the \textbf{Bank Names} dropdown list.

   \itempar{Instrument and Bank Matrix}{instruments!bank matrix}
   Instruments Bank Matrix.
   Shows the instruments that are in the currently selected bank
   (directory).

   \itempar{SELECT}{instruments!SELECT}
   Instruments SELECT.
   When this button is selected, then clicking on an instrument selects that
   instrument as the instrument for the current Part active in the main
   window.

   \itempar{RENAME}{instruments!RENAME}
   Instruments RENAME.
   When this button is selected, then clicking on a bank brings
   up a small dialog to rename the clicked-on bank.
   However, one might also experience the following warning message:

   \begin{verbatim}
      This instrument file cannot be changed
   \end{verbatim}

   \itempar{SAVE}{instruments!SAVE}
   Instruments SAVE.
   When this button is selected, then clicking on a bank saves
   the instruments as currently configured.
   However, one might also experience the following warning message:

   \begin{verbatim}
      This instrument file cannot be changed
   \end{verbatim}

   \itempar{DELETE}{instruments!DELETE}
   Instruments DELETE.
   Selecting this button and clicking an empty bank entry does nothing.
   Selecting this button and clicking an existing bank entry brings up a
   small dialog asking one if this bank is really to be deleted.
   However, one might also experience the following warning message:

   \begin{verbatim}
      This instrument file cannot be changed
   \end{verbatim}

   \itempar{SWAP}{instruments!SWAP}
   Instruments SWAP.
   Selecting this button, then selecting one bank, and then another,
   swaps the numbering and postion of the selected banks.
   However, one might also experience the following warning message:

   \begin{verbatim}
      This instrument file cannot be changed
   \end{verbatim}

   \itempar{Show synth engines}{instruments!show engines}
   If enabled, then the usage of each of the \textsl{Yoshimi} synthesis
   engines is indicated by color coding, as shown in the figure above.

   \itempar{Close}{instruments!Close}
   Closes the window.

   Here is a more conventional view of instruments, supplied with
   \textsl{Yoshimi}, shown in
   \figureref{fig:show_pads_bank}.

\begin{figure}[H]
   \centering 
%  \includegraphics[scale=0.75]{menu/Instrument/show-pads-bank.jpg}
   \includegraphics[scale=0.75]{1.3.6/show_pads_bank.png}
   \caption[Show Pads Instruments]{Show Pads Instruments}
   \label{fig:show_pads_bank}
\end{figure}

   Note that many of these Pads instruments also use the Add and Sub
   components as well.

\subsubsection{Menu / Instrument / Show Banks...}
\label{subsubsec:menu_instrument_show_banks}

   This menu entry brings up a dialog that shows all of the banks present in
   the current root.

\begin{figure}[H]
   \centering 
   \includegraphics[scale=0.75]{1.3.6/show_CA_banks.png}
   \caption[Show Banks]{Show Banks in Current Root}
   \label{fig:show_ca_banks}
\end{figure}

   This figure illustrates a setup where the installed banks were combined with
   banks downloaded from various web sites.
   The following list shows that the interface elements in the banks dialog
   are slightly different from the instruments dialog.

   \begin{enumber}
      \item \textbf{Roots}
      \item \textbf{Current Bank} (passive display element)
      \item \textbf{Instruments}
      \item \textbf{SELECT}
      \item \textbf{RENAME}
      \item \textbf{ADD}
      \item \textbf{DELETE}
      \item \textbf{SWAP}
      \item \textbf{Close}
   \end{enumber}

   \setcounter{ItemCounter}{0}      % Reset the ItemCounter for this list.

   \itempar{Roots}{banks!roots}
   Banks Roots.
   "Roots" button.
   Shows a list of directories that can serve as "root" directories.

   \itempar{current bank}{banks!current bank}
   \index{current!bank}
   Current Bank.  Simply indicates the current bank via color-highlighting.
   Note that one can left-click on a bank in this dialog to make it the
   current bank.  This setting is saved across \textsl{Yoshimi} restarts.

   \itempar{Instruments}{banks!instruments}
   Banks Instruments.
   \index{current!bank}
   Brings up a banks dialog that shows the instruments in the current bank.

   \itempar{SELECT}{banks!SELECT}
   Banks SELECT.
   When this button is selected, then clicking on a bank makes it the current
   bank.

   (Although we don't show a figure for it, note that some banks provide
   instruments with numbers in the extended program-change range (above
   127) prepended to the file-names.)

   \itempar{RENAME}{banks!RENAME}
   Banks RENAME.
   When this button is selected, then clicking on a bank brings
   up a small dialog to rename the clicked-on bank.
   However, one might also experience the following warning message:

   \begin{verbatim}
      This bank directory cannot be changed
   \end{verbatim}

   \itempar{ADD}{banks!ADD}
   Banks ADD.
   Selecting this button and clicking an empty bank entry brings up a small
   dialog to create a new empty bank name for that entry.
   If one clicks on an existing bank entry, then a small dialog comes up
   stating that the bank number selected is already in use.
   However, one might also experience the following warning message:

   \begin{verbatim}
      This bank directory cannot be changed
   \end{verbatim}

   \itempar{DELETE}{banks!DELETE}
   Banks DELETE.
   Selecting this button and clicking an empty bank entry does nothing.
   Selecting this button and clicking an existing bank entry brings up a
   small dialog asking one if this bank is really to be deleted.
   However, one might also experience the following warning message:

   \begin{verbatim}
      This bank directory cannot be changed
   \end{verbatim}

   \itempar{SWAP}{banks!SWAP}
   Banks SWAP.
   Selecting this button, then selecting one bank, and then another,
   swaps the numbering and postion of the selected banks.
   This button is good for minor reorganization of the bank numbers.

\subsubsection{Menu / Instrument / Show Root Paths...}
\label{subsubsec:menu_instrument_show_root_paths}

\begin{figure}[H]
   \centering 
   \includegraphics[scale=0.75]{menu/Instrument/show-banks-roots.jpg}
   \caption[Show Root Paths]{Show Root Paths}
   \label{fig:show_banks_roots}
\end{figure}

   \setcounter{ItemCounter}{0}      % Reset the ItemCounter for this list.

   \itempar{Add root directory...}{Root Paths!add directory}
   Show Root Paths Add Root Directory.
   To add a bank root path:

   \textsl{Yoshimi} (as installed by Debian Linux) provides a default bank at
   \texttt{/usr/share/yoshimi/banks}.
   To add one's own directory, navigate to "Yoshimi / Instrument / Show Root
   Paths ...".  Then click on "Add root directory...".

   Once selected, one will see that \texttt{/usr/share/yoshimi/banks}
   is marked with an asterisk.  One can select the new root directory,
   and make it current by clicking the "Make current" button.
   Then the Banks dialog will show all the banks in that directory, one bank
   per subdirectory (each subdirectory "is" a bank).

\begin{figure}[H]
   \centering 
   \includegraphics[scale=0.75]{menu/Instrument/new-directory.jpg}
   \caption{Add Root Directory}
   \label{fig:add_root_directory}
\end{figure}

   \itempar{Remove root directory...}{Root Paths!remove directory}
   Show Root Paths Remove Root Directory.
   If a path is selected, then this button is active, and can be used to
   delete the selected path from the "root paths" list.

   \itempar{Make current}{Root Paths!make current}
   Show Root Paths Make Current.
   \index{current!root}
   This button marks the currently-selected path as the "current root" path.

   \itempar{Open current}{Root Paths!open current}
   Show Root Paths Open Current.
   This button opens the current root path.

   \itempar{Change ID}{Root Paths!change ID}
   Show Root Paths Change ID.

   Values: \texttt{0* to 127}

   \textsl{
   We need to know more about how this ID can be used.
   Is it a way to make the path selectable via an extended MIDI control, or
   some other automation method?
   }

\subsection{Menu / Parameters}
\label{subsec:menu_parameters}

   \textsl{Yoshimi} stamps its parameter XML files with its own major and
   minor version numbers so it is possible to tell which version created the
   files, or whether they were created by \textsl{ZynAddSubFX}.

\begin{figure}[H]
   \centering 
   \includegraphics[scale=1.0]{menu/yoshimi-menu-parameters.jpg}
   \caption{Yoshimi Menu, Parameters}
   \label{fig:yoshimi_menu_parameters}
\end{figure}

   \begin{enumber}
      \item \textbf{Recent}
      \item \textbf{Open}
      \item \textbf{Save}
      \item \textbf{Clear}
   \end{enumber}

\subsubsection{Menu / Parameters / Recent}
\label{subsubsec:menu_parameters_recent}

   This menu entry provides access to parameter settings files that have
   been recently used.

\begin{figure}[H]
   \centering 
   \includegraphics[scale=1.0]{menu/Parameters/recent-parameters.jpg}
   \caption{Yoshimi Menu, Recent Parameters}
   \label{fig:yoshimi_menu_recent_parameters}
\end{figure}

   Selecting one of the items in this list causes it to be loaded.
   All the settings, including effects and instruments, are loaded.

\subsubsection{Menu / Parameters / Open}
\label{subsubsec:menu_parameters_open}

   Opens a standard \textsl{Yoshimi} dialog for selecting a
   \texttt{*.xmz} file.
   It is similar to
   \figureref{fig:open_instrument_dialog},
   as can be seen in the next figure.

\begin{figure}[H]
   \centering 
   \includegraphics[scale=0.75]{menu/Parameters/open-parameters.jpg}
   \caption{Yoshimi Menu, Open Parameters}
   \label{fig:yoshimi_menu_open_parameters}
\end{figure}

\subsubsection{Menu / Parameters / Save}
\label{subsubsec:menu_parameters_save}

   This menu entry provides a way to save parameter settings in
   a file.

   In parameter sets, \textsl{Yoshimi} will save named-but-disabled patches.
   Currently, \textsl{ZynAddSubFX} does not, so be aware when transferring
   data between the two synthesizers.

\begin{figure}[H]
   \centering 
   \includegraphics[scale=0.75]{menu/Parameters/save-parameters.jpg}
   \caption{Yoshimi Menu, Save Parameters}
   \label{fig:yoshimi_menu_save_parameters}
\end{figure}

   TODO:  What is the full extent of parameters saved?

   If nothing has changed, then the following dialog is shown.

\begin{figure}[H]
   \centering 
   \includegraphics[scale=0.75]{menu/Parameters/nothing-to-save.jpg}
   \caption{Yoshimi Menu, Nothing to Save}
   \label{fig:yoshimi_menu_nothing_to_save_parameters}
\end{figure}

\subsubsection{Menu / Parameters / Clear}
\label{subsubsec:menu_parameters_clear}

   Using this button brings up the following dialog.  Once clicked,
   \textsl{Yoshimi} seems to revert to its default "Simple Sound" setup.

\begin{figure}[H]
   \centering 
   \includegraphics[scale=0.75]{menu/Parameters/clear-parameters.jpg}
   \caption{Yoshimi Menu, Clear Parameters}
   \label{fig:yoshimi_menu_clear_parameters}
\end{figure}

   TODO:  What is the full extent of parameters cleared?

\subsection{Menu / Scales}
\label{subsec:menu_scales}

   \textsl{Yoshimi} is a microtonal synthesizer, and is capable of a wide
   range of microtonal scales.

   At present, we're not too experienced with this feature.

\begin{figure}[H]
   \centering 
   \includegraphics[scale=0.75]{menu/yoshimi-menu-scales.jpg}
   \caption{Yoshimi Menu, Scales}
   \label{fig:yoshimi_menu_scales}
\end{figure}

   \begin{enumber}
      \item \textbf{Load Scale Settings...}
      \item \textbf{Save Scale Settings...}
      \item \textbf{Show Scale Settings...}
   \end{enumber}

\subsubsection{Menu / Scales / Load}
\label{subsec:menu_scales_load}

\begin{figure}[H]
   \centering 
   \includegraphics[scale=0.75]{menu/Scales/open-scales.jpg}
   \caption{Yoshimi Menu, Open Scales}
   \label{fig:yoshimi_menu_open_scales}
\end{figure}

\begin{figure}[H]
   \centering 
   \includegraphics[scale=0.75]{menu/Scales/failed-to-load-scl-file-vice-xsz.jpg}
   \caption{Yoshimi Menu, Failed to Load Scales}
   \label{fig:yoshimi_menu_failed_to_load_scales}
\end{figure}

\subsubsection{Menu / Scales / Save}
\label{subsec:menu_scales_save}

   This dialog opens a stock file-dialog to allow the saving of
   \texttt{*.xsz} files.

\subsubsection{Menu / Scales / Show}
\label{subsec:menu_scales_show}

\begin{figure}[H]
   \centering 
   \includegraphics[scale=1.0]{menu/Scales/scale-settings-microtonal.jpg}
   \caption{Yoshimi Menu, Scales Settings}
   \label{fig:yoshimi_menu_scales_settings}
\end{figure}

\paragraph{Scales Basic Settings}
\label{paragraph:menu_scales_basic_settings}

   This item controls the micro-tonal capabilities of \textsl{Yoshimi} and
   some other settings related to tuning. 
   The last entry in the tunings list represents one octave.
   All other notes are deduced from these settings.

   \setcounter{ItemCounter}{0}      % Reset the ItemCounter for this list.

   \itempar{Microtonal}{Enable Microtonal}
   Enable Microtonal Scales.
   When disabled, the synthesizer will use equal-temperament, 12 notes per
   octave.  Otherwise, one can input any scale one desires.

   Values: \texttt{Off*, On}

   \itempar{"A" Freq.}{"A"}
   Frequency of the "A" Note.
   Sets the frequency of the "A" key. The standard is 440.0 Hz.

   Values: \texttt{440*}

   \itempar{"A" Note}{"A" MIDI}
   Sets the MIDI Value of the "A" Note.

   Values: \texttt{0 to 127, 69*}

   \itempar{Invert Keys}{keys}
   Allows the keys to be inverted, so that higher-valued keys play lower
   notes.

   Values: \texttt{Off*, On}

   \itempar{Center}{center}
   Center for Inverted Keys.
   This is the center where the notes frequencies are turned upside-down if
   \textbf{Invert keys} is enabled.
   If the center is 60, the note 59 will become 61, 58 will become 62, 61
   will become 59, and so on.

   Values: \texttt{0 to 127, 60*}

   \itempar{Name}{mapping}
   Name of the Mapping.
   For example, the default mapping is called "12tET".

   \itempar{Shift}{key shift}
   Key Shift.
   Shift the scale. If the scale is tuned to A, one can easily tune it to
   another key.

   Values: \texttt{-63 to 64, 0*}

   \itempar{Comment}{comment}
   Comment for Key Mapping.
   Provides a comment or a description of the scale.
   By default, this is "Equal Temperament 12 notes per octave.

   \itempar{Tunings}{tuning}
   Tunings.
   Here one can input a scale by entering all the tunings for one octave. 
   One can enter the tunings in two ways: 

   \begin{enumber}
      \item As the number of cents (1200 cents=1 octave) as a float number
      like "100.0", "123.234"
      \item As a proportion like "2/1" which represents one octave, "3/2" a
      perfect fifth, "5734/6561".  "2/1" is equal to "1200.0" cents.
   \end{enumber}

   The default is a series of values:
   \texttt{0100.0, 0200.0, ..., 1100.0, 2/1}.

   \itempar{Retune}{retune}
   Retune.
   TODO:  What does this button do?

   \itempar{nts./oct.}{Notes per Octave}
   Notes Per Octave.

   Values: \texttt{12*} (range not yet known)

   \itempar{Import .SCL file}{scale file}
   Import Scala files.
   Scala is a powerful application for experimentation with musical tunings
   (intonation scales, micro-tonal,...etc.). From its home page \cite{scala},
   one can download more than 2800 scales which one can import directly into
   \textsl{Yoshimi}.  Note that the zip file \textsl{must} be unzipped with
   the \texttt{-aa} ("autoconvert") option.

    \begin{verbatim}
      $ unzip -aa scales.zip
    \end{verbatim}

\begin{figure}[H]
   \centering 
   \includegraphics[scale=0.75]{menu/Scales/import-scl-file.jpg}
   \caption{Yoshimi Menu, Scales, Import File}
   \label{fig:yoshimi_menu_scales_import_file}
\end{figure}

   \itempar{Import .scl file}{scl file}
   This item is a standard file dialog for reading
   a \texttt{*.scl} file.

\begin{figure}[H]
   \centering 
   \includegraphics[scale=0.75]{menu/Scales/import-kbm-file.jpg}
   \caption{Yoshimi Menu, Scales, Import Keyboard Map}
   \label{fig:yoshimi_menu_scales_import_keyboard_map}
\end{figure}

   \itempar{Import .kbm file}{kbm file}
   This item is a standard file dialog for reading
   a \texttt{*.kbm} file.

   \itempar{Close, Scales Dialog}{close scales}

   The items related to the \textbf{Keyboard Mapping} are discussed
   separately in the next section.

\paragraph{Keyboard Mapping}
\label{paragraph:menu_scales_keyboard_mapping}

   One can set the MIDI keyboard mapping to scale-degree mapping.
   This is used if the scale has more or less than 12 notes/octave.
   One can enable the mapping by pressing the \textbf{ON} check-box.

   \begin{enumber}
      \item \textbf{ON}
      \item \textbf{First Note}
      \item \textbf{Last Note}
      \item \textbf{Midle Note}
      \item \textbf{Map}
      \item \textbf{Map Size}
   \end{enumber}

   \setcounter{ItemCounter}{0}      % Reset the ItemCounter for this list.

   \itempar{Scales!ON}{scales flag}

   Values: \texttt{Off*, On}

   \itempar{Scales!First Note}{first note}
   First MIDI Note Number.
   Keys below this value are ignored.

   Values: \texttt{0* to 127}

   \itempar{Scales!Last Note}{last note}
   Last MIDI Note Number.
   Keys above this value are ignored.

   Values: \texttt{0 to 127*}

   \itempar{Scales!Middle Note}{middle note}
   Middle note where scale-degree 0 is mapped to;
   the middle note represents the note where the formal octave starts.
   Note the misspelling of "middle".

   Values: \texttt{0 to 127*}

   \itempar{Scales!Map}{map}
   Scales map.  This is the input field where the mappings are entered.
   The numbers represent the order (degree) entered on
   \textbf{Tunings Input} field, with the first value being 0.
   This number must be less than the number of notes per octave (since
   the values start at 0).
   If one doesn't want a key to be mapped, one enters an "x" instead of a
   number.

   Values: \texttt{0 to 11}

   \itempar{Scales!Map Size}{map size}
   Provides the size of the scale-map.

   Values: \texttt{12}

\subsection{Menu / State}
\label{subsec:menu_state}

   \textsl{Yoshimi} state is saved in files with the extension
   \texttt{.state}.  These files are also XML files.

   TODO: What is the difference between "state" and "parameters"?
   Which one is all-inclusive?  What items are saved in each?

   \begin{enumber}
      \item \textbf{Save}
      \item \textbf{Load}
   \end{enumber}

   As the following figures show, state files are normally stored in the
   user's \texttt{.config/yoshimi/yoshimi.state} file.

\begin{figure}[H]
   \centering 
   \includegraphics[scale=1.0]{menu/State/save-state-file.jpg}
   \caption{Yoshimi Menu, State Save}
   \label{fig:yoshimi_menu_state_save}
\end{figure}

   This item is a standard \textsl{Yoshimi} file dialog.

\begin{figure}[H]
   \centering 
   \includegraphics[scale=1.0]{menu/State/load-state-file.jpg}
   \caption{Yoshimi Menu, State Load}
   \label{fig:yoshimi_menu_state_load}
\end{figure}

   This item is a standard \textsl{Yoshimi} file dialog.

%-------------------------------------------------------------------------------
% vim: ts=3 sw=3 et ft=tex
%-------------------------------------------------------------------------------
