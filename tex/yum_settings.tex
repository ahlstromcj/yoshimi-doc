%-------------------------------------------------------------------------------
% yum_settings
%-------------------------------------------------------------------------------
%
% \file        yum_settings.tex
% \library     Documents
% \author      Chris Ahlstrom
% \date        2015-05-15
% \update      2012-05-15
% \version     $Revision$
% \license     $XPC_GPL_LICENSE$
%
%     Provides the Settings section of yoshimi-advanced-reference-manual.tex,
%     which covers stock settings user-interface items.
%
%-------------------------------------------------------------------------------

\section{Stock Settings Elements}
\label{sec:stock_settings_elements}

   This section collects all of the settings values and small user-interface
   items that one will find for audio parameters in the \textsl{Yoshimi} GUI.
   Sometimes the labels and tool-tips in the application are a bit too brief to
   understand.  One will find their full meanings, their tricks, and usage
   notes in this section.
   This section also covers the sub-panels that provide the settings.
   Many of these sub-panels are used in many places in \textsl{Yoshimi},
   not only as user-interface elements, but as presets that can be saved and
   loaded.
   By describing the deep details of these sub-panels
   here, we can refer to them when
   describing how to set up specific sounds in
   \textsl{Yoshimi}.
   Much of this material comes from
   \url{http://sourceforge.net/zynaddsubfx/Doc}
   and has been reorganised and expanded.

\subsection{Settings Features}
\label{subsec:stock_settings_ui_features}

   This section notes some minor interface and synthesizer features that may
   be seen throughout \textsl{Yoshimi}.

\subsubsection{Mouse Features}
\label{subsubsec:stock_settings_mouse_features}

   \index{new!right-click}
   \index{mouse!right-click}

%  In the AddSynth dialogs and other dialogs with nested windows,
%  if one right-clicks on a button that opens a child
%  window, the current window is closed. Conversely, if one right-clicks on the
%  \textbf{Close} button of a child window, it will re-open the parent.  This
%  feature is especially useful when going up and down the AddSynth stack, and
%  avoids having a screenful of intermediate windows.
%  Note that this feature does \textsl{not} apply to the main window.

    The right mouse click is used for setting the default value of a control,
    where it acts like a "reset" button, and (combined with CTRL) is also
    used to bring up the \textbf{MIDI Learn} dialog for controls that can be
    MIDI-learned.

   When traversing window stacks, a right click will close one window as it
   opens the next. Closing such a window with the right button will re-open its
   parent.

\subsubsection{Tooltips}
\label{subsubsec:stock_settings_tooltips}

   Like many applications, \textsl{Yoshimi} provides tooltips to help the user
   navigate the many controls and data fields in the user-interface.  Many of
   the controls now have active tooltips that show the current value of the
   control when one hovers over it, so one no longer must click the control to
   see its value (and accidentally change it at the same time). Many have
   real-world data units, such as \textbf{dB}, \textbf{Hz}, \textbf{mS} etc.

\subsubsection{Title Bars}
\label{subsubsec:stock_settings_elements_title_bars}

   The title bars of all editing windows display both the part number and the
   current name of the instrument one is working on.  In the
   \textbf{ADDsynth Oscillator Editor}, one also sees the
   voice number of the oscillator one is editing.
   Title bars also include the kit entry number if that part has a kit enabled.

\subsubsection{Colour Coding}
\label{subsubsec:stock_settings_elements_color_coding}

   A GUI enhancement for \textsl{Yoshimi 1.3.5} is colour-coded identification
   of an instrument's use of ADD-, SUB-, and PAD-synth engines, no matter where
   in the instrument's kit they may be. This can be enabled/disabled in the
   mixer panel. It does slow down \textsl{Yoshimi}'s startup, but due to the
   banks reorganisation (done some time ago) it causes no delay in changing
   banks/instruments once \textsl{Yoshimi} is up and running.  Some saved
   instruments seem to have had their "Info" section corrupted.
   \textsl{Yoshimi} can detect this issue, and step over it to find the true
   status. Also, if one resaves the instrument, not only will the PADsynth
   status be restored, but ADDsynth and SUBsynth will be included, allowing a
   faster scan next time.

\subsubsection{Rotary Knobs}
\label{subsubsec:stock_settings_elements_knobs}

   \index{knobs}
   Visual rotary knobs are used for modifying numerical parameters in the
   user-interface.
   Horizontal, as well as vertical, mouse movements will adjust the knob.
   Mouse clicks also can adjust the knob.

   \begin{itemize}

      \item \textbf{Coarse Control}.
         \index{knobs!coarse control}
         \index{knobs!left-drag}
         When rotated using the left mouse button, the rotary knobs give a
         coarse control of the numerical settings of the knob.

      \item \textbf{Fine Control}.
         \index{knobs!fine control}
         \index{knobs!right-drag}
         When a knob is rotated using the middle mouse button, the rotary knobs
         give a finer control of the numerical settings of the knob.

      \item \textbf{Scroll Wheel}.
         \index{knobs!fine control, scroll}
         \index{knobs!scroll-wheel}
         One can also use the mouse scroll wheel to adjust rotary controls,
         which gives better control than using the mouse pointer.

      \item \textbf{Super Fine Control}.
         \index{knobs!super-fine, scroll}
         \index{knobs!scroll-wheel}
         If the Ctrl key is held at the same time as the wheel is scrolled, the
         control is \textsl{extremely} fine.
%        \index{knobs!super-fine, middle}
%        \index{knobs!middle-drag}
%        If the middle button is dragged, the
%        control is \textsl{extremely} fine.

   Note that the right mouse button can be used for setting the default value of a
   control or for the initiation of
   MIDI learn.

% Will:
% For everything except effects, a drag also gives fine adjustment. One day I
% might get effects to do this too - Effects are just weird in every way :(

      \item \textbf{Home Position}.
         \index{right-click}
         \index{mouse!right-click}
         \index{knobs!home position}
         \index{knobs!right-click}
         A right-click sends the knob to the home position (and sets its value
         to the home value). For setting like \textbf{Pan} and
         \textbf{Detune} it is the middle position; for other settings,
         it is whatever the default value is for that setting.

   \end{itemize}

   The fact that \textsl{every} control can now be homed with a right mouse
   click means there is no longer a need for the few "Zero" and "Reset" buttons
   dotted around, so they are all gone as of version 1.4.0.

\subsubsection{Sliders}
\label{subsubsec:stock_settings_elements_sliders}

   \index{right-click}
   \index{mouse!right-click}
   \index{sliders}
   \index{sliders!right-click}
   For both horizontal and vertical sliders, if one holds down the right mouse
   button, the thumb will go to it's default position.  The same thing will
   happen if one clicks on the track with the right button.

   \textsl{Yoshimi} has changed some rotary controls or rollers to sliders.
   These controls pack better without looking crowded, are easier to manage,
   with clearer indication.

   Since version 2.3.0 one will see colour changes of knob pointers and slider 'peg' centres if the controls are not at their default positions. The defaults are with black knob pointers and green slider pegs.
   \begin{figure}[H]
   \centering
   \includegraphics[scale=0.75]{2.3.0/defaults.png}
   \caption{Controls Changed}
   \label{fig:changed_controls}
\end{figure}

\subsubsection{Presets}
\label{subsubsec:stock_settings_elements_presets}

   \index{preset}
   The \textsl{Yoshimi} concept of presets is very powerful.

   Absolutely every user-interface section that has dark blue \textbf{C}
   and \textbf{P} buttons can be
   stored in the \texttt{presets} directory. That includes entire Addsynth
   engines! When one looks at the copy/paste buffer, one sees only items that
   are relevant to the group that the C/P buttons are in.

   As one wants to save, as well as load, these presets, it makes sense to copy
   all the default ones to preferred location
   \texttt{\textasciitilde/.local/share/yoshimi/presets}.
   That makes them fully accessible, but
   tucked away out of sight.  \textsl{Yoshimi} creates this directory at first
   time start up.
   Preset files allow one to save the
   settings for any of the components which support copy/paste operations.
   This is done with preset files (\texttt{.xpz}), which get stored in the
   folders indicated by \textbf{Paths / Preset Dirs...}.
   Note that the number of preset directories that can be set is limited to 128 (like roots and banks).

\subsubsection{Automation}
\label{subsubsec:stock_settings_elements_automation}

   In \textsl{Yoshimi 1.3.5}, a number of existing, as well as new features
   have come together to give much greater flexibility (especially for
   automation) using standard MIDI messages. These are:

   \begin{enumber}
      \item \textbf{NRPNs}
      \item \textbf{ZynAddSubFX controls}
      \item \textbf{Independent part control}
      \item \textbf{16, 32 or 64 parts}
      \item \textbf{Vector Control}
      \item \textbf{Direct part stereo audio output}
   \end{enumber}

   \setcounter{ItemCounter}{0}      % Reset the ItemCounter for this list.

   \itempar{NRPNs}{automation!NRPNs}
   NRPNs can handle individual bytes appearing in either order, and usually the
   same with the data bytes. Increment and decrement is also supported as
   graduated values for both data LSB and MSB. Additionally, the ALSA
   sequencer's 14-bit NRPN blocks are supported.

   \itempar{controls}{automation!controls}
   System and Insertion Effect controls are fully supported, with extensions
   to allow one to set the effect type and (for insertion effects) the
   destination part number.

   \itempar{Part control}{automation!part control}
   Independent part control enables one to change instrument, volume, pan, or
   indeed any other available control of just that part, without affecting any
   others that are receiving the same MIDI channel. This can be particularly
   interesting with multiply layered sounds. There are more extensions planned.

   \itempar{16/32/64 Parts}{automation!16/32/64 parts}
   With 32 and 64 parts, it helps to think of 2 or 4 rows of 16. When one
   saves a parameter block, the number of parts is also saved, and will be
   restored when one reloads.  By default each \textsl{column} has the same
   MIDI channel number, but these can be independently switched around, and
   by setting (say) number 17 taken right out of normal access.

   In tests, \textsl{compiling} for 64 parts compared with 16 parts increased
   processor load by a very small amount when \textsl{Yoshimi} was idling,
   but this becomes virtually undetectable once one has 8 or more instruments
   actually generating output. In normal use, selecting the different formats
   makes no detectable difference, but using the default 16 reduces clutter
   when one doesn't need the extras.

   \itempar{Vector control}{automation!vector control}
   Vector control is based on these parts columns, giving one either 2 (X
   only) or 4 (X + Y) instruments in this channel. Currently the vector
   CCs one sets up can (as inverse pairs) vary any combination of volume, pan,
   and filter cut-off.  More will be added.  To keep the processor load
   reasonable it pays to use fairly simple instruments, but if one has
   sufficient processing power, it would be theoretically possible to set up
   all 16 channels with quite independent vector behaviour!
   Also see
   \sectionref{subsec:vector_dialogs}, for a discussion of the vector
   configuration dialog, and \sectionref{sec:vector}, for an in-depth
   discussion of how vectors work.

   \itempar{Direct part audio}{automation!part audio}
   Direct part audio is JACK-specific, and allows one to apply further
   processing to just the defined part's audio output (which can still output
   to the main L+R if one wants). This setting is saved with patch sets.
   Currently it is only set in the mixer panel window, but it will also
   eventually come under MIDI direct part control.  Again, to reduce
   unnecessary clutter, part ports are only registered with JACK if they are
   both enabled, and set for direct output. However, once set they will remain
   in place for the session to avoid disrupting other applications that may
   have seen them.

\index{new!Volume Velocity Panning}
\subsection{Volume Velocity Panning}
\label{subsec:volume_panning}
   In \textsl{Yoshimi} V 1.7.2 this panel of controls within the amplitude
   groups of AddSynth, Voice, SubSynth and PadSynth has undergone considerable
   change. This can be seen in \figureref{fig:volume_pan_group}. All of these
   have the same actual controls although the settings vary slightly.
   Therefore, rather than repeat the description in all four sections of the
   manual they are detailed here.
\index{new!Panning Width}
   Both 'Volume' and 'Vel Sens' are now defined in actual dB terms.

   \setcounter{ItemCounter}{0}
   \itempar{Volume}{volume}
   Volume.
   Sets the overall/relative volume of the instrument.

   \itempar{Vel Sens}{vel sens}
   Velocity Sensing function.
   Velocity sensing is simply an exponential transformation from the note’s
   velocity to some parameter change.
   Observe \figureref{fig:velocity_sensing_function}.
   It shows how the velocity sensing controls affects the translation of a
   parameter over the whole range of possible note velocities.
   Turn the knob rightmost/maximum to disable this function.

\begin{figure}[H]
   \centering
   \includegraphics[scale=0.25]{zyn/velocity_sensing_function_velf.png}
   \caption{Velocity Sensing Function}
   \label{fig:velocity_sensing_function}
\end{figure}

   \itempar{Pan}{pan}
   Panning.
   This control no longer doubles up as a random panning switch, and is
   defined in terms of plus/minus percentage.

   \itempar{Rand}{random pan}
   Random Panning Switch.
   Switches the random function on or off.

   \itempar{Width}{random width}
   'Width.
   This is a new control also defined in terms of percentage.

   The way this works is that with the default random setting of 100\% the
   apparent position can be anywhere between extreme left and extreme right,
   as before. Set this to 50\% and the range will be between 50\% left of the
   centre position and 50\% right of the centre. However, unlike the previous
   implementation, the panning position is still active, and if one was to set
   that to 25\% left, then the actual range will be between 75\% left and 25\%
   right.

   Obviously it is not possible to go to more than 100\% so if the controls
   were set with Pan at 100\% right, and width was 100\% the actual range
   would be centre to 100\% right. The range is set before the actual
   randomisation so this will still produce an even distribution over the
   range, resulting in an apparent centre of 50\% right. That is, the same as
   if of both controls were set at 50\%

   As far as possible we maintain backwards compatibility, and in this case
   the fact of random panning being set will be correctly interpreted by older
   versions of \textsl{Yoshimi} but random position and width will be lost.

\begin{figure}[H]
   \centering
   \includegraphics[scale=1.0]{1.7.2/volume_pan_group.png}
   \caption{Volume/Velocity/Panning}
   \label{fig:volume_pan_group}
\end{figure}

\subsection{Filter Settings}
\label{subsec:filter_settings}

   This section describes filtering at a high level, in terms of frequency
   responses and other concepts of filtering.
   The end of this section covers a user interface used in filter settings.
   It is a stock-panel re-used in other user-interface elements.
   See \sectionref{subsubsec:filter_parameters_user_interface},
   if one is in a hurry.

   \textsl{Yoshimi}
   offers several different types of filters, which can be used to
   shape the spectrum of a signal. The primary parameters that affect the
   characteristics of the filter are the cutoff, resonance, filter stages, and
   the filter type.

   Filter stages are the number of times that this filter is applied in
   series. So, if this number is 1, one simply has this one filter. If it is
   two, the sound first passes the filter, and the results then pass the same
   filter again. In \textsl{Yoshimi}, the wetness is applied after all
   stages were passed.

\subsubsection{Filter Type}
\label{subsubsec:filter_type}

   \index{filter!type}
   A filter removes or attenuates frequency elements or tones from a signal.
   Filtering changes the character of a signal.

   The basic analog filters that \textsl{Yoshimi}
   offers are shown in \figureref{fig:basic_filter_types}, with
   the center frequency being marked by the red
   line. The state variable filters should look quite similar.

\begin{figure}[H]          % keep figure closer, requires the 'float' package
   \centering
   \includegraphics[scale=0.5]{zyn/zyn_filter_types_filter0.png}
   \caption[Basic Filter Types]{Filter Types, Yoshimi}
   \label{fig:basic_filter_types}
\end{figure}

   \begin{enumber}
      \item A \textbf{low-pass} filter makes the sound more muffled.
      \item A \textbf{band-pass} filter makes the sound more tone-like, and
         sometimes more penetrating, if the total energy in the passband is
         preserved as the bandwidth decreases.
      \item A \textbf{high-pass} filter makes the sound seem sharper or more
         strident.
   \end{enumber}

\subsubsection{Filter Cutoff}
\label{subsubsec:filter_cutoff}

   \index{filter!cutoff}
   The filter cutoff value determines which frequency marks the changing
   point for the filter. In a low pass filter, this value marks the point
   where higher frequencies begin to be attenuated.

\subsubsection{Filter Resonance}
\label{subsubsec:filter_resonance}

   \index{filter!Q}
   \index{filter!resonance}
   The resonance of a filter determines how much excess energy is present at
   the cutoff frequency. In \textsl{Yoshimi},
   this is represented by the Q-factor,
   which is defined to be the cutoff frequency divided by the bandwidth. In
   other words higher Q values result in a much more narrow resonant spike.

   The Q value of a filter affects how concentrated
   the signal’s energy is at the cutoff frequency. The result of differing Q
   values are shown in \figureref{fig:low_q_vs_high_q}.
   For many classical analog sounds, high Q values were used on sweeping
   filters. A simple high Q low pass filter modulated by a strong envelope is
   usually sufficient to get a good sound.

\begin{figure}[H]
   \centering
   \includegraphics[scale=0.5]{zyn/low_q_high_q_filter1.png}
   \caption[Low Q vs. High Q]{The Effect of the Q Value}
   \label{fig:low_q_vs_high_q}
\end{figure}

\subsubsection{Filter Stages}
\label{subsubsec:filter_stages}

   \index{filter!stages}
   \index{filter!order}
   The number of stages in a given filter describes how sharply it is able to
   make changes in the frequency response.
   The more stages, the sharper the filter.
   However, each added stage increases the processor time needed to make the
   filter calculation.

\begin{figure}[H]
   \centering
   \includegraphics[scale=0.5]{zyn/2_pole_8_pole_filter2.png}
   \caption[2 Pole vs. 8 Pole Filter]{The Effect of the Order of a Filter}
   \label{fig:2_pole_vs_8_pole_filter}
\end{figure}

   The affect of the order of the filter can be seen in the figure above.
   This is roughly synonymous with the number of stages of the filter. For
   more complex patches, it is important to realise that the extra sharpness
   in the filter does not come for free, as it requires many more
   calculations to be performed. This phenomena is the most visible in
   SUBsynth, where it is easy to need several \textsl{hundred} filter stages
   to produce a given note.

   There are different types of filters. The number of poles define what will
   happen at a given frequency. Mathematically, the filters are functions which
   have poles that correspond to that frequency. Usually, two poles mean that
   the function has more "steepness", and that one can set the exact value of
   the function at the poles by defining the "resonance value". Filters with
   two poles are also often referred to as \textsl{Butterworth Filters}.

   For the interested, functions having poles means that we are given a
   quotient of polynomials. The denominator has degree 1 or 2, depending on the
   filter having one or two poles. In the file \texttt{DSP/AnalogFilter.cpp},
   \texttt{computefiltercoefs()} sets the coefficients
   (depending on the filter type), and
   \texttt{singlefilterout()} shows
   the whole polynomial (in a formula where no quotient is needed).

\subsubsection{Filter Parameters User Interface}
\label{subsubsec:filter_parameters_user_interface}

\begin{figure}[H]
   \centering
%  \includegraphics[scale=1.0]{subpanels/Filter_Params.png}
%  \includegraphics[scale=0.5]{2.1.0/filter_parameters.png}
   \includegraphics[scale=0.5]{2.1.2/filter.png}
   \caption[Filter Parameters Sub-panel]{Stock Filter Parameters Sub-Panel}
   \label{fig:filter_parameters_subpanel}
\end{figure}

   The user interface for filter parameters is a small stock sub-panel that
   is re-used in a number of larger dialog boxes, as shown in the figure
   above.  This panel has changed slightly with version 1.5.0.
   Let's describe each item of this sub-panel.

%  \figureref{fig:filter_parameters_subpanel}.

\begin{figure}[H]
   \centering
   \includegraphics[scale=1.0]{bottom-panel/instrument-edit/ADD/filter-category.jpg}
   \caption[Filter Categories Dropdown]{Filter Categories, Dropdown Box}
   \label{fig:filter_categories_dropdown}
\end{figure}

   \begin{enumber}
      \item \textbf{Category}
      \item \textbf{Filter Type}
      \item \textbf{C.freq}
      \item \textbf{St.}
      \item \textbf{Q}
      \item \textbf{V.Sns.}
      \item \textbf{VF.Sns.}
      \item \textbf{Gain}
      \item \textbf{Freq.tr.}
      \item \textbf{-/+}
   \end{enumber}

   \setcounter{ItemCounter}{0}      % Reset the ItemCounter for this list.

   \itempar{Category}{filter!category}
   Determines the category of filter to be used.
   There are three categories of filters
   (as shown in the dropdown element shown in
   \figureref{fig:filter_categories_dropdown}).

\begin{enumber}                     % enumber is our arabic numbering style
   \item \textbf{Analog} (the default)
   \item \textbf{Formant}
   \item \textbf{StVarF}
\end{enumber}

   An \textbf{analog} filter
   \index{filter!analog}
   is one that approximates a filter that is based on
   a network of resistors, capacitors, and inductors.

   A \textbf{formant} filter
   \index{filter!formant}
   is a more complex kind of filter that acts a lot
   like the human vocal tract, allowing for sounds that
   are a bit like human voices.  For a description of how formants work,
   see \sectionref{subsubsec:concepts_basics_formants}.

   Using formant filters can be rather like pulling teeth.  Although Paul gave
   a pretty good description of how the vowels and formants interact and are
   laid out, there is an extremely important bit of information missing!  The
   filter lays out the sequence (or sequences if there are multiple vowels) but
   it is the filter \textsl{envelope} that sets the rate and degree to which
   these are traversed. Also, the richer the original harmonic content the more
   pronounced the effect will be (quite useless on sine wave).

   See \sectionref{paragraph:formant_parameters}~for the formant parameters,
   which may have slightly different names from what Paul used.

   A \textbf{state variable} ("StVarF") filter
   \index{filter!state variable}
   \index{filter!StVarF}
   \index{StVarF}
   is a type of active filter.
   The frequency of operation and the Q factor can be varied independently.
   This and the ability to switch between different filter responses make the
   state-variable filter widely used in analogue synthesizers.

   Values: \texttt{Analog*, Formant, StVarF}

   \itempar{Filter Type}{filter!type}
   Selects the type of filter to be used, such as high-pass, low-pass,
   and band-pass.
   See the dropdown element in \figureref{fig:filter_type_dropdown}.

\begin{figure}[H]
   \centering
   \includegraphics[scale=1.0]{bottom-panel/instrument-edit/ADD/filter-filtertype.jpg}
   \caption[Filter Type Dropdown]{Type of Filter Passband, Dropdown Box}
   \label{fig:filter_type_dropdown}
\end{figure}

   Values: \texttt{LPF1, HPF1, LPF2*, HPF2, BPF2, NF2, PkF2, LSh2, HSh2}

   \itempar{C.freq}{cutoff frequency}
   \index{center frequency}
   Cutoff frequency or center frequency.
   This item has various definitions in the literature.
   Usually it refers to the frequency at which the level
   drops to 3dB below the maximum level.
   In various dialogs, this value is the
   center frequency of the filter or the base position in
   a vowel's sequence.

   Values: \texttt{0 to 127, 90*}

   \itempar{St.}{filter!stages}
   Filter stages.
   The more filter stages applied to a signal, the stronger (in general) the
   filtering.
   It is the number of additional times the filter will be applied (in
   order to create a very steep roll-off, such as 48 dB/octave).
   This dropdown element is shown in
   \figureref{fig:filter_stage_dropdown}.
   Obviously, the more stages used, the more calculation-intensive the
   filter will be.  This should also increase the latency (lag) of the
   filter.

   \begin{figure}[H]
   \centering
   \includegraphics[scale=1.0]{bottom-panel/instrument-edit/ADD/filter-stages.jpg}
   \caption[Filter Stage Dropdown]{Filter Stage Dropdown}
   \label{fig:filter_stage_dropdown}
\end{figure}

   \itempar{Q}{resonance level}
   The level of resonance for the filter.
   It indicates a measure of the sharpness of a filter.
   The higher the Q, the sharper the filter.
   Generally, a higher Q value leads to a louder, more tonal
   affect for the filter.
   Note that some filter types might ignore this parameter.

   \itempar{V.Sns.}{velocity sensing amount}
   Velocity sensing amount for filter cutoff.
   Velocity sensing amount of the filter.

   Values: \texttt{0 to 127, 64*}

   \itempar{VF.Sns.}{velocity sensing function}
   \index{filter!velocity sensing function}
   Velocity sensing function of the filter.
   Set the amplitude of the velocity sensing.

   Values: \texttt{0 to 127, 64*}

   \itempar{Gain}{filter!gain}
   Filter gain.
   Additional gain/attenuation for a filter.
   Also described as the filter output gain/damping factor.

   Values: \texttt{0 to 127, 64*}

   \itempar{Freq.tr.}{frequency tracking amount}
   \index{filter!frequency tracking amount}
   Filter Frequency Tracking Amount.
   When this parameter is positive, higher note
   frequencies shift the filter’s cutoff frequency higher.
   For the filter frequency tracking knob, left is negative, middle is
   zero, and right is positive.

   Values: \texttt{0 to 127, 64*}

   \itempar{-/+}{filter!-/+}
   Filter tracking upgrade.
   \index{filter!tracking}
   Filter tracking could never quite reach 100\%,
   so if using it to get "notes" from noise it would go slightly out-of-tune.
   \textsl{Yoshimi} now has this new check box that changes its range so that
   instead of -100\% : 98.4\% it will track 0\% : 198.4\%.
   The icon will change to "0/+".

   This new feature is one of the first that actually change instrument files.
   However, the format is backwards compatible; older versions of
   \textsl{Yoshimi} simply ignore them.

   Also present in this sub-panel are the usual \textbf{C}opy
   and \textbf{P}aste buttons that call up a copy-parameters or
   a paste-parameters dialog.

\subsection{Stock Resonance Settings}
\label{subsec:stock_resonance_settings}

   \textsl{Yoshimi} provides for setting very arbitrary "resonance"
   settings for some sounds.  In fact, "resonance" is too limiting a word.
   A lot of control over the \textsl{spectrum} is possible.
   The following dialog is used by
   the ADDsynth editor, and the PADsynth editor.

   The resonance editor is brought on-screen via the
   \textbf{Resonance} button of the ADDsynth or PADsynth
   global part editors.

   The resonance effect acts as a "resonance box" or a filter with arbitrary
   frequency response. This produces very realistic sounds.
   The cursor location is shown below the graph (the frequency, kHz, and
   the amplitude, dB).

   Paul Nasca has a video on YouTube that includes a demonstration of how
   the resonance dialog works and affects the sound, if one cares to look for
   it.

\begin{figure}[H]
   \centering
%  \includegraphics[scale=0.8]{2.0/Resonance.png}
   \includegraphics[scale=0.75]{2.3.0/resonance.png}
   \caption{ADDsynth/PADsynth Resonance}
   \label{fig:addsynth_resonance}
\end{figure}

   \begin{enumber}
      \item \textbf{Graph Window}
      \item \textbf{Enable}
      \item \textbf{Max dB}
      \item \textbf{C.f.}
      \item \textbf{Oct.}
      \item \textbf{Prot.1st}
      \item \textbf{InterpPk}
      \item \textbf{KHz}
      \item \textbf{dB}
      \item \textbf{Clear}
      \item \textbf{Smooth}
      \item \textbf{RND1}
      \item \textbf{RND2}
      \item \textbf{RND3}
      \item \textbf{C}
      \item \textbf{P}
      \item \textbf{Close}
   \end{enumber}

   \setcounter{ItemCounter}{0}      % Reset the ItemCounter for this list.

   \itempar{Graph Window}{resonance!graph}
   Resonance Graph Window.
   Lets one draw the resonance frequency response in "freehand" mode.

   \itempar{Enable}{resonance!Enable}
   Resonance Enable.
   Turn the Resonance effect on.

   Values: \texttt{Off*, On}

   \itempar{Max dB}{resonance!max db}
   The Maximum Amplitude (dB) wheel.
   Sets the amount of resonance: lower values have little effect. Use the
   roller below to set it.

   Values: \texttt{1 to 90, 20*}

   \itempar{C.f.}{resonance!cf}
   Center Frequency (kHz).
   Sets the center frequency of the graph.
   The value is shown in the read-only text-box to the left.

   Values: \texttt{0 to 127, 64*} for \texttt{0.10 to 10.0, 1.0*}

   \itempar{Oct}{resonance!octaves}
   Number of Octaves.
   Sets the number of octaves the graph represents.
   The value is shown in the read-only text-box to the left.

   Values: \texttt{0 to 127, 64*} for \texttt{0 to 10, 5*}

   \itempar{Prot.1st}{resonance!first harmonic}
   Protect the fundamental Frequency.
   That is, do not damp the first harmonic.

   Values: \texttt{Off, On}

   \itempar{InterpPk}{resonance!interpolate peaks}
   Interpolate the resonance peaks.
   This setting used to be a weird one where the mouse button (left versus
   right) affected the kind of interpolation used,
   but also affects the next field as well, but in \textsl{Yoshimi} 1.3.9 the
   mechanism for interpolation has been made more clear.
   In addition, some of the controls have been changed to sliders for easier
   usage.

   This setting allows one to make resonance functions very easily.  To use
   it effectively, first, clear the graph using the \textbf{Clear} button.
   Click the left button on a position on the graph to create a peak (or do it
   more than once to create more peaks).

   First we show the manual edit that both the modified types were taken from, as shown here.

%  in \figureref{fig:addsynth_resonance_edit}.

   \begin{figure}[H]
      \centering
      \includegraphics[scale=0.8]{2.3.0/resonance_source.png}
      \caption{ADDsynth/PADsynth Resonance Edit}
      \label{fig:addsynth_resonance_edit}
   \end{figure}

   Click either the \textbf{InterpPk S} button (smooth interpolation) or the \textbf{InterpPk L}
   button (linear interpolation). \textsl{Yoshimi} will interpolate automatically between the peaks
   drawn, as shown in \figureref{fig:addsynth_resonance_interpolated}, which shows \textsl{smooth
   interpolation}.

   \begin{figure}[H]
      \centering
      \includegraphics[scale=0.8]{2.3.0/resonance_interpolated.png}
      \caption{ADDsynth/PADsynth Resonance Interpolated}
      \label{fig:addsynth_resonance_interpolated}
   \end{figure}

   Please do not confuse this "smoothing" with the processing done using the
   \textbf{Smooth} button discussed below.

   Also note that one can clear a part of the graph
   by dragging with the right mouse button. In fact, the \textbf{interpPk}
   functionality interpolates between non-zero values.
   Oh, and note that the \textbf{kHz} and \textbf{dB} fields update to match it.

   \itempar{KHz}{resonance!khz}
   The current frequency on graph.

   \itempar{dB}{resonance!db}
   The current level on graph window.

   Values: \texttt{-90 to +90}

   \itempar{Clear}{resonance!clear}
   Clear the resonance function.  (Used to be called "Zero".)
   Clear the graph.

   \itempar{Smooth}{resonance!smooth}
   Smooth the resonance function.
   Smooth the graph.
   This button causes each jagged portion of the graph to be smoothed.
   This smooth does not interpolate between the peaks, unlike the
   \textbf{InterpPk} functionality described earlier.  Compare the
   interpolation shown in \figureref{fig:addsynth_resonance_interpolated},
   and the smoothing shown in \figureref{fig:addsynth_resonance_smoothed}.

\begin{figure}[H]
   \centering
   \includegraphics[scale=0.8]{2.3.0/resonance_smoothed.png}
   \caption{ADDsynth/PADsynth Resonance Smoothed}
   \label{fig:addsynth_resonance_smoothed}
\end{figure}

   Note how the amplitude of the peaks is also reduced by the smoothing.
   Presumably, a frequency-smoothing window is applied to the peaks, thus making
   each new data-point a weighted average of the data-points around it.

   \itempar{RND1}{resonance!randomise}
   Randomise the resonance function, 1.
   RND1, RND2, RND3 are used to create random resonance functions.

   \itempar{RND2}{resonance!randomise}
   Randomise the resonance function, 2.

   \itempar{RND3}{resonance!randomise}
   Randomise the resonance function, 3.

   \itempar{C}{resonance!copy}
   Copy Dialog.

   \itempar{P}{resonance!paste}
   Paste Dialog.

   \itempar{Close}{resonance!close}
   Close.

\subsection{LFO Settings}
\label{subsec:lfo_settings}

   \textsl{Yoshimi} provides LFOs for its amplitude, frequency, and filtering
   functions.
   "LFO" means Low Frequency Oscillator. These oscillators are not used to make
   sounds by themselves, but they change parameters cyclically as a sound
   plays.

   LFOs are, as the name says, oscillators with, compared to the frequency of
   the sound, low frequency. They often appear in order to control the
   effect.

\subsubsection{LFO Basic Parameters}
\label{subsubsec:lfo_basic_parameters}

\begin{figure}[H]
   \centering
   \includegraphics[scale=0.65]{zyn/basic_parameters_lfo0.png}
   \caption[Basic LFO Parameters]{Basic LFO Parameters}
   \label{fig:basic_parameters_lfo}
\end{figure}

   \begin{enumber}
      \item \textbf{Delay}.
      \item \textbf{Start Phase}.
      \item \textbf{Frequency}.
      \item \textbf{Depth}.
   \end{enumber}

   The LFOs have some basic parameters (see
   \figureref{fig:basic_parameters_lfo}.

   \setcounter{ItemCounter}{0}      % Reset the ItemCounter for this list.

   \itempar{Delay}{LFO!delay}
   LFO Delay.
   This parameter sets how much time takes since the start of the note to
   start the cycling of the LFO.
   When the LFO starts, it has a certain position called "start phase".

   \itempar{Start Phase}{LFO!start phase}
   LFO Start Phase.
   The angular position at which a LFO waveform will start.

   \itempar{Frequency}{LFO!frequency}
   LFO Frequency.
   How fast the LFO is (i.e. how fast the parameter controlled by
   the LFO changes.)

   \itempar{Depth}{LFO!depth}
   LFO Depth.
   The amplitude of the LFO (i.e. how much the parameter is controlled by
   the LFO changes.)

\subsubsection{LFO Function}
\label{subsubsec:lfo_function}

   \index{LFO!shape}
   \index{LFO!type}
   \index{LFO!function}
   Another important additional LFO parameter is the shape or type of the
   LFO. There are many LFO Types that vary according to the function used to
   generate the LFO. \textsl{Yoshimi} supports the LFO shapes shown in
   \figureref{fig:types_of_lfo}.

\begin{figure}[H]
   \centering
   \includegraphics[scale=0.65]{zyn/types_of_lfo1.png}
   \caption[LFO Functions]{LFO Types, Shapes, or Functions}
   \label{fig:types_of_lfo}
\end{figure}

\subsubsection{LFO Randomness}
\label{subsubsec:lfo_randomness}

\begin{figure}[H]
   \centering
   \includegraphics[scale=0.75]{zyn/randomness_in_lfo2.png}
   \caption[LFO Randomisation]{LFO Randomisation}
   \label{fig:randomness_in_lfo}
\end{figure}

   \index{LFO!randomness}
   Another parameter is the LFO Randomness. It modifies the LFO amplitude or
   the LFO frequency at random. In \textsl{Yoshimi}
   one can choose how much the LFO
   frequency or LFO amplitude changes by this parameter.
   Observe \figureref{fig:randomness_in_lfo}.
   It shows some examples of randomness and how it changes the shape of a
   triangle LFO.

\subsubsection{LFO, More Settings}
\label{subsubsec:lfo_more_settings}

   Other settings are available as well.

   \index{LFO!continuous mode}
   Continuous mode: If this mode is used, the LFO will not start from "zero" on
   each new note, but it will be continuous. This is very useful if one
   applies on filters to make interesting sweeps.

   \index{LFO!stretch}
   Stretch: It controls how much the LFO frequency changes according to the
   note’s frequency. It can vary from negative stretch (the LFO frequency is
   decreased on higher notes) to zero (the LFO frequency will be the same
   on all notes) to positive stretch (the LFO frequency will be
   increased on higher notes).

\subsubsection{LFO User Interface Panels}
\label{subsubsec:lfo_user_interface_panels}

   \setcounter{ItemCounter}{0}      % Reset the ItemCounter for this list.

\begin{figure}[H]
   \centering
%  \includegraphics[scale=0.8]{2.0/Amplitude_LFO.png}
   \includegraphics[scale=0.8]{2.1.2/lfo.png}
   \caption[Amplitude LFO Sub-Panel]{Amplitude LFO Sub-Panel}
   \label{fig:amplitude_lfo}
\end{figure}

   In \textsl{Yoshimi}, LFO parameters are available for amplitude, filters,
   and frequency.  They all have essentially the same interface elements.
   Note \figureref{fig:amplitude_lfo}, which
   shows an example of an LFO stock sub-panel.

These parameters are:

   \begin{enumber}
      \item \textbf{On} (Only on AddSynth Voices)
      \item \textbf{Freq}
      \item \textbf{Depth}
      \item \textbf{Start}
      \item \textbf{Delay}
      \item \textbf{Amp. (Rand.)}
      \item \textbf{Freq (Rand.)}
      \item \textbf{Str}
      \item \textbf{BPM}
      \item \textbf{Cont.}
      \item \textbf{Type}
      \item \textbf{C} (copy)
      \item \textbf{P} (paste)
   \end{enumber}

   \setcounter{ItemCounter}{0}      % Reset the ItemCounter for this list.

   \itempar{On}{LFO!on} (Only on AddSynth Voices)

   Values: \texttt{Off*, On}

   \itempar{Freq}{LFO!frequency}
   LFO Frequency.
   This parameter varies from 0 to 85.25Hz.

   Values: \texttt{0 to 85.25, 6.259*}

   \itempar{Depth}{LFO!depth}
   \index{LFO!amount}
   LFO Depth.  Also called "LFO Amount". Uniquely, this control can be set
   greater than full depth, and the LFO envelope becomes re-entrant. A sine
   shape becomes two humps, as if rectified, triangle becomes two, etc.

   Values: \texttt{0* to 200\%}

   \itempar{Start}{LFO!starting phase}
   LFO Start Phase in degrees. If this knob is at the lowest value, the LFO Start
   Phase will be random.

   Values: \texttt{random/-167, to 179, 0*}

   \itempar{Delay}{LFO!delay}
   LFO Delay.

   Values: \texttt{0* to 4.00 sec}

   \itempar{Amp. (Rand.)}{LFO!amplitude randomness}
   LFO Amplitude Randomness.

   Values: \texttt{0.1* to 100\%}

   \itempar{Freq.(Rand.)}{LFO!frequency randomness}
   LFO Frequency Randomness.

   Values: \texttt{0.1* to 100\%}

   \itempar{Str}{LFO!stretch}
   LFO Stretch. See the image in
   \figureref{fig:amplitude_lfo}.
   It shows that the LFO stretch is set to zero.

   Values: \texttt{-100\% to 100\%, 0*}

   \itempar{BPM}{LFO!BPM sync mode}
   Turns the LFO frequency control into a ratio control related to the
   incoming MIDI timecode. New in \textsl{Yoshimi} V2.0 See
   \sectionref{subsubsec:bpm_and_frequency}\ for further details.

   Values: \texttt{Off*, On}

   \itempar{Cont.}{LFO!continuous mode}
   LFO Continous Mode.

   Values: \texttt{Off*, On}

   \itempar{Type}{LFO!type}
   LFO Function.

   Also present in this sub-panel are the usual \textbf{C}opy
   and \textbf{P}aste buttons that call up a copy-parameters or
   paste-parameters dialog.

   Values: \texttt{SINE*, TRI, SQR, R.up, R.dn, E1dn, E2dn, S\&H, RSQu, RSQd}

   Meaning: \texttt{Sine, Triangle, Square, Ramp Up, Ramp Down, Exponent 1 Down, Exponent 2 Down, Sample and Hold, Random Square Up, Random Square Down}

\begin{figure}[H]
   \centering
%   \includegraphics[scale=1.0]{bottom-panel/instrument-edit/ADD/lfo-function-type.jpg}
   \includegraphics[scale=1.0]{1.7.2/LFO_function_type.png}
   \caption[LFO Type Drop-down]{LFO Function Type Drop-down}
   \label{fig:lfo_function_type_dropdown}
\end{figure}

%   \itempar{Type}{LFO!function type}
%   LFO Type (or Shape, or Function).
   The various shapes of LFO functions are shown in
   \figureref{fig:types_of_lfo}.
   The values that can be selected are shown in
   \figureref{fig:lfo_function_type_dropdown}.

   \index{new!Extended LFO Types}
   The last three types are new from \textsl{Yoshimi} V 1.7.2. They are all
   based on the square wave type but with the following variations:

   \texttt{S\&H} alternates between the maximum depth and a random level
   from the 'depth' setting to the maximum. It also randomly skips some
   changes.

   \texttt{RSQu} maintains the frequency (doesn't skip any changes) but
   alternates between the depth control setting and a random level from that
   to the maximum depth.

   \texttt{RSQd} also maintains the frequency but alternates between the
   maximum depth and a random amount from the depth control setting and
   maximum.

   These extensions sound more like some special effect than normal LFOs, so
   it is likely they would be of most interest to those involved in
   experimental sounds and music.

   \itempar{C / P}{LFO!Copy Paste}
   Also present in this sub-panel are the usual \textbf{C}opy
   and \textbf{P}aste buttons that call up a copy-parameters or
   paste-parameters dialog.


\subsubsection{Filter LFO Sub-panel}
\label{subsubsec:filter_lfo_sub_panel}

   % Obtain sub-items from checklist
   The controls and layout for this are identical to the Amplitude LFO.
   \sectionref{subsubsec:lfo_user_interface_panels}

\subsubsection{Frequency LFO Sub-panel}
\label{subsubsec:frequency_lfo_sub_panel}

   The controls and layout for this are identical to the Amplitude LFO.
   \sectionref{subsubsec:lfo_user_interface_panels}
\iffalse
\begin{figure}[H]
   \centering
   \includegraphics[scale=1.0]{subpanels/Frequency_LFO.png}
   \caption[Frequency LFO Sub-Panel]{Frequency LFO Sub-Panel}
   \label{fig:frequency_lfo_subpanel}
\end{figure}

   This panel is basically identical to the Filter LFO panel described
   in the previous section.

\fi

\begin{flushleft}
  \begin{minipage}{\textwidth}
   \subsubsection{BPM and Frequency}
   \label{subsubsec:bpm_and_frequency}
   The BPM setting for LFOs and Effects changes the behaviour so that the frequency control
   produces stepped ratios against the incoming MIDI timecode.

   These are: \newline
    .\hspace{3em}1/16, 1/15, 1/14, 1/13, 1/12, 1/11, 1/10, 1/9, 1/8, 1/7, 1/6, 1/5, 1/4, 1/3, 1/2 \newline
    .\hspace{3em}2/3, 1/1, 3/2 \newline
    .\hspace{3em}2/1, 3/1, 4/1, 5/1, 6/1, 7/1, 8/1, 9/1, 10/1, 11/1, 12/1, 13/1, 14/1, 15/1, 16/1 \newline
    This will track timecode changes in real time, but rapid or extreme changes may
    introduce unwanted artefacts. Also, if there is no incoming timecode the BPM will be
    the figure set at the top of the main window (F.BPM).
  \end{minipage}
\end{flushleft}

\subsection{Envelope Settings}
\label{subsec:envelope_settings}

   Envelopes control how the amplitude, the frequency, or the filter changes
   over time.  The general envelope generator has four sections:
   \label{ref:ADSR}
   \begin{enumber}
      \item \textbf{Attack}.
         \label{ref:attack}
         \index{attack}
         \index{envelope!attack}
         The attack is the initial envelope response.
         It begins when the key for the note is first held down
         (at Note On).
         The volume starts at 0, and rises fast or slowly until a peak value.
         In \textsl{Yoshimi}, the attack is always linear.
      \item \textbf{Decay}
         \label{ref:decay}
         \index{decay}
         \index{envelope!decay}
         When the attack is at its highest value, it immediately begins
         to decay to the sustain value.  The decay can be fast or slow.
         The attack and decay together can be used to produce something like
         horn blips, for example.
      \item \textbf{Sustain}
         \label{ref:sustain}
         \index{sustain}
         \index{envelope!sustain}
         This is the level at which the parameter stays while the key is
         held down, i.e. until a Note Off occurs.
      \item \textbf{Release}
         \label{ref:release}
         \index{release}
         \index{envelope!release}
         When the key is released, the sound decays, either fast or slowly,
         until it is off (the volume is 0).
   \end{enumber}

   Together, these values are called "ADSR".
   The ADSR envelope generally controls the amplitude of the sound.
   In \textsl{Yoshimi},
   amplitude envelopes can be \textsl{linear} or \textsl{logarithmic}.

\begin{figure}[H]
   \centering
   \includegraphics[scale=0.65]{zyn/ADSR_envelope1.png}
   \caption{ADSR Envelope (Amplitude)}
   \label{fig:adsr_envelope_depiction}
\end{figure}

   See \figureref{fig:adsr_envelope_depiction},
   it shows a depiction of an ADSR envelope.
   The ADSR is mostly applied to amplitude envelopes.

\begin{figure}[H]
   \centering
   \includegraphics[scale=0.65]{zyn/ASR_frequency_envelope2.png}
   \caption{ASR Envelope, Frequency}
   \label{fig:asr_envelope_depiction}
\end{figure}

   Frequency envelopes control the frequency (more exactly, the pitch) of the
   oscillators. The following image depicts the stages of these envelopes.

   For frequency envelopes, a simpler form of envelope is used.
   This envelope is an ASR envelope, shown in
   \figureref{fig:asr_envelope_depiction}.
   The dotted line represents the real pitch of the sound without the envelope.
   The frequency envelopes are divided into 3 stages:

   \begin{enumber}
      \item \textbf{Attack}.
      This begins at the Note On. The frequency starts from a certain value and
      glides to the real frequency of the note.
      \item \textbf{Sustain}.
      The frequency stays the same during the sustain period.
      \item \textbf{Release}.
      This stage begins on Note Off and glides the frequency of the note to a
      certain value.
   \end{enumber}

\subsubsection{Amplitude Envelope Sub-Panel}
\label{subsubsec:amplitude_envelope_subpanel}

\begin{figure}[H]
   \centering
   \includegraphics[scale=1.0]{subpanels/Amplitude_Env.png}
   \caption[Amplitude Envelope Sub-Panel]{Amplitude Envelope Sub-Panel}
   \label{fig:amplitude_env}
\end{figure}

   \begin{enumber}
      \item \textbf{A.dt}
      \item \textbf{D.dt}
      \item \textbf{S.val}
      \item \textbf{R.dt}
      \item \textbf{Str}
      \item \textbf{L}
      \item \textbf{frcR}
      \item \textbf{C}
      \item \textbf{P}
      \item \textbf{E}
   \end{enumber}

   \setcounter{ItemCounter}{0}      % Reset the ItemCounter for this list.

   \itempar{A.dt}{envelope!attack time}
   Attack duration, attack time. (mS)

   Values: \texttt{0* to 41S}

   \itempar{D.dt}{envelope!decay time}
   Decay duration, decay time.

   Values: \texttt{0 to 41S, 127mS*}

   \itempar{S.val}{envelope!sustain value}
   Sustain value.
   This is the level at which the envelope will settle while the note is held down.
   The only stage that always remains defined is the Sustain, where the
   envelopes freezes until a Note Off event.

   Values: \texttt{-60dB to 0*}

   \itempar{R.dt}{envelope!release time}
   Release time.

   Values: \texttt{0 to 41S, 41.4mS*}

   \itempar{Str}{envelope!stretch}
   Stretch.
   How the envelope is stretched according to the note.
   Envelope Stretch means that, on lower notes, the envelope will be longer.
   On the higher notes the envelopes are shorter than lower notes. In the
   leftmost value, the stretch is zero. The rightmost use a stretch of 200\%;
   this means that the envelope is stretched about 4 times per octave.

   Values: \texttt{0.0\% to 100.1\%, 50.4\%*}

   \itempar{L}{envelope!linear}
   Linear envelope.
   If this option is set, the envelope is linear, otherwise, it will be
   logarithmic.

   Values: \texttt{Off*, On}

   \itempar{frcR}{envelope!forced release}
   Forced release.
   If this option is turned on, the release will go to the final value, even
   if the sustain stage is not reached, otherwise the normal release behaviour
   will be observed. Usually, this must be set.

   Also present in this sub-panel are the usual \textbf{C}opy
   and \textbf{P}aste buttons that call up a copy-parameters or
   paste-parameters dialog.

   Values: \texttt{Off, On*}

   \itempar{C}{copy}
   Copy to Clipboard/Preset.

   \itempar{P}{paste}
   Paste from Clipboard/Preset.

   \itempar{E}{edit}
   Amplitude Envelope Editing Window.
   Described in the next section.

\subsubsection{Envelope Settings}
\label{subsubsec:envelope_settings}

   This section describes the \textbf{Amplitude Envelope Editing} window.

\begin{figure}[H]
   \centering
   \includegraphics[scale=0.8]{2.3.0/FreeMode_off.png}
   \caption{Amplitude Envelope Editor}
   \label{fig:amplitude_envelope_editor}
\end{figure}

   \begin{enumber}
      \item \textbf{Graph Window}
      \item \textbf{FreeMode}
      \item \textbf{C}
      \item \textbf{P}
      \item \textbf{Close}
   \end{enumber}

   \setcounter{ItemCounter}{0}      % Reset the ItemCounter for this list.

   \itempar{FreeMode}{envelope!FreeMode enable}
   FreeMode Enable.
   Enables the envelope editor's Free Mode.
   See the next section for details.

   Values: \texttt{Off*, On} \\

\subsubsection{FreeMode Envelope Settings}
\label{subsubsec:FreeMode_envelope_settings}

   The envelope panels are parts that control a parameter (such as the
   frequencies) of a sound.  For all envelopes, there is a mode that allows the
   user to set an arbitrary number of stages and control points. This mode is
   called \textsl{FreeMode}.  The only stage that always remains defined is the
   Sustain, where the envelopes freezes until a Note Off event.  The FreeMode
   envelope editor has a separate window to set the parameters and controls.

   The main concept of the FreeMode editor window is the \textsl{control point}.
   \index{control point} One can move the points using the mouse. On the right
   side of the window, the total duration of the envelope is shown. If the
   mouse button is pressed on a control point, the duration of the stage at that
   point will be shown. The amplitude of these points (in decibels) will also be
   shown at the top right of the window.

   For an example of the stock FreeMode envelope editor, with
   FreeMode enabled, see \figureref{fig:amplitude_envelope_FreeMode}.

\begin{figure}[H]
   \centering
   \includegraphics[scale=0.8]{2.3.0/FreeMode_edit.png}
   \caption{Amplitude Envelope FreeMode Editor}
   \label{fig:amplitude_envelope_FreeMode}
\end{figure}

   All of the envelope editors have some common controls.

   \begin{enumber}
      \item \textbf{Graph Window}
      \item \textbf{FreeMode}
      \item \textbf{Add point}
      \item \textbf{Delete point}
      \item \textbf{Sust}
      \item \textbf{Stretch}
      \item \textbf{L}
      \item \textbf{frcR}
      \item \textbf{C}
      \item \textbf{P}
      \item \textbf{Close}
   \end{enumber}

   \setcounter{ItemCounter}{0}      % Reset the ItemCounter for this list.

   \itempar{E}{envelope!editor}
   Editor.  Graph Window.
   Shows a window with the real envelope shape and the option to convert to
   FreeMode to edit it.
   The envelope editor shows a window in which one can view and modify the
   detailed envelope shape, or convert it to FreeMode to edit it almost
   without restriction.
   By default, only the \textsl{FreeMode} button/checkbox is visible.

   If an envelope has FreeMode enabled, one can edit the graph of the
   envelope directly, select a point from the graph and move it. Notice that
   the original envelope panel then becomes a minimized version of the graph
   window. In this mini-graph, points can still be dragged and when doing so,
   the full size panel shows the total envelope time, while the mini-graph
   shows the segment time (although one can only read the mini-graph values
   if the enclosing window is considerably enlarged).

   In the FreeMode window only the line \textbf{before} the currently
   edited point of the envelope changes its duration. As the point is
   dragged, the text on the right shows the duration of this line. It will
   continue to show that value until another point is clicked. If the first
   point (or one of the duplicate points in the mini-display) is touched,
   the FreeMode graph will revert to showing the total envelope time.

   If the envelope doesn't have FreeMode mode enabled, it doesn't allow one
   to move the points directly; the envelope window is then useful only
   to see what happens if one changes the ADSR settings.

   \itempar{FreeMode}{envelope!FreeMode}
   FreeMode.  Provides a mode where completely arbitrary envelopes may be
   drawn.
   Actually, the envelopes aren't completely arbitrary, as the sustain
   section is always flat, and its duration corresponds with the duration
   the note is held down.
   When this mode is enabled, the rest of the controls shown in
   \figureref{fig:amplitude_envelope_FreeMode}\hspace{4pt}appear, and are
   described in the following paragraphs.

   Values: \texttt{Off*, On}

   \itempar{Add point}{envelope!add point}
   Add point.
   Provides a way to add a data point to the FreeMode envelope.
   It adds the point after the currently-selected point. One can select a
   point by clicking on it.

   \itempar{Delete point}{envelope!delete point}
   Delete point.
   Provides a way to delete the current data point from the FreeMode envelope.

   \itempar{Sust}{envelope!sustain}
   Sustain point.
   Sets the sustain point.
   The sustain point is shown using the yellow line.
   If the point is at 0, then sustain is disabled.

   \begin{enumber}
      \item{0} means that sustain is disabled, and the envelope goes through
      the entire sequence without being held at any point.
      \item{1} means the note is sustained at the end of the attack curve
      \item{2} means the note is sustained at the end of decay curve.
   \end{enumber}

   Values: \texttt{0, 1, 2*, (more)}

   In fact you can add up to 40 envelope points and the sustain point can be
   at any except the last one.

   \itempar{Stretch}{envelope!stretch}
   Envelope Stretch.
   How the envelope is stretched according to the note. On the higher notes the
   envelopes are shorter than lower notes, pivoting at 440Hz. At the leftmost
   value, the stretch is zero. The rightmost sets a stretch of 200\%; this
   means that the envelope is stretched about four times/octave.

   \itempar{L}{envelope!linear}
   Envelope Linear.
   This setting is only available in the amplitude envelope.
   If enabled, the envelope is linear.
   If not enabled, the envelope is logarithmic (dB).

   Values: \texttt{Off*, On}

   \itempar{frcR}{envelope!forced release}
   Forced Release.
   This means that if this option is turned on, the release will go
   immediately to the final value, even if the sustain stage is not reached,
   otherwise there will be the full release time.
   Usually, this must be set.

   Values: \texttt{Off*, On}

   \itempar{C/P}{envelope!copy/paste}
   the usual \textbf{C}opy and \textbf{P}aste buttons that call up a copy-parameters or paste-parameters dialogs.

   \itempar{Close}{envelope!close dialog}
   Close Dialog.

\subsubsection{Envelope Settings, Frequency}
\label{subsubsec:envelope_settings_for_frequency}

   These envelopes control the frequency (more exactly, the pitch) of the
   oscillators.
   Observe \figureref{fig:asr_envelope_depiction}.
   It depicts the stages of these envelopes.
   The dotted line represents the real pitch of the sound without the envelope.

   \noindent
   The frequency envelopes are divided into 3 stages: \newline
   attack (see \ref{ref:ADSR} item \ref{ref:attack}); \newline
   sustain (see \ref{ref:ADSR} item \ref{ref:sustain}); \newline
   release (see \ref{ref:ADSR} item \ref{ref:release}).

   One question to answer is:
   can the attack and release go in the opposite directions, or do the knob
   ranges prohibit this?

\begin{figure}[H]
   \centering
   \includegraphics[scale=1.0]{subpanels/Frequency_Env.png}
   \caption[Frequency Envelope Sub-Panel]{Frequency Envelope Sub-Panel}
   \label{fig:frequency_env}
\end{figure}

   \begin{enumber}
      \item \textbf{Enable} (present on some versions of this sub-panel).
      \item \textbf{A.val}
      \item \textbf{A.dt}
      \item \textbf{R.dt}
      \item \textbf{R.val}
      \item \textbf{Stretch}
      \item \textbf{frcR}
      \item \textbf{C}
      \item \textbf{P}
      \item \textbf{E}
   \end{enumber}

   For Frequency Envelopes the interface has the following parameters:

   \setcounter{ItemCounter}{0}      % Reset the ItemCounter for this list.

   \itempar{Enable}{envelope!enable}
   Enable the panel.

   \itempar{A.val}{envelope!attack value}
   Attack value. Starting Value.

   Values: \texttt{0 to 127, 64*}

   \itempar{A.dt}{envelope!attack time}
   Attack duration. Attack time.

   Values: \texttt{0 to 127, 40*}

   \itempar{R.dt}{envelope!release time}
   Release time.

   Values: \texttt{0 to 127, 60*}

   \itempar{R.val}{envelope!release value}
   Release Value.

   Values: \texttt{0 to 127, 64*}

   \itempar{Stretch}{envelope!stretch}
   Envelope Stretch.
   Envelope Stretch (make the envelope longer on low notes and shorter on high
   ones, pivoting at 440Hz).

   Values: \texttt{0 to 127, 64*}

   \itempar{frcR}{envelope!forced release}
   Forced release.
   If this option is turned on, the release will go immediately to the final
   value, otherwise the full release behaviour will be observed.

   Values: \texttt{Off, On*}

   Also present in this sub-panel are the usual \textbf{C}opy
   and \textbf{P}aste buttons that call up a copy-parameters or
   paste-parameters dialog, as well as a button
   to bring up the editor window.

\subsubsection{Envelope Settings for Filter}
\label{subsubsec:envelope_settings_for_filter}

   \noindent
   These filter envelopes are divided into 4 stages: \newline
     attack (see \ref{ref:ADSR} item \ref{ref:attack}); \newline
     decay (see \ref{ref:ADSR} item \ref{ref:decay}); \newline
     sustain (see \ref{ref:ADSR} item \ref{ref:sustain}); \newline
     release (see \ref{ref:ADSR} item \ref{ref:release}).

\begin{figure}[H]
   \centering
%   \includegraphics[scale=1.0]{subpanels/Filter_Env.png}
    \includegraphics[scale=0.5]{2.2.3/filtenv.png}
   \caption[Filter Envelope Sub-Panel]{Filter Envelope Sub-Panel}
   \label{fig:filter_env}
\end{figure}

   The items in this panel are:

   \begin{enumber}
      \item \textbf{A.value}
      \item \textbf{A.dt}
      \item \textbf{D.val}
      \item \textbf{D.dt}
      \item \textbf{R.dt}
      \item \textbf{R.val}
      \item \textbf{Stretch}
      \item \textbf{frcR}
   \end{enumber}

   Filter Envelopes has the following parameters:

   \setcounter{ItemCounter}{0}      % Reset the ItemCounter for this list.

   \itempar{A.value}{envelope!attack value}
   Attack Value.  Starting Value.

   Values: \texttt{0 to 127, 64*}

   \itempar{A.dt}{envelope!attack time}
   Attack Duration.  Attack Time.

   Values: \texttt{0 to 127, 40*}

   \itempar{D.val}{envelope!decay value}
   Decay Value.

   Values: \texttt{0 to 127, 64*}

   \itempar{D.dt}{envelope!decay time}
   Decay Duration.  Decay Time.

   Values: \texttt{0 to 127, 70*}

   \itempar{R.dt}{envelope!release time}
   Release time.

   Values: \texttt{0 to 127, 60*}

   \itempar{R.val}{envelope!release value}
   Release Value.

   Values: \texttt{0 to 127, 64*}

   \itempar{Stretch}{envelope!stretch}
   Stretch.
   Envelope Stretch (makes the envelope longer at low notes and shorter
   on high ones, pivoting at 440Hz).

   Values: \texttt{0 to 127, 64*}

   \itempar{frcR}{envelope!forced release}
   Forced Release.
   If this option is turned on, the release will go immediately to the final value,
   otherwise the full release behaviour will be observed. For the frequency envelope
   this is usually textbf{not} set.

   Also present in this sub-panel are the usual \textbf{C}opy
   and \textbf{P}aste buttons that call up a copy-parameters or
   paste-parameters dialog, as well as a button that bring up the editor
   window.

\subsubsection{Formant Filter Settings}
\label{subsubsec:formant_filter_settings}

   This window allows one to change most of the parameters of the formant
   filter.
   The formant filter is available in any filter panel by changing the
   \textbf{Category} to 'Formant', and then clicking the \textbf{Edit}
   button that will appear below the category drop-down list. (The filter
   panel must first be enabled if editing an AddSynth voice or SubSynth
   part.)

   One should keep in mind that these controls are an \textsl{addition} to the
   parent filter ones.

\begin{figure}[H]
   \centering
%  \includegraphics[scale=0.75]{zyn/formant_filter.png}
   \includegraphics[scale=0.75]{2.3.0/formant.png}
   \caption[Formant Filter Editor]{Formant Filter Editor Dialog}
   \label{fig:formant_filter_editor}
\end{figure}

   This editor dialog provides a lot of functionality:

   \begin{enumber}
      \item \textbf{Graph Window}
      \item \textbf{Formants}
      \item \textbf{Fr.Sl.}
      \item \textbf{Vw.Cl.}
      \item \textbf{C.f.}
      \item \textbf{Oct.}
      \item \textbf{Vowel no.}
      \item \textbf{Formant}
      \item \textbf{freq}
      \item \textbf{Q}
      \item \textbf{amp}
      \item \textbf{Seq.Size}
      \item \textbf{S.Pos.}
      \item \textbf{Vowel}
      \item \textbf{Stretch}
      \item \textbf{Neg Input}
   \end{enumber}

\paragraph{Formant Parameters}
\label{paragraph:formant_parameters}

   \itempar{Formants Graph Window}{formants!graph}
   The graph window shows the formant frequency envelope in red,
   and one or more vertical yellow lines at each formant's center frequency.
   As the mouse pointer is moved over the graph, other formant center
   frequencies may appear, highlighted in yellow these actually represent the
   formant that would be selected if one were to click the left mouse button.
   Various mouse actions will modify the formant graph in a manner that might
   be more intuitive than frobbing the spin and knob controls.

   \index{formants!mouse control}
   One can control some of the parameters by placing the mouse pointer over
   the yellow lines representing each formant.
   \index{formants!center frequency}
   Hold down the left: this action will change the formant's center frequency.
   \index{formants!amplitude control}
   The Left button down: moving vertically will change the amplitude.
   \index{formants!Q control}
   Holding the \texttt{Shift} key down: moving vertically changes the formant's Q factor.
   \index{formants!range control}
   Anywhere on the graph: the scroll-wheel changes the octave range, and
   holding \texttt{Shift} at the same time changes the center frequency.

   Thus, one can control most of the formant features one-handed, quickly,
   while also playing on a keyboard.
   If one's mouse has extra buttons on the sides (many haven't) these can
   be used to switch between the formants instead of moving the mouse across
   to the next one.
   While making changes, one sees the respective knobs/sliders moving
   too; these are still fully functional.

   Even though there may only be three or four formants per vowel enabled
   and active, \textsl{Yoshimi} will store (and recover) all twelve possible
   formants per vowel.
   All six vowels are always available, and from the \textbf{main} filter
   window a preset stores all of them, their formants and the sequence
   information.

   One should be aware though, that when actually in the formant editing
   window itself \textbf{only} the formants of the currently selected vowel
   will be stored as a preset. No other data at all.

   Another trap for the unwary is that one can still edit vowels that are
   not currently in use... and wonder why nothing seems to change. From
   \textsl{Yoshimi} V 2.2.3 there is now a warning of this in the form of
   a redish patch that appears behind the vowel number if it's not in use,
   i.e. not seen in any sequence position.

   An important detail is that, uniquely, there is no 'default' value for
   formant frequency. This value will be set randomly when the filter
   is created, and a pseudo-default value is set. When saved, it is the
   \textbf{current} postion that becomes the default value just for the
   saved copy.

   \itempar{Formants}{formants!number of}
   Number of Formants Used.

   Values:  \texttt{1 to 12, 3*}

   \itempar{Fr.Sl}{formants!slowness}
   Formant Slowness.
   Technically, this parameter prevents too-fast morphing between
   individual formants of adjacent vowels, but the user perception is
   that the vowels morph as a whole.

   Values:  \texttt{0 to 127, 64*}

   \itempar{Vw.Cl}{formants!vowel clearness}
   Vowel "Clearness".
   Sets how much the vowels are kept "clear",
   that is, how much "mixed" vowels are avoided.

   Values:  \texttt{0 to 127, 64*}

   \itempar{C.f}{formants!cf}
   Center Frequency.
   This slider changes the center frequency of the entire filter, in relative
   units.

   Values:  \texttt{0.09 to 10.00, 1.0*}

   \itempar{Oct}{formants!octaves}
   Number of Octaves.
   This slider controls the frequency range of the entire filter shifting and
   expanding / compressing each formant by a scaled amount.

   Values:  \texttt{0 to 10}

\paragraph{Formant Vowel Parameters}
\label{paragraph:formant_vowel_parameters}

   \itempar{Vowel no}{formants!vowel number}
   Vowel Number.
   The number of the current vowel.
   Each number represents a different vowel, and leads to a gross change in the
   shape of the formant spectrum. The display only shows the formants of the
   currently seleved vowel, and the yellow line identifies which one is being
   managed. There is no overal view.

   Values:  \texttt{1 to 6}

   \itempar{Formant}{formants!number}
   Formant Number.
   The current formant to be emphasised or modified.  The vertical marker in
   the graph moves as this value is changed.

   Values:  \texttt{1 to 12}

   \itempar{freq}{formants!frequency}
   Formant Frequency.
   The frequency of the current formant.
   This knob changes the frequency of the formant peak selected by the
   Formant Number control.

   Values:  \texttt{0 to 127}

   \itempar{Q}{formants!Q}
   Formant Resonance, Formant Q.
   The Q (resonance depth or bandwidth) of the current formant.
   Used to sharpen or make the current formant sound dull.

   Values:  \texttt{0 to 127}

   \itempar{amp}{formants!amplitude}
   Formant Amplitude.
   Controls the amplitude of the current formant.
   Initially, one will want to set this to the maximum value.

   Values:  \texttt{0 to 127}

\paragraph{Formant Sequence Parameters}
\label{paragraph:formant_sequence_parameters}

   The sequence represents the order in which each vowel will sound when
   traversing the input from the filter envelopes and LFO's.

   \itempar{Seq Size}{formants!seq size}
   Sequence Size.
   The number of vowels in the sequence.

   Values:  \texttt{1 to 6}

   \itempar{S.Pos}{formants!seq position}
   Sequence Position.
   The current position of the sequence.

   Values:  \texttt{1 to 8}

   \itempar{Vowel}{formants!vowel position}
   Vowel Position.
   The vowel used at the current position. One should keep in mind the same
   vowel can be used at several sequence postions!

   Values:  \texttt{1 to 6}

   \itempar{Stretch}{formants!seq stretch}
   How the sequence is stretched.
   This number gives the duration of the sequence relative to
   the pitch of the notes playing.

   Values:  \texttt{0 to 127}

   \itempar{Neg Input}{formants!reversed}
   Negative Input.
   If enabled, the input from the envelope or LFO control is reversed. The effect of this can be quite subtle.

   Values:  \texttt{off, on}

\subsection{Clipboard Presets}
\label{subsec:clipboard_presets}

   In many of the settings panels, there are buttons
   labelled \textbf{C}, \textbf{P}.

%   VVV doesn't make sense to me (will)
%    , and \textbf{E}.
%    \textbf{E} is the editor window, discussed in
%   section \ref{subsubsec:FreeMode_envelope_settings}

   \textbf{C} and \textbf{P} are the clipboard/preset copy and paste
   dialogs, respectively.
   These buttons allow cut-and-paste for shorter sections of the XML
   configuration.

   The first thing to be aware of is that one will only ever see items of a type
   relevant to the section one is copy/pasting. Also, in the \textsl{clipboard}
   there can only be one of each type (so making a fresh copy will overwrite any
   previous one). However, the clipboard can hold many \textsl{different} types at
   once. Once held here an entry can be pasted multiple times in any section that
   has a matching type.

   For example: one can copy an effect from system effects, then go to (say) part 4
   effects and paste it there. Alternatively it is possible to copy the whole of
   AddSynth from part 1 then paste it to AddSynth in part 6 kit item 4.

   However, one can't copy an amplitude envelope then paste it to a filter one as
   they are different types of envelopes.

   \index{preset!dialog}
   The preset dialog also provides a way
   \index{preset!file}
   to save a preset to a preset file.
   The naming convention for a preset file is
   \texttt{presetname.presettype.xpz}, where
   \textsl{presetname} is the name one types into the \textbf{Copy to Preset}
   name field, \textsl{presettype} is the name that appears in the
   \textbf{Type} field, and \textsl{xpz} is the file-extension for compressed
   XML preset files.

   The presets are stored in the current default preset directory,
   which is normally\\
   \texttt{\textasciitilde/.local/share/yoshimi/presets}.

   Preset directories can be added to the list, and
   the default preset directory can be changed.
   See \sectionref{subsec:menu_paths}.

\subsubsection{Clipboard/Preset Copy}
\label{subsubsec:clipboard_copy}
\index{subsubsec:clipboard!copy}

   Note that \figureref{fig:copy_to_clipboard/presets}\hspace {6 pt}shows an example of the copying
   dialog for the clipboard and presets.

\begin{figure}[H]
   \centering
   \includegraphics[scale=0.75]{2.3.0/copy.png}
   \caption[Copy to Clipboard/Presets]{Copy to Clipboard/Presets}
   \label{fig:copy_to_clipboard/presets}
\end{figure}

   \setcounter{ItemCounter}{0}      % Reset the ItemCounter for this list.

   \itempar{Type}{clipboard!copy type}
   Section type for copying.
   This field indicates the context (e.g "effect" - an effect envelope), also the
   name of the section from which the data will be copied.
   If the preset is saved/copied to a file, this field becomes the second part of
   the preset's file-name.

   \itempar{Preset list}{preset!list}
   Preset list.
   \index{preset!list}
   This item is actually a list of preset files available to be selected for
   this block of \textsl{Yoshimi} settings. Since \textsl{Yoshimi} V 1.6.0 the
   different preset root locations have been made selectable in a similar style
   as bank roots.

   \itempar{Copy to Preset}{preset!copy}
   Copy to preset.
   Provides a way to specify the preset (and, indirectly, the preset file)
   to which this data should be copied.

   To save to a preset, type the desired name of the setting.  This entry
   will enable this button.  When the button is pressed, the preset will
   be saved to the current preset directory.
   Be sure to set up a current preset directory where ordinary users have
   write permissions!
   A good choice for a preset directory is
   \texttt{\textasciitilde/.local/share/yoshimi/presets}.
   \index{preset files!.ADnoteParameters.xpz}
   The file-name of the preset will be a non-hidden file such as

   \begin{verbatim}
      my_preset.Paddsyth.xpz
   \end{verbatim}

   The middle part of this name is shown near the top of the preset dialog, as
   a cue.
   There is no way in \textsl{Yoshimi} to change this part of the file-name.
   And don't do it using file system commands!  Modify the first part of the
   file-name to distinguish it from other versions of the preset.
   Only the type-name will ever be visible in the \textsl{Yoshimi}
   presets \textbf{Type} field.

   There is also a legacy type for some engine presets, an example would be.
   \begin{verbatim}
      Crash.ADnoteparameters.xpz
   \end{verbatim}
   But from \textsl{Yoshimi}V 1.6.0 while one can still load this, it would be saved as
   \begin{verbatim}
      Crash.Paddsyth.xpz
   \end{verbatim}

   Note that \textsl{Yoshimi} ships with a number of non-hidden \texttt{.xpz}
   files.

   \itempar{Copy to Clipboard}{clipboard!copy}
   \index{preset!copy}
   Copies the section to the clipboard.

\subsubsection{Clipboard/Preset Paste}
\label{subsubsec:clipboard_paste}

   \index{clipboard!paste}
   \index{preset!paste}
   Observe \figureref{fig:paste_to_clipboard}.
   It shows an example of the pasting dialog for presets immediately prior to making
   the actual selection.

\begin{figure}[H]
   \centering
   \includegraphics[scale=0.75]{2.3.0/paste.png}
   \caption[Paste from Preset]{Paste from Preset}
   \label{fig:paste_to_clipboard}
\end{figure}

   \begin{enumber}
      \item \textbf{Type}
      \item \textbf{Preset list}
      \item \textbf{Paste from Preset}
      \item \textbf{Paste from Clipboard}
   \end{enumber}

   \setcounter{ItemCounter}{0}      % Reset the ItemCounter for this list.

   \itempar{Type}{clipboard!paste type}
   Clipboard/Preset type for pasting.
   This field indicates the context (e.g "effect") or name of the
   section to which the data will be copied.

   \itempar{Preset list}{preset!list}
   Preset list.

   \itempar{Paste from Preset}{preset!paste}
   Paste from preset.
   Provides a way to specify the preset from which this data should be copied.
   This remains deactivated until a selection has been made from the list.

   \itempar{Paste from Clipboard}{clipboard!paste}
   Clipboard to section. Note, this is also shown deactivated but this is due to there
   being no data in the clipboard for this type.

%-------------------------------------------------------------------------------
% vim: ts=3 sw=3 et ft=tex
%-------------------------------------------------------------------------------
