%-------------------------------------------------------------------------------
% yum_LV2
%-------------------------------------------------------------------------------
%
% \file        yum_LV2.tex
% \library     Documents
% \author      Chris Ahlstrom
% \date        2015-10-25
% \update      2015-10-25
% \version     $Revision$
% \license     $XPC_GPL_LICENSE$
%
%     Provides a description and discusson of Yoshimi LV2 support.
%
%-------------------------------------------------------------------------------

\section{LV2 Plug-in Support}
\label{sec:lv2_plugin}

   \textsl{Yoshimi} now runs as an LV2 plugin.

Supported features:

   \begin{enumber}
      \item Sample-accurate midi timing.
      \item State save/restore support via LV2\_State\_Interface.
      \item Working UI support via LV2\_External\_UI\_Widget.
      \item Programs interface support via LV2\_Programs\_Interface.
      \item Multi channel audio output. 'outl' and 'outr' have LV2 index 2
         and 3. All individual ports numbers start at 4.
   \end{enumber}

   Planned feature: Controls automation support. This will be a part of a
   common controls interface.

   Download and build the source code found at the
   \textsl{Yoshimi} site \cite{yoshimi},
   and one will find a file named
   \texttt{LV2\_Plugin/yoshimi\_lv2.so}

   The LV2 \textsl{Yoshimi} interface can be run in hosts such as
   Ardour 3, Carla, and QTractor.

   Apparently, \textsl{ZynAddSubFX} can also be used as an LV2 plugin with
   the help of the carla-lv2 project, by drag-and-dropping an
   \texttt{.xiz} or \texttt{.xmz} file into it.

   At some point we hope to document the process of setting up and using
   the \textsl{Yoshimi} LV2 plugin.

\subsection{LV2 Plug-in Issues}
\label{subsec:lv2_plugin_issues}

   Based on extensive testing of the current \textsl{Yoshimi} master, including
   LV2, using a test file that performed full bank and program setup from a MIDI
   source file as well as volume and pan settings, we made an unfortunate
   discovery.

   If \textsl{Yoshimi}'s internal buffer is \textsl{smaller} than the JACK
   buffer, it sounds quite horrible. If it's the same or greater, there's no
   problem.  This issue only affects Yoshimi LV2, and it doesn't apply to any of
   the released versions.

   The cause of the problem is that the LV2 code doesn't have the same looping
   structure that was added to the standalone routine to deal with exactly this
   situation (re-entering the audio 'construction' function until the JACK buffer
   is filled). This issue should be dealt with soon.

   There are valid reasons for wanting different sizes for these buffers. The
   internal buffer size affects latency and CPU load, and also alters the
   sound in subtle ways. It particularly affects the behaviour of filters.
   One day we may be able to stop this happening, but in the mean time we have to
   live with it. For my purposes we find a buffer size of 128 or 256 is best.

   Testing using the latest versions of Ardour, Muse,
   and Qtractor as LV2 hosts,
   showed that the one that performed the best in every respect was
   Qtractor. Muse got everything correct, but was prone to Xruns on program
   changes. Ardour was very problematical. There were no Xruns but it seemed to
   have odd timing issues. Also, on two occasions it managed to shorten the decay
   times of two of the instruments. How?

   The reference was the original MIDI file played into a stand-alone Yoshimi
   via aplaymidi. This also behaves identically to the file being sequenced by
   Rosegarden.

%-------------------------------------------------------------------------------
% vim: ts=3 sw=3 et ft=tex
%-------------------------------------------------------------------------------
