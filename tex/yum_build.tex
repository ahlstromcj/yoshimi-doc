%-------------------------------------------------------------------------------
% yum_build
%-------------------------------------------------------------------------------
%
% \file        yum_build.tex
% \library     Documents
% \author      Chris Ahlstrom
% \date        2015-07-10
% \update      2017-03-30
% \version     $Revision$
% \license     $XPC_GPL_LICENSE$
%
%     Provides the man page section of the manual.
%
%-------------------------------------------------------------------------------

\section{Building Yoshimi}
\label{sec:yoshimi_build}

   This section describes building and debugging \textsl{Yoshimi}.
   Building \textsl{Yoshimi} requires getting the source code, making sure
   all of the necessary dependencies are installed, and using CMake to set
   up the build.

   The source-code is located at its main location (\cite{yoshimi})
   or its alternate location (\cite{yoshimi2}).

   \textsl{Yoshimi} uses CMake as its build
   system \cite{zyncmake}.  CMake is a preprocessor that can generate
   project build setups for Visual Studio, UNIX make, and Xcode.

\subsection{Yoshimi Source Code}
\label{subsec:yoshimi_source_code}

   Get the source code version you want from SourceForge
   (\url{http://sourceforge.net/projects/yoshimi/files/2.2/}).
   Download the desired tar-ball and unpack it in your work area.

   Since SourceForge has had some issues, the \textsl{Yoshimi} team
   has wisely hosted the source code at another site as well,
   \url{https://github.com/abrolag/yoshimi}.  One can grab the whole
   git repository there using the following command in your work area:

   \begin{verbatim}
      $ git clone https://github.com/abrolag/yoshimi.git
   \end{verbatim}

   Please note that this code base comes with two files,
   \texttt{INSTALL} and \texttt{INSTALL\_CUSTOM} which elaborate on
   build options well beyond what is discussed here.

\subsection{Yoshimi Dependencies}
\label{subsec:yoshimi_dependencies}

   For all versions since \textsl{Yoshimi} 1.3.9, building \textsl{Yoshimi}
   requires C++11.  For GNU builds, this requires \textsl{gcc 4.7} and above.

   To save some wasted time, make sure the \textsl{development versions}
   of the following packages have been installed using your Linux
   distribution's package manager:

   \begin{itemize}
      \item \texttt{pkg-config}
      \item \texttt{libz}
      \item \texttt{fftw3f}
      \item \texttt{mxml}
      \item \texttt{ALSA} (libasound)
      \item \texttt{JACK}
      \item \texttt{fontconfig}
      \item \texttt{libcairo}
      \item \texttt{FLTK}
      \item \texttt{lv2}
   \end{itemize}

   These package names are from \textsl{Debian Jessie}:

   \begin{itemize}
      \item \texttt{automake}
      \item \texttt{build-essential}
      \item \texttt{cmake-curses-gui}
      \item \texttt{dssi-dev}
      \item \texttt{fluid}
      \item \texttt{libboost-dev}
      \item \texttt{libcairo2-dev}
      \item \texttt{libfftw3-dev}
      \item \texttt{libfltk1.3-dev}
      \item \texttt{libglu1-mesa-dev}
      \item \texttt{libjack-jackd2-dev}
      \item \texttt{libjpeg-dev}
      \item \texttt{libmxml-dev}
      \item \texttt{libncurses5-dev} (new dependency)
      \item \texttt{libreadline-dev} (new dependency)
      \item \texttt{libxft-dev}
      \item \texttt{libxinerama-dev}
      \item \texttt{libxml2-dev}
      \item \texttt{xutils-dev}
      \item \texttt{zlib1g.dev}
   \end{itemize}

	LV2 plugin adds one more dependency:

		\texttt{lv2-dev} with version 1.0.0 or greater.

   Other distros may have slightly different names or version numbers, and may
   even have these installed by default. If in doubt, try looking for just the
   main part of the name, but with the \texttt{-dev} extension where
   appropriate.

\subsection{Build It}
\label{subsec:yoshimi_build_it}

   The following instructions are for an in-source build.  An in-source
   build is simpler if you just want to build and install \textsl{Yoshimi}.

   We will also show how to set up for an out-source-build, which keeps
   the build products out of the way.

   The location of \texttt{CMakeList.txt} does not appear to be standard.
   Basically, the build is based in the project's
   \texttt{src} directory, instead of its root directory.
   And it is recommended to use an out-of-source build.

   \begin{enumber}
      \item Enter the source directory where the code was unpacked.
      \item Generate the project build-files:
            \begin{verbatim}
               $ mkdir build
               $ cd build
               $ cmake ..
               $ cmake .
            \end{verbatim}
      \item Build the code and install it (as root):
            \begin{verbatim}
               $ make
               # make install
            \end{verbatim}
   \end{enumber}

   Here is how to make an out-of-source debug build.  Despite what
   cmake documentation (and Googling) says, using a command like the
   following \textsl{does not work} unless you have \texttt{ccmake}
   installed.

   \begin{verbatim}
      $ cmake -DCMAKE_BUILD_TYPE=Debug ..
      $ ccmake
   \end{verbatim}

   In Debian Linux, install the \texttt{cmake-curses-gui}
   package to get access to \texttt{ccmake}.  Or use the shorter
   \texttt{cmake -DBuildForDebug=on ..} command below.

   \begin{enumber}
      \item Enter the source directory where the code was unpacked.
      \item Create a "Debug" or "Release" directory for an
            out-of-source build:
            \begin{verbatim}
               $ cd src
               $ mkdir Debug
            \end{verbatim}
      \item Generate the project build-files in the \texttt{Debug}
            directory.
            \begin{verbatim}
               $ cd Debug
               $ cmake -DBuildForDebug=on ..
               $ make
            \end{verbatim}
   \end{enumber}

   The output file, and executable name \textbf{\texttt{yoshimi}}
   is now ready to run (and be debugged).

   Here is a debugging use case we used in \textsl{Yoshimi 1.3.5.1} and
   slightly earlier versions.  Here is how to verify the bug:

   \begin{enumber}
      \item Run the following command:
         \begin{verbatim}
            $ yoshimi -a -A
         \end{verbatim}
      \item Navigate the following command path:
            Menu / Instrument / Show Banks
      \item Select the \textbf{RENAME} button.
      \item Select the bank (e.g. Arpeggios).
      \item In the file prompt that comes up, click \textbf{Cancel}.
      \item Observe a "Segmentation fault".
   \end{enumber}

   To avoid a lot of debugging, let \textsl{valgrind} find the bug for you.
   Install \textsl{valgrind}.  Then, in the \texttt{src/Debug} directory,
   run:

   \begin{verbatim}
      $ valgrind --log-file=yoshvalgrind.log ./yoshimi -a -A
   \end{verbatim}

   In the log file, one sees that the last good call was in the
   \texttt{Bank :: readOnlyBank()} function.  That would be a good place to
   put a breakpoint.

   However, even without \textsl{valgrind}, this particular bug is easy to
   find under the \textsl{debugger}.  The steps are simple:

   \begin{verbatim}
      $ cd src/Debug
      $ gdb ./yoshimi
      (gdb) r -a -A
   \end{verbatim}

   Then repeat the steps above that trigger the bug.
   One sees

   \begin{verbatim}
      Program received signal SIGSEGV, Segmentation fault.
   \end{verbatim}

   Issue the command "backtrace" at the \texttt{(gdb)} prompt.  There will
   be a list of stack frames starting at 0.  Frame 1 is in \textsl{Yoshimi},
   so issue the command "frame 1", and you'll see:

   \begin{verbatim}
      if (strlen(tmp) > 2) ...
   \end{verbatim}

   \texttt{tmp} is a null pointer here; we need to add an initial check for
   the null pointer there to avoid triggering the crash.

   Of course, the bug has long since been officially fixed.

\subsection{Yoshimi Code Policies}
\label{subsec:yoshimi_code_policies}

   Yes, we actually have \textsl{Yoshimi} code policies.
   Look how many there are! :)

   If the version string contains a 4th number this will always be just a
   bugfix, and will never have features added or changed from the main version.
   For example:

   \begin{itemize}
      \item yoshimi-1.3.5   Main version.
      \item yoshimi-1.3.5.1 First bugfix.
      \item yoshimi-1.3.5.2 Second bugfix. (Surely not!)
   \end{itemize}

   We won't accept fixes for spelling errors in the \textsl{code}.
   For a start, from bitter experience it is fatally easy to change two
   variables to the same name! Also, there's no point, after all they are only
   a mnemonic for memory addresses etc. 'volume' and 'LFO' could just as well
   be 'turnyfing' and 'derfingwotwiggles'.

   To avoid possible confusion, from now on 'master' will display the last
   released version number (without bugfix digits) with an 'M' suffix - unless
   it is a release candidate in which case the suffix will be rc[n].
   For example:

   \begin{itemize}
      \item Last release was yoshimi-1.3.5.2
      \item Master is shown as yoshimi-1.3.5 M
   \end{itemize}

   XML files created with this release will have:
   major version 3 and minor version 5.

   If using Fluid to edit GUI files, please close all windows and collapse all
   menus \textsl{before} the last save.  I know it's tedious, but it avoids
   storms of spurious 'changes' that make genuine ones harder to see.

%-------------------------------------------------------------------------------
% vim: ts=3 sw=3 et ft=tex
%-------------------------------------------------------------------------------
