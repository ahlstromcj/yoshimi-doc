%-------------------------------------------------------------------------------
% yoshimi-outline
%-------------------------------------------------------------------------------
%
% \file        yoshimi-outline.tex
% \library     Documents
% \author      Chris Ahlstrom
% \date        2015-04-21
% \update      2015-05-29
% \version     $Revision$
% \license     $XPC_GPL_LICENSE$
%
%     This document provides LaTeX outline documentation for yoshimi.
%     This document isn't complete, and will remain that way.  I used it
%     mostly to keep track of the figures and settings that needed to be
%     discussed.
%
%-------------------------------------------------------------------------------

\documentclass[
 11pt,
 twoside,
 a4paper,
 headinclude,
 footinclude,
 final                                 % versus draft
]{article}

%-------------------------------------------------------------------------------
% docs-structure
%-------------------------------------------------------------------------------
%
% \file        yoshimi-docs-structure.tex
% \library     Documents
% \author      Chris Ahlstrom
% \date        2015-04-20
% \update      2015-09-02
% \version     $Revision$
% \license     $XPC_GPL_LICENSE$
%
%     This "include file" provides LaTeX options for a document.
%
%     Note that enumitem is an extension of enumerate, and comes from
%     Debian's texlive-latex-recommended package.
%
%-------------------------------------------------------------------------------

\usepackage{enumitem}         % setting the whitespace between and within lists
\setlistdepth{9}
% \setlist{nosep}             % spacing around the list
\setlist{noitemsep}           % spacing within the list

% \usepackage[dvipsnames]{xcolor} % provide more colors?

\usepackage{color}            % provide colors?

% \usepackage[usenames,dvipsnames,svgnames,table]{xcolor}

\usepackage{nameref}          % Provide references by name instead of number
\usepackage[colorlinks=true,linkcolor=webgreen,filecolor=webbrown,citecolor=webgreen]{hyperref}
\definecolor{webgreen}{rgb}{0,.5,0}
\definecolor{webbrown}{rgb}{.6,0,0}

\usepackage{url}              % Required for including URLs
\usepackage{hyperref}         % Required for including hyperlinks
\usepackage{amsthm}           % Helps avoid "destination with same
% \usepackage{cleveref}       % identifier" warnings?
\usepackage[hypcap]{caption}  % make labels point to figure, not the caption
% \usepackage{hypcap}         % make labels point to figure, not the caption
\usepackage[pdftex]{graphicx} % Required for including images
\graphicspath{{../images/}}   % Set the default folder for images
\usepackage{float}            % For more control of location of Figures
\usepackage{geometry}         % Page & text layout
\geometry{
  letterpaper,
  top=2.5cm,
  bottom=2.5cm,
  left=2cm,
  right=2cm
}

\usepackage{longtable}        % For making multi-page tables
\usepackage{makeidx}          % For making an index

% This package isn't available easily on CentOS:
%
% \usepackage[subtle]{savetrees} % For tightening document vertical spacing

\hypersetup{                  % HYPERLINKS
% draft,                      % Uncomment removes links (e.g. for B&W printing)
 colorlinks=true,
 breaklinks=true,
% bookmarks=true,
 bookmarksnumbered,
 urlcolor=webbrown,
 linkcolor=blue,              % RoyalBlue
 citecolor=webgreen,
 pdftitle={},
 pdfauthor={\textcopyright},
 pdfsubject={},
 pdfkeywords={},
 pdfcreator={pdfLaTeX},
 pdfproducer={LaTeX with hyperref and ClassicThesis}
}

% Make an "enumber" style that makes all levels of enumerated lists show
% arabic numerals.

\newlist{enumber}{enumerate}{10}
\setlist[enumber]{nolistsep,label=\arabic*.}

% Make "paragraph" a fourth level, and make it shown in the table of
% contents.

\makeatletter
\renewcommand\paragraph{\@startsection{paragraph}{4}{\z@}%
   {-2.5ex\@plus -1ex \@minus -.25ex}%
   {1.25ex \@plus .25ex}%
   {\normalfont\normalsize\bfseries}}
\makeatother
\setcounter{secnumdepth}{4} % how many sectioning levels to assign numbers to
\setcounter{tocdepth}{4}    % how many sectioning levels to show in ToC

% Provide a way of counting user interface items without putting them in an
% enumberation.

\newcounter{ItemCounter}

% Makes a numbered paragraph out of an item, and allows two index entries
% for it.

\newcommand{\itempar}[2] {
   \stepcounter{ItemCounter}
   \textbf{\arabic{ItemCounter}. #1.}
   \index{#1}
   \index{#2}
}

% Provides for two forms of an option, as might be shown in a man page.

\newcommand{\optionpar}[2] {
   \textbf{\texttt{#1}} \textbf{\texttt{#2}} \\
   \index{#1}
   \index{#2}
}

% Now deprecated in preference to \itempar

\newcommand{\settingdesc}[2] {
   \textbf{#1}
   \index{#1}
   \index{#2}
}

% Make a full reference to a figure using its number, its name, and its page
% number.  Very useful if you have a hard-copy of the document to deal with.

\newcommand{\figureref}[1] {
   figure~\ref{#1}
   "\nameref{#1}"
   on page~\pageref{#1}\ignorespaces
}

% Make a full reference to a section using its number, its name, and its page
% number.  Very useful if you have a hard-copy of the document to deal with.

\newcommand{\sectionref}[1] {%
   section~\ref{#1}
   "\nameref{#1}"
   on page~\pageref{#1}\ignorespaces
}

% Make a full reference to a "paragraph"  using its number, its name, and
% its page number.  Very useful if you have a hard-copy of the document to
% deal with.

\newcommand{\paragraphref}[1] {%
   paragraph~\ref{#1}
   "\nameref{#1}"
   on page~\pageref{#1}\ignorespaces
}

% Make a full reference to a table using its number, its name, and its page
% number.  Very useful if you have a hard-copy of the document to deal with.

\newcommand{\tableref}[1] {%
   table~\ref{#1}
   "\nameref{#1}"
   on page~\pageref{#1}\ignorespaces
}

% An attempt to reduce excess vertical space.  Does not work.  See the
% top of yoshimi-user-manual.tex instead.
%
% \setlength{\parindent}{0pt}
% \setlength{\parskip}{0pt}

% Space between floats. \dblfloatsep for 2 column format.
% \setlength{\floatsep}{8pt}

% Space above and below in-line text floats
% \setlength{\intextsep}{8pt}

% Space above float caption
% \setlength{\abovecaptionskip}{8pt}

% Space below float caption
% \setlength{\belowcaptionskip}{8pt}

% Change the fragction of the page that can be filled with graphics from 0.7
% to 0.9.

\renewcommand\floatpagefraction{.9}
\renewcommand\dblfloatpagefraction{.9}
\renewcommand\topfraction{.9}
\renewcommand\dbltopfraction{.9}
\renewcommand\bottomfraction{.9}

\raggedbottom                          % avoid excessive vertical justification

%-------------------------------------------------------------------------------
% vim: ts=3 sw=3 et ft=tex
%-------------------------------------------------------------------------------
     % packages, macros, structure, & layout

\begin{document}

\title{Yoshimi User Interface Structure}
\author{Chris Ahlstrom\\
   (\texttt{ahlstromcj@gmail.com})}
\date{\today}
\maketitle

%%% \setcounter{tocdepth}{3}               % sections, subsections, and subsubs

% Change the paragraph style to remove indenting and put a line between eacp
% paragraph.  This could be moved up into the preamble, but then would
% affect the spacing of the TOC and LOF, LOT noted above.

\setlength{\parindent}{0pt}

%%% \setlength{\parskip}{1ex plus 0.5ex minus 0.2ex}

\section{Introduction}

Run Yoshimi so that it uses ALSA for audio and for MIDI:

\begin{verbatim}
   $ yoshimi -a -A
\end{verbatim}

You see a brief message, the splash screen, and then the
\textsl{Yoshimi} main window. The outline starts with the main window.

\section{Hierarchy}

% \flushleft

All files are in "images" or sub-directories of that directory.
Simple user-interface items such as 
\texttt{OK/Cancel}, \texttt{Yes/No}, or prompt strings
are not shown.

For values, we don't distinguish between integer-only values and values with
decimal points.  The user will figure out how the values change by trying
them.  Default values are marked with an asterisk.  They are achieved
by clearing the global parameters and the instrument parameters, as far as
we know.

\begin{enumber}

   \item \textbf{Main Window} \\
      Figure: Yosh-splash-OTW-1.png \\
      Figure: yoshimi-first-screen.jpg
   \item \textbf{Menu}
   \begin{enumber}                     % enumber is our arabic numbering style
      \item \textbf{Yoshimi} \\
         Figure: menu/yoshimi-menu-yoshimi.jpg
      \begin{enumber}
         \item \textbf{About...} \\
            Figure: menu/Yoshimi/yoshimi-about.jpg
         \item \textbf{New instance}
         \item \textbf{New instance with id...} \\
            Figure: menu/Yoshimi/yoshimi-instance-id.jpg
         \item \textbf{Settings...}
         \begin{enumber}
            \item \textbf{Main settings} \\
               Figure: menu/Yoshimi/yoshimi-settings-main.jpg
            \begin{enumber}
               \item \textbf{OscilSize} \\
                  Caption: "Adsynth Oscillator Size (samples)" \\
                  Values: \texttt{128, 256, 512, 1024*, 2048, 4096 8192, 16384} \\
                  Figure: menu/Yoshimi/main-oscilsize.jpg
               \item \textbf{Virtual Keyboard Layout} \\
                  Values: \texttt{Dvorak, QWERTY*, AZERTY} \\
                  Figure: menu/Yoshimi/main-virtual-kbd-layout.jpg
               \item \textbf{PADsynth interpolation} \\
                  Values: \texttt{Linear(fast)*, Cubic(slow)} \\
                  Figure: menu/Yoshimi/main-padsynth-interpolation.jpg
               \item \textbf{XML compression level} \\
                  Caption: "gzip compression level (0 - uncompressed)" \\
                  Values: \texttt{0 to 9, 3*}
               \item \textbf{Send reports to} \\
                  Values: \texttt{stderr, Console Window*} \\
                  Figure: menu/Yoshimi/main-send-reports-to.jpg
               \item \textbf{Session state save file}
               \item \textbf{Select}
                  Values: \texttt{~/.config/yoshimi/yoshimi.state} \\
                  Figure: menu/Yoshimi/main-nominate-session-save-state-file.jpg
               \item \textbf{Save and Close}
               \item \textbf{Close Unsaved}
            \end{enumber}
            \item \textbf{Presets dirs} \\
               Figure: menu/Yoshimi/yoshimi-settings-presets-dirs.jpg
            \begin{enumber}
               \item \textbf{Preset list}
               \item \textbf{Add preset directory...} \\
                  Figure: menu/Yoshimi/presets-add-a-preset-directory.jpg
               \item \textbf{Remove preset directory...}
               \item \textbf{Make default}
               \item \textbf{Save and Close}
               \item \textbf{Close Unsaved} (doesn't seem to work!)
            \end{enumber}
            \item \textbf{Jack} \\
               Figure: menu/Yoshimi/yoshimi-settings-jack.jpg
            \begin{enumber}
               \item \textbf{Jack Server} \\
               Values: \texttt{default*, name} name, as in "jackd --name"
               \item \textbf{Save and Close}
               \item \textbf{Close Unsaved} (does it work?)
            \end{enumber}
            \item \textbf{Alsa} \\
               Figure: menu/Yoshimi/yoshimi-settings-alsa.jpg
            \begin{enumber}
               \item \textbf{Alsa Midi Device} \\
                  Values: \texttt{default*} ???
               \item \textbf{Alsa Audio Device} \\
                  Values: \texttt{default*} ???
               \item \textbf{Samplerate} \\
                  Values: \texttt{96000, 48000*, 44100}
               \item \textbf{Period Size} \\
                  Values: \texttt{64, 128, 256, 512, 1024*}
               \item \textbf{Save and Close}
               \item \textbf{Close Unsaved} (does it work?)
            \end{enumber}
            \item \textbf{CCs} \\
               Caption: "MIDI CC preferences" \\
               Figure: menu/Yoshimi/yoshimi-settings-ccs.jpg
            \begin{enumber}
               \item \textbf{Enable Bank Root Change}
                  Values: \texttt{Off*, On}
               \item \textbf{Bank Root Change} \\
                  Values: \texttt{0*, to 127}
               \item \textbf{Bank Change} \\
                  Caption: "CC0 = msb, CC32 = lsb" \\
                  Values: \texttt{LSB, MSB*, Off}
               \item \textbf{Enable Program Change} \\
                  Values: \texttt{Off*, On}
               \item \textbf{Enable Part On Program Change} \\
                  Values: \texttt{Off*, On}
               \item \textbf{Enable Extended Program Change} \\
                  Values: \texttt{Off*, On}
               \item \textbf{Extended Program Change} \\
                  Values: \texttt{0-127, 110*}
               \item \textbf{Save and Close}
               \item \textbf{Close Unsaved} (doesn't seem to work!)
            \end{enumber}
         \end{enumber}
         \item \textbf{Exit} \\
            Figure: menu/Yoshimi/yoshimi-menu-exit-parameters-changed.jpg
      \end{enumber}                    % Yoshimi
      \item \textbf{Instrument}  \\
         Figure: menu/yoshimi-menu-instrument.jpg
      \begin{enumber}
         \item \textbf{Clear Instrument...} \\
            Figure: menu/Instrument/clear-instrument.jpg
         \item \textbf{Open Instrument...} \\
            Figure: menu/Instrument/open-instrument.jpg \\
            Figure: menu/Instrument/favorites-dropdown.jpg
         \item \textbf{Save Instrument...}
         \item \textbf{Show Instruments...} \\
            Figure: menu/Instrument/show-banks-directories.jpg
         \item \textbf{Show Banks...} \\
            Figure: menu/Instrument/show-CA-bank.jpg \\
            Figure: menu/Instrument/show-pads-bank.jpg
         \item \textbf{Show Root Paths...} \\
            Figure: menu/Instrument/show-banks-roots.jpg \\
            Figure: menu/Instrument/bank-root-paths.jpg (duplicate!)
            \begin{enumber}
               \item \textbf{Add root directory...} \\
                  Figure: menu/Instrument/new-directory.jpg
               \item \textbf{Remove root directory...}
               \item \textbf{Make current}
               \item \textbf{Open current}
               \item \textbf{Change ID} \\
                  Values: \texttt{0* to 127}
               \item \textbf{Save and Close} \\
                  Figure: menu/Instrument/nothing-to-save.jpg
               \item \textbf{Close Unsaved} (doesn't seem to work!)
            \end{enumber}
      \end{enumber}
      \item \textbf{Parameters} \\
         Figure: menu/yoshimi-menu-parameters.jpg
      \begin{enumber}
         \item \textbf{Recent} \\
            Figure: menu/Parameters/recent-parameters.jpg
         \item \textbf{Open} \\
            Figure: menu/Parameters/open-parameters.jpg
         \item \textbf{Save} \\
            Figure: menu/Parameters/nothing-to-save.jpg \\
            Figure: menu/Parameters/save-parameters.jpg
         \item \textbf{Clear} \\
            Figure: menu/Parameters/clear-parameters.jpg
      \end{enumber}
      \item \textbf{Scales} \\
         Figure: menu/yoshimi-menu-scales.jpg
      \begin{enumber}
         \item \textbf{Load Scale Settings...} \\
            Figure: menu/Scales/open-scales.jpg \\
            Figure: menu/Scales/failed-to-load-scl-file-vice-xsz.jpg
         \item \textbf{Save Scale Settings...} \\
            Figure: menu/Scales/scale-settings-microtonal.jpg
         \item \textbf{Show Scale Settings...}
         \begin{enumber}
            \item \textbf{Enable Microtonal} \\
               Values: \texttt{Off*, On}
            \item \textbf{"A" Freq.} \\
               Values: \texttt{440*}
            \item \textbf{"A" Note} \\
               Values: \texttt{0 to 127, 69*}
            \item \textbf{Invert Keys} \\
               Values: \texttt{Off*, On}
            \item \textbf{Center (for Invert Keys)} \\
               Caption: "Center where the notes frequencies are turned upside-down" \\
               Values: \texttt{0 to 127, 60*}
            \item \textbf{Name}
            \item \textbf{Shift} \\
               Values: \texttt{-63 to 64, 0*}
            \item \textbf{Comment}
            \item \textbf{Tunings}
            \item \textbf{Keyboard Mapping}
            \begin{enumber}
               \item \textbf{ON} \\
                  Values: \texttt{Off*, On}
               \item \textbf{First Note} \\
                  Caption: "First MIDI note number" \\
                  Values: \texttt{0* to 127}
               \item \textbf{Last Note} \\
                  Caption: "Last MIDI note number" \\
                  Values: \texttt{0 to 127*}
               \item \textbf{Midle Note} \\
                  Caption: "Middle note where scale degree 0 is mapped to" \\
                  Values: \texttt{0 to 127*}
               \item \textbf{Map} \\
                  Values: \texttt{0 to 11}
               \item \textbf{Map Size} \\
                  Values: \texttt{12}
            \end{enumber}
            \item \textbf{Retune}
            \item \textbf{nts./oct.} \\
               Caption: "Notes per octave" \\
               Values: \texttt{12*} (range not yet known)
            \item \textbf{Import .SCL file} \\
               Figure: menu/Scales/import-scl-file.jpg
            \item \textbf{Import .kbm file} \\
               Figure: menu/Scales/import-kbm-file.jpg
            \item \textbf{Close}
         \end{enumber}
      \end{enumber}
      \item \textbf{State} \\
         Figure: menu/yoshimi-menu-state.jpg
      \begin{enumber}
         \item \textbf{Save} \\
            Figure: menu/State/save-state-file.jpg
         \item \textbf{Load} \\
            Figure: menu/State/load-state-file.jpg
      \end{enumber}
   \end{enumber}
   \item \textbf{Top Panel} \\
      Figure: yoshimi-first-screen.jpg
   \begin{enumber}
      \item \textbf{Stop!} \\
         Caption: "Cease all sound immediately!"
      \item \textbf{Panel} \\
         Caption: "Mixer Panel Window" \\
         Figure: top-panel/yoshimi-panel-2x8.jpg
      \begin{enumber}
         \item \textbf{Part Section, 1 to 16}
         \begin{enumber}
            \item \textbf{Enable part} \\
               Values: \texttt{Off*, On}
            \item \textbf{Part name}
            \item \textbf{Volume ? Slider}
            \item \textbf{Meter display}
            \item \textbf{Volume ? Knob}
            \item \textbf{Channel} \\
               Caption: "Receive from MIDI channel" \\
               Values: \texttt{Ch1*, Ch2, ..., Ch16}
            \item \textbf{Main} \\
               Caption: "Set Audio Destination"
            \item \textbf{Edit} (does not work!) \\
               Caption: "Left mouse button: Part select
               Right mouse button: Instrument edit"
         \end{enumber}
         \item \textbf{Change to 1 x 16 / Change to 2 x 8}
         \item \textbf{Refresh}
         \item \textbf{Close}
      \end{enumber}
      \item \textbf{VirKbd} (see "Virtual Keyboard..." for details) \\
         Figure: top-panel/yoshimi-virtual-keyboard.jpg \\
         Figure: menu/Instrument/virtual-keyboard-controllers.jpg
      \item \textbf{Key Shift} \\
         Values: \texttt{-12 to 12, 0*}
      \item \textbf{Detune} \\
         Caption: "Global fine detune" \\
         Values: \texttt{0 to 127, 64*} (float)
      \item \textbf{Reset (Detune)}
      \item \textbf{Volume}
   \end{enumber}
   \item \textbf{Effects Panel} \\
      Figure: yoshimi-first-screen.jpg
   \begin{enumber}
      \item \textbf{System Effects} \\
         Figure: effects-panel/system-effects.jpg
      \begin{enumber}
         \item \textbf{Effect Number}
         \item \textbf{Effect Name} \\
            Figure: effects-panel/system-effects-selections.jpg
         \item \textbf{Send to} \\
            Figure: effects-panel/system-effects-send-to.jpg
         \item \textbf{C} \\
            Figure: effects-panel/system-effects-C-clipboard.jpg
         \item \textbf{P} \\
            Figure: effects-panel/system-effects-P-clipboard.jpg
      \end{enumber}
      \item \textbf{Insertion Effects} (see "Effects Edit") \\
         Figure: effects-panel/insertion-effects.jpg \\
         Figure: effects-panel/insertion-effects-to-part.jpg
      \begin{enumber}
         \item \textbf{C} \\
            Figure: effects-panel/system-effects-C-clipboard.jpg identical
         \item \textbf{P} \\
            Figure: effects-panel/system-effects-P-clipboard.jpg identical
      \end{enumber}
      \item \textbf{Reports} \\
         Figure: effects-panel/reports.jpg
   \end{enumber}
   \item \textbf{Bottom Panel} \\
      Figure: yoshimi-first-screen.jpg
   \begin{enumber}
      \item \textbf{Part} \\
         Values: \texttt{1 to 16}
      \item \textbf{Instrument Name}
      \item \textbf{Edit (Instrument Edit)} \\
         (more details in a separate section below) \\
         Figure: bottom-panel/edit-instrument.jpg
      \begin{enumber}
         \item \textbf{Type}
         \item \textbf{Author and Copyright}
         \item \textbf{Comments}
         \item \textbf{ADDsynth}
         \begin{enumber}
            \item \textbf{Enabled}
            \item \textbf{Edit}
         \end{enumber}
         \item \textbf{SUBsynth}
         \begin{enumber}
            \item \textbf{Enabled}
            \item \textbf{Edit}
         \end{enumber}
         \item \textbf{PADsynth}
         \begin{enumber}
            \item \textbf{Enabled}
            \item \textbf{Edit}
         \end{enumber}
         \item \textbf{Kit Edit}
         \item \textbf{Effects}
         \item \textbf{Rnd. Det.}
         \item \textbf{Close}
      \end{enumber}
      \item \textbf{Midi (channel)} \\
         Values: \texttt{1 to 16}
      \item \textbf{Mode} \\
         Values: \texttt{Poly, Mono, Legato}
      \item \textbf{Enabled} \\
         Values: \texttt{Off*, On}
      \item \textbf{Portamento} \\
         Caption: "Enable/disable the portamento" \\
         Values: \texttt{Off*, On}
      \item \textbf{Velocity Sens} \\
         Caption: "Velocity Sensing Function" \\
         Values: \texttt{0 to 127, 64*}
      \item \textbf{Velocity Offset} \\
         Caption: "Velocity Offset" \\
         Values: \texttt{0 to 127, 64*}
      \item \textbf{Pan} \\
         Caption: "Velocity Offset" \\
         Values: \texttt{0 to 127, 64*}
      \item \textbf{Pan (reset)} \\
         Caption: "Reset Pan to Middle (64)"
      \item \textbf{Volume}
      \item \textbf{Controllers} \\
         Figure: bottom-panel/controllers-dialog.jpg
      \begin{enumber}
         \item \textbf{Exp MWh} \\
            Caption: "Exponential Modulation Wheel" \\
            Values: \texttt{Off*, On}
         \item \textbf{ModWh} \\
            Caption: "Modulation Wheel Depth" \\
            Values: \texttt{0 to 127, 80*}
         \item \textbf{Exp BW} \\
            Caption: "Exponential Bandwidth Controller" \\
            Values: \texttt{Off*, On}
         \item \textbf{BwDepth} \\
            Caption: "Bandwidth Depth", \\
            Values: \texttt{0 to 127, 64*}
         \item \textbf{PanWdth} \\
            Caption: "Panning Depth" \\
            Values: \texttt{0 to 64*}
         \item \textbf{FltQ} \\
            Caption: "Filter Q Depth" \\
            Values: \texttt{0 to 127, 64*}
         \item \textbf{FitCut} \\
            Caption: "Filter Cutoff Depth" \\
            Values: \texttt{0 to 127, 64*}
         \item \textbf{Vol Rng} \\
            Caption: "Volume Range" \\
            Values: \texttt{64 to 127, 64*}
         \item \textbf{PWheelB.Rng} \\
            Caption: "Pitch Wheel Bend Range (cents)" \\
            Values: \texttt{-6400 to 6400, 200*}
         \item \textbf{Expr} \\
            Caption: "Expression Enable" \\
            Values: \texttt{Off, On*}
         \item \textbf{FMamp} \\
            Caption: "FM Amplitude Enable" \\
            Values: \texttt{Off, On*}
         \item \textbf{Vol} \\
            Caption: "Volume Enable" \\
            Values: \texttt{Off, On*}
         \item \textbf{Sustain} \\
            Caption: "Sustain Pedal Enable" \\
            Values: \texttt{Off, On*}
         \item \textbf{Resonance} (section)
         \begin{enumber}
            \item \textbf{CFdpth} \\
               Caption: "Center Frequency Controller Depth" \\
               Values: \texttt{0 to 127, 64*}
            \item \textbf{BWdpth} \\
               Caption: "Bandwidth Controller Depth" \\
               Values: \texttt{0 to 127, 64*}
         \end{enumber}
         \item \textbf{PortaMento} (section)
         \begin{enumber}
            \item \textbf{Rcv} \\
               Caption: "Receive Portamento Controllers" \\
               Values: \texttt{Off, On*}
            \item \textbf{Proprt} \\
               Caption: "Enable Proportional Portamento (over fixed portamento)" \\
               Values: \texttt{Off*, On}
            \item \textbf{time} \\
               Caption: "Portamento time" \\
               Values: \texttt{0 to 127, 64*}
            \item \textbf{t.dn/up} \\
               Caption: "Portamento Time Stretch (up/down)" \\
               Values: \texttt{0 to 127, 64*}
            \item \textbf{thresh} \\
               Caption: "Minimum or maximum difference of notes in order \\
                  to do the portamento (x 100 cents)" \\
               Values: \texttt{0 to 127, 3*}
            \item \textbf{th.type} \\
               Caption: "Threshold Type (min/max)" \\
               Values: \texttt{Off, On*}
            \item \textbf{Prp.Rate} \\
               Caption: "Distance required to double change from nonproportional \\
                  portamento time" \\
               Values: \texttt{0 to 127, 80*}, requires \textbf{Proprt.} =
               \texttt{On}
            \item \textbf{Prp.Depth} \\
               Caption: "The difference from nonproportional portamento" \\
               Values: \texttt{0 to 127, 90*}, requires \textbf{Proprt.} =
               \texttt{On}
         \end{enumber}
         \item \textbf{Reset all controllers}
         \item \textbf{Close}
      \end{enumber}
      \item \textbf{Minimum Note} \\
         Caption: "Minimum note the part receives" \\
         Values: \texttt{0* to 127}
      \item \textbf{Maximum Note} \\
         Caption: "Maximum note the part receives" \\
         Values: \texttt{0 to 127*}
      \item \textbf{m} \\
         Caption: "Set minimum note to last note played"
      \item \textbf{R} \\
         Caption: "Reset the minimum key to 0 and the maximum key to 127"
      \item \textbf{M} \\
         Caption: "Set maximum note to last pressed key"
      \item \textbf{Key Shift} \\
         Values: \texttt{-12 to 12, 0*}
      \item \textbf{Key Limit} \\
         Caption: "Maximum keys for this part" \\
         Values: \texttt{0 to 55, 15*}
      \item \textbf{System Effect Sends 1} \\
         Values: \texttt{0 to 127*}
      \item \textbf{SES 2} \\
         Values: \texttt{0 to 127*}
      \item \textbf{SES 3} \\
         Values: \texttt{0 to 127*}
      \item \textbf{SES 4} \\
         Values: \texttt{0 to 127*}
   \end{enumber}
   \item \textbf{Sound Meter}
   \begin{enumber}
      \item \textbf{Meter Bar}
         Figure: yoshimi-first-screen.jpg
   \end{enumber}

   % ------- End of Main Window Section ----------------------------

   \item \textbf{Generic File Dialog} \\
      Figure: menu/Instrument/favorites-dropdown.jpg
      \begin{enumber}
         \item \textbf{Show:}
         \item \textbf{Favorites}
         \begin{enumber}
            \item \textbf{Add to Favorites}
            \item \textbf{Manage Favorites}
            \item \textbf{File Systems}
            \item \textbf{/home/user}
            \item \textbf{/home/user/Home/Audio/yoshimi...}
         \end{enumber}
         \item \textbf{Folder button}
            Caption: "Create a new directory"
         \item \textbf{Preview} \\
            Values: \texttt{Off*, On}
         \item \textbf{Preview pane}
         \item \textbf{Filename:}
         \item \textbf{Directory buttons}
      \end{enumber}
   \item \textbf{Generic Copy Dialog}
      TODO
   \item \textbf{Generic Paste Dialog}
      TODO
   \item \textbf{Generic Freeform Envelope Dialog}
      TODO
   \item \textbf{Virtual Keyboard...} \\
      Figure: menu/Instrument/virtual-keyboard.jpg \\
      Figure: menu/Instrument/virtual-keyboard-controllers.jpg
      \begin{enumber}
         \item \textbf{Pwh}
Caption: "Pitch bend knob"
         \item \textbf{R} \\
            Caption: "Reset Pitch Bend"
         \item \textbf{Midi Channel} \\
            Values: \texttt{1* to 16}
         \item \textbf{Velocity} \\
            Values: \texttt{1 to 127, 100*}
         \item \textbf{Velocity} \\
            Caption: "Velocity Randomness" \\
            Values: \texttt{0* to 127}
         \item \textbf{Octave} \\
            Caption: "q2w3e4r5t6y Octave" \\
            Values: \texttt{1, 2*, 3, 4, 5}
         \item \textbf{"qwer.." Oct} \\
            Caption: "zsxdcfvgbh Octave" \\
            Values: \texttt{1, 2*, 3, 4, 5}
         \item \textbf{Controller} \\
            Figure: menu/Instrument/virtual-keyboard-controllers.jpg \\
            Values: \texttt{01:Mod.Wheel, 07:Volume, 10:Panning,
               11:Expression, 64:Sustain, 65:Portamento, 71:Filter Q,
               74:Filter Freq*, 75:Bandwidth, 76:FM Gain,
               77:Res.c.freq, 78:Res.bw.}
         \item \textbf{Cval} \\
            Caption: "Controller value" \\
            Values: \texttt{1 to 127, 96*}
         \item \textbf{Close}
      \end{enumber}
   \item \textbf{Instrument Edit Details} \\
      Figure: bottom-panel/edit-instrument.jpg \\
      Figure: bottom-panel/edit-instrument-drums.jpg \\
      Figure: bottom-panel/edit-instrument-type.jpg
   \begin{enumber}
      \item \textbf{ADDsynth Edit} \\
         ("ADsynth Global Parameters of the Instrument") \\
         Figure: bottom-panel/instrument-edit/ADD/ADDsynth-edit.jpg
      \begin{enumber}

         % This AMPLITUDE section is re-used (e.g in SUBsynth Parameters)

         \item \textbf{AMPLITIUDE} (section)
         \begin{enumber}
            \item \textbf{Volume} \\
               Values: \texttt{1 to 127, 64*}
            \item \textbf{Vel Sens} \\
               Caption: "Velocity Sensing function, rightmost/max to disable" \\
               Values: \texttt{1 to 127, 64*}
            \item \textbf{Pan} \\
               Caption: "Global panning, leftmost/zero gives random panning" \\
               Values: \texttt{1 to 127, 64*}
            \item \textbf{Rand} \\
               Caption: "Indicator for activation of random panning"
            \item \textbf{Reset (panning)} (red button)

            % Stock part: Amplitude Env (without Enable)

            \item \textbf{Amplitude Env}
            \begin{enumber}
               \item \textbf{A.dt} \\
                  Caption: "Attack Time" \\
                  Values: \texttt{0* to 127}
               \item \textbf{D.dt} \\
                  Caption: "Decay Time" \\
                  Values: \texttt{0 to 127, 44*}
               \item \textbf{S.val} \\
                  Caption: "Sustain Value" \\
                  Values: \texttt{0 to 127*}
               \item \textbf{R.dt} \\
                  Caption: "Release Time" \\
                  Values: \texttt{0 to 127, 25*}
               \item \textbf{Stretch} \\
                  Caption: "Envelope Stretch (on lower notes make the envelope longer") \\
                  Values: \texttt{0 to 127, 64*}
               \item \textbf{L} \\
                  Caption: "The envelope is linear" \\
                  Values: \texttt{Off*, On}
               \item \textbf{frcR} \\
                  Caption: "Forced Release"
                  Values: \texttt{Off*, On}
               \item \textbf{C} \\
                  Caption: "Copy to Clipboard/Preset" \\
                  Figure: bottom-panel/instrument-edit/ADD/copy-to-clipboard-preset.jpg
                  \begin{enumber}
                     \item \textbf{Type} ("envamplitude")
                     \item \textbf{Envelope list ?}
                     \item \textbf{Copy to Preset}
                     \item \textbf{Copy to Clipboard}
                  \end{enumber}
               \item \textbf{P} \\
                  Caption: "Paste from Clipboard/Preset" \\
                  bottom-panel/instrument-edit/ADD/ \\
                     Figure: paste-from-clipboard-preset.jpg
                  \begin{enumber}
                     \item \textbf{Type} ("envamplitude")
                     \item \textbf{Envelope list ?}
                     \item \textbf{Paste from Preset}
                     \item \textbf{Paste from Clipboard}
                  \end{enumber}
               \item \textbf{E} \\
                  Caption: "Amplitude Envelope Window" \\
                  Figure: bottom-panel/instrument-edit/ADD/amplitude-envelope.jpg
                  \begin{enumber}
                     \item \textbf{FreeMode}
                        Figure: bottom-panel/instrument-edit/ADD/amplitude-envelope.jpg \\
                        bottom-panel/instrument-edit/ADD/ \\
                           Figure: disable-free-mode-of-envelope.jpg \\
                        Values: \texttt{Off*, On}
                     \begin{enumber}
                        \item \textbf{Add point}
                        \item \textbf{Delete point}
                        \item \textbf{Sust} \\
                           Caption: "Sustain (0 is disabled)" \\
                           Values: \texttt{0, 1, 2*}
                        \item \textbf{Str.} \\
                           Envelope Stretch (on lower notes make the \\
                              envelope longer")
                        \item \textbf{L} \\
                           Caption: "The envelope is linear" \\
                           Values: \texttt{Off*, On}
                        \item \textbf{frcR} \\
                           Caption: "Forced Release"
                     \end{enumber}
                     \item \textbf{C}
                     \item \textbf{P}
                  \end{enumber}
            \end{enumber}

            % Stock part: Amplitude LFO (without Enable)

            \item \textbf{Amplitude LFO}
            \begin{enumber}
               \item \textbf{Freq.} \\
                  Caption: "LFO Frequency" \\
                  Values: \texttt{0 to 1, 0.63*}
               \item \textbf{Depth} \\
                  Caption: "LFO Amount" \\
                  Values: \texttt{0* to 127}
               \item \textbf{Start}
                  Caption: "LFO Startphase (leftmost is random)" \\
                  Values: \texttt{0 to 127, 64*}
               \item \textbf{Delay} \\
                  Caption: "LFO Delay" \\
                  Values: \texttt{0* to 127}
               \item \textbf{Str.} \\
                  Caption: "LFO Stretch" \\
                  Values: \texttt{0 to 127, 64*}
               \item \textbf{C.} \\
                  Caption: "Continuous LFO" \\
                  Values: \texttt{Off*, On}
               \item \textbf{A.R.} \\
                  Caption: "LFO Amplitude Randomness" \\
                  Values: \texttt{0* to 127}
               \item \textbf{F.R.} \\
                  Caption: "LFO Frequency Randomness" \\
                  Values: \texttt{0* to 127}
               \item \textbf{Type} \\
                  Figure: bottom-panel/instrument-edit/ADD/lfo-function-type.jpg \\
                  Values: \texttt{SINE*, TRI, SQR, R.up, R.dn, E1dn, E2dn}
               \item \textbf{C}
               \item \textbf{P}
            \end{enumber}

            \item \textbf{Stereo} \\
               Values: \texttt{Off, On*}
            \item \textbf{Rnd Grp} \\
               Caption: "How the harmonic amplitude is applied to voices that \\
                  use the same oscillator." \\
               Values: \texttt{Off*, On}
            \item \textbf{P.Str.} \\
               Caption: "Punch Strength" \\
               Values: \texttt{0* to 127}
            \item \textbf{P.t} \\
               Caption: "Punch Time (duration)" \\
               Values: \texttt{0 to 127, 64*}
            \item \textbf{P.Stc.} \\
               Caption: "Punch Stretch" \\
               Values: \texttt{0 to 127, 64*}
            \item \textbf{P.Vel.} \\
               Caption: "Punch Velocity Sensing" \\
               Values: \texttt{0 to 127, 72*}
         \end{enumber}

         \item \textbf{FILTER} (section)
         \begin{enumber}

            % Stock part: Filter Params (minus enable)

            \item \textbf{Filter Params}
            \begin{enumber}
               \item \textbf{Category} \\
                  Caption: "Filter Category" \\
                  Values: \texttt{Analog*, Formant, StVarF} \\
                  Figure: bottom-panel/instrument-edit/ADD/filter-category.jpg
               \item \textbf{FilterType} \\
                  Caption: "Filter Type" \\
                  Values: \texttt{LPF1, HPF1, LPF2*, HPF2, BPF2, NF2, PkF2,
                     LSh2, HSh2} \\
                  Figure: bottom-panel/instrument-edit/ADD/filter-filtertype.jpg
               \item \textbf{C.freq} \\
                  Caption: "Center frequency of the filter or the base position in
                     the vowel's sequence" \\
                  Values: \texttt{0 to 127, 90*}
               \item \textbf{Q} \\
                  Caption: "Filter resonance or bandwidth" \\
                  Values: \texttt{0 to 127, 40*}
               \item \textbf{V.SnsA.} \\
                  Caption: "Velocity sensing amount of the filter" \\
                  Values: \texttt{0 to 127, 64*}
               \item \textbf{V.Sns.} \\
                  Caption: "Velocity sensing function of the filter" \\
                  Values: \texttt{0 to 127, 64*}
               \item \textbf{freq.tr} \\
                  Caption: "Filter frequency tracking, left is negative, middle is \\
                     zero, and right is positive" \\
                  Values: \texttt{0 to 127, 64*}
               \item \textbf{gain} \\
                  Caption: "Filter output gain/damp" \\
                  Values: \texttt{0 to 127, 64*}
               \item \textbf{St} \\
                  Caption: "Filter stages (in order to increase dB/octave value and
                     the order of the filter)" \\
                  Values: \texttt{1x*, 2x, 3x, 4x, 5x} \\
                  Figure: bottom-panel/instrument-edit/ADD/filter-stages.jpg
               \item \textbf{C} (See elsewhere)\\
                  Caption: "Copy to Clipboard/Preset"
               \item \textbf{P} (See elsewhere)\\
                  Caption: "Paste from Clipboard/Preset"
            \end{enumber}

            % Stock part: Filter Env (without Enable)

            \item \textbf{Filter Env} \\
               Figure: bottom-panel/instrument-edit/ADD/ADDsynth-filter-envelope.jpg
            \begin{enumber}
               \item \textbf{A.val} \\
                  Caption: "Starting Value" \\
                  Values: \texttt{0 to 127, 64*}
               \item \textbf{A.dt} \\
                  Caption: "Attack Time" \\
                  Values: \texttt{0 to 127, 40*}
               \item \textbf{D.val} \\
                  Caption: "Decay Value" \\
                  Values: \texttt{0 to 127, 64*}
               \item \textbf{D.dt} \\
                  Caption: "Decay Time" \\
                  Values: \texttt{0 to 127, 70*}
               \item \textbf{R.dt} \\
                  Caption: "Release Time" \\
                  Values: \texttt{0 to 127, 60*}
               \item \textbf{R.val} \\
                  Caption: "Release Value" \\
                  Values: \texttt{0 to 127, 64*}
               \item \textbf{Stretch} \\
                  Caption: "Envelope Stretch (on lower notes make the envelope longer") \\
                  Values: \texttt{0* to 127}
               \item \textbf{L} \\
                  Caption: "The envelope is linear" \\
                  Values: \texttt{Off*, On}
               \item \textbf{frcR} \\
                  Caption: "Forced Release"
               \item \textbf{C} \\
                  Caption: "Copy to Clipboard/Preset" \\
                  Figure: bottom-panel/instrument-edit/ADD/copy-to-clipboard-preset.jpg
                  \begin{enumber}
                     \item \textbf{Type} ("envamplitude")
                     \item \textbf{Envelope list ?}
                     \item \textbf{Copy to Preset}
                     \item \textbf{Copy to Clipboard}
                  \end{enumber}
               \item \textbf{P} \\
                  Caption: "Paste from Clipboard/Preset" \\
                  bottom-panel/instrument-edit/ADD/ \\
                     Figure: paste-from-clipboard-preset.jpg
                  \begin{enumber}
                     \item \textbf{Type} ("envamplitude")
                     \item \textbf{Envelope list ?}
                     \item \textbf{Paste from Preset}
                     \item \textbf{Paste from Clipboard}
                  \end{enumber}
               \item \textbf{E} \\
                  Caption: "Filter Envelope Window" \\
                  Figure: bottom-panel/instrument-edit/ADD/amplitude-envelope.jpg
                  \begin{enumber}
                     \item \textbf{FreeMode} \\
                        Values: \texttt{Off*, On} \\
                        Figure: bottom-panel/instrument-edit/ADD/amplitude-envelope.jpg \\
                        Figure: bottom-panel/instrument-edit/ADD/disable-free-mode-of-envelope.jpg
                     \begin{enumber}
                        \item \textbf{Add point}
                        \item \textbf{Delete point}
                        \item \textbf{Sust} \\
                           Caption: "Sustain (0 is disabled)" \\
                           Values: \texttt{0, 1, 2*}
                        \item \textbf{Str.} \\
                           Caption: "Envelope Stretch (on lower notes make the
                              envelope longer)"
                        \item \textbf{L} \\
                           Caption: "The envelope is linear" \\
                           Values: \texttt{Off*, On}
                        \item \textbf{frcR} \\
                           Caption: "Forced Release"
                     \end{enumber}
                     \item \textbf{C}
                     \item \textbf{P}
                  \end{enumber}
            \end{enumber}

            % Stock part: Filter LFO (without Enable)

            \item \textbf{Filter LFO}
            \begin{enumber}
               \item \textbf{Freq.} \\
                  Caption: "LFO Frequency" \\
                  Values: \texttt{0 to 1, 0.64*}
               \item \textbf{Depth} \\
                  Caption: "LFO Amount" \\
                  Values: \texttt{0* to 127}
               \item \textbf{Start}
                  Caption: "LFO Startphase (leftmost is random)" \\
                  Values: \texttt{0 to 127, 64*}
               \item \textbf{Delay} \\
                  Caption: "LFO Delay" \\
                  Values: \texttt{0* to 127}
               \item \textbf{Str.} \\
                  Caption: "LFO Stretch" \\
                  Values: \texttt{0 to 127, 64*}
               \item \textbf{C.} \\
                  Caption: "Continuous LFO" \\
                  Values: \texttt{Off*, On}
               \item \textbf{A.R.} \\
                  Caption: "LFO Amplitude Randomness" \\
                  Values: \texttt{0* to 127}
               \item \textbf{F.R.} \\
                  Caption: "LFO Frequency Randomness" \\
                  Values: \texttt{0* to 127}
               \item \textbf{Type} \\
                  Figure: bottom-panel/instrument-edit/ADD/lfo-function-type.jpg \\
                  Values: \texttt{SINE*, TRI, SQR, R.up, R.dn, E1dn, E2dn}
               \item \textbf{C} \\
                  Caption: "Copy to Clipboard/Preset"
               \item \textbf{P} \\
                  Caption: "Paste from Clipboard/Preset"
            \end{enumber}
         \end{enumber}
 
         % Almost, but not quite, a stock part.

         \item \textbf{FREQUENCY} (section)
         \begin{enumber}
            \item \textbf{Detune} \\
               Values: \texttt{-35.00 to 35.00, 0*} \\
               Figure: bottom-panel/instrument-edit/ADD/frequency-detune-type.jpg
            \item \textbf{FREQUENCY slider} \\
               Caption: "Fine Detune (cents)"
            \item \textbf{Octave} \\
               Values: \texttt{-8 to 7, 0*}
            \item \textbf{RelBW} \\
               Caption: "Bandwidth - how the relative fine detune of the voice is changed" \\
               Values: \texttt{0 to 127, 64*}

            % Stock part: Frequency Env (without Enable)

            \item \textbf{Frequency Env} \\
               Figure: bottom-panel/instrument-edit/ADD/ADDsynth-frequency-envelope.jpg

            \begin{enumber}
               \item \textbf{A.val} \\
                  Caption: "Attack Value" \\
                  Values: \texttt{0 to 127, 64*}
               \item \textbf{A.dt} \\
                  Caption: "Attack Time" \\
                  Values: \texttt{0 to 127, 50*}
               \item \textbf{R.dt} \\
                  Caption: "Release Time" \\
                  Values: \texttt{0 to 127, 60*}
               \item \textbf{R.val} \\
                  Caption: "Release Value" \\
                  Values: \texttt{0 to 127, 64*}
               \item \textbf{Stretch} \\
                  Caption: "Envelope Stretch (on lower notes make the envelope longer") \\
                  Values: \texttt{0* to 127}
               \item \textbf{frcR} \\
                  Caption: "Forced Release" \\
                  Values: \texttt{Off*, On}
               \item \textbf{C} \\
                  Caption: "Copy to Clipboard/Preset" \\
                  Figure: bottom-panel/instrument-edit/ADD/copy-to-clipboard-preset.jpg
                  \begin{enumber}
                     \item \textbf{Type} ("envamplitude")
                     \item \textbf{Envelope list ?}
                     \item \textbf{Copy to Preset}
                     \item \textbf{Copy to Clipboard}
                  \end{enumber}
               \item \textbf{P} \\
                  Caption: "Paste from Clipboard/Preset" \\
                  bottom-panel/instrument-edit/ADD/ \\
                     Figure: paste-from-clipboard-preset.jpg
                  \begin{enumber}
                     \item \textbf{Type} ("envamplitude")
                     \item \textbf{Envelope list ?}
                     \item \textbf{Paste from Preset}
                     \item \textbf{Paste from Clipboard}
                  \end{enumber}
               \item \textbf{E} \\
                  Caption: "Frequency Envelope Window" \\
                  Figure: bottom-panel/instrument-edit/ADD/amplitude-envelope.jpg
                  \begin{enumber}
                     \item \textbf{FreeMode}
                  Figure: bottom-panel/instrument-edit/ADD/amplitude-envelope.jpg \\
                        bottom-panel/instrument-edit/ADD/ \\
                           Figure: disable-free-mode-of-envelope.jpg \\
                        Values: \texttt{Off*, On}
                     \begin{enumber}
                        \item \textbf{Add point}
                        \item \textbf{Delete point}
                        \item \textbf{Sust} \\
                           Caption: "Sustain (0 is disabled)" \\
                           Values: \texttt{0, 1, 2*}
                        \item \textbf{Str.} \\
                           Envelope Stretch (on lower notes make the \\
                              envelope longer")
                        \item \textbf{L} \\
                           Caption: "The envelope is linear" \\
                           Values: \texttt{Off*, On}
                        \item \textbf{frcR} \\
                           Caption: "Forced Release"
                     \end{enumber}
                     \item \textbf{C}
                     \item \textbf{P}
                  \end{enumber}
            \end{enumber}

            % Stock part: Frequency LFO (without Enable)

            \item \textbf{Frequency LFO}
            \begin{enumber}
               \item \textbf{Freq.} \\
                  Caption: "LFO Frequency" \\
                  Values: \texttt{0 to 1, 0.54*}
               \item \textbf{Depth} \\
                  Caption: "LFO Amount" \\
                  Values: \texttt{0* to 127}
               \item \textbf{Start}
                  Caption: "LFO start phase (leftmost is random)" \\
                  Values: \texttt{0 to 127, 64*}
               \item \textbf{Delay}
                  Caption: "LFO Delay" \\
                  Values: \texttt{0* to 127}
               \item \textbf{Str.}
                  Caption: "Envelope Stretch (on lower notes makes the
                     envelope longer)" \\
                  Values: \texttt{0 to 127, 64*}
               \item \textbf{C.}
                  Caption: "COntinuous LFO" \\
                  Values: \texttt{Off*, On}
               \item \textbf{A.R.}
                  Caption: "LFO Amplitude Randomness" \\
                  Values: \texttt{0* to 127}
               \item \textbf{F.R.}
                  Caption: "LFO Frequency Randomness" \\
                  Values: \texttt{0* to 127}
               \item \textbf{Type} (use other? picture of dropdown)
                  Caption: "LFO Function" \\
                  Values: \texttt{SINE*, TRI, SQR, R.up, R.dn, E1dn, E2dn}
               \item \textbf{C}
               \item \textbf{P}
            \end{enumber}

            \item \textbf{Detune Type} \\
               Caption: "Frequency Detune Type" \\
               Values: \texttt{L35cents, L10cents, E100cents, E1200cents} \\
               Figure: bottom-panel/instrument-edit/ADD/frequency-detune-type.jpg
            \item \textbf{Coarse det.} \\
               Caption: "Coarse Detune" \\
               Values: \texttt{-64 to 63, 0*}
         \end{enumber}

         \item \textbf{Show Voice Parameters} \\
            Caption: "ADsynth Voice Parameters" \\
            Figure: bottom-panel/instrument-edit/ADD/ADDsynth-voice-parameters.jpg \\
            Figure: bottom-panel/instrument-edit/ADD/modulator-type.jpg
            \begin{enumber}
               \item \textbf{Voice Number} \\
                  Values: \texttt{1* to 8}
               \item \textbf{On} \\
                  Values: \texttt{Off, On}
               \item \textbf{Delay} \\
                  Caption: "Volume" !?
                  Values: \texttt{0* to 127}
               \item \textbf{R.} \\
                  Caption: "Resonance On/Off" \\
                  Values: \texttt{Off, On*}
               \item \textbf{AMPLITUDE} \\
                  Values: \texttt{Off, On*}
               \begin{enumber}
                  \item \textbf{Minus} \\
                     Caption: "Invert volume control action" \\
                     Values: \texttt{Off*, On}
                  \item \textbf{Volume} \\
                     Caption: "Volume" \\
                     Values: \texttt{0 to 127, 90?}
                  \item \textbf{Vel Sens} \\
                     Caption: "Velocity-sensing function - rightmost/max
                        disables" \\
                     Values: \texttt{0 to 127*}
                  \item \textbf{Pan} \\
                     Caption: "Voice panning - leftmost/0 gives random
                        panning" \\
                     Values: \texttt{0 to 127, 64*}
                  \item \textbf{Pan randomness indicator} \\
                     Caption: "Voice random panning On/Off"
                  \item \textbf{Pen reset} (red button)
                     Caption: "Center panning" \\
                  \item \textbf{Amplitude Env, Stock + Enable}
                  \item \textbf{Amplitude LFO, Stock + Enable}
               \end{enumber}
               \item \textbf{FILTER}
               \begin{enumber}
                  \item \textbf{Enable} \\
                     Caption: "Enable Filter" \\
                     Values: \texttt{Off*, On}
                  \item \textbf{Bypass Global F.} \\
                     Caption: "Bypass Global Filter" \\
                     Values: \texttt{Off*, On}
                  \item \textbf{Filter Params, Stock}
                  \item \textbf{Filter Env, Stock + Enable}
                  \item \textbf{Filter LFO, Stock + Enable}
               \end{enumber}
               \item \textbf{MODULATOR}
               \begin{enumber}

                  \item \textbf{Type:} \\
                     Figure: bottom-panel/instrument-edit/ADD/modulator-type.jpg \\
                     \begin{enumber}
                        \item \textbf{OFF}
                        \item \textbf{MORPH}
                        \item \textbf{RING}
                        \item \textbf{PM}
                        \item \textbf{FM}
                        \item \textbf{PITCH}
                     \end{enumber}
                  \item \textbf{External Mod.} \\
                     Values: \texttt{OFF*, Other voices?}
                  \item \textbf{Mod AMPLITUDE}
                     \begin{enumber}
                        \item \textbf{Vol} \\
                           Caption: "Volume" \\
                           Values: \texttt{0 to 127, 90*}
                        \item \textbf{V.Sns}
                           Caption: "Velocity Sensing Function -
                              rightmost/max to disable" \\
                           Values: \texttt{0 to 127, 64*}
                        \item \textbf{F.Damp}
                           Caption: "Modulator Damp at higher frequency" \\
                           Values: \texttt{0 to 127, 90*}
                        \item \textbf{Amplitude Env, Stock + Enable}
                     \end{enumber}
                  \item \textbf{Mod FREQUENCY}
                     \begin{enumber}
                        \item \textbf{Detune slider}
                           Caption: "Fine Detune (cents)" \\
                           Values: \texttt{-35.00 to 35.00, 0*} \\
                        \item \textbf{Detune Type}
                           Caption: "Fine Detune (cents)" \\
                           Values: \texttt{L35cents, L10cents, E100cents, E1200cents} \\
                           Figure: bottom-panel/instrument-edit/ADD/frequency-detune-type.jpg
                        \item \textbf{Octave} \\
                           Caption: "" \\
                           Values: \texttt{-8 to 7, 0*}
                        \item \textbf{Coarse Det.}
                           Caption: "Coarse Detune" \\
                           Values: \texttt{-64 to 63, 0*}
                        \item \textbf{Filter Env, Stock + Enable}
                     \end{enumber}
                  \item \textbf{Mod Oscilator}
                     \begin{enumber}
                        \item \textbf{Change} \\
                           Caption: "ADDsynth Oscillator Editor" (see below)\\
                           Figure: bottom-panel/instrument-edit/ADD/ADDsynth-oscillator-editor.jpg
                           % We need to add a section for this complex
                           % dialog box!

                        \item \textbf{Use} \\
                           Values: \texttt{Internal*, Other oscillators?}
                        \item \textbf{Phase} \\
                           Values: \texttt{0 to 360 (0 to 2PI)}
                        \item \textbf{Waveform graph}
                     \end{enumber}

               \end{enumber}

               % Similar to the ADDsynth Edit's FREQUENCY section.
 
               \item \textbf{FREQUENCY} (section) \\
                  Caption: "Frequency section, almost a stock part"
               \begin{enumber}
                  \item \textbf{Detune} \\
                     Values: \texttt{-35.00 to 35.00, 0*} \\
                     Figure: bottom-panel/instrument-edit/ADD/frequency-detune-type.jpg
                  \item \textbf{FREQUENCY slider} \\
                  \item \textbf{440 Hz} \\         % addition to stock
                     Caption: "" \\
                     Values: \texttt{Off*, On}
                  \item \textbf{Eq.T.} \\          % addition to stock
                     Caption: "" \\
                     Values: \texttt{0 to 127?}
                  \item \textbf{Octave} \\
                     Caption: "" \\
                     Values: \texttt{-8 to 7, 0*}
                  % \item \textbf{RelBW}           % not present here
                  \item \textbf{Detune Type}
                     Caption: "Fine Detune (cents)" \\
                     Values: \texttt{L35cents, L10cents, E100cents, E1200cents} \\
                     Figure: bottom-panel/instrument-edit/ADD/frequency-detune-type.jpg
                  \item \textbf{Coarse det.} \\
                     Caption: "Coarse Detune" \\
                     Values: \texttt{-64 to 63, 0*}
                  \item \textbf{Frequency Env, Stock + Enable}
                  \item \textbf{Frequency LFO, Stock + Enable}
                  \item \textbf{Voice Oscillator}
                  \begin{enumber}
                     \item \textbf{Phase} \\
                        Values: \texttt{0 to 360 (0 to 2PI)}
                     \item \textbf{Use} \\
                        Values: \texttt{Internal*, Other oscillators?}
                     \item \textbf{Waveform graph}
                     \item \textbf{Change} \\
                        Caption: "ADDsynth Oscillator Editor" (see below)\\
                        Figure: bottom-panel/instrument-edit/ADD/ADDsynth-oscillator-editor.jpg
                     \item \textbf{Sound} \\
                        Caption: "Oscillator Type (sound/noise)" \\
                        Figure: bottom-panel/instrument-edit/ADD/voice-oscillator-sound-dropdown.jpg
                     \item \textbf{Unison} \\
                        Values: \texttt{Off*, On}
                        \begin{enumber}
                           \item \textbf{Size} \\
                              Caption: "Number of unison sub-voices" \\
                              Values: \texttt{2* to 50}
                           \item \textbf{Frequency Spread} \\
                              Caption: "Frequency spread of the unison (cents)" \\
                              Values: \texttt{0 to 200, 44.6*}
                           \item \textbf{Ph.rnd} \\
                              Caption: "Phase randomness" \\
                              Values: \texttt{0 to 127*}
                           \item \textbf{Stereo} \\
                              Caption: "Stereo Spread" \\
                              Values: \texttt{0 to 127, 64*}
                           \item \textbf{Vibrato} \\
                              Caption: "Vibrato" \\
                              Values: \texttt{0 to 127, 64*}
                           \item \textbf{V.speed} \\
                              Caption: "Vibrato Average Speed" \\
                              Values: \texttt{0 to 127, 64*}
                           \item \textbf{Invert} \\
                              Caption: "Phase Invert" \\
                              Values: \texttt{None*, Random, 50\%, 33\%, 25\%, 20\%} \\
                              Figure: bottom-panel/instrument-edit/ADD/voice-oscillator-phase-invert-dropdown.jpg
                        \end{enumber}
                        \item \textbf{Current Voice} \\
                           Caption: "Current Voice" \\
                           Values: \texttt{1* to 8}
                        \item \textbf{C} \\
                           Caption: "Copy D note parameters"
                        \item \textbf{P} \\
                           Caption: "Paste D note parameters"
                     \item \textbf{Close Window}
                  \end{enumber}
               \end{enumber}

            \end{enumber}

         \item \textbf{Show Voice List} \\
            Caption: "ADsynth Voices List (Voices 1 to 8)" \\
            Figure: bottom-panel/instrument-edit/ADD/ADDsynth-voices-list.jpg
            \begin{enumber}
               \item \textbf{Number (1 to 8)} \\
                  Values: \texttt{Off, On}
               \item \textbf{Vol} \\
                  Caption: "Volume" \\
                  Values: \texttt{0 to 127, 100*}
               \item \textbf{Pan} \\
                  Caption: "Voice Panning (0/leftmost is Random)" \\
                  Values: \texttt{0 to 127, 64*}
               \item \textbf{R.} \\
                  Caption: "Resonance On/Off" \\
                  Values: \texttt{Off, On*}
               \item \textbf{Detune} \\
                  Caption: "Fine Detune (cents)" \\
                  Values: \texttt{-35 to 35, 0*}
               \item \textbf{Vib. Depth} \\
                  Caption: "Frequency LFO Amount" \\
                  Values: \texttt{0 to 127, 40*}
               \item \textbf{Hide Voice List} \\
            \end{enumber}

         \item \textbf{Resonance} \\
            Figure: bottom-panel/instrument-edit/ADD/ADDsynth-resonance.jpg
            \begin{enumber}
               \item \textbf{Enable} \\
                  Values: \texttt{Off*, On}
               \item \textbf{Max dB (wheel)} \\
                  Caption: "The Maximum Amplitude (dB)" \\
                  Values: \texttt{1 to 90, 20*}
               \item \textbf{C.f. (knob)} \\
                  Caption: "Center Frequency (kHz)" \\
                  Values: \texttt{0 to 127, 64*} for \texttt{0.10 to 10.0, 1.0*}
               \item \textbf{Oct.} \\
                  Caption: "Number of Octaves" \\
                  Values: \texttt{0 to 127, 64*} for \texttt{0 to 10, 5*}
               \item \textbf{P.1st} \\
                  Caption: "Protect the fundamental Frequency (do not damp
                     the first harmonic)" \\
                  Values: \texttt{Off, On}

               % This one is a weird one where mouse movement affects it,
               % but also affects the next field as well.  Oh, kHz and dB.

               \item \textbf{InterpP} \\
                  Caption: "Interpolate the peaks"
               \item \textbf{KHz} \\
                  Caption: "The current frequency on graph"
               \item \textbf{dB} \\
                  Caption: "The current level on graph" \\
                  Values: \texttt{-90 to +90}
               \item \textbf{Zero} \\
                  Caption: "Clear the resonance function"
               \item \textbf{Smooth} \\
                  Caption: "Smooth the resonance function"
               \item \textbf{RND1} \\
                  Caption: "Randomize the resonance function"
               \item \textbf{RND2} \\
                  Caption: "Randomize the resonance function"
               \item \textbf{RND3} \\
                  Caption: "Randomize the resonance function"
               \item \textbf{C}
               \item \textbf{P}
               \item \textbf{Close}
            \end{enumber}

         \item \textbf{C}
         \item \textbf{P}
         \item \textbf{Close}
      \end{enumber}

      \item \textbf{SUBsynth Edit} \\
         Figure: bottom-panel/instrument-edit/SUB/harmonic-type.jpg \\
         Figure: bottom-panel/instrument-edit/SUB/mag-type.jpg \\
         Figure: bottom-panel/instrument-edit/SUB/start-type.jpg \\
         Figure: bottom-panel/instrument-edit/SUB/SUBsynth-parameters.jpg

         % This AMPLITUDE section is re-used (e.g in SUBsynth Parameters)

         \item \textbf{AMPLITIUDE} (section)
         \begin{enumber}
            \item \textbf{Volume} \\
               Values: \texttt{1 to 127, 64*}
            \item \textbf{Vel Sens} \\
               Caption: "Velocity Sensing function, rightmost/max to disable" \\
               Values: \texttt{1 to 127, 64*}
            \item \textbf{Pan} \\
               Caption: "Global panning, leftmost/zero gives random panning" \\
               Values: \texttt{1 to 127, 64*}
            \item \textbf{Rand} \\
               Caption: "Indicator for activation of random panning"
            \item \textbf{Reset (panning)} (red button)

            % Stock part: Amplitude Env (without Enable)

            \item \textbf{Amplitude Env}
         \end{enumber}
         \item \textbf{Harmonics magnitude}
         \begin{enumber}
            \item \textbf{xxxxxx}
         \end{enumber}
         \item \textbf{BANDWIDTH}
         \begin{enumber}
            \item \textbf{xxxxxx}
         \end{enumber}
         \item \textbf{Harmonics bandwidth}
         \begin{enumber}
            \item \textbf{xxxxxx}
         \end{enumber}
         \item \textbf{FREQUENCY}
         \begin{enumber}
            \item \textbf{xxxxxx}
         \end{enumber}
         \item \textbf{OVERTONES}
         \begin{enumber}
            \item \textbf{xxxxxx}
         \end{enumber}
         \item \textbf{FILTER}
         \begin{enumber}
            \item \textbf{Enabled}
            \item \textbf{Filter Params}
            \begin{enumber}
               \item \textbf{xxxxxx}
            \end{enumber}
            \item \textbf{Filter Env}
            \begin{enumber}
               \item \textbf{xxxxxx}
            \end{enumber}
            \item \textbf{xxxxxx}
         \end{enumber}
      \item \textbf{PADsynth Edit} \\
         Figure: bottom-panel/instrument-edit/PAD/amp-mode.jpg \\
         Figure: bottom-panel/instrument-edit/PAD/amp-multiplier.jpg \\
         Figure: bottom-panel/instrument-edit/PAD/bandwidth-scale.jpg \\
         Figure: bottom-panel/instrument-edit/PAD/base.jpg \\
         Figure: bottom-panel/instrument-edit/PAD/base-type.jpg \\
         Figure: bottom-panel/instrument-edit/PAD/export-dialog.jpg \\
         Figure: bottom-panel/instrument-edit/PAD/full-upper-lower.jpg \\
         Figure: bottom-panel/instrument-edit/PAD/harmonic-content-adaptive-harmonic-type.jpg \\
         Figure: bottom-panel/instrument-edit/PAD/harmonic-content-base-function.jpg \\
         Figure: bottom-panel/instrument-edit/PAD/harmonic-content-editor.jpg \\
         Figure: bottom-panel/instrument-edit/PAD/harmonic-content-filter.jpg \\
         Figure: bottom-panel/instrument-edit/PAD/harmonic-content-mag-type.jpg \\
         Figure: bottom-panel/instrument-edit/PAD/harmonic-content-modulation.jpg \\
         Figure: bottom-panel/instrument-edit/PAD/harmonic-content-osc-spectrum-adjust.jpg \\
         Figure: bottom-panel/instrument-edit/PAD/harmonic-content-waveshaping-function.jpg \\
         Figure: bottom-panel/instrument-edit/PAD/number-of-octaves.jpg \\
         Figure: bottom-panel/instrument-edit/PAD/overtones-position.jpg \\
         Figure: bottom-panel/instrument-edit/PAD/PADsynth-parameters-envelopes-LFOs.jpg \\
         Figure: bottom-panel/instrument-edit/PAD/PADsynth-parameters-harmonic-structure.jpg \\
         Figure: bottom-panel/instrument-edit/PAD/resonance.jpg \\
         Figure: bottom-panel/instrument-edit/PAD/sample-size.jpg \\
         Figure: bottom-panel/instrument-edit/PAD/smp-per-octave.jpg \\
         Figure: bottom-panel/instrument-edit/PAD/spectrum-mode.jpg
      \item \textbf{Kit Edit} \\
         Figure: bottom-panel/instrument-edit/Kit/instrument-kit-edit.jpg
      \item \textbf{Effects Edit} \\
         Figure: bottom-panel/instrument-edit/Effects/dynfilter-parameters.jpg \\
         Figure: bottom-panel/instrument-edit/Effects/effects-edit-alienwah.jpg \\
         Figure: bottom-panel/instrument-edit/Effects/effects-edit-chorus.jpg \\
         Figure: bottom-panel/instrument-edit/Effects/effects-edit-distortion.jpg \\
         Figure: bottom-panel/instrument-edit/Effects/effects-edit-dynfilter.jpg \\
         Figure: bottom-panel/instrument-edit/Effects/effects-edit-echo.jpg \\
         Figure: bottom-panel/instrument-edit/Effects/effects-edit-eq.jpg \\
         Figure: bottom-panel/instrument-edit/Effects/effects-edit-none.jpg \\
         Figure: bottom-panel/instrument-edit/Effects/effects-edit-phaser.jpg \\
         Figure: bottom-panel/instrument-edit/Effects/effects-edit-reverb.jpg
   \end{enumber}                    % Instrument Edit Details

\end{enumber}

\end{document}

%-------------------------------------------------------------------------------
% vim: ts=3 sw=3 et ft=tex
%-------------------------------------------------------------------------------
